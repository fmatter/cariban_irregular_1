\section{Explaining conservativeness: a network morphology approach}
\label{sec:motivations}
Perhaps the most well-known contribution regarding conservativeness, innovativeness, and (ir-){}re\-gu\-la\-ri\-ty in the lexicon is \posscite{bybee1985morphology} network model of morphology, which seems well-suited for the data at hand.
It aims \dbqu{to account for cross-linguistic, diachronic and acquisition patterns in complex morphological systems} \parencite[428]{bybee1995regular}.
It does so by modeling shared morphological properties such as inflectional patterns as emerging from connections of differing strength between lexemes.
A classic example is a network of \dbqu{strong} English verbs with \obj{strɪŋ}--\obj{strʌŋ} at the center and pairs like \obj{rɪŋ}--\obj{rʌŋ}, \obj{spɪn}--\obj{spʌn}, or \obj{stɪk}--\obj{stʌk} at its periphery.
This network is attracting new verbs in certain dialects, like \obj{sniːk}--\obj{snʌk} or \obj{brɪŋ}--\obj{brʌŋ} \parencite[129--130]{bybee1985morphology}.
These verbs are recruited based on the lexical connection they form with prototypical members of the group, and accordingly develop irregular or \dbqu{strong} past tense forms.

As possible bases of these connections between lexemes, \textcite[118]{bybee1985morphology} suggests the criteria of semantic, phonological, and morphological similarity; the English strong verbs are an example for a phonologically motivated network.
Another important factor in the model is frequency, since more frequent words have a higher lexical strength \parencite[119]{bybee1985morphology}.
This diminishes the influence from other lexemes, meaning that high-frequency items are more likely to resist innovations.
For the Cariban first person patterns, the model would predict that a) semantically\slash{}phonologically\slash\hspace{0pt}morphologically similar verbs will be affected by person marker extensions, and b) high-frequency verbs will tend to resist these extensions and thus remain conservative.

When considering the groups of verbs with innovative first person markers (those not in \cref{tab:overview}), there are several factors which could serve as the thread connecting a lexical network.
The most obvious similarity is that they all have a reflex of the detransitivizer \detrz \parentext{see e.g. \textcite[112]{meira1998proto} for Taranoan}, a hallmark of \gl{s_a_} verbs \pcref{sec:split}.
%This was already noted by \textcite{meira1998proto} for the group of \PTar verbs taking irregular first person \rc{w-}:
%\begin{quotebox}{\parencite[112]{meira1998proto}}
%	This category includes a small number of stems, among which ‘to go’, ‘to come’, ‘to say’, [...]\footnote{The original list includes \qu{to go down} and \qu{to defecate}. While these verbs are indeed underived \gl{s_a_} verbs in \trio, no irregular first person \rc{w-} can be reconstructed. As their inclusion in the list was most likely an error, we omit them here.} and the copula. These are basically the verbs that are not synchronically or diachronically detransitivized, yet belong to the A conjugation.
%\end{quotebox} % ‘to go down’, ‘to defecate’,
Since all derived \gl{s_a_} verbs begin with reflexes of \rc{ə} or \rc{e} \parentext{see e.g. \textcite[153]{alves2017arara} for \arara}, phonologically based networks are a second possibility.
%A more trivial connection between these other verbs is that they are all \gl{s_a_} verbs, and thus share inflectional (morphological) patterns.
%In the following analysis, the \gl{1}\gl{s_a_} (pre-innovation) prefix will serve as a proxy for fhese inflectional patterns.
%This is made possible by the fact that \gl{s_a_} verbs all had a reflex of \rc{w-} at the time of innovation, making the first person prefix representative for the inflectional class.
%The exception to this is \akuriyo, where innovative \gl{1}\gl{s_a_} \obj{k-} and idiosyncratic \obj{p-} on \qu{to go down} created inflectional subclasses of \gl{s_a_} verbs.
A third common trait is the \gl{s_a_} (sub-)class membership, represented by pre-innovation first person \rc{w-} (\obj{t-} in \akuriyo).
There are no obvious semantic patterns, which is not surprising given the absence of a semantic basis in the split-\gl{s} system \pcref{sec:split}.
Thus, for each extension, there are three hypotheses as to what connected the members of the responsible network: a reflex of \gl{detrz}, their stem-initial phoneme, or a specific \gl{1}\gl{s_a_} prefix.

It is intuitively obvious that many of the conservative verbs in \cref{tab:overview} are high-frequency verbs, which would cause conservativeness according to the network model.
Going beyond intuition is difficult due to the lack of frequency counts for individual lexemes for any Cariban language.
The only statement in the literature is \posscite[75]{courtz2008carib} claim of \kalina underived \gl{s_a_} verbs being the most frequent ones: \dbqu{It is difficult […] to imagine an intransitive or transitive origin for some of the most frequent middle verbs}.
This claim is supported neither by frequency counts nor accompanied by a list of verbs, but that list is likely identical with the five first columns of \cref{tab:overview}, all underived \gl{s_a_} verbs in \kalina.
To improve the situation, a count of \gl{s_a_} verbs in three glossed \apalai texts from \textcite{koehns1994textos} will serve as a second source of frequency information, the results of which are shown in \cref{tab:apalaicounts}.
The count data agree with the above interpretation of \posscite{courtz2008carib} claim: defining \dbqu{high frequency} as having an above average count yields the exact same five verbs.
While it is uncertain that the interpretation of \posscite{courtz2008carib} claim and the small \apalai sample are truly representative of discourse patterns in the Cariban (proto-)languages under discussion, the absence of alternatives necessitates their use as a tool for modeling frequency.

\begin{table}[h]
\centering
\caption[Frequency counts of \gl{s_a_} verbs in \apalai]{Frequency counts of \gl{s_a_} verbs in three \apalai texts from \textcite{koehns1994textos} (163 \gl{s_a_} verbs, 1070 words)}
\label{tab:apalaicounts}
\begin{tabular}[t]{@{}lrr}
\mytoprule
                              Verb &  Count & \% \gl{s_a_} verb tokens \\
\mymidrule
                 \obj{a} \qu{be-1} &     49 &                  30.06\% \\
               \obj{eʃi} \qu{be-2} &     30 &                  18.40\% \\
                 \obj{ka} \qu{say} &     26 &                  15.95\% \\
                 \obj{ɨto} \qu{go} &     23 &                  14.11\% \\
              \obj{oepɨ} \qu{come} &     13 &                   7.98\% \\
       \obj{e-poreʔka} \qu{arrive} &      3 &                   1.84\% \\
           \obj{ot-urupo} \qu{ask} &      2 &                   1.23\% \\
              \obj{ot-uʔ} \qu{eat} &      2 &                   1.23\% \\
     \obj{os-enakũnuʔ} \qu{choke} &      2 &                   1.23\% \\
          \obj{e-unopɨ} \qu{laugh} &      1 &                   0.61\% \\
    \obj{at-akĩma} \qu{pack bags} &      1 &                   0.61\% \\
    \obj{at-ankɨema} \qu{be happy} &      1 &                   0.61\% \\
      \obj{os-ereh} \qu{be amazed} &      1 &                   0.61\% \\
\obj{e-metɨka} \qu{lose loincloth} &      1 &                   0.61\% \\
       \obj{e-tuarima} \qu{suffer} &      1 &                   0.61\% \\
            \obj{e-puka} \qu{fall} &      1 &                   0.61\% \\
          \obj{os-eporɨ} \qu{meet} &      1 &                   0.61\% \\
         \obj{ot-ɨrɨʔka} \qu{land} &      1 &                   0.61\% \\
         \obj{ot-ɨʔka} \qu{finish} &      1 &                   0.61\% \\
            \obj{ot-uru} \qu{talk} &      1 &                   0.61\% \\
    \obj{at-apiaka} \qu{divide up} &      1 &                   0.61\% \\
         \obj{e-sɨrɨʔma} \qu{move} &      1 &                   0.61\% \\
\mybottomrule
\end{tabular}
\end{table}

%\textcite[153]{alves2017arara}: \dbqu{Observe que todos os 12 verbos indexados pelo morfema \obj{k-} [\gl{s_a_}] apresentam a raiz iniciada pela vogal /o/}.

Thus, each of the three hypotheses for possible network factors can be combined with frequency: the members of the lexical network formed by the factor are predicted to undergo innovation, but high-frequency verbs are exempt.
The resulting six hypotheses for possible explanations were tested by predicting the expected behavior of each verb in each extension, illustrated in \cref{tab:ptir-predictions} for \PTir.
For example, \rc{eʔi} \qu{to be} is expected to participate (\checkmark) in innovations spreading in a phonologically defined network (being \rc{e}-initial), as well as in an inflectionally defined one (sharing \rc{w-} with other \gl{s_a_} verbs).
On the other hand, a network based on the detransitivizer would predict it to not take on new suffixes (×).
If frequency is taken into account, it is expected to remain conservative regardless of the basis of the network.

\begin{table}
\centering
\caption{Predictions for \PTir}
\label{tab:ptir-predictions}
\begin{tabular}[t]{@{}lllllll@{}}
\mytoprule
{} &      \rc{a} &    \rc{eʔi} &  \rc{əʔepɨ} &   \rc{təmɨ} &     \rc{ka} &               \rc{epɨ} \\
{} &     \qu{be} &     \qu{be} &   \qu{come} &     \qu{go} &    \qu{say} & \qu{bathe (\gl{intr})} \\
\mymidrule
\gl{detrz}                       &           × &           × &  \checkmark &           × &           × &             \checkmark \\
\gl{detrz}+freq                  &           × &           × &           × &           × &           × &             \checkmark \\
phono (\envr{}{\rc{ə}, \obj{e}}) &           × &  \checkmark &  \checkmark &           × &           × &             \checkmark \\
phono+freq                       &           × &           × &           × &           × &           × &             \checkmark \\
infl (\rc{w-})                   &  \checkmark &  \checkmark &  \checkmark &  \checkmark &  \checkmark &             \checkmark \\
infl+freq                        &           × &           × &           × &           × &           × &             \checkmark \\
\mybottomrule
\end{tabular}
\end{table}

These predictions were checked against the data in \cref{tab:overview}, counting verbs which had their behavior (in-)correctly predicted.
The resulting scores are illustrated for the extension of \PTir \rc{t-} in \cref{tab:ptir-evaluations} and summed up for all extensions in \cref{tab:resultsoverview}.
Notably, the scores in \cref{tab:resultsoverview} only refer to the seven verbs which are attested as resisting at least one extension.
For each extension, there were also many run-of-the-mill \gl{s_a_} verbs which were all affected, except for the \akuriyo \obj{e}-initial verbs.%
\footnote{While there are a few \gl{s_a_} verbs not transparently derived from transitive verbs \parencites[252]{triomeira1999}[222]{meira2000split}[30]{gildea2007greenberg}, which are not featured in \cref{tab:overview}, these are mostly \rc{ə}-initial and were likely productively derived at some point.
	The verbs to which this does not apply, like \trio \obj{wa} \qu{to dance} \parencites[252]{triomeira1999}, are all instances of \gl{s_p_} verbs switching classes.
	Since none of them is attested as being an \gl{s_a_} verb at the time of a person marker extension, they are not relevant for this study.}
To illustrate, if one adds 1'000 simulated derived \gl{s_a_} verbs per language to the data -- a conservative estimate based on \posscite{courtz2008carib} \kalina dictionary -- all six explanations consistently predict the behavior of 99.99+\% verbs correctly.
However, the present investigation is restricted to the edge cases, since the available data simply does not allow such large-scale tests for Cariban languages.

\begin{table}[h]
\centering
\caption{Evaluating predictions for \PTir}
\label{tab:ptir-evaluations}
\begin{tabular}[t]{@{}lllllllr}
\mytoprule
{} &     \rc{ka} & \rc{ɨtə[mɨ]} &      \rc{a} &    \rc{eʔi} &  \rc{əʔepɨ} &    \rc{epɨ} &  Score \\
{} &    \qu{say} &      \qu{go} &   \qu{be-1} &   \qu{be-2} &   \qu{come} &  \qu{bathe} &        \\
\mymidrule
\gl{detrz}+freq &  \checkmark &   \checkmark &  \checkmark &  \checkmark &  \checkmark &  \checkmark & 100.0\% \\
phono+freq      &  \checkmark &   \checkmark &  \checkmark &  \checkmark &  \checkmark &  \checkmark & 100.0\% \\
infl+freq       &  \checkmark &   \checkmark &  \checkmark &  \checkmark &  \checkmark &  \checkmark & 100.0\% \\
\gl{detrz}      &  \checkmark &   \checkmark &           × &  \checkmark &  \checkmark &  \checkmark &  83.3\% \\
phono           &  \checkmark &            × &           × &  \checkmark &  \checkmark &  \checkmark &  66.7\% \\
infl            &           × &            × &           × &           × &           × &  \checkmark &  16.7\% \\
\mybottomrule
\end{tabular}
\end{table}
\begin{table}[h]
\centering
\caption{Overview of prediction accuracy}
\label{tab:resultsoverview}
\begin{tabular}[t]{@{}lrrrrrr}
\mytoprule
{} &  \gl{detrz} &  \gl{detrz}+freq &  phono &  phono+freq &   infl &  infl+freq \\
\mymidrule
\PWai \rc{k-}     &      100.0\% &           100.0\% &  60.0\% &      100.0\% &  20.0\% &     100.0\% \\
\PPek \rc{k-}     &      100.0\% &           100.0\% &  71.4\% &      100.0\% &   0.0\% &      71.4\% \\
\PTir \rc{t-}     &       83.3\% &           100.0\% &  66.7\% &      100.0\% &  16.7\% &     100.0\% \\
\akuriyo \obj{k-} &       66.7\% &            83.3\% & 100.0\% &      100.0\% & 100.0\% &     100.0\% \\
\carijo \obj{j-}  &       60.0\% &            60.0\% & 100.0\% &       60.0\% &  40.0\% &      60.0\% \\
\yukpa \obj{j-}   &       33.3\% &            33.3\% & 100.0\% &       33.3\% &  66.7\% &      33.3\% \\
\mybottomrule
\end{tabular}
\end{table}

The extent of the extensions in both \PWai and \PPek is fully predicted by the presence or absence of a detransitivizer.
In both cases, only the underived%
\footnote{Note that for \PPek, the idiosyncratic evolution of \rc{e-pɨ} \qu{to bathe (\gl{intr})} to \rc{ipɨ}  made the verb morphologically opaque.}
\gl{s_a_} verbs were not affected, all other \gl{s_a_} verbs taking \rc{k-}.
Not shown in \cref{tab:resultsoverview} are subsequent evolutions in the Pekodian daughter languages, which largely support a detransitivizer-based explanation:
First, both \ikpeng and \bakairi regularized the paradigm to use forms with detransitivizer for first person \pcref{sec:pekodian}, and both introduced \obj{k-}.%
\footnote{If one instead assumes that first person \rc{w-ebɨ-} and \rc{k-əd-ebɨ-} already co-existed in \PPek, the clear correlation between \rc{k-} and the detransitivizer remains.}
Second, the development of \PPek \rc{ɨtən} \qu{to go} to \ikpeng \obj{aran} may have led to reanalysis of \obj{ar} as a detransitivizer, accompanied by the introduction of \obj{k-}.

The extent of three extensions (in \akuriyo, \carijo, and \yukpa) is correctly predicted by phonological criteria.
As discussed in \cref{sec:akuriyo}, \akuriyo \obj{k-} only appears on \obj{ə}-initial verbs.
In \carijo, the extension of \obj{j-} affected all \obj{e}- and \obj{ə}-initial verbs, including \obj{eh} \qu{to come} or \obj{et͡ʃi} \qu{to be}, which do not have a detransitivizing prefix.
Only \obj{ka} \qu{say}, \obj{təmə} \qu{go}, and \obj{a} \qu{be-1} did not take on \obj{j-}.
Similarly, the extension of \yukpa \obj{j-} can succinctly be characterized as affecting all vowel-initial verbs; the only attested conservative verb is C-initial \obj{to} \qu{to go}.

Inflectional morphology as a network basis only played a potential role in the case of \akuriyo.
However, this explanation only works if \obj{t-} and \obj{t͡ʃ-} are analyzed as distinct morphemes.
Since they can also be seen phonologically conditioned allomorphs, the prediction is identical to the phonological one.

When additionally considering the assumed conservative effects of frequency, prediction accuracy was improved in 8 cases, stagnated in 7 cases, and worsened in three cases.
These three cases where the tentative model of verb frequency arrives at incorrect predictions are found in \carijo and \yukpa, the only languages to feature innovative markers on the reflexes of \rc{eti} \qu{be-2} or \rc{a[p]} \qu{be-1}.
Overall, including frequency in the model led to improvements, up to 100\% prediction accuracy for all three potential factors in \PTir, as well as for the phonological criterion in \PPek and the inflection criterion in \PWai.

In summary, one can conclude that the patterns of most extensions are correctly predicted not by a single explanation, but rather by 3 to 4 different ones.
The exceptions are \carijo and \yukpa, where a phonologically defined lexical network emerges as the unambiguous winner, while frequency-based explanations fare much worse.
%It may be worth noting that the three extensions which we do not reconstruct to a proto-language (ti.e., which are more recent) are best explained by phonological conditioning factors.
For the other extensions, the network model gives no unambiguous answer to the question of what (combination of) factors caused innovative first person markers to spread the way they did.
This in turn is due to the fact that three of the factors in the model -- detransitivizer, phonology, frequency -- strongly converge in their predictions:
The most frequent \gl{s_a_} verbs are at the same time those without a detransitivizer, and therefore mostly of a different phonological shape than regular \gl{s_a_} verbs.