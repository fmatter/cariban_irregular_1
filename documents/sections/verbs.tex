\section{Resistant verbs from a comparative perspective}
\label{sec:verbs}
In \cref{sec:extensions}, we introduced six distinct extensions of personal prefixes into \gl{1}\gl{s_a_} territory, and identified verbs resistant to each innovation.
The set of unaffected verbs was rather small in most cases, and there is a considerable etymological overlap between the (proto-)language-specific verb groups.
In this section, we present these verbs from a comparative perspective and discuss their reconstructability.
\cref{sec:be} treats both roots of the copula \rc{eti}/\obj{a(p)} \qu{to be}, \cref{sec:say} \rc{ka(ti)} \qu{to say}, \cref{sec:go} \rc{ɨtə(mə)} \qu{to go}, and \cref{sec:come} \rc{(ət)jəpɨ} \qu{to come}; these are all verbs that \textcite{gildea2007greenberg} reconstructed as \gl{s_a_} verbs that were not derived from transitive verbs.
\cref{sec:godown} takes a look at \rc{ɨpɨtə} \qu{to go down}, which is resistant in \PTir and \PPek; \cref{sec:bathe} investigates \PPek \rc{ɨpɨ} \qu{to bathe}, \cref{sec:shit} \trio \qu{to defecate}, and \cref{sec:movement} the \akuriyo movement verbs.
The \obj{e}-initial verbs not affected by the extension of \rc{k-} in \akuriyo \pcref{sec:akuriyo} will not be discussed here, as they are a large and phonologically coherent group.


%\begin{table}
%	\centering
%	\caption{XXX}
%	\label{tab:XXX}
%	\begin{tabular}{@{}llllllllllllll@{}}
%	\mytoprule
%& \PPeks \rc{k-} & \acs{ara_lg} & \acs{ikp_lg} & \acs{bak_lg} & \PWais \rc{k-} & \acs{wai_lg} & \acs{hix_lg} & \PTirs \rc{t-} & \acs{tri_lg} & \acs{aku_lg} & \acs{aku_lg} \obj{k-} & \acs{car_lg} \obj{j(i)-} & \acs{yuk_lg} \obj{j-} \\
%\rc{ɨtə(mə)} \qu{go} & × & × & \checkmark & × & × & (\checkmark) & × & × & × & × & × & × & × \\
%\rc{ka(ti)} \qu{say} & × & × & × & × & × & × & × & × & × & × & × & ? & ? \\
%\rc{ət-epɨ} \qu{come} & \checkmark & -- & \checkmark & \checkmark & -- & -- & -- & × & × & -- & -- & -- & ? \\
%\rc{jəpɨ} \qu{come} & × & × & -- & -- & -- & -- & -- & × & × & × & × & \checkmark & ? \\
%\rc{ɨptə} \qu{go down} & × & × & ? & \checkmark & -- & -- & -- & × & × & × & × & ? & -- \\
%\PPeks\rc{ipɨ} \qu{bathe} & × & × & × & × & -- & -- & -- & -- & -- & -- & -- & -- & -- \\
%\rc{eti} \qu{be} & × & × & × & × & × & × & × & × & × & × & × & \checkmark & \checkmark \\
%\rc{a(p)} \qu{be} & × & -- & -- & × & × & × & × & × & × & × & × & × & \checkmark \\
%	\mybottomrule
%	\end{tabular}
%\end{table}


\subsection{\rc{eti}/\obj{a(p)} \qu{to be}}
\label{sec:be}
For a comprehensive comparative overview for this verb, we refer the reader to \textcite[375--382]{gildea2018reconstructing}, who reconstructs two distinct roots serving as verbs \qu{to be} in modern Cariban languages.
One is the older copula \rc{a(p)}, which can be reconstructed as already having various irregularities in \PC.
The other is a root \rc{eti} reconstructed by \textcites{meira2009property}{gildea2018reconstructing} as originally meaning \qu{to dwell, live}, but serving as a copula in \PC.\footnote{Such a stative, locative source is also suggested by the existence of \obj{it͡ʃi} \qu{to lie down} in \arara \parencite[196]{alves2017arara}.}
Various modern languages use reflexes of these two roots in a suppletive manner, conditioned by person and\slash{}or \gl{tam} value.
Both roots preserved \gl{1}\gl{s_a_} \rc{w-} in \PPek, \PWai, and \PTir (\crefrange{sec:pekodian}{sec:taranoan}).
\akuriyo \obj{eʔi} was not a potential target of innovative \obj{k-} \pcref{sec:akuriyo}, due to the verb being \obj{e}-initial.
\carijo innovated \obj{j-}, but only in the \obj{et͡ʃi} root allomorph \exref{carcop.mivida-12}; the \obj{a} root preserves \obj{w-} \exref{carcop.car-25}.
\yukpa innovated \obj{j-} for reflexes of both \rc{a(p)} and \rc{eti}, which are preserved as encliticized auxiliaries in certain constructions \exref{yukcop}.

\pex<carcop>\carijo
\a<mivida-12>
\begingl
\glpreamble iretibə et͡ʃinəme gərə jet͡ʃiɨ//
\gla ireti-bə et͡ʃi-nə=me gərə j-et͡ʃi-ɨ//
\glb then-from be-\gl{inf}=\gl{attrz} still \gl{1}-be-\gl{pfv}//
\glft \qu{Then I was already grown up.} \parencite[][177]{robayo1989rame}//
\endgl
\a<car-25>
\begingl
\gla əji-marə-ne w-a-e//
\glb \gl{2}-\gl{com}-\gl{pl} \gl{1}-be-\gl{npst}//
\glft \qu{I am with you all.} \parencite[][42]{guerrero2016karihona}//
\endgl
\xe

\ex<yukcop> cliticized copula forms in \yukpa \parencite[143--144]{meira2006syntactic}\\
\begin{tabular}[t]{@{}lll@{}}
& \gl{npst} & \gl{pst}\\
\gl{1} & \obj{=j-a(-s)}&\obj{=j-e}\\
\gl{2} & \obj{=mak(o)}&\obj{=m-e}\\
\gl{3} & \obj{=mak(o)}&\obj{=n-e}\\
\end{tabular}
\xe

\subsection{\rc{ka(ti)} \qu{to say}}
\label{sec:say}
Most reflexes of this verb are simply \obj{ka}, but a fleeting syllable \rc{ti} is reconstructed by \textcite{gildea2007greenberg}, illustrated with modern imperative forms in \exref{kati}.

\pex<kati> \PC \rc{kati-kə} \qu{say!}
\a \apalai \obj{kaʃi-ko} \parencite[35]{koehn1986apalai}
\a \wayana \obj{kai-kə} \parencite[181]{wayanatavares2005}
\a \hixka \obj{kas-ko} \parencite[128]{hixkaryanaderby1985}
\a \panare \obj{kah-kə} \parencite[102]{mattei1994diccionario}
\xe

This verb was not affected by the extensions found in \PPek, \PWai, and \PTir.
It was not a potential target of \akuriyo \obj{k-}; we do not know the first person forms of its \carijo and \yukpa reflexes.

As briefly mentioned in \cref{sec:waiwaian}, \textcite{hixkaryanaderby1985} analyzes this verb as transitive in \hixka.
This analytical choice is not only motivated by avoiding an idiosyncratic intransitive first person prefix \obj{ɨ-}, with the usual prefix being \obj{k-}.
\hixka \obj{ka} also behaves like a transitive verb in other ways, for instance by showing the complementary distribution of the third person marker \obj{n-} and preceding objects -- in this case direct speech or ideophones \exref{hixsay}.

\pex<hixsay>\hixka
\a<hix-119>
\begingl
\glpreamble onɨ wyaro nkekonɨ bɨryekomo, tɨyonɨ wya//
\gla onɨ wjaro n-ka-jakonɨ bɨrʲekomo tɨ-jonɨ wja//
\glb this like \gl{3}-say-\gl{rem}.\gl{cont} boy \gl{cor}-mother \gl{obl}//
\glft \qu{This is what the boy said to his mother.} \parencite[][36]{hixkaryanaderby1985}//
\endgl
\a<ordpl-35>
\begingl
\glpreamble moro ha, ketxkoná hatá.//
\gla moro ha ka-jat͡ʃkonɨ hatɨ//
\glb \gl{med}.\gl{dem}.\gl{inan} \gl{ints} say-\gl{rem}.\gl{cont}.\gl{pl} \gl{hsy}//
\glft \qu{“That one there” they said.} \parencite[][14]{derbyshire1965textos}//
\endgl
\xe
In \exref{hixsay.hix-119}, the prefix \obj{n-} occurs because there is no preceding object (\qu{he said it like this}).
In \exref{hixsay.ordpl-35}, it does not occur, because \qu{they said} is preceded by direct speech.
This complementary distribution is otherwise only found with transitive verbs \parencite[59--60]{gildea1998}.
The verb shows the same pattern, albeit inconsistently, in \trio \parencite[267]{triocarlin2004}.

Further comparative evidence also points to \rc{ka(ti)} \qu{to say} showing transitive traits:
\trio \obj{ka} is characterized as the only intransitive verb being able to take the causative suffix \obj{-po} and the agentive nominalizer \obj{-ne} \parencite[263, 169]{triomeira1999}.
The exceptionality of \obj{ka} \qu{to say} taking \obj{-po} \qu{\gl{caus}.\gl{tr}} has also been noted for \kalina \parencite[82]{courtz2008carib} and \wayana \parencite[258]{wayanatavares2005}.
Reflexes of another transitive causativizer \rc{-metipo} \parencite{gildea2015valency} are found with \obj{ka} in \apalai \parencite[51]{koehn1986apalai} and \waiwai \parencite[52]{waiwaihawkins1998}.
\panare has innovated a gnomic or nonspecific verbal suffix \obj{-ɲe} from the agent nominalizer \rc{-ne} \parencite[184]{gildea1998}.
Its occurrence on \obj{ka} leads \parencite[214]{panarepayne2013} to analyze the verb as transitive, in contrast to \textcite[102]{mattei1994diccionario}, who categorizes it as intransitive.

Our classification of \qu{to say} as an intransitive verb is supported primarily by prefix patterns:
\kalina offers a minimal pair between transitive \obj{ka} \qu{to remove} and intransitive \obj{ka} \qu{to say}, \obj{sikai} \qu{I took it away} vs \obj{wɨkai} \qu{I said} \parencite[288, 45]{courtz2008carib}.\footnote{Interestingly, the \kalina causativized form \obj{kapo} \qu{to make say} does not have the regular \gl{1}>\gl{3} prefix \obj{s(i)-}, but irregular \obj{w(ɨ)-} \parencite[430]{courtz2008carib}.}
Similarly, \qu{to say} in Pekodian languages has a reflex of \gl{1}\gl{s} \rc{w-} \pcref{sec:pekodian}, and not \gl{1}>\gl{3} \obj{s-} (\bakairi) or \rc{ini-} (\PXin).
Additionally, the \gl{s_a_} class marker \obj{w-} occurs on nominalizations in \kalina \exref{kar-84}, and it is probably reflected in vowel length in the \trio \parencite[333]{triomeira1999} and \wayana \parencite[196]{wayanatavares2005} participles.

\ex<kar-84>\kalina \parencite[][202]{courtz2008carib}\\
\begingl
\glpreamble Òmakon \`wa oti ywykàpo kaiko.//
\gla o-ʔma-kon ʔwa oti ɨ-wɨ-ka-ʔpo kai-ko//
\glb \gl{2}-child-\gl{pl} \gl{obl} greeting \gl{1}-\gl{s_a_}-say-\gl{pst}.\gl{nmlz} say-\gl{imp}//
\glft \qu{Pass my greetings to your children.}//
\endgl
\xe
%
Summing up, this verb could be reconstructed as being intransitive based on its prefixes, but transitive based on some suffixes.
\hixka has lost the main intransitive criteria, making its reflex look more like a transitive verb.

\subsection{\rc{ɨtə(mə)} \qu{to go}}
\label{sec:go}
This verb is reconstructed by \textcite{gildea2007greenberg} as \rc{tə(mə)}, with the second syllable only rarely occurring, as in the case of \rc{ka(ti)} \qu{to say}.
It is true that many reflexes are clearly \obj{t}-initial, for example \hixka \obj{ntoje} \qu{he went} \parencite[27]{hixkaryanaderby1985}, \trio \obj{təkə} \qu{go!} \parencite[246]{triomeira1999}, or \wayana \obj{kuptəm} \qu{we went} \parencite[195]{wayanatavares2005}.
However, a comparison of selected forms with unambiguously C-initial \rc{ka(ti)} \qu{to say} suggests that an initial vowel \rc{ɨ} should be reconstructed \exref{gosaycomp},\footnote{Many inflected forms, like e.g.\ \trio \obj{wɨtənne} or \arara \obj{wɨdolɨ} \qu{I went} \parencites[43]{triomeira1999}[153]{alves2017arara} are ambiguous, since an epenthetic \obj{ɨ} breaks up CC clusters on the prefix-verb boundary.} which was subsequently reanalyzed as part of the prefixes in many languages.

\ex<gosaycomp>
\begin{tabular}[t]{@{}lllll@{}}
& go-\gl{imp} & go-\gl{neg} & say-\gl{imp} & say-\gl{neg} \\
\wayana & \obj{ɨtə-kə} & \obj{ɨtə-ra} & \obj{kai-kə} & \obj{ka-ra} \\
\hixka & \obj{ɨto-ko} & \obj{ɨto-hra} & \obj{kas-ko} & \obj{ka-hra} \\
\apalai & \obj{ɨto-ko} & \obj{ɨto-pɨra} & \obj{kaʃi-ko} & \obj{ka-ra} \\
\end{tabular}\\
\parencites[66, 98]{camargo2010wayana}[235, 258]{wayanatavares2005}[47, 54 194]{hixkaryanaderby1985}[65]{derbyshire1965textos}[kuruaz 033, 055]{koehns1994textos}[100]{camargo2002lexico}[107]{koehn1986apalai}
\xe
%

This verb was not affected by the extensions found in \PPek, \PWai, \PTir, \carijo, and \yukpa.
It was not a potential target of \akuriyo \obj{k-}.

\subsection{\rc{(ət-)jəpɨ} \qu{to come}}
\label{sec:come}
\definecolor{come1}{RGB}{161.0,201.0,244.0}
\definecolor{come2}{RGB}{255.0,180.0,130.0}
\definecolor{come3}{RGB}{141.0,229.0,161.0}
\definecolor{come4}{RGB}{255.0,159.0,155.0}
\begin{table}
	\centering
	\caption{Reflexes of \qu{to come}}
    \label{tab:come}
	\begin{tabular}{@{}lll@{}}
	\mytoprule
	Language & Form & Source\\
	\mymidrule
\kaxui & \colorbox{come3}{\obj{ehɨ}} & \perscomm{Spike Gildea}\\
\kaxui & \colorbox{come1}{\obj{johɨ}} & \perscomm{Spike Gildea}\\
\kaxui & \colorbox{come2}{\obj{oohɨ}} & \perscomm{Spike Gildea}\\
\arara & \colorbox{come3}{\obj{ebɨ}} & \textcite[150]{alves2017arara}\\
\arara & \colorbox{come4}{\obj{odebɨ}} & \textcite[150]{alves2017arara}\\
\ikpeng & \colorbox{come4}{\obj{arep}} & \textcite[265]{ikpengpacheco2001}\\
\bakairi & \colorbox{come4}{\obj{əewɨ}} & \textcite[4]{meira2003bakairi}\\
\trio & \colorbox{come4}{\obj{əepɨ}} & \textcite[294]{triomeira1999}\\
\trio & \colorbox{come3}{\obj{epɨ}} & \textcite[294]{triomeira1999}\\
\akuriyo & \colorbox{come3}{\obj{epɨ}} & \textcite[168]{meira1998proto}\\
\carijo & \colorbox{come3}{\obj{ehɨ}} & \textcite[178]{robayo2000avance}\\
\apalai & \colorbox{come4}{\obj{oepɨ}} & \textcite[37]{koehn1986apalai}\\
\kalina & \colorbox{come2}{\obj{opɨ}} & \textcite[429]{courtz2008carib}\\
\maqui & \colorbox{come3}{\obj{ehə}} & \textcite[438]{maquiritaricaceres2011}\\
\akawaio & \colorbox{come4}{\obj{əsipɨ}} & \textcite[160]{stegeman2014akawaio}\\
\akawaio & \colorbox{come1}{\obj{jepɨ}} & \textcite[125]{akawaiocaesar2003}\\
\ingariko & \colorbox{come1}{\obj{jepə}} & \textcite[415]{cruz2005fonologia}\\
\ingariko & \colorbox{come1}{\obj{jə}} & \textcite[299]{cruz2005fonologia}\\
\patamona & \colorbox{come1}{\obj{jepɨ}} & \perscomm{Spike Gildea}\\
\patamona & \colorbox{come1}{\obj{jəpɨ}} & \perscomm{Spike Gildea}\\
\pemon & \colorbox{come1}{\obj{jepɨ}} & \textcite[102]{alvarez2000construcciones}\\
\macushi & \colorbox{come1}{\obj{ipɨ}} & \textcite[32]{macushiabbott1991}\\
\panare & \colorbox{come2}{\obj{əpɨ}} & \textcite[65]{panarepayne2013}\\
\yawarana & \colorbox{come2}{\obj{əpɨ}} & \textcite[68]{mendez1959yawarana}\\
\mapoyo & \colorbox{come3}{\obj{epɨ}} & \textcite[74]{muller1975mapoyo}\\
\uxc & \colorbox{come2}{\obj{ee}} & \textcite[182]{meira2005southern}\\
\mybottomrule
	\end{tabular}
\end{table}
This verb is reconstructed as monomorphemic \rc{ətepɨ} by \textcite[30]{gildea2007greenberg}, but an inspection of all the attested reflexes \pcref{tab:come} suggests a more complex story.
In particular, the Pemongan languages and one of its allomorphs in \kaxui suggest a root \colorbox{come1}{\rc{jəpɨ}}.
However, the one form in \kaxui only occurs in the Progressive \exref{kaxprog.kax-82}, and \obj{j(-)} may be a reflex the \settwo third person marker \rc{i-} combined with  the \gl{s_a_} class marker \rc{w-}, as suggested by the occurrence of the latter with other person values \exref{kaxprog.kax-54}.
On the other hand, while C-initial verbs do show a clear reflex of third person \rc{i-} \exref{kaxprog.kax-77}, regular V-initial verbs do not show \obj{j-}, but Ø \exref{kaxprog.kax-83}.
Thus, it seems more likely that the \obj{j} is indeed part of the verb, rather than an outcome of \rc{i-w-}.

\pex<kaxprog>\kaxui \perscomm{Spike Gildea}
\a<kax-82>
\begingl
\gla johɨ-rɨ//
\glb \gl{3}.come-\gl{prog}//
\glft \qu{S/he is coming.}//
\endgl
\a<kax-54>
\begingl
\gla o-w-ohɨ-rɨ//
\glb \gl{2}-\gl{s_a_}-come-\gl{prog}//
\glft \qu{You are coming.}//
\endgl
\a<kax-77>
\begingl
\gla i-nkɨ-rɨ//
\glb \gl{3}-sleep-\gl{prog}//
\glft \qu{S/he is sleeping.}//
\endgl
\a<kax-83>
\begingl
\gla Ø-osone-rɨ//
\glb \gl{3}-dream-\gl{prog}//
\glft \qu{S/he is dreaming.}//
\endgl
\xe

There is further evidence supporting the reconstruction of both \rc{j} and \rc{ə}.
First, there are many forms reflecting \colorbox{come2}{\rc{əpɨ}}, occurring in different branches of the family.\footnote{\obj{e} is the regular outcome of \rc{ə} in \uxc; \rc{e} became \obj{i} \parencite[176]{meira2005southern}.}
Second, there are also forms reflecting \colorbox{come3}{\rc{epɨ}}, and umlaut of \rc{ə} to \rc{e} is conditioned by preceding \rc{j} \parencite{meira2010origin}.
That is, a unifying account of these forms starts from a root \rc{jəpɨ}, which is the subject to two major sound changes: \begin{inlinelist}
 \item \rc{j}-loss
 \item \rc{ə}-umlaut	
 \end{inlinelist}.
However, these sound changes appear to have applied irregularly, and not always in the same order.
For example, the \kalina form \obj{opɨ} can only be explained if \rc{j} was lost before the umlaut of \rc{ə} to \rc{e}, which happened elsewhere after \rc{j}.
On the other hand, forms like \maqui \obj{ehə} must be the result of \phonl{\rc{ə}}{\rc{e}}{\rc{j}}, with subsequent loss of \rc{j}.

we have so far ignored the forms which form the basis for \posscite{gildea2007greenberg} \rc{ətepɨ}; these are compatible with our reconstructed root \rc{jəpɨ}, with the addition of the detransitivizer \rc{ət(e)-}: \colorbox{come4}{\rc{ətjəpɨ}}.
In fact, the \obj{i} in the \akawaio form \obj{əsipɨ} is better explained by a sequence \rc{jə}, which has the same outcome \obj{i} in the \macushi reflex of \rc{jəpɨ}, \obj{ipɨ}.
While the semantics of combining a detransitivizer with an intransitive verb do not really make sense, some historical \gl{s_p_} verbs are attested as adding the detransitivizer to become \gl{s_a_} verbs.
For example, \gl{s_p_} \rc{wənɨkɨ} \qu{to sleep} becomes \trio \obj{əənɨkɨ} \parencite[252]{triomeira1999} and \kalina \obj{əʔnɨkɨ} \parencite[429]{courtz2008carib}, both \gl{s_a_}.
\waiwai \qu{go to sleep} can be \obj{wɨnɨk} \parencite[30]{waiwaihawkins1998} or \obj{et-wɨnɨk} \parencite[204]{hawkins1953waiwai}.
The parallels to \qu{to sleep} end here, since bare \rc{jəpɨ} \qu{to come} apparently already was an \gl{s_a_} verb, as evidenced by its status in \kaxui, \kalina, \panare \exref{pan-128}, \arara, \trio, and \akuriyo.

\ex<pan-128>\panare \parencite[][65]{panarepayne2013}\\
\begingl
\gla ju-w-əəpɨ-n ka=m kanoʔ//
\glb \gl{3}-\gl{s_a_}-come-\gl{nspec} \gl{q}=\gl{2}.\gl{aux} rain//
\glft \qu{Do you think it is gonna rain?}//
\endgl
\xe

While these sound changes and the addition of \rc{ət(e)-} do account for the majority of the forms in \cref{tab:come},\footnote{The only exception is the \apalai form \obj{oepɨ}, where the detransitivizer would have the reflex \obj{os-} \parencite[506]{meira2010origin}. While \obj{oepɨ} would be a regular outcome of \rc{ə-jəpɨ}, the \envr{}{C} allomorph of the detransitivizer is \obj{e-}. The form may be due to borrowing from \trio, which has lost intervocalic \rc{t} to create \obj{əepɨ}. Alternatively, \apalai \obj{oepɨ} could be a fossilized loan \wayana, which has lost its reflex of \rc{ətjəpɨ}, but where regular sound changes would also have resulted in the loss of intervocalic \rc{t} \parencite[63]{wayanatavares2005}.}
the distribution within the family is rather chaotic.
Not only do very closely related languages show different forms, like \yawarana and \mapoyo, but distinct forms can even be found within the same language, usually conditioned by different prefixes.
This was discussed in \cref{sec:pekodian} for \arara, which has reflexes of \rc{ətjəpɨ} and \rc{jəpɨ} within the same paradigm.
A similar situation is found in \trio, where the \setone paradigm shows a reflex of \rc{ətjəpɨ} for first, but of \rc{epɨ} (< \rc{jəpɨ}) for the other persons \exref{tricome}.\footnote{While the \gl{1+2} form is a regular outcome of \rc{kɨt-epɨ}, the second person form is mysterious.}

\ex<tricome> \trio \parencite[294]{triomeira1999}\\
\begin{tabular}[t]{@{}ll@{}}
\gl{1} & \obj{w-\colorbox{come4}{əepɨ}} \\
\gl{2} &  \obj{mən-\colorbox{come3}{epɨ}} \\ 
\gl{1+2} &  \obj{ke-\colorbox{come3}{epɨ}} \\
\gl{3} &  \obj{n-\colorbox{come3}{epɨ}} \\
\end{tabular}
\xe
%
Summing up, this verb is highly irregular, both from a synchronic and diachronic perspective.
It seems that reflexes of the detransitivizer \rc{ət(e)-} were optionally added to an \gl{s_a_} verb root \rc{jəpɨ}, which further underwent umlaut and loss of \rc{j}, but in no systematic manner, resulting in the chaotic picture in \cref{tab:come}.

As discussed in \cref{sec:pekodian}, innovative \rc{k-} was introduced on the \arara reflex of \rc{jəpɨ}, but not on the \ikpeng and \bakairi reflexes of \rc{ətjəpɨ}.
Both reflexes of \rc{ətjəpɨ} (\trio) and of \rc{jəpɨ} (\akuriyo) resisted the introduction of \rc{t-} in \PTir.
\carijo \obj{ehɨ} shows innovative \obj{j-}, rather than conservative \obj{w-} \exref{car-18}.
It is unknown whether there is a \yukpa reflex of this verb, and it was fully replaced in \PWai by \rc{omokɨ} \qu{to come}.

\ex<car-18>\carijo \parencite[][102]{guerrero2019carijo}\\
\begingl
\gla əji-wa-e j-eh-ɨ//
\glb \gl{2}-search-\gl{sup} \gl{1}-come-\gl{pfv}//
\glft \qu{I came looking for you.}//
\endgl
\xe

\subsection{\rc{ɨpɨtə} \qu{to go down}}
\label{sec:godown}
Reflexes of this verb were not affected by the extensions of \rc{t-} in \PTir \pcref{sec:taranoan} and \obj{k-} in \akuriyo \pcref{sec:akuriyo}.
Rather, an irregular first person form \rc{p-ɨptə} can be reconstructed for \PTir.
While it originally resisted the extension of \rc{k-} in \PPek, \bakairi subsequently introduced it \pcref{sec:pekodian}.
It was also affected by the extension of \obj{j-} in \carijo \exref{car-32}.

\ex<car-32>\carijo \perscommpar{David Felipe Guerrero}\\
\begingl
\gla irə wat͡ʃinakano tae j-ehɨtə-e//
\glb \gl{inan}.\gl{ana} body.of.water along.bounded \gl{1}-go.down-\gl{npst}//
\glft \qu{…I go down through that guachinacán.}//
\endgl
\xe
%
The situation in \yukpa is unclear, as it is an open question whether \obj{ew(uh)tu} \qu{to go down} \parencite{meira2003primeras} is cognate.

\begin{table}
	\centering
	\caption{Reflexes of \rc{ɨpɨtə} \qu{to go down}}
	\label{tab:descend_cog}
	\begin{tabular}{@{}llll@{}}
	\mytoprule
Language & Verb & Class & Source \\
\mymidrule
\kaxui & \obj{ɨhɨto} & \gl{s_p_} & \perscomm{Spike Gildea} \\
\hixka & \obj{hto} & ? & \textcite[196]{hixkaryanaderby1979}\\
\waiwai & \obj{hto} & -- & \textcite[55]{waiwaihawkins1998}\\
\PPek & \rc{ɨptə} & \gl{s_a_} & \cref{sec:pekodian}\\
\PTir & \rc{ɨhtə}  (\gl{1} \rc{p-})& \gl{s_a_} & \cref{sec:taranoan} \\
\carijo & \obj{ehɨtə} & -- & \textcite[118]{guerrero2019carijo}\\
\wayana & \obj{ɨptə} & \gl{s_a_}/\gl{s_p_} & \textcite[44]{camargo2010wayana}\\
\apalai & \obj{ɨhto} & \gl{s_p_} & \textcite[99]{camargo2002lexico} \\
\kalina & (\obj{onɨʔto}) & \gl{s_a_} & \textcite[263]{courtz2008carib}\\
\maqui & \obj{əʔtə} & \gl{s_p_} & \textcite[450]{maquiritaricaceres2011}\\
\kapon & (\obj{uʔtə}) & -- & \textcite[139]{stegeman2014akawaio}\\
\pemon & (\obj{uʔtə}) & -- & \textcite[139]{alvarez2008clausulas}\\
\macushi & (\obj{autɨ}) & -- & \textcite[34]{macushiabbott1991}\\
\panare & \obj{əhtə} & \gl{s_a_} & \textcite[88]{mattei1994diccionario}\\
\yawarana & \obj{əhtə} & -- & \textcite[68]{mendez1959yawarana}\\
\waimiri & \obj{ɨtɨ} & -- & \textcite[58]{bruno1996dictionary}\\
	\mybottomrule
	\end{tabular}
\end{table}


A comparative view \pcref{tab:descend_cog} shows that while a form \rc{ɨpɨtə} can be reconstructed to \PC, the class of this verb is unclear at first sight.
The reflexes are fairly evenly split between \gl{s_a_} and \gl{s_p_} -- in those languages that preserve the split-\gl{s} system.
In one language, \wayana, the verb shows traits of both classes, leading us to consider it a \dbqu{mixed} verb.
It takes the first and second person \gl{s_p_} markers \obj{j-} and \obj{əw-} \parencite[200]{wayanatavares2005}, but the \gl{1+2}\gl{s_a_} marker \obj{kut-} \parencite[206]{wayanatavares2005}.
It also shows the \gl{s_a_} class marker \obj{w-} in nominalizations \exref{waygodown.way-73}, but behaves like an \gl{s_p_} verb in taking a second person prefix in imperatives \exref{waygodown.way-71}.

\pex<waygodown>\wayana \parencite[][200]{wayanatavares2005}
\a<way-73>
\begingl
\glpreamble ïwïptëë//
\gla ɨ-w-ɨptə-rɨ//
\glb \gl{1}-\gl{s_a_}-go.down-\gl{nmlz}//
\glft \qu{my going down}//
\endgl
\a<way-71>
\begingl
\gla əw-ɨptə-k//
\glb \gl{2}-go.down-\gl{imp}//
\glft \qu{Go down!}//
\endgl
\xe
%
Its causativized form is \obj{ɨptəka} \parencite[255]{wayanatavares2005}, containing a reflex of the transitivizer \rc{-ka}, which was restricted to \gl{s_p_} verbs in \PC \parencite{gildea2019overview}.

Further comparative evidence suggests that \rc{ɨpɨtə} \qu{to go down} was originally an \gl{s_p_} verb which switched its class to \gl{s_a_} in some languages, incompletely so in \wayana.
The \arara causativized form is \obj{eniptoŋ} \parencite[66]{alves2017arara}, and \kalina has a cognate form \obj{enɨʔto} \parencite[263]{courtz2008carib}; \obj{onɨʔto} \qu{to go down} in \cref{tab:descend_cog} is a detransitivized form thereof, lit.\ \qu{to get oneself down}.
Both causativized forms contain a reflex of the (rare) transitivizer \rc{en-}, which was usually found with \gl{s_p_} verbs \parencite{gildea2019overview}.
Besides the irregular first person \obj{p-}, \trio \obj{ɨhtə} shows other irregularities, in particular in its causativized forms \parencite[263]{triomeira1999}.
Thus, it seems that this verb was originally \gl{s_p_}, but then switched its class in four and a half languages of the family, for so far unknown reasons.




\subsection{\PPek \rc{ipɨ} \qu{to bathe}}
\label{sec:bathe}
This verb only emerged as resisting an extension in the case of \PPek \pcref{sec:pekodian}.
Verbs for intransitive \qu{to bathe} are usually based on transitive verbs in Cariban languages, which are reflexes of \rc{pɨ}, or \rc{kupi} in Pemongan, \panare, \kalina, and \maqui\footnote{For \maqui, note that while intransitive \obj{eʔhi} points to \rc{e-kupi}, transitive \obj{ɨhɨ} looks like a reflex of \rc{pɨ}.} \pcref{tab:bathe}.
As we have shown in \cref{sec:pekodian}, \PPek can be reconstructed as having the pair \rc{ipɨ} (\gl{intr}) / \rc{ɨp(ɨ)} (\gl{tr}).
Thus, while \PPek \qu{to bathe (\gl{tr})} has perfectly regular cognates in other languages of the family, intransitive \qu{to bathe} is divergent in this branch, changing \rc{e-} to \rc{i}.
The reasons for this are unknown; we are not aware of \rc{i-} as a regular development of the detransitivizer in Pekodian, see also \textcite[506]{meira2010origin}.
However, it should be noted that other languages also show unexpected developments in this verb, consider the apparent addition of \obj{ew-} in \hixka or the chaotic mixture of \rc{pɨ} and \rc{kupi} in languages spoken in Venezuela.

\begin{table}
	\centering
	\caption{Intransitive and transitive \qu{to bathe}}
    \label{tab:bathe}
	\begin{tabular}{@{}llll@{}}
	\mytoprule
	Language & \gl{intr} & \gl{tr} & Source\\
	\mymidrule
\kaxui & \obj{eehɨ} & \obj{ɨhɨ} & \perscomm{Spike Gildea}\\
\hixka & \obj{ewehɨ} & \obj{ɨhɨ} & \textcites[198]{hixkaryanaderby1979}\\
\waiwai & \obj{ejepu} & \obj{pɨ} & \textcites[166, 192]{waiwaihawkins1998}\\
\arara & \obj{ibɨ} & \obj{ɨp} & \textcites[150, 162]{alves2017arara}\\
\ikpeng & \obj{ip} & \obj{ɨp} & \textcites[103]{ikpengpacheco1997}[123]{campetela1997analise}\\
\bakairi & \obj{i} & \obj{ɨ} & \textcites[4]{meira2003bakairi}[285]{meira2005bakairi}\\
\trio & \obj{epɨ} & \obj{pɨ} & \textcites[697]{triomeira1999}\\
\akuriyo & \obj{epɨ} & \obj{pɨ} & \textcites[87]{gildea1994akuriyo}\\
\wayana & \obj{epɨ} & \obj{upɨ} & \textcites[24, 52]{camargo2010wayana}\\
\apalai & \obj{epɨ} & \obj{pɨ} & \textcites[218]{meira2000split}\\
\kalina & \obj{ekupi} & \obj{kupi} & \textcites[304]{courtz2008carib}\\
\maqui & \obj{eʔhi} & \obj{ɨhɨ} & \textcites[439, 454]{maquiritaricaceres2011}\\
\kapon & \obj{ekuʔpi} & \obj{kuʔpi} & \textcites[37]{stegeman2014akawaio}\\
\pemon & \obj{ekupɨ} & \obj{pɨ} & \textcites[34, 129]{pemondearmellada1944dic}\\
\panare & \obj{akupɨ} & \obj{ɨpɨ}/\obj{kupɨ} & \textcites[8, 294]{mattei1994diccionario}\\
\mybottomrule
	\end{tabular}
\end{table}

\subsection{\trio \obj{weka}/\obj{oeka} \qu{to defecate}}
\label{sec:shit}
As in the case of \qu{to go down} \pcref{sec:godown}, \rc{weka} \qu{to defecate} was likely originally \gl{s_p_}, with most languages showing an \gl{s_p_} reflex \pcref{tab:shit_cog}.
Another parallel to \qu{to go down} is the mixed \gl{s_a_}/\gl{s_p_} status in \wayana.
The \bakairi reflex \obj{əeke} seems to contain a reflex of \rc{ət-}, which would explain its status as an \gl{s_a_} verb.
While the class memberships of \panare \obj{aiʔka} and \obj{iʔka} \qu{to defecate} are both unknown, the first form seems to contain a reflex of the detransitivizer, but not the second form.

\begin{table}
	\centering
	\caption{\rc{weka} \qu{to defecate} as a class-switching \gl{s_p_} verb}
	\label{tab:shit_cog}
	\begin{tabular}{@{}llll@{}}
	\mytoprule
Language & Verb & Class & Source \\
\mymidrule
\kalina & \obj{uweka} & \gl{s_p_} & \textcites[418]{courtz2008carib} \\
\maqui & \obj{weka} & \gl{s_p_} & \textcites[455]{maquiritaricaceres2011} \\
\kaxui & \obj{weka} & \gl{s_p_} & \perscomm{Spike Gildea} \\
\arara & \obj{watke} & \gl{s_p_} & \textcite[44]{souza1993arara}\\
\ikpeng & \obj{atke} & \gl{s_p_} & \textcite[118]{alves2013verbo}\\
\wayana & \obj{uika} & \gl{s_a_}/\gl{s_p_} & \textcite[86, 206]{wayanatavares2005}\\
\trio & \obj{weka} (\gl{1} \obj{koeka}) & \gl{s_a_}  & \textcites[294]{triomeira1999} \\
\bakairi & \obj{əeke} & \gl{s_a_} & \parencite{meira2005bakairi} \\
\apalai & \obj{weka} & ? & \textcites[96]{camargo2002lexico} \\
\panare & \obj{(a)iʔka} & ? & \textcite[319]{mattei1994diccionario}\\
	\mybottomrule
	\end{tabular}
\end{table}



\subsection{The \akuriyo movement verbs}
\label{sec:movement}
As discussed in \cref{sec:akuriyo}, four \akuriyo \gl{s_a_} verbs are attested as not having the first person marker \obj{t͡ʃ-} in \textcite{gildea1994akuriyo}; instead, they have \obj{w-} or its phonologically reduced form ∅.
These verbs share two properties: they are all \obj{e-}initial, and they are all movement verbs.
However, there are also \obj{e-}initial movement verbs with a first person marker \obj{t͡ʃ-} \exref{aku-106}.
As for their etymology, we can only speak to \obj{erama} \qu{to return}, which is a detransitivized form of \obj{rama} \qu{to put back}, a pair which is also found in other Cariban languages.
For the other three verbs, we have not encountered potential cognates, but given their \obj{e-}initial nature and their semantics, it seems plausible that they, too, are derived from transitive verbs with meanings like \qu{to put up} etc.
In any case, \akuriyo is only sparsely documented, and there are many open questions about it.

\ex<aku-106>\akuriyo \parencite[][84]{gildea1994akuriyo}\\
\begingl
\gla t͡ʃ-ewai//
\glb \gl{1}-sit.down//
\glft \qu{I sat down.}//
\endgl
\xe