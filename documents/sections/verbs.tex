\subsection{Conservative verbs in comparison}
\label{sec:verbs}
In \cref{sec:extensions}, six incomplete extensions of personal prefixes into \gl{1}\gl{s_a_} territory were presented, and the verbs resistant to these innovations were identified.
The set of unaffected verbs is rather small in most cases, and (proto-)languages show a considerable degree of overlap in what verbs are unaffected.
Here,  resistant verbs are investigated from a comparative perspective:
\rc{ka[ti]} \qu{to say} \pcref{sec:say}, \rc{ɨtə[mə]} \qu{to go} \pcref{sec:go}, both roots of the copula \rc{eti}/\obj{a[p]} \pcref{sec:be}, \rc{(ət)jəpɨ} \qu{to come} \pcref{sec:come},  \rc{ɨpɨtə} \qu{to go down} \pcref{sec:godown}, and  \cref{sec:bathe} investigates \rc{e-pɨ} \qu{to bathe}, of which the \PPek reflex \rc{ipɨ} resisted the extension of \rc{k-}.
The \obj{e}-initial verbs not affected by the extension of \obj{k-} in \akuriyo \pcref{sec:akuriyo} will not be discussed here, as they are a large and phonologically coherent group.

\subsubsection{\rc{ka[ti]} \qu{to say}}
\label{sec:say}
Most reflexes of this verb are simply \obj{ka}, but a fleeting syllable \rc{ti} is reconstructed by \textcite{gildea2007greenberg}, best visible in the imperative forms of some languages.% \exref{kati}.
%
%\ex<kati> \apalai\\
%\obj{kaʃi-ko} \qu{say!}\\
%\parencite[35]{koehn1986apalai}
%\xe
%
\cref{tab:say} shows a comparison of the longest attested forms for each language.%
\footnote{Cognate segments in \crefrange{tab:say}{tab:bathe} were aligned automatically with LingPy \parencite{lingpy268}, for easier recognition of correspondences.}
%
\begin{table}
\centering
\caption[Reflexes of \rc{ka[ti]} \qu{to say}]{Reflexes of \rc{ka[ti]} \qu{to say} \parencites[4]{meira2003bakairi}[48]{franchetto2008absolutivo}[209]{ikpengpacheco2001}[153]{alves2017arara}[182]{hixkaryanaderby1985}[113]{meira1998proto}[107]{koehn1986apalai}[26]{waiwaihawkins1998}[66]{camargo2010wayana}[59]{macushiabbott1991}[123]{swiggers2010gramatica}[430]{courtz2008carib}[125]{akawaiocaesar2003}[102]{mattei1994diccionario}[63; p.c., Spike Gildea]{largo2011yukpa}}
\label{tab:say}
\begin{tabular}[t]{@{}llllll@{}}
\mytoprule
Language & \multicolumn{5}{l}{Form} \\
\midrule
\kaxui   &   \obj{ka[s]} &  k &  a &  s &    \\
\PWai    &    \rc{ka[s]} &  k &  a &  s &    \\
\hixka   &   \obj{ka[h]} &  k &  a &  h &    \\
\waiwai  &   \obj{ka[s]} &  k &  a &  s &    \\
\PPek    &       \rc{ke} &  k &  e &    &    \\
\arara   &      \obj{ke} &  k &  e &    &    \\
\ikpeng  &      \obj{ke} &  k &  e &    &    \\
\bakairi &      \obj{ke} &  k &  e &    &    \\
\PTir    &       \rc{ka} &  k &  a &    &    \\
\trio    &      \obj{ka} &  k &  a &    &    \\
\akuriyo &      \obj{ka} &  k &  a &    &    \\
\carijo  &      \obj{ka} &  k &  a &    &    \\
\wayana  &   \obj{ka[i]} &  k &  a &    &  i \\
\apalai  &  \obj{ka[ʃi]} &  k &  a &  ʃ &  i \\
\kalina  &      \obj{ka} &  k &  a &    &    \\
\kapon   &      \obj{ka} &  k &  a &    &    \\
\pemon   &      \obj{ka} &  k &  a &    &    \\
\macushi &      \obj{ka} &  k &  a &    &    \\
\panare  &   \obj{ka[h]} &  k &  a &  h &    \\
\uxc     &      \obj{ki} &  k &  i &    &    \\
\yukpa   &      \obj{ka} &  k &  a &    &    \\
\bottomrule
\end{tabular}
\end{table}%
%
This verb was not affected by any of the extensions found in \PPek, \PWai, \PTir, \akuriyo, or \carijo (\crefrange{sec:pekodian}{sec:carijo}).
The first person form of its \yukpa reflex \obj{ka} is unattested.

As mentioned in \cref{sec:waiwaian}, \textcite{hixkaryanaderby1985} analyzes this verb as transitive in \hixka.
This analytical choice is not only motivated by a desire to avoid an idiosyncratic intransitive first person prefix \obj{ɨ-} instead of regular \obj{kɨ-}.
\hixka \obj{ka} also shows the complementary distribution of third person  \obj{n-} and preceding objects typical of transitive verbs in Cariban \parencite[60--81]{gildea1998}.
Due to the semantics of \qu{to say}, these objects are either ideophones or direct speech \exref{hixsay}.

\pex<hixsay>\hixka
\a<hix-119>
\begingl
\glpreamble onɨ wyaro nkekonɨ bɨryekomo, tɨyonɨ wya//
\gla onɨ wjaro n-ka-jakonɨ bɨrʲekomo tɨ-jonɨ wja//
\glb this like \gl{3}-say-\gl{rem}.\gl{cont} boy \gl{cor}-mother \gl{obl}//
\glft \qu{This is what the boy said to his mother.} \parencite[][36]{hixkaryanaderby1985}//
\endgl
\a<ordpl-35>
\begingl
\glpreamble moro ha, ketxkoná hatá.//
\gla moro ha ka-jat͡ʃkonɨ hatɨ//
\glb \gl{med}.\gl{dem}.\gl{inan} \gl{ints} say-\gl{rem}.\gl{cont}.\gl{pl} \gl{hsy}//
\glft \qu{“That one there” they said.} \parencite[][14]{derbyshire1965textos}//
\endgl
\xe
In \exref{hixsay.hix-119}, the prefix \obj{n-} occurs because there is no preceding object (\qu{he said it like this}).
It does not occur in \exref{hixsay.ordpl-35}, because \qu{they said} is preceded by direct speech.
Looking beyond \hixka, at least the \trio cognate shows the same pattern, albeit inconsistently so \parencite[267]{triocarlin2004}.

Derivational suffixes also point to \rc{ka[ti]} \qu{to say} being transitive:
\trio \obj{ka} is characterized as the only intransitive verb being able to take the causative suffix \obj{-po} and the agentive nominalizer \obj{-ne} \parencite[263, 169]{triomeira1999}.
The exceptionality of \obj{ka} \qu{to say} taking \obj{-po} \qu{\gl{caus}.\gl{tr}} has also been noted for \kalina \parencite[82]{courtz2008carib} and \wayana \parencite[258]{wayanatavares2005}.
The agent nominalizer \rc{-ne} gave rise to the \panare gnomic verbal suffix \obj{-ɲe} on transitive verbs \parencite[184--185]{gildea1998}.
The occurrence of \obj{-ɲe} on \obj{ka} likely led \textcite[214]{panarepayne2013} to categorize it as transitive, contrasting with the intransitive analysis by \textcite[102]{mattei1994diccionario}.
Finally, reflexes of the causativizer \rc{-metipo}, usually restricted to transitive verbs \parencite{gildea2015valency}, are found with \obj{ka} in \apalai \parencite[51]{koehn1986apalai} and \waiwai \parencite[52]{waiwaihawkins1998}.

The categorization of \qu{to say} as an intransitive verb is supported primarily by its person prefixes.
\kalina offers a minimal pair between transitive \obj{ka} \qu{to remove} and intransitive \obj{ka} \qu{to say}, \obj{sikai} \qu{I took it away} vs \obj{wɨkai} \qu{I said} \parencite[288, 45]{courtz2008carib}.
Similarly, \qu{to say} has a reflex of \gl{1}\gl{s_a_} \rc{w-} in Pekodian languages \pcref{sec:pekodian}, rather than \gl{1}>\gl{3} \obj{s-} (\bakairi) or \rc{ini-} (\PXin).
Additionally, languages which differentiate direct prefixes from \gl{s_a_} prefixes by the presence of \obj{i} \parencite[495]{meira2010origin} show \obj{ɨ} instead of \obj{i} for this verb, see \akuriyo in \exref{sayintr.aku-154}, as well as \textcites[294]{triomeira1999}[195]{wayanatavares2005}[288]{ikpengpacheco2001}[150]{alves2017arara}[168]{hoff1968carib} for cognate forms in other such languages.
Finally, the \gl{s_a_} class marker \obj{w-} occurs on nominalizations in \kalina \exref{sayintr.kar-84}, and it is probably reflected as vowel length in the \trio \parencite[333]{triomeira1999} and \wayana \parencite[196]{wayanatavares2005} participles.

\pex<sayintr>
\a<aku-154> \akuriyo \parencite[][113]{meira1998proto}\\
\begingl
\gla mɨ-ka//
\glb \gl{2}-say//
\glft \qu{You said.}//
\endgl
\a<kar-84> \kalina \parencite[][202]{courtz2008carib}\\
\begingl
\glpreamble Òmakon `wa oti ywykàpo kaiko.//
\gla o-ʔma-kon ʔwa oti ɨ-wɨ-ka-ʔpo kai-ko//
\glb \gl{2}-child-\gl{pl} \gl{obl} greeting \gl{1}-\gl{s_a_}-say-\gl{pst}.\gl{nmlz} say-\gl{imp}//
\glft \qu{Pass my greetings to your children.}//
\endgl
\xe

In summary, this verb can be reconstructed as being intransitive based on its (inflectional) prefixes, but transitive based on some (derivational) suffixes.
\hixka has lost the main intransitive criteria, making its reflex look more like a transitive verb.

\subsubsection{\rc{ɨtə[mə]} \qu{to go}}
\label{sec:go}
This verb is reconstructed by \textcite{gildea2007greenberg} as \rc{tə[mə]}, like \rc{ka[ti]} \qu{to say} with a fleeting second syllable.
It is true that many reflexes are clearly \obj{t}-initial, for example \hixka \obj{ntoje} \qu{he went} \parencite[27]{hixkaryanaderby1985}, \trio \obj{təkə} \qu{go!} \parencite[246]{triomeira1999}, or \wayana \obj{kuptəm} \qu{we went} \parencite[195]{wayanatavares2005}.
However, once one considers all forms of the various reflexes of this verb \pcref{tab:go}, an initial vowel \rc{ɨ} must clearly be reconstructed -- in contrast to unambiguously C-initial \rc{ka[ti]} \qu{to say}.%
\footnote{As indicated by the brackets in \cref{tab:go}, there are many languages where the initial vowel is only present in some forms.
Also, the prefix-verb boundary in many inflected forms like e.g.\ \trio \obj{wɨtənne} or \arara \obj{wɨdolɨ} \qu{I went} \parencites[43]{triomeira1999}[153]{alves2017arara} is ambiguous, since an epenthetic \obj{ɨ} breaks up potential CC clusters.
Still, when one considers unambiguous forms, the contrast with \rc{ka[ti]} becomes very clear.}
This verb was not affected by the any of the extensions in \cref{sec:extensions}.

%The \ikpeng form \obj{aran} is compatible with our reconstruction \rc{ɨtən} when considering that \ikpeng \obj{a} is an attested outcome of \rc{ə}: \begin{inlinelist}
%	\item \obj{akari} \qu{dog} in \cref{tab:pxinw} above
%	\item \obj{anma} \qu{path} \parencite[24]{ikpengpacheco2001} from \PC \rc{ətema} \parencite[12]{gildea2007greenberg}
%	\item \obj{jaj} \qu{tree} \parencite[98]{ikpengpacheco2001} from \PC \rc{jəje}
%\end{inlinelist}.
%This attested change of \rc{ə} to \obj{a} need only be preceded by a assimilatory lowering of initial \rc{ɨ} to \rc{ə}, to yield the form \obj{aran} from \rc{ɨtən}.

\begin{table}[h]
\centering
\caption[Reflexes of \rc{ɨtə[mə]} \qu{to go}]{Reflexes of \rc{ɨtə[mə]} \qu{to go} \parencites[291]{cruz2005fonologia}[292]{triomeira1999}[195]{wayanatavares2005}[87]{gildea1994akuriyo}[80, 153]{alves2017arara}[27, 248]{hixkaryanaderby1985}[45, 62]{waiwaihawkins1998}[54, 80]{ikpengpacheco2001}[112, 374]{von1892bakairi}[181, 216]{maquiritaricaceres2011}[112]{meira1998proto}[168]{hoff1968carib}[139]{meira2006syntactic}[4]{caceres2018yawarana}[74]{muller1975mapoyo}[198]{mattei1994diccionario}[48, 50]{macushiabbott1991}[172]{garcia2006diccionario}[6]{franchetto2002kuikuro}[99; p.c., Spike Gildea]{camargo2002lexico}}
\label{tab:go}
\begin{tabular}[t]{@{}lllllll@{}}
\mytoprule
Language &             Form &    &    &    &    &    \\
\mymidrule
\kaxui    &     \obj{to[mo]} &    &  t &  o &  m &  o \\
\PWai     &    \rc{[ɨ]to[m]} &  ɨ &  t &  o &  m &    \\
\hixka    &      \obj{[ɨ]to} &  ɨ &  t &  o &    &    \\
\waiwai   &   \obj{[e]to[m]} &  e &  t &  o &  m &    \\
\PPek     &        \rc{ɨtən} &  ɨ &  t &  ə &  n &    \\
\arara    &        \obj{ɨdo} &  ɨ &  d &  o &    &    \\
\arara    &         \obj{to} &    &  t &  o &    &    \\
\ikpeng   &       \obj{aran} &  a &  r &  a &  n &    \\
\ikpeng   &        \obj{ero} &  e &  r &  o &    &    \\
\bakairi  &      \obj{[ɨ]tə} &  ɨ &  t &  ə &    &    \\
\PTir     &        \rc{təmɨ} &    &  t &  ə &  m &  ɨ \\
\trio     &      \obj{tə[n]} &    &  t &  ə &  n &    \\
\akuriyo  &  \obj{[ə]tə[mɨ]} &  ə &  t &  ə &  m &  ɨ \\
\carijo   &       \obj{təmə} &    &  t &  ə &  m &  ə \\
\wayana   &   \obj{[ɨ]tə[m]} &  ɨ &  t &  ə &  m &    \\
\apalai   &        \obj{ɨto} &  ɨ &  t &  o &    &    \\
\kalina   &         \obj{to} &    &  t &  o &    &    \\
\kalina   &       \obj{[ɨ]ʔ} &  ɨ &  ʔ &    &    &    \\
\maqui    &    \obj{ɨtə[mə]} &  ɨ &  t &  ə &  m &  ə \\
\ingariko &        \obj{ətə} &  ə &  t &  ə &    &    \\
\pemon    &      \obj{[e]tə} &  e &  t &  ə &    &    \\
\macushi  &      \obj{[a]tɨ} &  a &  t &  ɨ &    &    \\
\panare   &      \obj{tə[n]} &    &  t &  ə &  n &    \\
\yawarana &         \obj{tə} &    &  t &  ə &    &    \\
\mapoyo   &         \obj{tə} &    &  t &  ə &    &    \\
\uxc      &      \obj{[e]te} &  e &  t &  e &    &    \\
\yukpa    &         \obj{to} &    &  t &  o &    &    \\
\mybottomrule
\end{tabular}
\end{table}
%
%\ex<gosaycomp>
%\begin{tabular}[t]{@{}lllll@{}}
%& go-\gl{imp} & go-\gl{neg} & say-\gl{imp} & say-\gl{neg} \\
%\wayana & \obj{ɨtə-kə} & \obj{ɨtə-ra} & \obj{kai-kə} & \obj{ka-ra} \\
%\hixka & \obj{ɨto-ko} & \obj{ɨto-hra} & \obj{kas-ko} & \obj{ka-hra} \\
%\apalai & \obj{ɨto-ko} & \obj{ɨto-pɨra} & \obj{kaʃi-ko} & \obj{ka-ra} \\
%\end{tabular}\\
%\parencites[66, 98]{camargo2010wayana}[235, 258]{wayanatavares2005}[47, 54 194]{hixkaryanaderby1985}[65]{derbyshire1965textos}[kuruaz 033, 055]{koehns1994textos}[100]{camargo2002lexico}[107]{koehn1986apalai}
%\xe
%%

\subsubsection{\rc{eti} and \rc{a[p]} \qu{to be}}
\label{sec:be}
For a comprehensive comparative overview of these two roots, readers are referred to \textcite[375--382]{gildea2018reconstructing}; they will not be discussed in detail here.
\rc{a[p]} is the original copula and can be reconstructed as already having various irregularities in \PC.
\rc{eti} is reconstructed by \textcites{meira2009property}{gildea2018reconstructing} as originally meaning \qu{to dwell, live}, but serving as a copula already in \PC.\footnote{Such a stative, locative source is also suggested by the existence of \obj{it͡ʃi} \qu{to lie down} in \arara \parencite[196]{alves2017arara}.}
Various modern languages use reflexes of these two roots in a suppletive manner, conditioned by person and\slash{}or \gl{tam}.
Both roots preserved \gl{1}\gl{s_a_} \rc{w-} in \PPek, \PWai, and \PTir (\crefrange{sec:pekodian}{sec:taranoan}).
\akuriyo \obj{a} was not affected by the extension of \obj{k-} \pcref{sec:akuriyo}, while \obj{eʔi} is not attested in a first-person form. 
\carijo innovated \obj{j-}, but only in the reflex of \rc{eti} \exref{mivida-12}; the \obj{a} root preserves \obj{w-} \pcref{sec:carijo}.
\yukpa introduced \obj{j-} to the reflexes of both \rc{a[p]} and \rc{eti}, which are preserved as encliticized auxiliaries in certain constructions \exref{yukcop}.

\ex<mivida-12>\carijo \parencite[][177]{robayo1989rame}\\
\begingl
\glpreamble iretibə et͡ʃinəme gərə jet͡ʃiɨ//
\gla ireti-bə et͡ʃi-nə=me gərə j-et͡ʃi-ɨ//
\glb then-from be-\gl{inf}=\gl{attrz} still \gl{1}-be-\gl{pfv}//
\glft \qu{Then I was already grown up.}//
\endgl
\xe

\ex<yukcop> \yukpa \parencite[143--144]{meira2006syntactic}\\
\begin{tabular}[t]{@{}lll@{}}
	& \gl{npst} & \gl{pst}\\
	\gl{1} & \obj{=j-a(-s)}&\obj{=j-e}\\
	\gl{2} & \obj{=mak(o)}&\obj{=m-e}\\
	\gl{3} & \obj{=mak(o)}&\obj{=n-e}\\
\end{tabular}
\xe

\subsubsection{\rc{(ət-)jəpɨ} \qu{to come}}
\label{sec:come}

\begin{table}
\centering
\caption[Reflexes of \rc{(ət-)jəpɨ} \qu{to come}]{Reflexes of \rc{(ət-)jəpɨ} \qu{to come} \parencites[32]{macushiabbott1991}[102]{alvarez2000construcciones}[125]{akawaiocaesar2003}[299, 415]{cruz2005fonologia}[438]{maquiritaricaceres2011}[178]{robayo2000avance}[168]{meira1998proto}[74]{muller1975mapoyo}[294]{triomeira1999}[150]{alves2017arara}[37]{koehn1986apalai}[265]{ikpengpacheco2001}[160]{stegeman2014akawaio}[4]{meira2003bakairi}[65]{panarepayne2013}[68]{mendez1959yawarana}[429]{courtz2008carib}[182; p.c., Spike Gildea]{meira2005southern}}
\label{tab:come}
\begin{tabular}[t]{@{}lllllllll@{}}
\mytoprule
Language &         Form &    &    &    &    &     &    &    \\
\mymidrule
\kaxui    &   \obj{oohɨ} &    &    &    &    &  oo &  h &  ɨ \\
\kaxui    &   \obj{johɨ} &    &    &    &  j &   o &  h &  ɨ \\
\kaxui    &    \obj{ehɨ} &    &    &    &    &   e &  h &  ɨ \\
\PPek     &   \rc{ədepɨ} &  ə &  d &  - &    &   e &  p &  ɨ \\
\PPek     &     \rc{epɨ} &    &    &    &    &   e &  p &  ɨ \\
\arara    &    \obj{ebɨ} &    &    &    &    &   e &  b &  ɨ \\
\arara    &  \obj{odebɨ} &  o &  d &  - &    &   e &  b &  ɨ \\
\ikpeng   &   \obj{arep} &  a &  r &  - &    &   e &  p &    \\
\bakairi  &   \obj{əewɨ} &  ə &    &  - &    &   e &  w &  ɨ \\
\PTir     &   \rc{əʔepɨ} &  ə &  ʔ &  - &    &   e &  p &  ɨ \\
\trio     &    \obj{epɨ} &    &    &    &    &   e &  p &  ɨ \\
\trio     &   \obj{əepɨ} &  ə &    &  - &    &   e &  p &  ɨ \\
\akuriyo  &   \obj{eepɨ} &    &    &    &    &  ee &  p &  ɨ \\
\carijo   &    \obj{ehɨ} &    &    &    &    &   e &  h &  ɨ \\
\apalai   &   \obj{oepɨ} &  o &    &  - &    &   e &  p &  ɨ \\
\kalina   &    \obj{opɨ} &    &    &    &    &   o &  p &  ɨ \\
\maqui    &    \obj{ehə} &    &    &    &    &   e &  h &  ə \\
\akawaio  &  \obj{əsipɨ} &  ə &  s &  - &    &   i &  p &  ɨ \\
\akawaio  &   \obj{jepɨ} &    &    &    &  j &   e &  p &  ɨ \\
\ingariko &     \obj{jə} &    &    &    &  j &   ə &    &    \\
\ingariko &   \obj{jepə} &    &    &    &  j &   e &  p &  ə \\
\patamona &   \obj{jepɨ} &    &    &    &  j &   e &  p &  ɨ \\
\patamona &   \obj{jəpɨ} &    &    &    &  j &   ə &  p &  ɨ \\
\pemon    &   \obj{jepɨ} &    &    &    &  j &   e &  p &  ɨ \\
\panare   &    \obj{əpɨ} &    &    &    &    &   ə &  p &  ɨ \\
\yawarana &    \obj{əpɨ} &    &    &    &    &   ə &  p &  ɨ \\
\mapoyo   &    \obj{epɨ} &    &    &    &    &   e &  p &  ɨ \\
\uxc      &     \obj{ee} &    &    &    &    &  ee &    &    \\
\mybottomrule
\end{tabular}
\end{table}

This verb is reconstructed as \rc{ətepɨ} by \textcite[30]{gildea2007greenberg}, but an inspection of all attested reflexes \pcref{tab:come} points to a more complex story.
Crucially, the majority do not reflect the \rc{ət} part of their reconstruction, and many forms are ostensibly reflexes of \rc{əpɨ}, \rc{jepɨ}, or \rc{jəpɨ}, rather than \rc{epɨ}.
A unifying account of all these forms is achieved by reconstructing a \PC form \rc{(ət-)jəpɨ}, morphologically segmentable into a detransitivizing prefix and a root \rc{jəpɨ}.

Only the Pemongan languages and \kaxui point to an \rc{j}-initial root, and \kaxui \obj{johɨ} is very rare in contrast to more frequent \obj{o(o)hɨ}.
It only occurs in the third person form of the Progressive, meaning that the \obj{j} may be a reflex of third person \rc{i-}.
As \cref{tab:kaxprog} shows, regular \obj{o}-initial verbs have no third person prefix, while \obj{i-} occurs with C-initial \obj{to[mo]} \qu{to go}.
Even if \qu{to come} once shared this \obj{i-}, the circumstances are strongly in favor of a \obj{j}-initial \kaxui root, allowing the reconstruction of \PC \rc{jəpɨ}.
%
\begin{table}[h]
\centering
\caption{\kaxui \obj{johɨ} \qu{to come} compared with other verbs (Spike Gildea, p.c.)}
\label{tab:kaxprog}
\begin{tabular}[t]{@{}llll@{}}
\mytoprule
{} &                   \qu{to come} &                   \qu{to dream} &                    \qu{to go} \\
\mymidrule
\gl{1}   &  \obj{{\normalfont ∅}-w-oohɨ-} &  \obj{{\normalfont ∅}-w-osone-} &  \obj{{\normalfont ∅}-wɨ-to-} \\
\gl{2}   &                 \obj{o-w-ohɨ-} &                \obj{o-w-osone-} &                 \obj{o-w-to-} \\
\gl{1+2} &                \obj{ku-w-ohɨ-} &               \obj{ku-w-osone-} &                \obj{kɨ-w-to-} \\
\gl{3}   &    \obj{{\normalfont ∅}-johɨ-} &    \obj{{\normalfont ∅}-osone-} &                   \obj{i-to-} \\
\mybottomrule
\end{tabular}
\end{table}%
%\pex<kaxprog>\kaxui \perscomm{Spike Gildea}
%\a<kax-82>
%\begingl
%\gla johɨ-rɨ//
%\glb \gl{3}.come-\gl{prog}//
%\glft \qu{S/he is coming.}//
%\endgl
%\a<kax-54>
%\begingl
%\gla o-w-ohɨ-rɨ//
%\glb \gl{2}-\gl{s_a_}-come-\gl{prog}//
%\glft \qu{You are coming.}//
%\endgl
%\a<kax-77>
%\begingl
%\gla i-nkɨ-rɨ//
%\glb \gl{3}-sleep-\gl{prog}//
%\glft \qu{S/he is sleeping.}//
%\endgl
%\a<kax-83>
%\begingl
%\gla Ø-osone-rɨ//
%\glb \gl{3}-dream-\gl{prog}//
%\glft \qu{S/he is dreaming.}//
%\endgl
%\xe
%
Most of the (non-Pemongan) morphologically complex forms corresponding to \posscite{gildea2007greenberg} \rc{ətepɨ} show no segmental trace of \rc{j}, but the \obj{i} in \akawaio \obj{əsipɨ} is likely a reflex of the sequence \rc{jə}.
This analysis is supported by the reflex \obj{ipɨ} from bare \rc{jəpɨ} in very closely related \macushi.

As for the many forms seemingly reflecting \rc{əpɨ} and \rc{epɨ} rather than \obj{jəpɨ}, they are distributed widely in the family, sometimes even co-occurring in the same language.
A unifying account requires the root \rc{jəpɨ} to undergo two major sound changes: \begin{inlinelist}
 \item \rc{j}-loss
 \item \rc{ə}-umlaut after \rc{j}
 \end{inlinelist}.
While both sound changes are found in other contexts throughout the family \parencite{meira2010origin}, they appear to have applied irregularly to this verb, and not always in the same order.
For example, the \kalina form \obj{opɨ} can only be explained if \rc{j} was lost before the umlaut of \rc{ə} to \rc{e}, which would have been triggered by \rc{j}.
On the other hand, forms like \maqui \obj{ehə} must be the result of \phonl{\rc{ə}}{\rc{e}}{\rc{j}}, with subsequent loss of \rc{j}.
The \akuriyo form \obj{eepɨ} looks like a reflex thereof as well, but the length is unexpected, and is analyzed by \textcite[]{meira1998proto} as a coalescence of \rc{e} and \rc{əe}.

%Other languages -- across the family -- have forms reflecting the same root without the initial consonant, i.e.\ \rc{əpɨ}: \kalina, \panare, \yawarana, and \uxc.\footnote{\obj{e} is the regular outcome of \rc{ə} in \uxc; \rc{e} became \obj{i} \parencite[176]{meira2005southern}.}
%Also found across the family are forms reflecting \rc{epɨ}, in \kaxui, \arara, \trio, \carijo, \maqui, and \mapoyo.

While a root \rc{jəpɨ}, the two sound changes, and the optional addition of \rc{ət-} do account for the majority of the forms in \cref{tab:come},\footnote{
	Apart from aforementioned \akuriyo \obj{eepɨ}, another diachronically irregular form is \apalai \obj{oepɨ}, where one would expect \rc{ət-epɨ} to yield \obj{os-epɨ} \parencite[506]{meira2010origin}.
	Similarly, while \obj{oepɨ} would be a regular outcome of a hypothetical \rc{ə-jəpɨ}, the \envr{}{C} allomorph of the detransitivizer is \obj{e-} in \apalai.
	One possibility is that the form is due to borrowing from \trio, which has lost intervocalic \rc{t} to create \obj{əepɨ}.
	Alternatively, \apalai \obj{oepɨ} could be a fossilized loan from \wayana, which has replaced its reflex of \rc{ətjəpɨ}, but where intervocalic \rc{t} was also regularly lost \parencite[63]{wayanatavares2005}.}
the distribution within the family is rather chaotic.
In addition to the seemingly unordered distribution of \rc{əpɨ} and \rc{epɨ}, forms with and without \rc{ət-} can be found within the same language, usually conditioned by different prefixes.
This was briefly discussed in \cref{sec:pekodian} for \arara (and \PPek) and in \cref{sec:taranoan} for \trio (and \PTar).
To illustrate, the \trio \setone paradigm shows a reflex of \rc{ətepɨ} (< \rc{ətjəpɨ}) for first, but of \rc{epɨ} (< \rc{jəpɨ}) for the other persons \exref{tricome}.%
\footnote{While the \gl{1+2} form is a regular outcome of \rc{kɨt-epɨ}, the second person form is mysterious \parencite[115]{meira1998proto}.}
It should be noted that forms with and without \rc{ət-} in different languages are not triggered by the same person values.


\ex<tricome> \trio \parencite[294]{triomeira1999}\\
\begin{tabular}[t]{@{}ll@{}}
	\gl{1} & \obj{w-əepɨ} \\
	\gl{2} &  \obj{mən-epɨ} \\ 
	\gl{1+2} &  \obj{ke-epɨ} \\
	\gl{3} &  \obj{n-epɨ} \\
\end{tabular}
\xe

The interpretation of the \rc{ət} part as a detransitivizer is based on its form and its paradigmatically conditioned occurrence in some languages.
Although the combination of a detransitivizer and an intransitive verb seems semantically illogical, some historical \gl{s_p_} verbs are attested as adding the detransitivizer to become \gl{s_a_} verbs.
For example, the \PC \gl{s_p_} verb \rc{wɨnɨkɨ} \qu{to sleep} becomes \trio \obj{əənɨkɨ} \parencite[252]{triomeira1999} and \kalina \obj{əʔnɨkɨ} \parencite[429]{courtz2008carib}, both \gl{s_a_}.
Also, \waiwai \qu{go to sleep} can be \obj{wɨnɨk} \parencite[30]{waiwaihawkins1998} or \obj{et-wɨnɨk} \parencite[204]{hawkins1953waiwai}.
The parallels to \qu{to sleep} end here, since bare \rc{jəpɨ} \qu{to come} apparently already was an \gl{s_a_} verb, as evidenced the class membership of its reflexes in \kaxui, \kalina, \arara, \trio, and \panare \exref{pan-128}.

\ex<pan-128>\panare \parencite[][65]{panarepayne2013}\\
\begingl
\gla ju-w-əəpɨ-n ka=m kanoʔ//
\glb \gl{3}-\gl{s_a_}-come-\gl{nspec} \gl{q}=\gl{2}.\gl{aux} rain//
\glft \qu{Do you think it is gonna rain?}//
\endgl
\xe

Summing up, this verb is highly irregular, both from a synchronic and diachronic perspective.
The scenario suggested here involves reflexes of the detransitivizer \rc{ət(e)-} being optionally added to an \gl{s_a_} verb root \rc{jəpɨ}, which further underwent umlaut and loss of \rc{j}, but in no systematic manner, resulting in the chaotic picture in \cref{tab:come}.
As discussed in \cref{sec:pekodian}, innovative \rc{k-} was introduced on the \ikpeng and \bakairi reflexes of \rc{ətjəpɨ}, but not on the \arara reflex of \rc{jəpɨ}.
Reflexes of \rc{ətjəpɨ} (\trio) and of \rc{ətjəpɨ}/\rc{jəpɨ} (\akuriyo) resisted the introduction of \PTir \rc{t-}.
\carijo \obj{ehɨ} shows innovative \obj{j-}, rather than conservative \obj{w-} \exref{car-18}.
It is unknown whether there is a \yukpa reflex of this verb, and it was fully replaced in \PWai by \rc{omokɨ} \qu{to come}.

\ex<car-18>\carijo \parencite[][102]{guerrero2019carijo}\\
\begingl
\gla əji-wa-e j-eh-ɨ//
\glb \gl{2}-search-\gl{sup} \gl{1}-come-\gl{pfv}//
\glft \qu{I came looking for you.}//
\endgl
\xe

\subsubsection{\rc{ɨpɨtə} \qu{to go down}}
\label{sec:godown}
Reflexes of this verb were not affected by the extensions of \rc{k-} in \PPek \pcref{sec:pekodian} and \obj{k-} in \akuriyo \pcref{sec:akuriyo}.
The resistance against the former extension was subsequently overcome in \bakairi; its fate in \ikpeng is unknown.
When \akuriyo extended \obj{k-}, the verb already had a first person form irregularly inflected with \obj{p-}, inherited from \PTir.
One might think that it was also affected by the extensions of \obj{j-} in \carijo \exref{gowhere.car-32} and \yukpa \exref{gowhere.yuk-14}.

\pex<gowhere>
\a<car-32> \carijo \perscommpar{David Felipe Guerrero}\\
\begingl
\gla irə wat͡ʃinakano tae j-ehɨtə-e//
\glb \gl{inan}.\gl{ana} body.of.water along.bounded \gl{1}-go.down-\gl{npst}//
\glft \qu{…I go down through that guachinacán.}//
\endgl
\a<yuk-14> \yukpa \parencite[][]{meira2003primeras}\\
\begingl
\glpreamble aw yéwtu//
\gla aw j-ewuhtu//
\glb \gl{1}\gl{pro} \gl{1}-go.down//
\glft \qu{I went down.}//
\endgl
\xe
%
However, a broader comparative perspective reveals a much more complicated story \pcref{tab:godown}.%
\footnote{In \cref{tab:godown}, parenthesized forms indicate uncertainty about cognacy status. The reconstruction of \PPek \rc{ɨptə} treats the suffixed elements found in the daughter languages as non-cognate.
\textcite{meira2005southern} identify no correspondence between \bakairi \obj{gɨ} and \ikpeng \obj{ŋ}, and at least the addition of a final \obj{ŋ} in \PXin is attested elsewhere:
		\begin{inlinelist}
			\item \PC \rc{əne} \qu{to see}, \arara and \ikpeng \obj{eneŋ} %\parencites[8]{gildea2007greenberg}[56]{alves2017arara}[25]{ikpengpacheco2001}
			\item \PC \rc{əta} \qu{to hear}, \arara \obj{taŋ}, \ikpeng \obj{iraŋ} %\parencites[8]{gildea2007greenberg}[144]{alves2017arara}[270]{ikpengpacheco2001}
			\item \PC \rc{ənə} \qu{to eat meat}, \arara \obj{oŋoŋ} \qu{to bite} %\parencites[8]{gildea2007greenberg}[57]{alves2017arara}
		\end{inlinelist} \parencites[8]{gildea2007greenberg}[56, 144, 57]{alves2017arara}[25, 270]{ikpengpacheco2001}.}
It turns out that while a verb\rc{ɨpɨtə} can be reconstructed to \PC, different (proto-)languages do not agree about its class.
Its reflexes in languages that preserve the split-\gl{s} system are distributed fairly evenly between \gl{s_a_} and \gl{s_p_}.

\begin{table}[h]
\centering
\caption[Reflexes of \rc{ɨpɨtə} \qu{to go down}]{Reflexes of \rc{ɨpɨtə} \qu{to go down} \parencites{meira2003primeras}[196]{hixkaryanaderby1979}[55]{waiwaihawkins1998}[118]{guerrero2019carijo}[44]{camargo2010wayana}[99]{camargo2002lexico}[263]{courtz2008carib}[450]{maquiritaricaceres2011}[139]{stegeman2014akawaio}[139]{alvarez2008clausulas}[34]{macushiabbott1991}[88]{mattei1994diccionario}[68]{mendez1959yawarana}[58]{bruno1996dictionary}[84]{gildea1994akuriyo}[153]{alves2017arara}[137]{von1892bakairi}[116; p.c., Angela Fabíola Alves Chagas, Spike Gildea]{meira1998proto}}
\label{tab:godown}
\begin{tabular}[t]{@{}llllllllllllllll@{}}
\mytoprule
Language &            Form &                  Class &    &    &    &    &    &    &    &    &    &    &    &    &    \\
\mymidrule
\PPar     &      \rc{ɨɸɨto} &              \gl{s_p_} &    &    &    &    &  ɨ &  ɸ &  ɨ &  t &  o &    &    &    &    \\
\kaxui    &     \obj{ɨhɨto} &              \gl{s_p_} &    &    &    &    &  ɨ &  h &  ɨ &  t &  o &    &    &    &    \\
\PWai     &        \rc{hto} &                      ? &    &    &    &    &    &  h &    &  t &  o &    &    &    &    \\
\hixka    &       \obj{hto} &                      ? &    &    &    &    &    &  h &    &  t &  o &    &    &    &    \\
\waiwai   &       \obj{hto} &                      – &    &    &    &    &    &  h &    &  t &  o &    &    &    &    \\
\PPek     &       \rc{ɨptə} &              \gl{s_a_} &    &    &    &    &  ɨ &  p &    &  t &  ə &    &    &    &    \\
\arara    &     \obj{iptoŋ} &              \gl{s_a_} &    &    &    &    &  i &  p &    &  t &  o &  - &  ŋ &    &    \\
\ikpeng   &     \obj{iptoŋ} &                      ? &    &    &    &    &  i &  p &    &  t &  o &  - &  ŋ &    &    \\
\bakairi  &     \obj{ɨtəgɨ} &              \gl{s_a_} &    &    &    &    &  ɨ &    &    &  t &  ə &  - &    &  g &  ɨ \\
\PTar     &      \rc{ɨpɨtə} &                      ? &    &    &    &    &  ɨ &  p &  ɨ &  t &  ə &    &    &    &    \\
\PTir     &       \rc{ɨhtə} &              \gl{s_a_} &    &    &    &    &  ɨ &  h &    &  t &  ə &    &    &    &    \\
\trio     &      \obj{ɨhtə} &              \gl{s_a_} &    &    &    &    &  ɨ &  h &    &  t &  ə &    &    &    &    \\
\akuriyo  &      \obj{ɨhtə} &              \gl{s_a_} &    &    &    &    &  ɨ &  h &    &  t &  ə &    &    &    &    \\
\carijo   &     \obj{ehɨtə} &                      – &    &    &    &    &  e &  h &  ɨ &  t &  ə &    &    &    &    \\
\wayana   &      \obj{ɨptə} &  \gl{s_a_} / \gl{s_p_} &    &    &    &    &  ɨ &  p &    &  t &  ə &    &    &    &    \\
\apalai   &      \obj{ɨhto} &              \gl{s_p_} &    &    &    &    &  ɨ &  h &    &  t &  o &    &    &    &    \\
\kalina   &    \obj{onɨʔto} &            (\gl{s_a_}) &  o &  - &  n &  - &  ɨ &  ʔ &    &  t &  o &    &    &    &    \\
\maqui    &      \obj{əʔtə} &              \gl{s_p_} &    &    &    &    &  ə &  ʔ &    &  t &  ə &    &    &    &    \\
\kapon    &    \obj{(uʔtə)} &                      – &    &    &    &    &    &    &    &    &    &    &    &    &    \\
\pemon    &    \obj{(uʔtə)} &                      – &    &    &    &    &    &    &    &    &    &    &    &    &    \\
\macushi  &    \obj{(autɨ)} &                      – &    &    &    &    &    &    &    &    &    &    &    &    &    \\
\panare   &      \obj{əhtə} &              \gl{s_a_} &    &    &    &    &  ə &  h &    &  t &  ə &    &    &    &    \\
\yawarana &      \obj{əhtə} &                      – &    &    &    &    &  ə &  h &    &  t &  ə &    &    &    &    \\
\yukpa    &  \obj{(ewuhtu)} &                      – &    &    &    &    &    &    &    &    &    &    &    &    &    \\
\waimiri  &       \obj{ɨtɨ} &                      – &    &    &    &    &    &    &  ɨ &  t &  ɨ &    &    &    &    \\
\mybottomrule
\end{tabular}
\end{table}

The verb shows traits of both classes in \wayana, necessitating an analysis as a \dbqu{mixed} verb in a synchronic description of that language.
It takes the first and second person \gl{s_p_} markers \obj{j-} and \obj{əw-} \parencite[200]{wayanatavares2005}, but the \gl{1+2}\gl{s_a_} marker \obj{kut-} \parencite[206]{wayanatavares2005}.
It also shows the \gl{s_a_} class marker \obj{w-} in nominalizations \exref{waygodown.way-73}, but behaves like an \gl{s_p_} verb in taking a second person prefix in imperatives \exref{waygodown.way-71}.

\pex<waygodown>\wayana \parencite[][200]{wayanatavares2005}
\begin{multicols}{2}
\a<way-73>
\begingl
\glpreamble ïwïptëë//
\gla ɨ-w-ɨptə-rɨ//
\glb \gl{1}-\gl{s_a_}-go.down-\gl{nmlz}//
\glft \qu{my going down}//
\endgl
\a<way-71>
\begingl
\gla əw-ɨptə-k//
\glb \gl{2}-go.down-\gl{imp}//
\glft \qu{Go down!}//
\endgl
\end{multicols}
\xe
%
Its causativized form is \obj{ɨptə-ka} \parencite[255]{wayanatavares2005}; the \PC  causativizer \rc{-ka} was restricted to \gl{s_p_} verbs \parencite{gildea2019overview}.
These patterns point to a scenario where the verb was a regular member of the \gl{s_p_} class in pre-\wayana, but partially switched to the \gl{s_a_} class, taking a \gl{1+2}\gl{s_a_} prefix and the \gl{s_a_} class marker.
This in turn implies that \gl{s_a_} reflexes of this verb in other languages fully switched from \gl{s_p_} in their inflectional patterns.

\wayana-external comparative evidence supports this hypothesis:
The \arara causativized form is \obj{eniptoŋ} \parencite[66]{alves2017arara}, and \kalina has a cognate form \obj{enɨʔto} \parencite[263]{courtz2008carib}; \obj{onɨʔto} \qu{to go down} in \cref{tab:godown} is a detransitivized form thereof, lit.\ \qu{to get oneself down}.
Both causativized forms contain a reflex of the transitivizer \rc{en-}, which was usually found with \gl{s_p_} verbs \parencite{gildea2019overview}.
\trio \obj{ɨhtə} has irregular causativized forms that also feature a reflex of \rc{en-} \parencite[263]{triomeira1999}.
Thus, it appears that this verb was originally \gl{s_p_}, but then switched its class in four \dbqu{and a half} languages of the family, for so far unknown reasons.

These circumstances make it impossible to answer the question of whether \qu{to go down} was affected by some extensions.
For \PTir, one cannot establish a relative chronology of the verb class switch, the introduction of the idiosyncratic marker \rc{p-}, and the extension of \rc{t-}.
For \PWai, both its first person form and its inflectional class are unknown.
For \carijo and \yukpa, one cannot rule out the possibility that the verb switched classes before the breakdown of the split-\gl{s} system.
While no language-internal evidence supports that scenario, \qu{to go down} clearly has an inclination to switch classes; in the case of \carijo, that could have already happened at the \PTar stage.
In all four cases, the verb could have had \gl{s_a_} status at the time of the extension, resisting it and preserving the old prefix, but it could also have had \gl{s_p_} status and thus not even have been a potential target.

On the other hand, the class switch happened before the split of \PTir, and therefore this verb resisted the extension of \akuriyo \obj{k-} as an \gl{s_a_} verb.
Likewise, it is very likely that the class switch took place before the extension of \PPek \rc{k-}.
Otherwise, the newly-turned-\gl{s_a_} verb would have taken on conservative and lexically heavily restricted \rc{w-}, either in \PPek, \PXin, or \arara.

\subsubsection{\rc{e-pɨ} \qu{to bathe}}
\label{sec:bathe}
This verb resisted the extensions of \PPek \rc{k-} \pcref{sec:pekodian} and \akuriyo \obj{k-} \pcref{sec:akuriyo}.
It took on new \gl{1}\gl{s_a_} prefixes in \PTir \parentext{\trio \obj{s-epɨ-}, \akuriyo \obj{t͡ʃ-epɨ-} \parencites[292]{triomeira1999}[87]{gildea1994akuriyo}} and \PWai \parentext{\hixka \obj{k-ewehɨ-}, \waiwai \obj{k-ejeɸu-} \parencites[195]{hixkaryanaderby1985}[166]{waiwaihawkins1998}}.
The first person form of its \carijo reflex \obj{ehɨ} \parencites[72]{koch1908hiana} is unknown; an unattested \yukpa cognate may exist.

Verbs for intransitive \qu{to bathe} are usually typical \gl{s_a_} verbs in Cariban languages, derived with a detransitivizer from a transitive root.
These roots are reflexes of \rc{[ɨ]pɨ}, or \rc{kupi} in some Venezuelan languages \pcref{tab:bathe}.
\PPek can be reconstructed as having the pair \rc{ipɨ} (\gl{intr}) / \rc{ɨp} (\gl{tr}).
Thus, while \PPek \qu{to bathe (\gl{tr})} has perfectly regular cognates in other languages of the family, \qu{to bathe (\gl{intr})} diverges by changing \rc{e-} to \rc{i}.
This is an irregular development, since there are no attested reflexes of a Pekodian detransitivizer \rc{i-} \parencite[506]{meira2010origin}; its cause is unknown.
However, other languages also show unexpected developments in this verb, like the glide insertions in Waiwaian or the distribution of \rc{[ɨ]pɨ} and \rc{kupi} in Venezuelan languages.

\begin{table}
\caption[Comparison of intransitive and transitive \qu{to bathe}]{Comparison of intransitive and transitive \qu{to bathe} \parencites[198]{hixkaryanaderby1979}[192, 203]{waiwaihawkins1998}[150, 162]{alves2017arara}[103]{ikpengpacheco1997}[123]{campetela1997analise}[4]{meira2003bakairi}[285]{meira2005bakairi}[697]{triomeira1999}[87]{gildea1994akuriyo}[24, 52]{camargo2010wayana}[218]{meira2000split}[304]{courtz2008carib}[439, 454]{maquiritaricaceres2011}[37]{stegeman2014akawaio}[34, 129]{pemondearmellada1944dic}[8, 294; p.c., Spike Gildea]{mattei1994diccionario}}
\label{tab:bathe}
\small
\centering
\begin{subtable}[t]{.49\linewidth}
\caption{Reflexes of \rc{e-pɨ} \qu{to bathe (\gl{intr})}}
\label{tab:bathe_intr_1}
\begin{tabular}[t]{@{}llllllll@{}}
\mytoprule
Language &         Form & \multicolumn{6}{l}{Alignment} \\
\midrule
\kaxui   &   \obj{eehɨ} &           &    &  ee &  - &  h &  ɨ \\
\hixka   &  \obj{ewehɨ} &         e &  w &   e &  - &  h &  ɨ \\
\waiwai  &  \obj{ejeɸu} &         e &  j &   e &  - &  ɸ &  u \\
\arara   &    \obj{ibɨ} &           &    &   i &  - &  b &  ɨ \\
\ikpeng  &     \obj{ip} &           &    &   i &  - &  p &    \\
\bakairi &      \obj{i} &           &    &   i &    &    &    \\
\trio    &    \obj{epɨ} &           &    &   e &  - &  p &  ɨ \\
\akuriyo &    \obj{epɨ} &           &    &   e &  - &  p &  ɨ \\
\wayana  &    \obj{epɨ} &           &    &   e &  - &  p &  ɨ \\
\apalai  &    \obj{epɨ} &           &    &   e &  - &  p &  ɨ \\
\bottomrule
\end{tabular}
\caption{Reflexes of \rc{e-kupi} \qu{to bathe (\gl{intr})}}
\label{tab:bathe_intr_2}
\begin{tabular}[t]{@{}lllllllll@{}}
\mytoprule
Language &          Form & \multicolumn{7}{l}{Alignment} \\
\midrule
\kalina &   \obj{ekupi} &         e &  - &  k &  u &    &  p &  i \\
\maqui  &    \obj{eʔhi} &         e &  - &    &    &  ʔ &  h &  i \\
\kapon  &  \obj{ekuʔpi} &         e &  - &  k &  u &  ʔ &  p &  i \\
\pemon  &   \obj{ekupɨ} &         e &  - &  k &  u &    &  p &  ɨ \\
\bottomrule
\end{tabular}
\caption{Reflexes of \rc{ə-kupi} \qu{to bathe (\gl{intr})}}
\label{tab:bathe_intr_3}
\begin{tabular}[t]{@{}llllllll@{}}
\mytoprule
Language &         Form & \multicolumn{6}{l}{Alignment} \\
\midrule
\panare &  \obj{akupɨ} &         a &  - &  k &  u &  p &  ɨ \\
\bottomrule
\end{tabular}
\end{subtable}
\begin{subtable}[t]{.49\linewidth}\caption{Reflexes of \rc{(ɨ)pɨ} \qu{to bathe (\gl{tr})}}
\label{tab:bathe_tr_1}
\begin{tabular}[t]{@{}lllll@{}}
\mytoprule
Language &       Form & \multicolumn{3}{l}{Alignment} \\
\midrule
\kaxui   &  \obj{ɨhɨ} &         ɨ &  h &  ɨ \\
\hixka   &  \obj{ɨhɨ} &         ɨ &  h &  ɨ \\
\waiwai  &   \obj{pɨ} &           &  p &  ɨ \\
\arara   &   \obj{ɨp} &         ɨ &  p &    \\
\ikpeng  &   \obj{ɨp} &         ɨ &  p &    \\
\bakairi &    \obj{ɨ} &           &    &  ɨ \\
\trio    &   \obj{pɨ} &           &  p &  ɨ \\
\akuriyo &   \obj{pɨ} &           &  p &  ɨ \\
\wayana  &  \obj{upɨ} &         u &  p &  ɨ \\
\apalai  &   \obj{pɨ} &           &  p &  ɨ \\
\maqui   &  \obj{ɨhɨ} &         ɨ &  h &  ɨ \\
\pemon   &   \obj{pɨ} &           &  p &  ɨ \\
\panare  &  \obj{ɨpɨ} &         ɨ &  p &  ɨ \\
\bottomrule
\end{tabular}
\caption{Reflexes of \rc{kupi} \qu{to bathe (\gl{tr})}}
\label{tab:bathe_tr_2}
\begin{tabular}[t]{@{}lllllll@{}}
\mytoprule
Language &         Form & \multicolumn{5}{l}{Alignment} \\
\midrule
\kalina &   \obj{kupi} &         k &  u &    &  p &  i \\
\kapon  &  \obj{kuʔpi} &         k &  u &  ʔ &  p &  i \\
\panare &   \obj{kupɨ} &         k &  u &    &  p &  ɨ \\
\bottomrule
\end{tabular}
\end{subtable}\end{table}


\subsection{Summary}
\label{sec:verbsummary}
In \cref{sec:verbs}, the verbs which were unaffected by the extensions in \cref{sec:extensions} were reconstructed, and affected reflexes in the languages under discussion were identified.
\cref{tab:overview} gives an overview of what verbs were affected by which extensions (except for \obj{e}-initial \akuriyo verbs unaffected by the extension of \obj{k-}, as they are a large and predictable group).
In some cases, the verb does not occur at all or just not in first person forms (–), in others the first person form is unknown (?), and in the case of \qu{to go down} the question of affectedness is not meaningfully answerable (\textsc{n/a}), for any of the reasons discussed in \cref{sec:godown}.
Every \checkmark stands for a verb affected by an extension, while × represents conservatively inflected verbs.
This overview makes clear just how strongly these verbs tend to resist person marker extensions, as they do so in different languages.

\begin{table}
\centering
\caption{Overview of extensions and (un-)affected verbs}
\label{tab:overview}
\begin{tabular}[t]{@{}llllllll@{}}
\mytoprule
{} & \rc{ka[ti]} &  \rc{ɨtə[mə]} &   \rc{a[p]} &    \rc{eti} & \rc{(ət-)jəpɨ} &    \rc{ɨpɨtə} &   \rc{e-pɨ} \\
{} &    \qu{say} &       \qu{go} &   \qu{be-1} &   \qu{be-2} &      \qu{come} &  \qu{go down} &  \qu{bathe} \\
\midrule
\PWai \rc{k-}     &           × &             × &           × &           × &              – &  \textsc{n/a} &  \checkmark \\
\quad \hixka      &           × &             × &           × &           × &              – &  \textsc{n/a} &  \checkmark \\
\quad \waiwai     &           × &  (\checkmark) &           × &           × &              – &  \textsc{n/a} &  \checkmark \\
\PPek \rc{k-}     &           × &             × &           × &           × &              × &             × &           × \\
\quad \arara      &           × &             × &           × &           × &              × &             × &           × \\
\quad \ikpeng     &           × &    \checkmark &           – &           × &     \checkmark &             ? &           × \\
\quad \bakairi    &           × &             × &           × &           × &     \checkmark &    \checkmark &           × \\
\PTir \rc{t-}     &           × &             × &           × &           × &              × &  \textsc{n/a} &  \checkmark \\
\quad \trio       &           × &             × &           × &           × &              × &  \textsc{n/a} &  \checkmark \\
\quad \akuriyo    &           × &             × &           × &           ? &              × &  \textsc{n/a} &  \checkmark \\
\akuriyo \obj{k-} &           × &             × &           × &           ? &              × &             × &           × \\
\carijo \obj{j-}  &           × &             × &           × &  \checkmark &     \checkmark &  \textsc{n/a} &           ? \\
\yukpa \obj{j-}   &           ? &             × &  \checkmark &  \checkmark &              – &  \textsc{n/a} &           – \\
\bottomrule
\end{tabular}
\begin{legendlist}\item[\checkmark] affected
\item[×] not affected
\item[?] unknown first person prefix
\item[–] does not occur
\item[(\checkmark)] affected with surviving old marker
\item[\textsc{n/a}] not meaningfully answerable
\end{legendlist}\end{table}

It is astonishing that the same 1-7 verbs retain their old first person marker in 6 distinct developments, while a plethora of regular \gl{s}\textsubscript{(\gl{a})} verbs take on innovative markers.
This suggests that there is something about these verbs which causes them to behave in such a way.
Possible answers to the question of what makes them different from regular \gl{s_a_} verbs will be discussed in \cref{sec:motivations}, using \posscite{bybee1985morphology} network model of morphology.