\section{Resistant verbs from a comparative perspective}
\label{sec:verbs}
In \cref{sec:extensions}, we introduced six distinct extensions of personal prefixes into \gl{1}\gl{s_a_} territory, and identified verbs resistant to each innovation.
The set of unaffected verbs was rather small in most cases, and (proto-)languages show a considerable degree of overlap in what verbs are unaffected.
In this section, we present these verbs from a comparative perspective.
\cref{sec:be} treats both roots of the copula \rc{eti}/\obj{a[p]} \qu{to be}, \cref{sec:say} \rc{ka[ti]} \qu{to say}, \cref{sec:go} \rc{ɨtə[mə]} \qu{to go}, and \cref{sec:come} \rc{(ət)jəpɨ} \qu{to come}; these are all verbs that \textcite{gildea2007greenberg} reconstruct as \gl{s_a_} verbs that were not derived from transitive verbs.
\cref{sec:godown} takes a look at \rc{ɨpɨtə} \qu{to go down}, which is resistant in \PTir and \PPek, and  \cref{sec:bathe} investigates \rc{e-pɨ} \qu{to bathe}, of which the \PPek reflex \rc{i-pɨ} resisted the extension of \rc{k-}.
The \obj{e}-initial verbs not affected by the extension of \rc{k-} in \akuriyo \pcref{sec:akuriyo} will not be discussed here, as they are a large and phonologically coherent group.

\subsection{\rc{eti} and \rc{a[p]} \qu{to be}}
\label{sec:be}
For a comprehensive comparative overview for this verb, we refer the reader to \textcite[375--382]{gildea2018reconstructing}, who reconstructs two distinct roots serving as verbs \qu{to be} in modern Cariban languages.
One is the older copula \rc{a[p]}, which can be reconstructed as already having various irregularities in \PC.
The other is a root \rc{eti} reconstructed by \textcites{meira2009property}{gildea2018reconstructing} as originally meaning \qu{to dwell, live}, but serving as a copula in \PC.\footnote{Such a stative, locative source is also suggested by the existence of \obj{it͡ʃi} \qu{to lie down} in \arara \parencite[196]{alves2017arara}.}
Various modern languages use reflexes of these two roots in a suppletive manner, conditioned by person and\slash{}or \gl{tam} value.
Both roots preserved \gl{1}\gl{s_a_} \rc{w-} in \PPek, \PWai, and \PTir (\crefrange{sec:pekodian}{sec:taranoan}).
\akuriyo \obj{a} was not affected by the extension of \obj{k-} \pcref{sec:akuriyo}, while \obj{eʔi} is not attested in a first-person form. 
\carijo innovated \obj{j-}, but only in the \obj{et͡ʃi} root allomorph \exref{mivida-12}; the \obj{a} root preserves \obj{w-} \pcref{sec:carijo}.
\yukpa innovated \obj{j-} for reflexes of both \rc{a[p]} and \rc{eti}, which are preserved as encliticized auxiliaries in certain constructions \exref{yukcop}.

\ex<mivida-12>\carijo \parencite[][177]{robayo1989rame}\\
\begingl
\glpreamble iretibə et͡ʃinəme gərə jet͡ʃiɨ//
\gla ireti-bə et͡ʃi-nə=me gərə j-et͡ʃi-ɨ//
\glb then-from be-\gl{inf}=\gl{attrz} still \gl{1}-be-\gl{pfv}//
\glft \qu{Then I was already grown up.}//
\endgl
\xe

\ex<yukcop> \yukpa \parencite[143--144]{meira2006syntactic}\\
\begin{tabular}[t]{@{}lll@{}}
& \gl{npst} & \gl{pst}\\
\gl{1} & \obj{=j-a(-s)}&\obj{=j-e}\\
\gl{2} & \obj{=mak(o)}&\obj{=m-e}\\
\gl{3} & \obj{=mak(o)}&\obj{=n-e}\\
\end{tabular}
\xe

\subsection{\rc{ka[ti]} \qu{to say}}
\label{sec:say}
Most reflexes of this verb are simply \obj{ka}, but a fleeting syllable \rc{ti} is reconstructed by \textcite{gildea2007greenberg}, best visible in the imperative forms of some languages \exref{kati}.

\ex<kati> \apalai\\
\obj{kaʃi-ko} \qu{say!}\\
\parencite[35]{koehn1986apalai}
\xe
%
A comparison of the longest forms attested in each language is shown in \cref{tab:say}.
Segments of cognate forms in this and other tables were aligned automatically with LingPy \parencite{lingpy268}, for easier recognition of correspondences.
%
\begin{table}
\centering
\caption[Reflexes of \rc{ka(ti)} \qu{to say}]{Reflexes of \rc{ka(ti)} \qu{to say} \parencites[4]{meira2003bakairi}[48]{franchetto2008absolutivo}[209]{ikpengpacheco2001}[153]{alves2017arara}[182]{hixkaryanaderby1985}[113]{meira1998proto}[107]{koehn1986apalai}[26]{waiwaihawkins1998}[66]{camargo2010wayana}[59]{macushiabbott1991}[123]{swiggers2010gramatica}[430]{courtz2008carib}[125]{akawaiocaesar2003}[102; p.c., Spike Gildea]{mattei1994diccionario}}
\label{tab:say}
\begin{tabular}[t]{@{}llllll@{}}
\toprule
Language &          Form & \multicolumn{4}{l}{Alignment} \\
\midrule
\kaxui   &   \obj{ka[s]} &         k &  a &  s &    \\
\hixka   &   \obj{ka[h]} &         k &  a &  h &    \\
\waiwai  &   \obj{ka[s]} &         k &  a &  s &    \\
\arara   &      \obj{ke} &         k &  e &    &    \\
\ikpeng  &      \obj{ke} &         k &  e &    &    \\
\bakairi &      \obj{ke} &         k &  e &    &    \\
\trio    &      \obj{ka} &         k &  a &    &    \\
\akuriyo &      \obj{ka} &         k &  a &    &    \\
\carijo  &      \obj{ka} &         k &  a &    &    \\
\wayana  &   \obj{ka[i]} &         k &  a &    &  i \\
\apalai  &  \obj{ka[ʃi]} &         k &  a &  ʃ &  i \\
\kalina  &      \obj{ka} &         k &  a &    &    \\
\kapon   &      \obj{ka} &         k &  a &    &    \\
\pemon   &      \obj{ka} &         k &  a &    &    \\
\macushi &      \obj{ka} &         k &  a &    &    \\
\panare  &   \obj{ka[h]} &         k &  a &  h &    \\
\uxc     &      \obj{ki} &         k &  i &    &    \\
\bottomrule
\end{tabular}
\end{table}
%
This verb was not affected by any of the extensions found in \PPek, \PWai, \PTir, \akuriyo, or \carijo (\crefrange{sec:pekodian}{sec:carijo}).
We do not know the first person prefix of its \yukpa reflex \obj{ka}.

As briefly mentioned in \cref{sec:waiwaian}, \textcite{hixkaryanaderby1985} analyzes this verb as transitive in \hixka.
This analytical choice is not only motivated by avoiding an idiosyncratic intransitive first person prefix \obj{ɨ-}:
\hixka \obj{ka} also behaves like a transitive verb in other ways, for instance by showing the complementary distribution of the third person marker \obj{n-} and preceding objects -- in this case direct speech or ideophones \exref{hixsay}.

\pex<hixsay>\hixka
\a<hix-119>
\begingl
\glpreamble onɨ wyaro nkekonɨ bɨryekomo, tɨyonɨ wya//
\gla onɨ wjaro n-ka-jakonɨ bɨrʲekomo tɨ-jonɨ wja//
\glb this like \gl{3}-say-\gl{rem}.\gl{cont} boy \gl{cor}-mother \gl{obl}//
\glft \qu{This is what the boy said to his mother.} \parencite[][36]{hixkaryanaderby1985}//
\endgl
\a<ordpl-35>
\begingl
\glpreamble moro ha, ketxkoná hatá.//
\gla moro ha ka-jat͡ʃkonɨ hatɨ//
\glb \gl{med}.\gl{dem}.\gl{inan} \gl{ints} say-\gl{rem}.\gl{cont}.\gl{pl} \gl{hsy}//
\glft \qu{“That one there” they said.} \parencite[][14]{derbyshire1965textos}//
\endgl
\xe
In \exref{hixsay.hix-119}, the prefix \obj{n-} occurs because there is no preceding object (\qu{he said it like this}).
In \exref{hixsay.ordpl-35}, it does not occur, because \qu{they said} is preceded by direct speech.
This complementary distribution is otherwise only found with transitive verbs \parencite[59--60]{gildea1998}.
The verb shows the same pattern, albeit inconsistently, in \trio \parencite[267]{triocarlin2004}.

Further comparative evidence also points to \rc{ka[ti]} \qu{to say} showing transitive traits:
\trio \obj{ka} is characterized as the only intransitive verb being able to take the causative suffix \obj{-po} and the agentive nominalizer \obj{-ne} \parencite[263, 169]{triomeira1999}.
The exceptionality of \obj{ka} \qu{to say} taking \obj{-po} \qu{\gl{caus}.\gl{tr}} has also been noted for \kalina \parencite[82]{courtz2008carib} and \wayana \parencite[258]{wayanatavares2005}.
The agent nominalizer \rc{-ne} gave rise to the \panare gnomic verbal suffix \obj{-ɲe} on transitive verbs \parencite[184--185]{gildea1998}.
The occurrence of \obj{-ɲe} on \obj{ka} has led to differing categorizations as transitive \parencite[214]{panarepayne2013} or intransitive \parencite[102]{mattei1994diccionario}.
Reflexes of another transitive causativizer \rc{-metipo} \parencite{gildea2015valency} are found with \obj{ka} in \apalai \parencite[51]{koehn1986apalai} and \waiwai \parencite[52]{waiwaihawkins1998}.

Our classification of \qu{to say} as an intransitive verb is supported primarily by prefix patterns:
\kalina offers a minimal pair between transitive \obj{ka} \qu{to remove} and intransitive \obj{ka} \qu{to say}, \obj{sikai} \qu{I took it away} vs \obj{wɨkai} \qu{I said} \parencite[288, 45]{courtz2008carib}.\footnote{Interestingly, the \kalina causativized form \obj{kapo} \qu{to make say} does not have the regular \gl{1}>\gl{3} prefix \obj{s(i)-}, but irregular \obj{w(ɨ)-} \parencite[430]{courtz2008carib}.}
Similarly, \qu{to say} in Pekodian languages has a reflex of \gl{1}\gl{s} \rc{w-} \pcref{sec:pekodian}, and not \gl{1}>\gl{3} \obj{s-} (\bakairi) or \rc{ini-} (\PXin).
Additionally, the \gl{s_a_} class marker \obj{w-} occurs on nominalizations in \kalina \exref{kar-84}, and it is probably reflected in vowel length in the \trio \parencite[333]{triomeira1999} and \wayana \parencite[196]{wayanatavares2005} participles.

\ex<kar-84>\kalina \parencite[][202]{courtz2008carib}\\
\begingl
\glpreamble Òmakon \`wa oti ywykàpo kaiko.//
\gla o-ʔma-kon ʔwa oti ɨ-wɨ-ka-ʔpo kai-ko//
\glb \gl{2}-child-\gl{pl} \gl{obl} greeting \gl{1}-\gl{s_a_}-say-\gl{pst}.\gl{nmlz} say-\gl{imp}//
\glft \qu{Pass my greetings to your children.}//
\endgl
\xe
%
Summing up, this verb could be reconstructed as being intransitive based on its prefixes, but transitive based on some suffixes.
\hixka has lost the main intransitive criteria, making its reflex look more like a transitive verb.
It is not attested as being affected by any of the person marker extensions under discussion.

\subsection{\rc{ɨtə[mə]} \qu{to go}}
\label{sec:go}
This verb is reconstructed by \textcite{gildea2007greenberg} as \rc{tə[mə]}, like \rc{ka[ti]} \qu{to say} with a fleeting second syllable.
It is true that many reflexes are clearly \obj{t}-initial, for example \hixka \obj{ntoje} \qu{he went} \parencite[27]{hixkaryanaderby1985}, \trio \obj{təkə} \qu{go!} \parencite[246]{triomeira1999}, or \wayana \obj{kuptəm} \qu{we went} \parencite[195]{wayanatavares2005}.
However, once one considers all forms of the various reflexes of this verb \pcref{tab:go}, an initial vowel \rc{ɨ} must clearly be reconstructed -- in contrast to unambiguously C-initial \rc{ka[ti]} \qu{to say}.%
\footnote{As indicated by the brackets in \cref{tab:go}, there are many languages where the initial vowel is only present in some forms.
Also, the prefix-verb boundary in many inflected forms like e.g.\ \trio \obj{wɨtənne} or \arara \obj{wɨdolɨ} \qu{I went} \parencites[43]{triomeira1999}[153]{alves2017arara} is ambiguous, since an epenthetic \obj{ɨ} breaks up potential CC clusters.}
This verb was not affected by the any of the extensions discussed in \cref{sec:extensions}.

\begin{table}
\centering
\caption[Reflexes of \rc{ɨtə(mə)} \qu{to go}]{Reflexes of \rc{ɨtə(mə)} \qu{to go} \parencites[291]{cruz2005fonologia}[292]{triomeira1999}[195]{wayanatavares2005}[87]{gildea1994akuriyo}{alves2017arara}[27, 248]{hixkaryanaderby1985}[45, 62]{waiwaihawkins1998}[54, 80]{ikpengpacheco2001}[112, 374]{von1892bakairi}[181, 216]{maquiritaricaceres2011}[112]{meira1998proto}[168]{hoff1968carib}[139]{meira2006syntactic}[4]{caceres2018yawarana}[74]{muller1975mapoyo}[198]{mattei1994diccionario}[48, 50]{macushiabbott1991}[172]{garcia2006diccionario}[6; p.c., Spike Gildea]{franchetto2002kuikuro}}
\label{tab:go}
\begin{tabular}[t]{@{}lllllll@{}}
\toprule
Language &             Form & \multicolumn{5}{l}{Alignment} \\
\midrule
\kaxui    &     \obj{to(mo)} &           &  t &  o &  m &  o \\
\hixka    &      \obj{(ɨ)to} &         ɨ &  t &  o &    &    \\
\waiwai   &   \obj{(e)to(m)} &         e &  t &  o &  m &    \\
\arara    &        \obj{ɨdo} &         ɨ &  d &  o &    &    \\
\ikpeng   &       \obj{aran} &         a &  r &  a &  n &    \\
\ikpeng   &        \obj{ero} &         e &  r &  o &    &    \\
\bakairi  &      \obj{(ɨ)tə} &         ɨ &  t &  ə &    &    \\
\trio     &      \obj{tə(n)} &           &  t &  ə &  n &    \\
\akuriyo  &  \obj{(ə)tə(mɨ)} &         ə &  t &  ə &  m &  ɨ \\
\carijo   &       \obj{təmə} &           &  t &  ə &  m &  ə \\
\wayana   &   \obj{(ɨ)tə(m)} &         ɨ &  t &  ə &  m &    \\
\kalina   &         \obj{to} &           &  t &  o &    &    \\
\kalina   &       \obj{(ɨ)ʔ} &         ɨ &  ʔ &    &    &    \\
\maqui    &    \obj{ɨtə(mə)} &         ɨ &  t &  ə &  m &  ə \\
\ingariko &        \obj{ətə} &         ə &  t &  ə &    &    \\
\pemon    &      \obj{(e)tə} &         e &  t &  ə &    &    \\
\macushi  &      \obj{(a)tɨ} &         a &  t &  ɨ &    &    \\
\panare   &      \obj{tə(n)} &           &  t &  ə &  n &    \\
\yawarana &         \obj{tə} &           &  t &  ə &    &    \\
\mapoyo   &         \obj{tə} &           &  t &  ə &    &    \\
\uxc      &      \obj{(e)te} &         e &  t &  e &    &    \\
\yukpa    &         \obj{to} &           &  t &  o &    &    \\
\bottomrule
\end{tabular}
\end{table}
%
%\ex<gosaycomp>
%\begin{tabular}[t]{@{}lllll@{}}
%& go-\gl{imp} & go-\gl{neg} & say-\gl{imp} & say-\gl{neg} \\
%\wayana & \obj{ɨtə-kə} & \obj{ɨtə-ra} & \obj{kai-kə} & \obj{ka-ra} \\
%\hixka & \obj{ɨto-ko} & \obj{ɨto-hra} & \obj{kas-ko} & \obj{ka-hra} \\
%\apalai & \obj{ɨto-ko} & \obj{ɨto-pɨra} & \obj{kaʃi-ko} & \obj{ka-ra} \\
%\end{tabular}\\
%\parencites[66, 98]{camargo2010wayana}[235, 258]{wayanatavares2005}[47, 54 194]{hixkaryanaderby1985}[65]{derbyshire1965textos}[kuruaz 033, 055]{koehns1994textos}[100]{camargo2002lexico}[107]{koehn1986apalai}
%\xe
%%

\subsection{\rc{(ət-)jəpɨ} \qu{to come}}
\label{sec:come}

\begin{table}[h]
\centering
\caption[Reflexes of \rc{(ət-)jəpɨ} \qu{to come}]{Reflexes of \rc{(ət-)jəpɨ} \qu{to come} \parencites[32]{macushiabbott1991}[102]{alvarez2000construcciones}[125]{akawaiocaesar2003}[299, 415]{cruz2005fonologia}[438]{maquiritaricaceres2011}[178]{robayo2000avance}[168]{meira1998proto}[74]{muller1975mapoyo}[294]{triomeira1999}[113, 150]{alves2017arara}[37]{koehn1986apalai}[265]{ikpengpacheco2001}[160]{stegeman2014akawaio}[4]{meira2003bakairi}[65]{panarepayne2013}[68]{mendez1959yawarana}[429]{courtz2008carib}[182; p.c., Spike Gildea]{meira2005southern}}
\label{tab:come}
\begin{tabular}[t]{@{}lllllllll@{}}
\mytoprule
Language &          Form &    &    &    &    &     &    &    \\
\mymidrule
\kaxui    &  \obj{o[o]hɨ} &    &    &    &    &  oo &  h &  ɨ \\
\kaxui    &    \obj{johɨ} &    &    &    &  j &   o &  h &  ɨ \\
\kaxui    &     \obj{ehɨ} &    &    &    &    &   e &  h &  ɨ \\
\PPek     &      \rc{epɨ} &    &    &    &    &   e &  p &  ɨ \\
\PPek     &    \rc{ədepɨ} &  ə &  d &  - &    &   e &  p &  ɨ \\
\arara    &      \obj{ep} &    &    &    &    &   e &  p &    \\
\arara    &   \obj{odebɨ} &  o &  d &  - &    &   e &  b &  ɨ \\
\arara    &     \obj{ebɨ} &    &    &    &    &   e &  b &  ɨ \\
\ikpeng   &    \obj{arep} &  a &  r &  - &    &   e &  p &    \\
\bakairi  &    \obj{əewɨ} &  ə &    &  - &    &   e &  w &  ɨ \\
\PTir     &    \rc{əʔepɨ} &  ə &  ʔ &  - &    &   e &  p &  ɨ \\
\PTir     &      \rc{epɨ} &    &    &    &    &   e &  p &  ɨ \\
\trio     &     \obj{epɨ} &    &    &    &    &   e &  p &  ɨ \\
\trio     &  \obj{əe[pɨ]} &  ə &    &  - &    &   e &  p &  ɨ \\
\akuriyo  &    \obj{eepɨ} &    &    &    &    &  ee &  p &  ɨ \\
\carijo   &   \obj{eh[ɨ]} &    &    &    &    &   e &  h &  ɨ \\
\apalai   &    \obj{oepɨ} &  o &    &  - &    &   e &  p &  ɨ \\
\kalina   &     \obj{opɨ} &    &    &    &    &   o &  p &  ɨ \\
\maqui    &     \obj{ehə} &    &    &    &    &   e &  h &  ə \\
\akawaio  &    \obj{jepɨ} &    &    &    &  j &   e &  p &  ɨ \\
\akawaio  &   \obj{əsipɨ} &  ə &  s &  - &    &   i &  p &  ɨ \\
\ingariko &    \obj{jepə} &    &    &    &  j &   e &  p &  ə \\
\ingariko &      \obj{jə} &    &    &    &  j &   ə &    &    \\
\patamona &    \obj{jəpɨ} &    &    &    &  j &   ə &  p &  ɨ \\
\patamona &    \obj{jepɨ} &    &    &    &  j &   e &  p &  ɨ \\
\pemon    &    \obj{jepɨ} &    &    &    &  j &   e &  p &  ɨ \\
\panare   &     \obj{əpɨ} &    &    &    &    &   ə &  p &  ɨ \\
\yawarana &     \obj{əpɨ} &    &    &    &    &   ə &  p &  ɨ \\
\mapoyo   &     \obj{epɨ} &    &    &    &    &   e &  p &  ɨ \\
\uxc      &      \obj{ee} &    &    &    &    &  ee &    &    \\
\mybottomrule
\end{tabular}
\end{table}

This verb is reconstructed as \rc{ətepɨ} by \textcite[30]{gildea2007greenberg}, but an inspection of all the attested reflexes \pcref{tab:come} suggests a somewhat more complex story.
Crucially, the \rc{ət} part of their reconstruction is not reflected in the majority of forms, and many languages ostensibly have reflexes of \rc{əpɨ}, \rc{jepɨ}, or \rc{jəpɨ} for the \rc{epɨ} part.
We analyze these forms as going back to a \PC verb of the form \rc{(ət-)jəpɨ}, morphologically segmentable into a detransitivizer and a root \rc{jəpɨ}.

Evidence for the originally \rc{j}-initial nature of the root is found in the Pemongan languages and \kaxui, although the simultaneous presence of \obj{oohɨ} in \kaxui raises the question of whether the \obj{j} is part of the root in \obj{johɨ}.
This form only occurs in the Progressive \exref{kaxprog.kax-82}, and indeed \obj{j(-)} may be a reflex of \rc{i-w-} \qu{\gl{3}-\gl{s_a_}-}, as the \gl{s_a_} class marker \rc{w-} is present with other person values \exref{kaxprog.kax-54}.
However, while C-initial verbs do show a clear reflex of third person \rc{i-} \exref{kaxprog.kax-77}, regular V-initial \gl{s_a_} verbs do not show \obj{j-}, but Ø \exref{kaxprog.kax-83}.
Thus, we are confident that this \obj{j} is indeed part of the root, rather than an outcome of \rc{i-w-}, which in turn allows us to reconstruct \rc{j} back to \PC.

\pex<kaxprog>\kaxui \perscomm{Spike Gildea}
\a<kax-82>
\begingl
\gla johɨ-rɨ//
\glb \gl{3}.come-\gl{prog}//
\glft \qu{S/he is coming.}//
\endgl
\a<kax-54>
\begingl
\gla o-w-ohɨ-rɨ//
\glb \gl{2}-\gl{s_a_}-come-\gl{prog}//
\glft \qu{You are coming.}//
\endgl
\a<kax-77>
\begingl
\gla i-nkɨ-rɨ//
\glb \gl{3}-sleep-\gl{prog}//
\glft \qu{S/he is sleeping.}//
\endgl
\a<kax-83>
\begingl
\gla Ø-osone-rɨ//
\glb \gl{3}-dream-\gl{prog}//
\glft \qu{S/he is dreaming.}//
\endgl
\xe
%
Most forms corresponding to \posscite{gildea2007greenberg} \rc{ətepɨ} do not show evidence for this \rc{j}.
However, we interpret the \obj{i} in the \akawaio form \obj{əsipɨ} as additional evidence of the sequence \rc{jə}, which has the same outcome \obj{i} in the \macushi reflex of bare \rc{jəpɨ}, \obj{ipɨ}.

Turning to the many forms apparently reflecting \rc{əpɨ} and \rc{epɨ}, we find that both are distributed widely in the family, sometimes even co-occurring in the same language.
A unifying account of these forms requires the root \rc{jəpɨ} to undergo two major sound changes: \begin{inlinelist}
 \item \rc{j}-loss
 \item \rc{ə}-umlaut after \rc{j}
 \end{inlinelist}.
Both phenomena are found in other contexts throughout the family \parencite{meira2010origin}.
For this particular verb, these sound changes appear to have applied irregularly, and not always in the same order.
For example, the \kalina form \obj{opɨ} can only be explained if \rc{j} was lost before the umlaut of \rc{ə} to \rc{e}, which would have been triggered by \rc{j}.
On the other hand, forms like \maqui \obj{ehə} must be the result of \phonl{\rc{ə}}{\rc{e}}{\rc{j}}, with subsequent loss of \rc{j}.
The \akuriyo form \obj{eepɨ} looks like a reflex thereof as well, but the length is unexpected, and is analyzed by \textcite[]{meira1998proto} as reflecting an earlier diphthong \rc{əe}.

%Other languages -- across the family -- have forms reflecting the same root without the initial consonant, i.e.\ \rc{əpɨ}: \kalina, \panare, \yawarana, and \uxc.\footnote{\obj{e} is the regular outcome of \rc{ə} in \uxc; \rc{e} became \obj{i} \parencite[176]{meira2005southern}.}
%Also found across the family are forms reflecting \rc{epɨ}, in \kaxui, \arara, \trio, \carijo, \maqui, and \mapoyo.

Our interpretation of the \rc{ət} portion as a detransitivizer is primarily based on its form, as well as its paradigmatically conditioned occurrence in some languages.
Although the semantics of combining a detransitivizer with an intransitive verb do not really make sense, some historical \gl{s_p_} verbs are attested as adding the detransitivizer to become \gl{s_a_} verbs.
For example, the \PC \gl{s_p_} verb \rc{wɨnɨkɨ} \qu{to sleep} becomes \trio \obj{əənɨkɨ} \parencite[252]{triomeira1999} and \kalina \obj{əʔnɨkɨ} \parencite[429]{courtz2008carib}, both \gl{s_a_}.
Also, \waiwai \qu{go to sleep} can be \obj{wɨnɨk} \parencite[30]{waiwaihawkins1998} or \obj{et-wɨnɨk} \parencite[204]{hawkins1953waiwai}.
The parallels to \qu{to sleep} end here, since bare \rc{jəpɨ} \qu{to come} apparently already was an \gl{s_a_} verb, as evidenced by its status in \kaxui, \kalina, \panare \exref{pan-128}, \arara, and \trio.

\ex<pan-128>\panare \parencite[][65]{panarepayne2013}\\
\begingl
\gla ju-w-əəpɨ-n ka=m kanoʔ//
\glb \gl{3}-\gl{s_a_}-come-\gl{nspec} \gl{q}=\gl{2}.\gl{aux} rain//
\glft \qu{Do you think it is gonna rain?}//
\endgl
\xe

While these sound changes and the addition of \rc{ət-} do account for the majority of the forms in \cref{tab:come},\footnote{
Apart from aforementioned \akuriyo \obj{eepɨ}, another irregular form is \apalai \obj{oepɨ}, where the detransitivizer would have the reflex \obj{os-} \parencite[506]{meira2010origin}.
While \obj{oepɨ} would be a regular outcome of \rc{ə-jəpɨ}, the \envr{}{C} allomorph of the detransitivizer is \obj{e-} in \apalai.
The form may be due to borrowing from \trio, which has lost intervocalic \rc{t} to create \obj{əepɨ}.
Alternatively, \apalai \obj{oepɨ} could be a fossilized loan from \wayana, which has replaced its reflex of \rc{ətjəpɨ}, but where regular sound changes would also have resulted in the loss of intervocalic \rc{t} \parencite[63]{wayanatavares2005}.}
the distribution within the family is rather chaotic.
Apart from the distribution of \rc{əpɨ} and \rc{epɨ}, forms with and without \rc{ət-} can be found within the same language, usually conditioned by different prefixes.
This was briefly discussed in \cref{sec:pekodian} for \arara and in \cref{sec:taranoan} for \trio (and \PTir and \PTar).
To illustrate, the \trio \setone paradigm shows a reflex of \rc{ətepɨ} (< \rc{ətjəpɨ}) for first, but of \rc{epɨ} (< \rc{jəpɨ}) for the other persons \exref{tricome}.\footnote{While the \gl{1+2} form is a regular outcome of \rc{kɨt-epɨ}, the second person form is mysterious.}%
\footnote{It should be noted that the paradigmatic distributions of forms with and without \rc{ət-} in different languages share no discernible patterns.}

\ex<tricome> \trio \parencite[294]{triomeira1999}\\
\begin{tabular}[t]{@{}ll@{}}
\gl{1} & \obj{w-əepɨ} \\
\gl{2} &  \obj{mən-epɨ} \\ 
\gl{1+2} &  \obj{ke-epɨ} \\
\gl{3} &  \obj{n-epɨ} \\
\end{tabular}
\xe

Summing up, this verb is highly irregular, both from a synchronic and diachronic perspective.
It seems that reflexes of the detransitivizer \rc{ət(e)-} were optionally added to an \gl{s_a_} verb root \rc{jəpɨ}, which further underwent umlaut and loss of \rc{j}, but in no systematic manner, resulting in the chaotic picture in \cref{tab:come}.

As discussed in \cref{sec:pekodian}, innovative \rc{k-} was introduced on the \ikpeng and \bakairi reflexes of \rc{ətjəpɨ}, but not on the \arara reflex of \rc{jəpɨ}.
Both reflexes of \rc{ətjəpɨ} (\trio) and of \rc{ətjəpɨ} and/or \rc{epɨ} (\akuriyo) resisted the introduction of \rc{t-} in \PTir.
\carijo \obj{ehɨ} shows innovative \obj{j-}, rather than conservative \obj{w-} \exref{car-18}.
It is unknown whether there is a \yukpa reflex of this verb, and it was fully replaced in \PWai by \rc{omokɨ} \qu{to come} and was thus not a potential target of innovative \rc{k-}.

\ex<car-18>\carijo \parencite[][102]{guerrero2019carijo}\\
\begingl
\gla əji-wa-e j-eh-ɨ//
\glb \gl{2}-search-\gl{sup} \gl{1}-come-\gl{pfv}//
\glft \qu{I came looking for you.}//
\endgl
\xe

\subsection{\rc{ɨpɨtə} \qu{to go down}}
\label{sec:godown}
Reflexes of this verb were not affected by the extensions of \rc{k-} in \PPek \pcref{sec:pekodian} and \obj{k-} in \akuriyo \pcref{sec:akuriyo}.
The resistance against the former extension was subsequently overcome in \bakairi; its fate in \ikpeng is unknown.
When \akuriyo extended \obj{k-}, the verb already had an irregularly inflected first person form with \obj{p-}, inherited from \PTir.

At first sight, it may seem that it also was affected by the independent extensions of \obj{j-} in \carijo \exref{gowhere.car-32} and \yukpa \exref{gowhere.yuk-14}.

\pex<gowhere>
\a<car-32> \carijo \perscommpar{David Felipe Guerrero}\\
\begingl
\gla irə wat͡ʃinakano tae j-ehɨtə-e//
\glb \gl{inan}.\gl{ana} body.of.water along.bounded \gl{1}-go.down-\gl{npst}//
\glft \qu{…I go down through that guachinacán.}//
\endgl
\a<yuk-14> \yukpa \parencite[][]{meira2003primeras}\\
\begingl
\glpreamble aw yéwtu//
\gla aw j-ewuhtu//
\glb \gl{1}\gl{pro} \gl{1}-go.down//
\glft \qu{I went down.}//
\endgl
\xe
%
However, a look at the comparative picture suggests a much more complicated situation.
\cref{tab:godown} shows all attested forms, including verb class membership where applicable; parenthesized forms indicate uncertainty about cognacy status.
It turns out that while a form \rc{ɨpɨtə} can be reconstructed to \PC, different (proto-)languages do not agree about the class of this verb.
Its reflexes in languages that preserve the split-\gl{s} system are split fairly evenly between \gl{s_a_} and \gl{s_p_}.

\begin{table}
\centering
\caption[Reflexes of \rc{ɨpɨtə} \qu{to go down}]{Reflexes of \rc{ɨpɨtə} \qu{to go down} \parencites[44, 70]{souza1993arara}[118]{chagas2013verbo}[143]{alves2017arara}[21, 164]{ikpengpacheco2001}[9]{desouza2010arara}[40]{campetela1997analise}}
\label{tab:godown}
\begin{tabular}[t]{@{}llll@{}}
\mytoprule
Language &              Form &                            Source &              Class \\
\mymidrule
\PPar     &        \rc{ɨɸɨto} &                                   &            \gl{sp} \\
\kaxui    &       \obj{ɨhɨto} &                  pc[Spike Gildea] &            \gl{sp} \\
\PWai     &          \rc{hto} &                                   &                  ? \\
\hixka    &         \obj{hto} &          hixkaryanaderby1979[196] &                  ? \\
\waiwai   &         \obj{hto} &             waiwaihawkins1998[55] &                  – \\
\PPek     &         \rc{ɨptə} &                                   &            \gl{sa} \\
\bakairi  &       \obj{ɨtəgɨ} &               von1892bakairi[137] &            \gl{sa} \\
\arara    &       \obj{iptoŋ} &               alves2017arara[153] &            \gl{sa} \\
\ikpeng   &       \obj{iptoŋ} &  pc[Angela Fabíola Alves Chagas] &            \gl{sa} \\
\PTar     &        \rc{ɨpɨtə} &                                   &                  ? \\
\carijo   &       \obj{ehɨtə} &           guerrero2019carijo[118] &                  – \\
\PTir     &         \rc{ɨhtə} &                                   &            \gl{sa} \\
\trio     &        \obj{ɨhtə} &               meira1998proto[116] &            \gl{sa} \\
\akuriyo  &      \obj{ɨ[h]tə} &             gildea1994akuriyo[84] &            \gl{sa} \\
\wayana   &        \obj{ɨptə} &             camargo2010wayana[44] &  \gl{sa} / \gl{sp} \\
\apalai   &        \obj{ɨhto} &             camargo2002lexico[99] &            \gl{sp} \\
\kalina   &      \obj{onɨʔto} &              courtz2008carib[263] &          (\gl{sa}) \\
\maqui    &        \obj{əʔtə} &       maquiritaricaceres2011[450] &            \gl{sp} \\
\kapon    &      \obj{(uʔtə)} &          stegeman2014akawaio[139] &                  – \\
\pemon    &      \obj{(uʔtə)} &         alvarez2008clausulas[139] &                  – \\
\macushi  &      \obj{(autɨ)} &             macushiabbott1991[34] &                  – \\
\panare   &        \obj{əhtə} &         mattei1994diccionario[88] &            \gl{sa} \\
\yawarana &        \obj{əhtə} &            mendez1959yawarana[68] &                  – \\
\yukpa    &  \obj{(ew[uh]tu)} &                 meira2003primeras &                  – \\
\waimiri  &         \obj{ɨtɨ} &           bruno1996dictionary[58] &                  – \\
\mybottomrule
\end{tabular}
\end{table}

In one language, \wayana, the verb shows traits of both classes, leading us to consider it a \dbqu{mixed} verb in our synchronic analysis of \wayana.
It takes the first and second person \gl{s_p_} markers \obj{j-} and \obj{əw-} \parencite[200]{wayanatavares2005}, but the \gl{1+2}\gl{s_a_} marker \obj{kut-} \parencite[206]{wayanatavares2005}.
It also shows the \gl{s_a_} class marker \obj{w-} in nominalizations \exref{waygodown.way-73}, but behaves like an \gl{s_p_} verb in taking a second person prefix in imperatives \exref{waygodown.way-71}.

\pex<waygodown>\wayana \parencite[][200]{wayanatavares2005}
\a<way-73>
\begingl
\glpreamble ïwïptëë//
\gla ɨ-w-ɨptə-rɨ//
\glb \gl{1}-\gl{s_a_}-go.down-\gl{nmlz}//
\glft \qu{my going down}//
\endgl
\a<way-71>
\begingl
\gla əw-ɨptə-k//
\glb \gl{2}-go.down-\gl{imp}//
\glft \qu{Go down!}//
\endgl
\xe
%
Its causativized form is \obj{ɨptə-ka} \parencite[255]{wayanatavares2005}; the restriction of \rc{-ka} to \gl{s_p_} verbs in \PC \parencite{gildea2019overview} points to \gl{s_p_} membership.
These patterns lead us to posit the hypothesis that the verb was a regular member of the \gl{s_p_} class in pre-\wayana, but partially switched to the \gl{s_a_} class.
This in turn implies that all reflexes of this verb with \gl{s_a_} status at some point switched from the \gl{s_p_} class.

\wayana-external comparative evidence supports this hypothesis:
The \arara causativized form is \obj{eniptoŋ} \parencite[66]{alves2017arara}, and \kalina has a cognate form \obj{enɨʔto} \parencite[263]{courtz2008carib}; \obj{onɨʔto} \qu{to go down} in \cref{tab:godown} is a detransitivized form thereof, lit.\ \qu{to get oneself down}.
Both causativized forms contain a reflex of the transitivizer \rc{en-}, which was usually found with \gl{s_p_} verbs \parencite{gildea2019overview}.
Besides the irregular first person \obj{p-}, \trio \obj{ɨhtə} shows other irregularities, in particular in its causativized forms \parencite[263]{triomeira1999}.
Thus, it seems that this verb was originally \gl{s_p_}, but then switched its class in four and a half languages of the family, for so far unknown reasons.

These circumstances make it impossible to answer the question of whether \qu{to go down} was affected by the extensions in \PWai, \PTir, \carijo, and \yukpa.
For \PTir, we cannot establish a relative chronology of the verb class change, the introduction of the idiosyncratic marker \rc{p-}, and the extension of \rc{t-}.
For \PWai, we lack knowledge not only about the first person form, but even about class membership (in \hixka).
For \carijo and \yukpa, we cannot know whether the verb potentially switched class before the breakdown of the entire split-\gl{s} system.
While there is no language-internal evidence for such a switch, \qu{to go down} does have an inclination for class switches; in the case of \carijo, that could have already happened at the \PTar stage.
In all four cases, it is possible that the verb had \gl{s_a_} status at the time of the extension, resisting and keeping its old marker, but it is also possible that it was not even a potential target due to its \gl{s_p_} status.
On the other hand, the class change must have happened before the split-up of \trio and \akuriyo, and therefore this verb resisted the extension of \akuriyo \obj{k-}.
Likewise, it seems very likely that the class change took place before the extension of \PPek \rc{k-}.
Otherwise, the newly-turned-\gl{s_a_} verb would have taken on archaic and lexically heavily restricted \rc{w-}, either in \PTar, \PXin, or \arara. 

\subsection{\rc{e-pɨ} \qu{to bathe}}
\label{sec:bathe}
This verb resisted extensions in \PPek \pcref{sec:pekodian} and \akuriyo \pcref{sec:akuriyo}, but is a regularly inflected \gl{s_a_} verb elsewhere.
Verbs for intransitive \qu{to bathe} are usually detransitivized derivations of transitive roots in Cariban languages, which are reflexes of \rc{pɨ}, or \rc{kupi} in some Venezuelan languages \pcref{tab:bathe}.
As we have shown in \cref{sec:pekodian}, \PPek can be reconstructed as having the pair \rc{ipɨ} (\gl{intr}) / \rc{ɨp(ɨ)} (\gl{tr}).
Thus, while \PPek \qu{to bathe (\gl{tr})} has perfectly regular cognates in other languages of the family, intransitive \qu{to bathe} is divergent in this branch, changing \rc{e-} to \rc{i}.
The reasons for this are entirely mysterious; we are not aware of \rc{i-} as a reflex of the detransitivizer in Pekodian, see also \textcite[506]{meira2010origin}.
However, it should be noted that other languages also show unexpected developments in this verb, considering the apparent glide insertion in Waiwaian or the chaotic distribution of \rc{pɨ} and \rc{kupi} in Venezuelan languages.

\begin{table}
\centering
\caption{Transitive and intransitive \qu{to bathe} \parencites[198]{hixkaryanaderby1979}[192, 203]{waiwaihawkins1998}[150, 162]{alves2017arara}[103]{ikpengpacheco1997}[123]{campetela1997analise}[4]{meira2003bakairi}[285]{meira2005bakairi}[697]{triomeira1999}[87]{gildea1994akuriyo}[24, 52]{camargo2010wayana}[218]{meira2000split}[304]{courtz2008carib}[439, 454]{maquiritaricaceres2011}[37]{stegeman2014akawaio}[34, 129]{pemondearmellada1944dic}[8, 294; p.c., Spike Gildea]{mattei1994diccionario}}
\label{tab:bathe}
\begin{tabular}[t]{@{}lll@{}}
\toprule
Language &               \gl{tr} &     \gl{intr} \\
\midrule
\kaxui   &             \obj{ɨhɨ} &    \obj{eehɨ} \\
\hixka   &             \obj{ɨhɨ} &   \obj{ewehɨ} \\
\waiwai  &              \obj{pɨ} &    \obj{ejeh} \\
\arara   &              \obj{ɨp} &     \obj{ibɨ} \\
\ikpeng  &              \obj{ɨp} &      \obj{ip} \\
\bakairi &               \obj{ɨ} &       \obj{i} \\
\trio    &              \obj{pɨ} &     \obj{epɨ} \\
\akuriyo &              \obj{pɨ} &     \obj{epɨ} \\
\wayana  &             \obj{upɨ} &     \obj{epɨ} \\
\apalai  &              \obj{pɨ} &     \obj{epɨ} \\
\kalina  &            \obj{kupi} &   \obj{ekupi} \\
\maqui   &             \obj{ɨhɨ} &    \obj{eʔhi} \\
\kapon   &           \obj{kuʔpi} &  \obj{ekuʔpi} \\
\pemon   &              \obj{pɨ} &   \obj{ekupɨ} \\
\panare  &  \obj{kupɨ}/\obj{ɨpɨ} &   \obj{akupɨ} \\
\bottomrule
\end{tabular}
\end{table}

%\subsection{\trio \obj{weka}/\obj{oeka} \qu{to defecate}}
%\label{sec:shit}
%The comparative picture for the resistant \trio \gl{s_a_} verb \qu{to defecate} \pcref{tab:defecate} shows that \rc{weka} was clearly originally an \gl{s_p_} verb.
%As in the case of \qu{to go down} \pcref{sec:godown}, this verb has a mixed \gl{s_a_}/\gl{s_p_} status in \wayana.
%The \bakairi reflex \obj{əeke} seems to contain a reflex of \rc{ət-}, which would explain its status as an \gl{s_a_} verb.
%While the class membership of \panare \obj{aiʔka} and \obj{iʔka} \qu{to defecate} is unknown, the former seems to contain a reflex of the detransitivizer, while the latter only consists of the root, meaning that one would assume \gl{s_a_} and \gl{s_p_} membership, respectively.
%
%\begin{table}
\centering
\caption[\rc{weka} \qu{to defecate} as another class-switching \gl{s_p_} verb]{\rc{weka} \qu{to defecate} as another class-switching \gl{s_p_} verb \parencites[418]{courtz2008carib}[455]{maquiritaricaceres2011}[44]{souza1993arara}[118]{alves2013verbo}[86, 206]{wayanatavares2005}[294]{triomeira1999}{meira2005bakairi}[96]{camargo2002lexico}[319; p.c., Spike Gildea]{mattei1994diccionario}}
\label{tab:defecate}
\begin{tabular}[t]{@{}lll@{}}
\toprule
Language &         Form &                  Class \\
\midrule
\apalai  &   \obj{weka} &                      ? \\
\arara   &  \obj{watke} &              \gl{s_p_} \\
\bakairi &   \obj{əeke} &              \gl{s_a_} \\
\ikpeng  &   \obj{atke} &              \gl{s_p_} \\
\kalina  &  \obj{uweka} &              \gl{s_p_} \\
\kaxui   &   \obj{weka} &              \gl{s_p_} \\
\maqui   &   \obj{weka} &              \gl{s_p_} \\
\panare  &   \obj{iʔka} &                      ? \\
\panare  &  \obj{aiʔka} &                      ? \\
\trio    &   \obj{weka} &              \gl{s_a_} \\
\wayana  &   \obj{uika} &  \gl{s_a_} / \gl{s_p_} \\
\bottomrule
\end{tabular}
\end{table}
%
%\subsection{The \akuriyo movement verbs}
%\label{sec:movement}
%As discussed in \cref{sec:akuriyo}, four \akuriyo \gl{s_a_} verbs are attested as not having the first person marker \obj{t͡ʃ-} in \textcite{gildea1994akuriyo}; instead, they have \obj{w-} or its phonologically reduced form ∅.
%These verbs share two properties: they are all \obj{e-}initial, and they are all movement verbs.
%However, the same dataset also contains \obj{e-}initial movement verbs with a first person marker \obj{t͡ʃ-} \exref{aku-106}.
%As for the etymology of these four verbs, we can only speak to \obj{erama} \qu{to return}, which is a detransitivized form of \obj{rama} \qu{to put back}, a pair which is also found in other Cariban languages.
%For the other three verbs, we have not encountered potential cognates, but given their \obj{e-}initial nature and their semantics, it seems plausible that they, too, are derived from transitive verbs with meanings like \qu{to put up} etc.
%In any case, \akuriyo is only sparsely documented, and there are many open questions about it; the \akuriyo data in \textcite{gildea1994akuriyo} are not fully reliable.
%
%\ex<aku-106>\akuriyo \parencite[][84]{gildea1994akuriyo}\\
%\begingl
%\gla t͡ʃ-ewai//
%\glb \gl{1}-sit.down//
%\glft \qu{I sat down.}//
%\endgl
%\xe
