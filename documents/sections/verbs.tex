\subsection{Conservative verbs in comparison}
\label{sec:verbs}
In \cref{sec:extensions}, six incomplete extensions of personal prefixes into \gl{1}\gl{s_a_} territory and the verbs unaffected by them were identified.
This set is rather small in most cases, and many of the verbs remain conservative in different (proto-)languages.
Here, these verbs are investigated from a comparative perspective:
\rc{ka[ti]} \qu{to say} \pcref{sec:say}, \rc{ɨtə[mə]} \qu{to go} \pcref{sec:go}, both roots of the copula \rc{eti} and \rc{a[p]} \pcref{sec:be}, \rc{(ət)jəpɨ} \qu{to come} \pcref{sec:come},  \rc{ɨpɨtə} \qu{to go down} \pcref{sec:godown}, and \rc{e-pɨ} \qu{to bathe} \pcref{sec:bathe}.
The large and phonologically coherent group of \obj{e}-initial verbs not affected by the extension of \akuriyo \obj{k-} \pcref{sec:akuriyo} will not be discussed.

\subsubsection{\rc{ka[ti]} \qu{to say}}
\label{sec:say}
This verb was not affected by any of the extensions in \PPek, \PWai, \PTir, \akuriyo, or \carijo (\crefrange{sec:pekodian}{sec:carijo}), while no first person form of its \yukpa reflex \obj{ka} is attested.
Most reflexes are simply \obj{ka}, but a fleeting syllable \rc{ti} is reconstructed by \textcite{gildea2007greenberg}, visible in the imperative forms of some languages. % \exref{kati}.
%
%\ex<kati> \apalai\\
%\obj{kaʃi-ko} \qu{say!}\\
%\parencite[35]{koehn1986apalai}
%\xe
%
\cref{tab:say} shows a comparison of the longest attested forms for each language.%
\footnote{Cognate segments in \crefrange{tab:say}{tab:bathe} were aligned automatically with LingPy \parencite{lingpy268}, for improved exposition of correspondences.
Brackets indicate segments not present in all forms.}
%
\begin{table}
\centering
\caption[Reflexes of \rc{ka[ti]} \qu{to say}]{Reflexes of \rc{ka[ti]} \qu{to say} \parencites[4]{meira2003bakairi}[48]{franchetto2008absolutivo}[209]{ikpengpacheco2001}[153]{alves2017arara}[182]{hixkaryanaderby1985}[113]{meira1998proto}[107]{koehn1986apalai}[26]{waiwaihawkins1998}[66]{camargo2010wayana}[59]{macushiabbott1991}[123]{swiggers2010gramatica}[430]{courtz2008carib}[125]{akawaiocaesar2003}[102]{mattei1994diccionario}[63; p.c., Spike Gildea]{largo2011yukpa}}
\label{tab:say}
\begin{tabular}[t]{@{}llllll@{}}
\mytoprule
Language & \multicolumn{5}{l}{Form} \\
\midrule
\kaxui   &   \obj{ka[s]} &  k &  a &  s &    \\
\PWai    &    \rc{ka[s]} &  k &  a &  s &    \\
\hixka   &   \obj{ka[h]} &  k &  a &  h &    \\
\waiwai  &   \obj{ka[s]} &  k &  a &  s &    \\
\PPek    &       \rc{ke} &  k &  e &    &    \\
\arara   &      \obj{ke} &  k &  e &    &    \\
\ikpeng  &      \obj{ke} &  k &  e &    &    \\
\bakairi &      \obj{ke} &  k &  e &    &    \\
\PTir    &       \rc{ka} &  k &  a &    &    \\
\trio    &      \obj{ka} &  k &  a &    &    \\
\akuriyo &      \obj{ka} &  k &  a &    &    \\
\carijo  &      \obj{ka} &  k &  a &    &    \\
\wayana  &   \obj{ka[i]} &  k &  a &    &  i \\
\apalai  &  \obj{ka[ʃi]} &  k &  a &  ʃ &  i \\
\kalina  &      \obj{ka} &  k &  a &    &    \\
\kapon   &      \obj{ka} &  k &  a &    &    \\
\pemon   &      \obj{ka} &  k &  a &    &    \\
\macushi &      \obj{ka} &  k &  a &    &    \\
\panare  &   \obj{ka[h]} &  k &  a &  h &    \\
\uxc     &      \obj{ki} &  k &  i &    &    \\
\yukpa   &      \obj{ka} &  k &  a &    &    \\
\bottomrule
\end{tabular}
\end{table}%
%


As mentioned in \cref{sec:waiwaian}, \textcite{hixkaryanaderby1985} analyzes \hixka \obj{ka[s]} as transitive, a choice not only motivated by a desire to avoid an idiosyncratic intransitive first person prefix \obj{ɨ-} instead of regular \obj{kɨ-}.
The verb also shows the complementary distribution of third person \obj{n-} and preceding objects typical of transitive verbs in Cariban \parencite[60--81]{gildea1998}.
Due to its semantics, these objects are either ideophones or direct speech \exref{hixsay}.

\pex<hixsay>\hixka
\a<hix-119>
\begingl
\glpreamble onɨ wyaro nkekonɨ bɨryekomo, tɨyonɨ wya//
\gla onɨ wjaro n-ka-jakonɨ bɨrʲekomo tɨ-jonɨ wja//
\glb this like \gl{3}-say-\gl{rem}.\gl{cont} boy \gl{cor}-mother \gl{obl}//
\glft \qu{This is what the boy said to his mother.} \parencite[][36]{hixkaryanaderby1985}//
\endgl
\a<ordpl-35>
\begingl
\glpreamble moro ha, ketxkoná hatá.//
\gla moro ha ka-jat͡ʃkonɨ hatɨ//
\glb \gl{med}.\gl{dem}.\gl{inan} \gl{ints} say-\gl{rem}.\gl{cont}.\gl{pl} \gl{hsy}//
\glft \qu{“That one there” they said.} \parencite[][14]{derbyshire1965textos}//
\endgl
\xe
In \exref{hixsay.hix-119}, the prefix \obj{n-} occurs because there is no preceding object (\qu{he said it like this}), while does not occur in \exref{hixsay.ordpl-35} where the verb is preceded by direct speech.
Outside of \hixka, at least the \trio reflex shows the same pattern, albeit inconsistently so \parencite[267]{triocarlin2004}.

Reflexes of \rc{ka[ti]} \qu{to say} also show transitive patterns in their derivational suffixes:
In \trio, it is the only intransitive verb (inconsistently) taking transitive \obj{-po} (\gl{caus}) and \obj{-ne} (\gl{agt}.\gl{nmlz}) \parencite[263, 169]{triomeira1999}.
It also exceptionally takes the former suffix in \kalina \parencite[82]{courtz2008carib} and \wayana \parencite[258]{wayanatavares2005}.
The agent nominalizer \rc{-ne} became the \panare inflectional suffix \obj{-ɲe} on transitive verbs \parencite[184--185]{gildea1998}.
The combination of \obj{ka} and \obj{-ɲe} likely led \textcite[214]{panarepayne2013} to categorize it as transitive, disagreeing with  \textcite[102]{mattei1994diccionario}.
Finally, reflexes of the transitive causativizer \rc{-metipo} \parencite{gildea2015valency} are found with \obj{ka} in \apalai \parencite[51]{koehn1986apalai} and \waiwai \parencite[52]{waiwaihawkins1998}.

Arguments in favor of intransitive \qu{to say} primarily come from its inflectional prefixes.
\kalina has a minimal pair between transitive \obj{ka} \qu{to remove} and intransitive \obj{ka} \qu{to say}, \obj{sikai} \qu{I took it away} vs \obj{wɨkai} \qu{I said} \parencite[288, 45]{courtz2008carib}.
Similarly, \PPek \rc{ke} \qu{to say} had \gl{1}\gl{s_a_} \rc{w-} \pcref{sec:pekodian}, rather than \gl{1}>\gl{3} \obj{s-} (\bakairi) or \rc{ini-} (\PXin).
Additionally, languages differentiating transitive from \gl{s_a_} prefixes by the presence of \obj{i} \parencite[495]{meira2010origin} have \obj{ɨ}-final prefixes, see \akuriyo in \exref{sayintr.aku-154}, as well as \textcites[294]{triomeira1999}[195]{wayanatavares2005}[288]{ikpengpacheco2001}[150]{alves2017arara}[168]{hoff1968carib} for cognate forms in other such languages.
Finally, the \gl{s_a_} class marker \obj{w-} occurs on nominalizations in \kalina \exref{sayintr.kar-84}, and is probably reflected as vowel length in the \trio \parencite[333]{triomeira1999} and \wayana \parencite[196]{wayanatavares2005} participles.

\pex<sayintr>
\a<aku-154> \akuriyo \parencite[][113]{meira1998proto}\\
\begingl
\gla mɨ-ka//
\glb \gl{2}-say//
\glft \qu{You said.}//
\endgl
\a<kar-84> \kalina \parencite[][202]{courtz2008carib}\\
\begingl
\glpreamble Òmakon `wa oti ywykàpo kaiko.//
\gla o-ʔma-kon ʔwa oti ɨ-wɨ-ka-ʔpo kai-ko//
\glb \gl{2}-child-\gl{pl} \gl{obl} greeting \gl{1}-\gl{s_a_}-say-\gl{pst}.\gl{nmlz} say-\gl{imp}//
\glft \qu{Pass my greetings to your children.}//
\endgl
\xe

In summary, this verb can be reconstructed as being intransitive based on its (inflectional) prefixes, but transitive based on some (derivational) suffixes.
\hixka has lost the main intransitive criteria, making its reflex look more like a transitive verb.

\subsubsection{\rc{ɨtə[mə]} \qu{to go}}
\label{sec:go}
This verb was not affected by the any of the extensions in \cref{sec:extensions}.
\textcite{gildea2007greenberg} reconstruct it as C-initial \rc{tə[mə]}, like \rc{ka[ti]} \qu{to say} with a fleeting second syllable.
While many reflexes are unambiguously \obj{t}-initial \parentext{e.g. \hixka \obj{ntoje} \qu{he went} \parencite[27]{hixkaryanaderby1985}, \trio \obj{təkə} \qu{go!} \parencite[246]{triomeira1999}, or \wayana \obj{kuptəm} \qu{we went} \parencite[195]{wayanatavares2005}}, the contrast with C-initial \rc{ka[ti]} becomes clear once all forms are considered \pcref{tab:go}.
\obj{ɨ} is predominant and can tentatively be reconstructed, although languages throughout the family reflect \rc{ə}.
In many languages \obj{ɨ} only occurs in certain contexts; but even languages with only C-initial roots show ambiguous inflected forms like \trio \obj{wɨtənne} \qu{I went} \parencite[43]{triomeira1999}, since epenthetic \obj{ɨ} breaks up CC clusters on prefix-verb boundaries.



%The \ikpeng form \obj{aran} is compatible with our reconstruction \rc{ɨtən} when considering that \ikpeng \obj{a} is an attested outcome of \rc{ə}: \begin{inlinelist}
%	\item \obj{akari} \qu{dog} in \cref{tab:pxinw} above
%	\item \obj{anma} \qu{path} \parencite[24]{ikpengpacheco2001} from \PC \rc{ətema} \parencite[12]{gildea2007greenberg}
%	\item \obj{jaj} \qu{tree} \parencite[98]{ikpengpacheco2001} from \PC \rc{jəje}
%\end{inlinelist}.
%This attested change of \rc{ə} to \obj{a} need only be preceded by a assimilatory lowering of initial \rc{ɨ} to \rc{ə}, to yield the form \obj{aran} from \rc{ɨtən}.

\begin{table}[h]
\centering
\caption[Reflexes of \rc{ɨtə[mə]} \qu{to go}]{Reflexes of \rc{ɨtə[mə]} \qu{to go} \parencites[291]{cruz2005fonologia}[292]{triomeira1999}[195]{wayanatavares2005}[87]{gildea1994akuriyo}[80, 153]{alves2017arara}[27, 248]{hixkaryanaderby1985}[45, 62]{waiwaihawkins1998}[54, 80]{ikpengpacheco2001}[112, 374]{von1892bakairi}[181, 216]{maquiritaricaceres2011}[112]{meira1998proto}[168]{hoff1968carib}[139]{meira2006syntactic}[4]{caceres2018yawarana}[74]{muller1975mapoyo}[198]{mattei1994diccionario}[48, 50]{macushiabbott1991}[172]{garcia2006diccionario}[6]{franchetto2002kuikuro}[99; p.c., Spike Gildea]{camargo2002lexico}}
\label{tab:go}
\begin{tabular}[t]{@{}lllllll@{}}
\mytoprule
Language &             Form &    &    &    &    &    \\
\mymidrule
\kaxui    &     \obj{to[mo]} &    &  t &  o &  m &  o \\
\PWai     &    \rc{[ɨ]to[m]} &  ɨ &  t &  o &  m &    \\
\hixka    &      \obj{[ɨ]to} &  ɨ &  t &  o &    &    \\
\waiwai   &   \obj{[e]to[m]} &  e &  t &  o &  m &    \\
\PPek     &        \rc{ɨtən} &  ɨ &  t &  ə &  n &    \\
\arara    &        \obj{ɨdo} &  ɨ &  d &  o &    &    \\
\arara    &         \obj{to} &    &  t &  o &    &    \\
\ikpeng   &       \obj{aran} &  a &  r &  a &  n &    \\
\ikpeng   &        \obj{ero} &  e &  r &  o &    &    \\
\bakairi  &      \obj{[ɨ]tə} &  ɨ &  t &  ə &    &    \\
\PTir     &        \rc{təmɨ} &    &  t &  ə &  m &  ɨ \\
\trio     &      \obj{tə[n]} &    &  t &  ə &  n &    \\
\akuriyo  &  \obj{[ə]tə[mɨ]} &  ə &  t &  ə &  m &  ɨ \\
\carijo   &       \obj{təmə} &    &  t &  ə &  m &  ə \\
\wayana   &   \obj{[ɨ]tə[m]} &  ɨ &  t &  ə &  m &    \\
\apalai   &        \obj{ɨto} &  ɨ &  t &  o &    &    \\
\kalina   &         \obj{to} &    &  t &  o &    &    \\
\kalina   &       \obj{[ɨ]ʔ} &  ɨ &  ʔ &    &    &    \\
\maqui    &    \obj{ɨtə[mə]} &  ɨ &  t &  ə &  m &  ə \\
\ingariko &        \obj{ətə} &  ə &  t &  ə &    &    \\
\pemon    &      \obj{[e]tə} &  e &  t &  ə &    &    \\
\macushi  &      \obj{[a]tɨ} &  a &  t &  ɨ &    &    \\
\panare   &      \obj{tə[n]} &    &  t &  ə &  n &    \\
\yawarana &         \obj{tə} &    &  t &  ə &    &    \\
\mapoyo   &         \obj{tə} &    &  t &  ə &    &    \\
\uxc      &      \obj{[e]te} &  e &  t &  e &    &    \\
\yukpa    &         \obj{to} &    &  t &  o &    &    \\
\mybottomrule
\end{tabular}
\end{table}
%
%\ex<gosaycomp>
%\begin{tabular}[t]{@{}lllll@{}}
%& go-\gl{imp} & go-\gl{neg} & say-\gl{imp} & say-\gl{neg} \\
%\wayana & \obj{ɨtə-kə} & \obj{ɨtə-ra} & \obj{kai-kə} & \obj{ka-ra} \\
%\hixka & \obj{ɨto-ko} & \obj{ɨto-hra} & \obj{kas-ko} & \obj{ka-hra} \\
%\apalai & \obj{ɨto-ko} & \obj{ɨto-pɨra} & \obj{kaʃi-ko} & \obj{ka-ra} \\
%\end{tabular}\\
%\parencites[66, 98]{camargo2010wayana}[235, 258]{wayanatavares2005}[47, 54 194]{hixkaryanaderby1985}[65]{derbyshire1965textos}[kuruaz 033, 055]{koehns1994textos}[100]{camargo2002lexico}[107]{koehn1986apalai}
%\xe
%%

\subsubsection{\rc{eti} and \rc{a[p]} \qu{to be}}
\label{sec:be}
\rc{a[p]} is the older copula and already had various irregularities in \PC \parencite{gildea2018reconstructing}.
\rc{eti} is reconstructed by \textcites{meira2009property}{gildea2018reconstructing} as originally meaning \qu{to dwell, live}, but serving as a copula already in \PC.\footnote{Such a stative, locative source is also suggested by the existence of \obj{it͡ʃi} \qu{to lie down} in \arara \parencite[196]{alves2017arara}.}
Reflexes of these roots are used suppletively, conditioned by person and\slash{}or \gl{tam}.
Both roots preserved \gl{1}\gl{s_a_} \rc{w-} in \PPek, \PWai, and \PTir (\crefrange{sec:pekodian}{sec:taranoan}).
\akuriyo \obj{a} was not targeted by the extension of \obj{k-} \pcref{sec:akuriyo}, while no first person form of \obj{eʔi} is attested. 
\carijo innovated \obj{j-}, but only in the reflex of \rc{eti} \exref{mivida-12}; the \obj{a} root preserves \obj{w-} \pcref{sec:carijo}.
\yukpa introduced \obj{j-} to the reflexes of both \rc{a[p]} and \rc{eti}, which are preserved as encliticized auxiliaries  \exref{yukcop}.

\ex<mivida-12>\carijo \parencite[][177]{robayo1989rame}\\
\begingl
\glpreamble iretibə et͡ʃinəme gərə jet͡ʃiɨ//
\gla ireti-bə et͡ʃi-nə=me gərə j-et͡ʃi-ɨ//
\glb then-from be-\gl{inf}=\gl{attrz} still \gl{1}-be-\gl{pfv}//
\glft \qu{Then I was already grown up.}//
\endgl
\xe

\ex<yukcop> \yukpa \parencite[143--144]{meira2006syntactic}\\
\begin{tabular}[t]{@{}lll@{}}
	& \gl{npst} & \gl{pst}\\
	\gl{1} & \obj{=j-a(-s)}&\obj{=j-e}\\
	\gl{2} & \obj{=mak(o)}&\obj{=m-e}\\
	\gl{3} & \obj{=mak(o)}&\obj{=n-e}\\
\end{tabular}
\xe
%
A comprehensive comparative overview of both roots is given by \textcite[375--382]{gildea2018reconstructing}; they will not be discussed in detail here.

\subsubsection{\rc{(ət-)epɨ} \qu{to come}}
\label{sec:come}
Innovative \rc{k-} was introduced on the \ikpeng and \bakairi reflexes of \rc{ət-epɨ}, but not on the \arara reflex of \rc{epɨ} \pcref{sec:pekodian}.
The reflex of \rc{ət-epɨ} resisted the introduction of \PTir \rc{t-} \pcref{sec:taranoan}.
\carijo \obj{ehɨ} shows innovative \obj{j-}, rather than conservative \obj{w-} \exref{car-18}.
No \yukpa reflex of this verb is attested, and it was fully replaced in \PWai by \rc{omokɨ} \qu{to come}.

\ex<car-18>\carijo \parencite[][102]{guerrero2019carijo}\\
\begingl
\gla əji-wa-e j-eh-ɨ//
\glb \gl{2}-search-\gl{sup} \gl{1}-come-\gl{pfv}//
\glft \qu{I came looking for you.}//
\endgl
\xe
%
\posscite{gildea2007greenberg} reconstruction \rc{ətepɨ} can be segmented into an optional prefix \rc{ət-} and a root \rc{epɨ}, since reflexes can be grouped into those with a reflex of \rc{ət-} and those without \pcref{tab:come}.
As seen in \cref{tab:comepara}, this division can exist within a single paradigm.

\begin{table}
\centering
\caption[Reflexes of \rc{(ət-)jəpɨ} \qu{to come}]{Reflexes of \rc{(ət-)jəpɨ} \qu{to come} \parencites[32]{macushiabbott1991}[102]{alvarez2000construcciones}[125]{akawaiocaesar2003}[299, 415]{cruz2005fonologia}[438]{maquiritaricaceres2011}[178]{robayo2000avance}[168]{meira1998proto}[74]{muller1975mapoyo}[294]{triomeira1999}[150]{alves2017arara}[37]{koehn1986apalai}[265]{ikpengpacheco2001}[160]{stegeman2014akawaio}[4]{meira2003bakairi}[65]{panarepayne2013}[68]{mendez1959yawarana}[429]{courtz2008carib}[182; p.c., Spike Gildea]{meira2005southern}}
\label{tab:come}
\begin{tabular}[t]{@{}lllllllll@{}}
\mytoprule
Language &         Form &    &    &    &    &     &    &    \\
\mymidrule
\kaxui    &   \obj{oohɨ} &    &    &    &    &  oo &  h &  ɨ \\
\kaxui    &   \obj{johɨ} &    &    &    &  j &   o &  h &  ɨ \\
\kaxui    &    \obj{ehɨ} &    &    &    &    &   e &  h &  ɨ \\
\PPek     &   \rc{ədepɨ} &  ə &  d &  - &    &   e &  p &  ɨ \\
\PPek     &     \rc{epɨ} &    &    &    &    &   e &  p &  ɨ \\
\arara    &    \obj{ebɨ} &    &    &    &    &   e &  b &  ɨ \\
\arara    &  \obj{odebɨ} &  o &  d &  - &    &   e &  b &  ɨ \\
\ikpeng   &   \obj{arep} &  a &  r &  - &    &   e &  p &    \\
\bakairi  &   \obj{əewɨ} &  ə &    &  - &    &   e &  w &  ɨ \\
\PTir     &   \rc{əʔepɨ} &  ə &  ʔ &  - &    &   e &  p &  ɨ \\
\trio     &    \obj{epɨ} &    &    &    &    &   e &  p &  ɨ \\
\trio     &   \obj{əepɨ} &  ə &    &  - &    &   e &  p &  ɨ \\
\akuriyo  &   \obj{eepɨ} &    &    &    &    &  ee &  p &  ɨ \\
\carijo   &    \obj{ehɨ} &    &    &    &    &   e &  h &  ɨ \\
\apalai   &   \obj{oepɨ} &  o &    &  - &    &   e &  p &  ɨ \\
\kalina   &    \obj{opɨ} &    &    &    &    &   o &  p &  ɨ \\
\maqui    &    \obj{ehə} &    &    &    &    &   e &  h &  ə \\
\akawaio  &  \obj{əsipɨ} &  ə &  s &  - &    &   i &  p &  ɨ \\
\akawaio  &   \obj{jepɨ} &    &    &    &  j &   e &  p &  ɨ \\
\ingariko &     \obj{jə} &    &    &    &  j &   ə &    &    \\
\ingariko &   \obj{jepə} &    &    &    &  j &   e &  p &  ə \\
\patamona &   \obj{jepɨ} &    &    &    &  j &   e &  p &  ɨ \\
\patamona &   \obj{jəpɨ} &    &    &    &  j &   ə &  p &  ɨ \\
\pemon    &   \obj{jepɨ} &    &    &    &  j &   e &  p &  ɨ \\
\panare   &    \obj{əpɨ} &    &    &    &    &   ə &  p &  ɨ \\
\yawarana &    \obj{əpɨ} &    &    &    &    &   ə &  p &  ɨ \\
\mapoyo   &    \obj{epɨ} &    &    &    &    &   e &  p &  ɨ \\
\uxc      &     \obj{ee} &    &    &    &    &  ee &    &    \\
\mybottomrule
\end{tabular}
\end{table}

\begin{table}
\centering
\caption[\rc{ət-epɨ} \qu{to come} in paradigms]{\rc{ət-epɨ} \qu{to come} in paradigms \parencites[113, 150, 153, 156]{alves2017arara}[294; p.c., Spike Gildea]{triomeira1999}}
\label{tab:comepara}
\begin{tabular}[t]{@{}llll@{}}
\mytoprule
{} &          \kaxui &          \arara &          \trio \\
\mymidrule
\gl{1}   &   \obj{w-oohɨ-} &    \obj{w-ebɨ-} &  \obj{w-əepɨ-} \\
\gl{2}   &   \obj{m-oohɨ-} &  \obj{m-odebɨ-} &  \obj{mən-epɨ} \\
\gl{1+2} &  \obj{kɨs-ohɨ-} &  \obj{kud-ebɨ-} &   \obj{ke-epɨ} \\
\gl{3}   &    \obj{n-ehɨ-} &    \obj{t-ebɨ-} &    \obj{n-epɨ} \\
\mybottomrule
\end{tabular}
\end{table}

Long \rc{ət-epɨ} lost \rc{t} in \trio, \apalai, and \bakairi, but only the first development is due to regular sound changes \parencite[31--32]{meira1998proto}.
\akuriyo reflects \rc{əte} as \obj{ee}, where \obj{eepɨ} cannot be a reflex of \rc{epɨ} due the otherwise unexpected vowel length \parencite[114--115]{meira1998proto}.
Other languages coalesced \rc{əe} to \rc{əə}, with \rc{əəpɨ} being reflected in \uxc \parencite[452]{gildea2012classification}, \kalina, \kaxui, \panare, and \yawarana.
\rc{e}-initial transitive verbs detransitivized with \rc{ət-} underwent the same development in \kalina, creating minimal pairs for length \parencite[509--510]{meira2010origin}, while \rc{ət-e} > \rc{əə} is irregular at least in \kaxui and \panare.



The Pemongan languages and \kaxui point to \rc{jəpɨ} rather than \rc{epɨ}, although \kaxui \obj{johɨ} is very rare in contrast to more frequent \obj{o(o)hɨ}.
It only occurs in the third person form of the Progressive, meaning that the \obj{j} may be a reflex of third person \rc{i-}.
However, regular \obj{o}-initial \kaxui verbs have no third person prefix \pcref{tab:kaxprog}, while \obj{i-} occurs with C-initial \obj{to[mo]} \qu{to go}, suggesting that \obj{j} is indeed part of the root.
Putative \PC \rc{jəpɨ} could yield \rc{epɨ} via the two widespread sound changes \phonl{\rc{ə}}{\rc{e}}{\rc{j}}{} and \phonl{\rc{j}}{∅}{\#} \parencite{meira2010origin}.
%Still, one would expect evidence of the sequence \rc{jə} in \yawarana, \mapoyo, or \panare, and the \kaxui form \obj{ehɨ} would also not be accounted for
The argument against this reconstruction is that \phonl{\rc{ə}}{\rc{e}}{\rc{j}} did not happen in \PPar, and only inconsistently in \PPP and \mapoyo-\yawarana \parencites[501--502]{meira2010origin}{gildea2010story}.
Thus, it does not explain \kaxui \obj{ehɨ}, nor the total absence of the sequence \rc{jə} in \panare and \mapoyo-\yawarana.
%
\begin{table}[h]
\centering
\caption{\kaxui \obj{johɨ} \qu{to come} compared with other verbs (Spike Gildea, p.c.)}
\label{tab:kaxprog}
\begin{tabular}[t]{@{}llll@{}}
\mytoprule
{} &                   \qu{to come} &                   \qu{to dream} &                    \qu{to go} \\
\mymidrule
\gl{1}   &  \obj{{\normalfont ∅}-w-oohɨ-} &  \obj{{\normalfont ∅}-w-osone-} &  \obj{{\normalfont ∅}-wɨ-to-} \\
\gl{2}   &                 \obj{o-w-ohɨ-} &                \obj{o-w-osone-} &                 \obj{o-w-to-} \\
\gl{1+2} &                \obj{ku-w-ohɨ-} &               \obj{ku-w-osone-} &                \obj{kɨ-w-to-} \\
\gl{3}   &    \obj{{\normalfont ∅}-johɨ-} &    \obj{{\normalfont ∅}-osone-} &                   \obj{i-to-} \\
\mybottomrule
\end{tabular}
\end{table}%

The clear segmentability of \rc{ət-} in combination with its form suggest that it is a detransitivizing prefix.
Although the combination of a detransitivizer and an intransitive verb makes no sense semantically, some historical \gl{s_p_} verbs are attested as adding the detransitivizer to become \gl{s_a_} verbs.
For example, the \PC \gl{s_p_} verb \rc{wɨnɨkɨ} \qu{to sleep} becomes \trio \obj{əənɨkɨ} \parencite[252]{triomeira1999} and \kalina \obj{əʔnɨkɨ} \parencite[429]{courtz2008carib}, both \gl{s_a_}.
\waiwai \qu{to sleep} can be \obj{wɨnɨk} \parencite[30]{waiwaihawkins1998} or \obj{et-wɨnɨk} \parencite[204]{hawkins1953waiwai}.
However, unlike \qu{to sleep}, bare \rc{epɨ} \qu{to come} apparently was an \gl{s_a_} verb already (although its reflexes in languages with split-\gl{s} mostly co-occur with reflexes of \rc{ət-epɨ}).

\subsubsection{\rc{ɨpɨtə} \qu{to go down}}
\label{sec:godown}
Reflexes of this verb were not affected by the extensions of \rc{k-} in \PPek \pcref{sec:pekodian} and \obj{k-} in \akuriyo \pcref{sec:akuriyo}.
Its resistance against the former extension was later broken in \bakairi, while its fate in \ikpeng is unknown.
When \akuriyo extended \obj{k-}, the verb already had a first person form irregularly inflected with \obj{p-}, inherited from \PTir.
At first sight, it was also affected by the extensions of \obj{j-} in \carijo \exref{gowhere.car-32} and \yukpa \exref{gowhere.yuk-14}.

\pex<gowhere>
\a<car-32> \carijo \perscommpar{David Felipe Guerrero}\\
\begingl
\gla irə wat͡ʃinakano tae j-ehɨtə-e//
\glb \gl{inan}.\gl{ana} body.of.water along.bounded \gl{1}-go.down-\gl{npst}//
\glft \qu{…I go down through that guachinacán.}//
\endgl
\a<yuk-14> \yukpa \parencite[][]{meira2003primeras}\\
\begingl
\glpreamble aw yéwtu//
\gla aw j-ewuhtu//
\glb \gl{1}\gl{pro} \gl{1}-go.down//
\glft \qu{I went down.}//
\endgl
\xe
%
However, a family-wide perspective reveals a more complicated story \pcref{tab:godown}.%
\footnote{
The cognacy status of parenthesized forms in \cref{tab:godown} is uncertain.
The reconstruction of \PPek \rc{ɨptə} treats the additional elements in daughter languages as non-cognate.
\textcite{meira2005southern} identify no correspondence between \bakairi \obj{gɨ} and \ikpeng \obj{ŋ}, and at least the addition of a final \obj{ŋ} in \PXin is attested elsewhere:
		\begin{inlinelist}
			\item \PC \rc{əne} \qu{to see}, \arara and \ikpeng \obj{eneŋ} %\parencites[8]{gildea2007greenberg}[56]{alves2017arara}[25]{ikpengpacheco2001}
			\item \PC \rc{əta} \qu{to hear}, \arara \obj{taŋ}, \ikpeng \obj{iraŋ} %\parencites[8]{gildea2007greenberg}[144]{alves2017arara}[270]{ikpengpacheco2001}
			\item \PC \rc{ənə} \qu{to eat meat}, \arara \obj{oŋoŋ} \qu{to bite} %\parencites[8]{gildea2007greenberg}[57]{alves2017arara}
		\end{inlinelist} \parencites[8]{gildea2007greenberg}[56, 144, 57]{alves2017arara}[25, 270]{ikpengpacheco2001}.}
While a verb \rc{ɨpɨtə} can be reconstructed to \PC, different (proto-)languages do not agree about its class.
Its reflexes in languages that preserve the split-\gl{s} system are distributed fairly evenly between \gl{s_a_} and \gl{s_p_}.

\begin{table}[h]
\centering
\caption[Reflexes of \rc{ɨpɨtə} \qu{to go down}]{Reflexes of \rc{ɨpɨtə} \qu{to go down} \parencites{meira2003primeras}[196]{hixkaryanaderby1979}[55]{waiwaihawkins1998}[118]{guerrero2019carijo}[44]{camargo2010wayana}[99]{camargo2002lexico}[263]{courtz2008carib}[450]{maquiritaricaceres2011}[139]{stegeman2014akawaio}[139]{alvarez2008clausulas}[34]{macushiabbott1991}[88]{mattei1994diccionario}[68]{mendez1959yawarana}[58]{bruno1996dictionary}[84]{gildea1994akuriyo}[153]{alves2017arara}[137]{von1892bakairi}[116; p.c., Angela Fabíola Alves Chagas, Spike Gildea]{meira1998proto}}
\label{tab:godown}
\begin{tabular}[t]{@{}llllllllllllllll@{}}
\mytoprule
Language &            Form &                  Class &    &    &    &    &    &    &    &    &    &    &    &    &    \\
\mymidrule
\PPar     &      \rc{ɨɸɨto} &              \gl{s_p_} &    &    &    &    &  ɨ &  ɸ &  ɨ &  t &  o &    &    &    &    \\
\kaxui    &     \obj{ɨhɨto} &              \gl{s_p_} &    &    &    &    &  ɨ &  h &  ɨ &  t &  o &    &    &    &    \\
\PWai     &        \rc{hto} &                      ? &    &    &    &    &    &  h &    &  t &  o &    &    &    &    \\
\hixka    &       \obj{hto} &                      ? &    &    &    &    &    &  h &    &  t &  o &    &    &    &    \\
\waiwai   &       \obj{hto} &                      – &    &    &    &    &    &  h &    &  t &  o &    &    &    &    \\
\PPek     &       \rc{ɨptə} &              \gl{s_a_} &    &    &    &    &  ɨ &  p &    &  t &  ə &    &    &    &    \\
\arara    &     \obj{iptoŋ} &              \gl{s_a_} &    &    &    &    &  i &  p &    &  t &  o &  - &  ŋ &    &    \\
\ikpeng   &     \obj{iptoŋ} &                      ? &    &    &    &    &  i &  p &    &  t &  o &  - &  ŋ &    &    \\
\bakairi  &     \obj{ɨtəgɨ} &              \gl{s_a_} &    &    &    &    &  ɨ &    &    &  t &  ə &  - &    &  g &  ɨ \\
\PTar     &      \rc{ɨpɨtə} &                      ? &    &    &    &    &  ɨ &  p &  ɨ &  t &  ə &    &    &    &    \\
\PTir     &       \rc{ɨhtə} &              \gl{s_a_} &    &    &    &    &  ɨ &  h &    &  t &  ə &    &    &    &    \\
\trio     &      \obj{ɨhtə} &              \gl{s_a_} &    &    &    &    &  ɨ &  h &    &  t &  ə &    &    &    &    \\
\akuriyo  &      \obj{ɨhtə} &              \gl{s_a_} &    &    &    &    &  ɨ &  h &    &  t &  ə &    &    &    &    \\
\carijo   &     \obj{ehɨtə} &                      – &    &    &    &    &  e &  h &  ɨ &  t &  ə &    &    &    &    \\
\wayana   &      \obj{ɨptə} &  \gl{s_a_} / \gl{s_p_} &    &    &    &    &  ɨ &  p &    &  t &  ə &    &    &    &    \\
\apalai   &      \obj{ɨhto} &              \gl{s_p_} &    &    &    &    &  ɨ &  h &    &  t &  o &    &    &    &    \\
\kalina   &    \obj{onɨʔto} &            (\gl{s_a_}) &  o &  - &  n &  - &  ɨ &  ʔ &    &  t &  o &    &    &    &    \\
\maqui    &      \obj{əʔtə} &              \gl{s_p_} &    &    &    &    &  ə &  ʔ &    &  t &  ə &    &    &    &    \\
\kapon    &    \obj{(uʔtə)} &                      – &    &    &    &    &    &    &    &    &    &    &    &    &    \\
\pemon    &    \obj{(uʔtə)} &                      – &    &    &    &    &    &    &    &    &    &    &    &    &    \\
\macushi  &    \obj{(autɨ)} &                      – &    &    &    &    &    &    &    &    &    &    &    &    &    \\
\panare   &      \obj{əhtə} &              \gl{s_a_} &    &    &    &    &  ə &  h &    &  t &  ə &    &    &    &    \\
\yawarana &      \obj{əhtə} &                      – &    &    &    &    &  ə &  h &    &  t &  ə &    &    &    &    \\
\yukpa    &  \obj{(ewuhtu)} &                      – &    &    &    &    &    &    &    &    &    &    &    &    &    \\
\waimiri  &       \obj{ɨtɨ} &                      – &    &    &    &    &    &    &  ɨ &  t &  ɨ &    &    &    &    \\
\mybottomrule
\end{tabular}
\end{table}

The verb shows traits of both classes in \wayana, making it a \dbqu{mixed} verb synchronically.
It takes the first and second person \gl{s_p_} markers \obj{j-} and \obj{əw-} \parencite[200]{wayanatavares2005}, but the \gl{1+2}\gl{s_a_} marker \obj{kut-} \parencite[206]{wayanatavares2005}.
It also shows the \gl{s_a_} class marker \obj{w-} in nominalizations \exref{waygodown.way-73}, but behaves like an \gl{s_p_} verb in taking a second person prefix in imperatives \exref{waygodown.way-71}.

\pex<waygodown>\wayana \parencite[][200]{wayanatavares2005}
\begin{multicols}{2}
\a<way-73>
\begingl
\glpreamble ïwïptëë//
\gla ɨ-w-ɨptə-rɨ//
\glb \gl{1}-\gl{s_a_}-go.down-\gl{nmlz}//
\glft \qu{my going down}//
\endgl
\a<way-71>
\begingl
\gla əw-ɨptə-k//
\glb \gl{2}-go.down-\gl{imp}//
\glft \qu{Go down!}//
\endgl
\end{multicols}
\xe
%
Its causativized form is \obj{ɨptə-ka} \parencite[255]{wayanatavares2005}; the \PC  causativizer \rc{-ka} was restricted to \gl{s_p_} verbs \parencite{gildea2019overview}.
These patterns point to \qu{to go down} being a regular \gl{s_p_} verb in pre-\wayana, but partially switching to the \gl{s_a_} class by taking a \gl{1+2}\gl{s_a_} prefix and the \gl{s_a_} class marker.
This in turn implies that (inflectionally defined) \gl{s_a_} reflexes in other languages fully switched from \gl{s_p_}.

\wayana-external evidence supports this hypothesis:
The \arara causativized form is \obj{eniptoŋ} \parencite[66]{alves2017arara}, and \kalina has a cognate form \obj{enɨʔto} \parencite[263]{courtz2008carib}; \obj{onɨʔto} \qu{to go down} in \cref{tab:godown} is a detransitivized form thereof, lit.\ \qu{to get oneself down}.
These forms reflect the transitivizer \rc{en-}, occurring with \gl{s_p_} verbs in \PC \parencite{gildea2019overview}.
\trio \obj{ɨhtə} has irregular causativized forms, also with a reflex of \rc{en-} \parencite[263]{triomeira1999}.
In conclusion, \rc{ɨpɨtə} \qu{to go down} was an \gl{s_p_} verb in \PC, but for unknown reasons switched classes in four \dbqu{and a half} languages of the family.

This makes it impossible to tell whether it was affected by most extensions under discussion:
For \PTir, one cannot establish a relative chronology of the class switch, the introduction of idiosyncratic \gl{1}\gl{s_a_} \rc{p-}, and the extension of \rc{t-}.
Its first person form and its inflectional class in \PWai are unknown.
For \carijo and \yukpa, one cannot rule out a verb class switch before the breakdown of the split-\gl{s} system.
While no language-internal evidence supports such a switch, \qu{to go down} is clearly inclined to do so; \carijo may even have inherited it as \gl{s_a_} from \PTar.
In all four cases, the verb could have had \gl{s_a_} or \gl{s_p_} status at the time of the extension, so it is unknown whether it even was a potential target.

On the other hand, a class switch is reconstructible to \PTir, so it was an \gl{s_a_} verb when \akuriyo introduced \obj{k-}.
Likewise, the class switch most likely took place before the extension of \PPek \rc{k-}.
Otherwise, the newly-turned-\gl{s_a_} verb would have taken on conservative and lexically heavily restricted \rc{w-}, either in \PPek, \PXin, or \arara.

\subsubsection{\rc{e-pɨ} \qu{to bathe}}
\label{sec:bathe}
This verb resisted the extensions of \PPek \rc{k-} \pcref{sec:pekodian} and \akuriyo \obj{k-} \pcref{sec:akuriyo}.
It took on new \gl{1}\gl{s_a_} prefixes in \PTir \parentext{\trio \obj{s-epɨ-}, \akuriyo \obj{t͡ʃ-epɨ-} \parencites[292]{triomeira1999}[87]{gildea1994akuriyo}} and \PWai \parentext{\hixka \obj{k-ewehɨ-}, \waiwai \obj{k-ejeɸu-} \parencites[195]{hixkaryanaderby1985}[166]{waiwaihawkins1998}}.
The first person form of its \carijo reflex \obj{ehɨ} \parencites[72]{koch1908hiana} is unknown; an unattested \yukpa cognate may exist.

Verbs for intransitive \qu{to bathe} are typical \gl{s_a_} verbs in most Cariban languages, derived with a detransitivizer from a transitive root.
These roots are reflexes of \rc{[ɨ]pɨ}, or \rc{kupi} in some Venezuelan languages \pcref{tab:bathe}.
\PPek can be reconstructed as having the pair \rc{ipɨ} (\gl{intr}) / \rc{ɨp} (\gl{tr}) \pcref{sec:pekodian}.
Thus, while \PPek \qu{to bathe (\gl{tr})} has perfectly regular cognates in other languages of the family, \qu{to bathe (\gl{intr})} changed \rc{e-} to \rc{i}.
This is an irregular development, since there are no attested reflexes of a Pekodian detransitivizer \rc{i-} \parencite[506]{meira2010origin}; its cause is unknown.
%However, other languages also show unexpected developments in this verb, like the glide insertions in Waiwaian or the distribution of \rc{[ɨ]pɨ} and \rc{kupi} in Venezuelan languages.

\begin{table}
\caption[Comparison of intransitive and transitive \qu{to bathe}]{Comparison of intransitive and transitive \qu{to bathe} \parencites[198]{hixkaryanaderby1979}[192, 203]{waiwaihawkins1998}[150, 162]{alves2017arara}[103]{ikpengpacheco1997}[123]{campetela1997analise}[4]{meira2003bakairi}[285]{meira2005bakairi}[697]{triomeira1999}[87]{gildea1994akuriyo}[24, 52]{camargo2010wayana}[218]{meira2000split}[304]{courtz2008carib}[439, 454]{maquiritaricaceres2011}[37]{stegeman2014akawaio}[34, 129]{pemondearmellada1944dic}[8, 294; p.c., Spike Gildea]{mattei1994diccionario}}
\label{tab:bathe}
\small
\centering
\begin{subtable}[t]{.49\linewidth}
\caption{Reflexes of \rc{e-pɨ} \qu{to bathe (\gl{intr})}}
\label{tab:bathe_intr_1}
\begin{tabular}[t]{@{}llllllll@{}}
\mytoprule
Language &         Form & \multicolumn{6}{l}{Alignment} \\
\midrule
\kaxui   &   \obj{eehɨ} &           &    &  ee &  - &  h &  ɨ \\
\hixka   &  \obj{ewehɨ} &         e &  w &   e &  - &  h &  ɨ \\
\waiwai  &  \obj{ejeɸu} &         e &  j &   e &  - &  ɸ &  u \\
\arara   &    \obj{ibɨ} &           &    &   i &  - &  b &  ɨ \\
\ikpeng  &     \obj{ip} &           &    &   i &  - &  p &    \\
\bakairi &      \obj{i} &           &    &   i &    &    &    \\
\trio    &    \obj{epɨ} &           &    &   e &  - &  p &  ɨ \\
\akuriyo &    \obj{epɨ} &           &    &   e &  - &  p &  ɨ \\
\wayana  &    \obj{epɨ} &           &    &   e &  - &  p &  ɨ \\
\apalai  &    \obj{epɨ} &           &    &   e &  - &  p &  ɨ \\
\bottomrule
\end{tabular}
\caption{Reflexes of \rc{e-kupi} \qu{to bathe (\gl{intr})}}
\label{tab:bathe_intr_2}
\begin{tabular}[t]{@{}lllllllll@{}}
\mytoprule
Language &          Form & \multicolumn{7}{l}{Alignment} \\
\midrule
\kalina &   \obj{ekupi} &         e &  - &  k &  u &    &  p &  i \\
\maqui  &    \obj{eʔhi} &         e &  - &    &    &  ʔ &  h &  i \\
\kapon  &  \obj{ekuʔpi} &         e &  - &  k &  u &  ʔ &  p &  i \\
\pemon  &   \obj{ekupɨ} &         e &  - &  k &  u &    &  p &  ɨ \\
\bottomrule
\end{tabular}
\caption{Reflexes of \rc{ə-kupi} \qu{to bathe (\gl{intr})}}
\label{tab:bathe_intr_3}
\begin{tabular}[t]{@{}llllllll@{}}
\mytoprule
Language &         Form & \multicolumn{6}{l}{Alignment} \\
\midrule
\panare &  \obj{akupɨ} &         a &  - &  k &  u &  p &  ɨ \\
\bottomrule
\end{tabular}
\end{subtable}
\begin{subtable}[t]{.49\linewidth}\caption{Reflexes of \rc{(ɨ)pɨ} \qu{to bathe (\gl{tr})}}
\label{tab:bathe_tr_1}
\begin{tabular}[t]{@{}lllll@{}}
\mytoprule
Language &       Form & \multicolumn{3}{l}{Alignment} \\
\midrule
\kaxui   &  \obj{ɨhɨ} &         ɨ &  h &  ɨ \\
\hixka   &  \obj{ɨhɨ} &         ɨ &  h &  ɨ \\
\waiwai  &   \obj{pɨ} &           &  p &  ɨ \\
\arara   &   \obj{ɨp} &         ɨ &  p &    \\
\ikpeng  &   \obj{ɨp} &         ɨ &  p &    \\
\bakairi &    \obj{ɨ} &           &    &  ɨ \\
\trio    &   \obj{pɨ} &           &  p &  ɨ \\
\akuriyo &   \obj{pɨ} &           &  p &  ɨ \\
\wayana  &  \obj{upɨ} &         u &  p &  ɨ \\
\apalai  &   \obj{pɨ} &           &  p &  ɨ \\
\maqui   &  \obj{ɨhɨ} &         ɨ &  h &  ɨ \\
\pemon   &   \obj{pɨ} &           &  p &  ɨ \\
\panare  &  \obj{ɨpɨ} &         ɨ &  p &  ɨ \\
\bottomrule
\end{tabular}
\caption{Reflexes of \rc{kupi} \qu{to bathe (\gl{tr})}}
\label{tab:bathe_tr_2}
\begin{tabular}[t]{@{}lllllll@{}}
\mytoprule
Language &         Form & \multicolumn{5}{l}{Alignment} \\
\midrule
\kalina &   \obj{kupi} &         k &  u &    &  p &  i \\
\kapon  &  \obj{kuʔpi} &         k &  u &  ʔ &  p &  i \\
\panare &   \obj{kupɨ} &         k &  u &    &  p &  ɨ \\
\bottomrule
\end{tabular}
\end{subtable}\end{table}


\subsection{Summary}
\label{sec:verbsummary}
In \cref{sec:verbs}, the verbs which were unaffected by the extensions in \cref{sec:extensions} were reconstructed, and affected reflexes in the languages under discussion were identified.
\cref{tab:overview} gives an overview of what verbs were affected by which extensions (except for \obj{e}-initial \akuriyo verbs unaffected by the extension of \obj{k-}, as they are a large and predictable group).
In some cases, the verb does not occur, or at least not in a first person form (–), in others that form is unknown (?), and the question of affectedness is often not meaningfully answerable (\textsc{n/a}) for \qu{to go down}.
Every \checkmark stands for a verb affected by an extension, while × represents conservatively inflected verbs, making clear how strongly these verbs tend to resist person marker extensions in different languages.
\cref{sec:motivations} will explore explanations for the fact that the same 1-7 verbs retained their old \gl{1}\gl{s_a_} marker in 6 independent developments, while a plethora of regular \gl{s}\textsubscript{(\gl{a})} verbs took on new markers.

\begin{table}
\centering
\caption{Overview of extensions and (un-)affected verbs}
\label{tab:overview}
\begin{tabular}[t]{@{}llllllll@{}}
\mytoprule
{} & \rc{ka[ti]} &  \rc{ɨtə[mə]} &   \rc{a[p]} &    \rc{eti} & \rc{(ət-)jəpɨ} &    \rc{ɨpɨtə} &   \rc{e-pɨ} \\
{} &    \qu{say} &       \qu{go} &   \qu{be-1} &   \qu{be-2} &      \qu{come} &  \qu{go down} &  \qu{bathe} \\
\midrule
\PWai \rc{k-}     &           × &             × &           × &           × &              – &  \textsc{n/a} &  \checkmark \\
\quad \hixka      &           × &             × &           × &           × &              – &  \textsc{n/a} &  \checkmark \\
\quad \waiwai     &           × &  (\checkmark) &           × &           × &              – &  \textsc{n/a} &  \checkmark \\
\PPek \rc{k-}     &           × &             × &           × &           × &              × &             × &           × \\
\quad \arara      &           × &             × &           × &           × &              × &             × &           × \\
\quad \ikpeng     &           × &    \checkmark &           – &           × &     \checkmark &             ? &           × \\
\quad \bakairi    &           × &             × &           × &           × &     \checkmark &    \checkmark &           × \\
\PTir \rc{t-}     &           × &             × &           × &           × &              × &  \textsc{n/a} &  \checkmark \\
\quad \trio       &           × &             × &           × &           × &              × &  \textsc{n/a} &  \checkmark \\
\quad \akuriyo    &           × &             × &           × &           ? &              × &  \textsc{n/a} &  \checkmark \\
\akuriyo \obj{k-} &           × &             × &           × &           ? &              × &             × &           × \\
\carijo \obj{j-}  &           × &             × &           × &  \checkmark &     \checkmark &  \textsc{n/a} &           ? \\
\yukpa \obj{j-}   &           ? &             × &  \checkmark &  \checkmark &              – &  \textsc{n/a} &           – \\
\bottomrule
\end{tabular}
\begin{legendlist}\item[\checkmark] affected
\item[×] not affected
\item[?] unknown first person prefix
\item[–] does not occur
\item[(\checkmark)] affected with surviving old marker
\item[\textsc{n/a}] not meaningfully answerable
\end{legendlist}\end{table}