\section{Cariban verbal person marking and marker extensions}
The archaic person markers discussed in this paper are the result of incomplete person marker extensions, i.e., modifications to the ancestral prefix system.
This section provides the necessary background for said ancestral system \pcref{sec:pc_person}, the split-\gl{s} system which will be relevant later \pcref{sec:split}, and the concept of person marker extensions \pcref{sec:extensions_intro}.

\subsection{Verbal person marking in \PC}
\label{sec:pc_person}
\PC is reconstructed by \textcite{gildea1998} as using a person paradigm called \setone in its independent verb forms, shown in \cref{tab:pcpers}.\footnote{I use standard IPA symbols in my transcription of Cariban languages, with the exception of coronal rhotics, which I simply represent with \ort{r}, rather than \ort{ɽ} for \wayana or \ort{ɾ̠} for \maqui etc. In languages with strong morphophonological processes and/or subphonemic orthography I show the original transcription in an additional surface line when presented in an interlinearized glossed example. I follow \textcite{gildea2018reconstructing} in using \ort{ə} for the proto-vowel reconstructed by \textcite{meira2005southern}, although it was likely more back \parencite{gildea2010story}.}
The choice of person marker in transitive verbs can be characterized as being conditioned by a basic person hierarchy \fbox{\gl{1}/\gl{2} > \gl{3}}.
The locuphoric markers, those referring to \gl{sap} participants, had two forms, an \gl{a}-oriented one for direct (\gl{sap}>\gl{3}) scenarios and a \gl{p}-oriented one for inverse (\gl{3}>\gl{sap}) scenarios.
This was not the case for third person referents, which only had one marker, since third person only surfaced in nonlocal (\gl{3}>\gl{3}) scenarios, and third-person prefixes did not distinguish different different third persons.
Local scenarios were expressed in a non-transparent manner, by using the \gl{1+2} prefix \rc{k-} in both cases.\footnote{The presence of a \gl{1+2} person value implies that of a \gl{1+3} value. This is expressed with a free pronoun combined with third person morphology in Cariban languages, so it is not represented in this table.}

\begin{table}
	\centering
	\caption{\PC \setone (main clause) person markers \parencites[495]{meira2010origin}[497]{gildea2016referential}}
	\label{tab:pcpers}
\begin{subtable}[b]{.49\linewidth}
\caption{Transitive}
\label{tab:pctrans}
\centering
	\begin{tabular}{@{}lllll@{}}
	\mytoprule
\gl{a}/\gl{p}&		\gl{1}	&	\gl{2}		&	\gl{1+2}	&	\gl{3}	\\
\mymidrule
\gl{1}	&		&	\rc{k-}	&				&	\rc{t(i)-}		\\	
\gl{2}	&	\rc{k-}			&&				&	\rc{m(i)-}		\\
\gl{1+2}&		&				&				&	\rc{kɨt(i)-}		\\
\gl{3}	&	\rc{u(j)-}	&	\rc{ə(j)-}	&	\rc{k-}			&	\rc{n(i)-}		\\
	\mybottomrule
	\end{tabular}
\end{subtable}%
\begin{subtable}[b]{.49\linewidth}
\caption{Intransitive}
\label{tab:pcintrans}
\centering
\begin{tabular}{@{}lll@{}}
\mytoprule
& \gl{s_a_} & \gl{s_p_}  \\
\mymidrule
\gl{1} & \rc{w-} & \rc{u(j)-} \\
\gl{2} & \rc{m-} & \rc{ə(j)-}\\
\gl{1+2} & \rc{kɨt-} & \rc{k-}\\
\gl{3} & \rc{n-} & \rc{n(i)-}\\
\mybottomrule
\end{tabular}	
\end{subtable}
\end{table}

Formally identical or etymologically related markers occured in intransitive verbs, which showed a split-\gl{s} system \pcref{tab:pcintrans}.
That is, \gl{s_a_} verbs took similar markers as the \gl{a}-oriented ones in transitive verbs, with the exception of first person (\gl{1}>\gl{3} \rc{t(i)-} vs \gl{1}\gl{s_a_} \rc{w-}), as well as the absence of \rc{i} after all \gl{s_a_} prefixes.
On the other hand, \gl{s_p_} verbs took markers fully identical to the \gl{p}-oriented ones.
The third person marker in \gl{s_p_} verbs was identical to the one in \gl{3}>\gl{3} scenarios (\rc{n(i)-)}, while the \gl{3}\gl{s_a_} marker did not have an \rc{i}, like the other \gl{s_a_} prefixes.

As in many other split-\gl{s} systems found in languages of the world, the intransitive verbal lexicon was divided into the two classes, with verbs inherently being \gl{s_a_} or \gl{s_p_} verbs.
That is, while in transitive verbs, the choice of the person marker had a crucial semantic contribution, it is predictable in intransitive verbs.
This is illustrated with modern \kalina data in \exref[kar1]{kar2}.

\pex<kar1> \kalina
\a<kar-60>
\begingl
\gla mi-kupi-ja//
\glb \gl{2}>\gl{3}-bathe-\gl{prs}//
\glft \qu{You bathe him/her.} \parencite[][160]{hoff1968carib}//
\endgl
\a<kar-62>
\begingl
\gla a-kupi-ja//
\glb \gl{3}>\gl{2}-bathe-\gl{prs}//
\glft \qu{S/he bathes you.} \parencite[][63]{yamada2011evidentiality}//
\endgl
\xe
%
In \exref{kar1}, the choice between the second person \gl{a}- and \gl{p}-oriented markers \obj{mi-} and \obj{a-} depends on the scenario:
The transitive verb \obj{kupi} \qu{to bathe} takes \obj{mi-} in \gl{2}>\gl{3} scenarios \exref{kar1.kar-60}, but \obj{a-} in \gl{3}>\gl{2} scenarios \exref{kar1.kar-62}.
While intransitive verbs show the same (or very similar) person markers, they contribute no semantic difference here \exref{kar2}.

\pex<kar2> \kalina
\a<kar-59>
\begingl
\gla sipi tɨnka-rɨ m-ekema-non hen//
\glb net pull-\gl{nmlz} \gl{2}-be.afraid-\gl{prs}.\gl{uncert} eh?//
\glft \qu{You're afraid to pull up the net, aren't you? } \parencite[][253]{courtz2008carib}//
\endgl
\a<kar-61>
\begingl
\gla aj-awoi-ja//
\glb \gl{2}-get.up-\gl{prs}//
\glft \qu{You are getting up.} \parencite[][167]{hoff1968carib}//
\endgl
\xe
%
Rather, \obj{ekema} \qu{to be afraid} takes an \gl{a}-oriented marker, since it is an \gl{s_a_} verb \exref{kar2.kar-59}, while the \gl{s_p_} verb \obj{awomɨ} \qu{to get up} takes a \gl{p}-oriented marker \exref{kar2.kar-61}.\footnote{The root \obj{awomɨ} \qu{to get up} is subject to syllable reduction and assimilation to the prefix-initial \obj{j}.}

In fact, the split-\gl{s} system is not only fully lexically conditioned, but there are clear semantic mismatches between class membership and semantics:
\qu{to be afraid} with an \dbqu{agentive} marker can hardly be considered a volitional act, in fact  \qu{you're afraid of pulling up the net} clearly has a second person patient (experiencer).
Similarly, \qu{to get up} with a \dbqu{patientive} marker is a clear semantic mismatch as well.
These mismatches are not isolated cases, as will be discussed in \cref{sec:split}.

\subsection{Defining features and origins of the split-\gl{s} system}
\label{sec:split}
As seen in the previous section, the split-\gl{s} defined two inflectional classes for intransitive verbs within the \setone system.
%\kalina paradigms illustrating the split for each person are given in \cref{tab:kalinaparadigms}.
%\gl{3}\gl{s_a_} differs from \gl{3}\gl{s_p_} by \begin{inlinelist}
%\item not showing the allomorph \obj{ni-} before consonants
%\item not triggering the umlaut of verb-initial \phon{\obj{o}}{\obj{e}}
%\end{inlinelist}, which is shown in \cref{tab:kalinaparadigms}.
%
%\begin{table}[h]
%	\centering
%	\caption{\kalina \gl{s_a_} and \gl{s_p_} person paradigms \parencites[178]{hoff1968carib}[430, 73]{courtz2008carib}[481]{meira2010origin}}
%	\label{tab:kalinaparadigms}
%	\begin{tabular}{@{}lll@{}}
%	\mytoprule
%Person & \gl{s_a_} \obj{opɨ} \qu{to come} & \gl{s_p_} \obj{[o/e]kanumɨ} \qu{to run}\\
%	\mymidrule
%\gl{1} & \obj{w-opɨ-} & \obj{j-ekanumɨ-}\\
%\gl{2} & \obj{m-opɨ-} & \obj{aj-ekanumɨ-}\\
%\gl{1+2} & \obj{kɨt-opɨ-} & \obj{k-okanumɨ-}\\
%\gl{3} & \obj{n-opɨ-} & \obj{n-ekajumɨ-}\\
%	\mybottomrule
%	\end{tabular}
%\end{table}
%
However, there were some other morphological criteria distinguishing \gl{s_a_} from \gl{s_p_} verbs in \PC:
\begin{inlinelist}
\item presence vs absence of the \gl{s_a_} marker \rc{w-}
\item absence vs presence of the second person prefix \rc{ə(j)-} in imperatives
\item presence vs absence of a derivational detransitivizing prefix
\end{inlinelist}.

Many languages show an \gl{s_a_} class marker in deverbalized forms, which can be reconstructed to \PC as \rc{w-}.\footnote{See \textcite[227]{meira2000split}, who identifies reflexes of this morpheme as having \dbqu{no purpose other than being \qu{class markers}, without any obvious semantic or functional load}.}
With \gl{s_a_} verbs, \rc{w-} occurred immediately between the possessive prefixes and the verb stem, while \gl{s_p_} verbs took the bare prefixes.
Reflexes of \rc{w-} in languages from different branches are illustrated in \cref{tab:participles} for participles, and in \cref{tab:nmlz} for nominalizations.
%\kaxui does not preserve \rc{w-} in its participles (\obj{tehurkat͡ʃe} \qu{fallen}, not \ungr{tɨwehurkat͡ʃe}), while \apalai only preserves it there, but not in nominalizations (\obj{jepɨtopo} \qu{my bathing place}, not \ungr{joepɨtopo}).

\begin{table}
	\centering
	\caption{Participles of \gl{s_a_} and \gl{s_p_} verbs \parencites[118, 207]{alves2017arara}[30, 42]{camargo2002lexico}[430, 433]{hoff1968carib}[333, 334]{triomeira1999}[400]{wayanatavares2005}[39]{schuring2018kaxuyana}[232, 244]{panarepayne2013}}
	\label{tab:participles}
\begin{table}
\centering
\caption[Participles of \gl{s_a_} and \gl{s_p_} verbs]{Participles of \gl{s_a_} and \gl{s_p_} verbs \parencites[39]{schuring2018kaxuyana}[118, 207]{alves2017arara}[333, 334]{triomeira1999}[400]{wayanatavares2005}[35]{koehn1986apalai}[kuruaz-154]{koehns1994textos}[430, 433]{hoff1968carib}[232, 244]{panarepayne2013}}
\label{tab:participles}
\begin{tabular}[t]{@{}lll@{}}
\toprule
Language &                               \gl{s_a_} &                        \gl{s_p_} \\
\midrule
\kaxui  &         \obj{t-ehurka-t͡ʃe} \qu{fallen} &       \obj{tɨ-jaʔ-so} \qu{burnt} \\
\arara  &         \obj{t-\emp{o-}ep-te} \qu{come} &      \obj{t-oregrum-te} \qu{sad} \\
\trio   &    \obj{tɨ-\emp{w-}əturu-e} \qu{talked} &       \obj{t-əpəə-se} \qu{tired} \\
\wayana &     \obj{tə-\emp{w-}epɨ-he} \qu{bathed} &    \obj{t-onopɨ-he} \qu{painted} \\
\apalai &        \obj{t-\emp{o-}ɨto-se} \qu{gone} &   \obj{t-ɨhto-se} \qu{gone down} \\
\kalina &  \obj{tu-\emp{w-}oʔka-se} \qu{come out} &       \obj{t-okari-se} \qu{told} \\
\panare &  \obj{t-\emp{o-}tatɨhpə-se} \qu{wailed} &  \obj{tɨ-sɨrɨke-t͡ʃe} \qu{tired} \\
\bottomrule
\end{tabular}
\end{table}
\end{table}

\begin{table}
	\centering
	\caption{Nominalizations of \gl{s_a_} and \gl{s_p_} verbs \parencites[49, 74]{schuring2018kaxuyana}[97]{alves2017arara}[246]{triomeira1999}[130, 409]{wayanatavares2005}[90]{koehn1986apalai}[ner2-003]{koehns1994textos}[135, 392]{hoff1968carib}[390]{panarepayne2013}[23]{mattei1994diccionario}}
	\label{tab:nmlz}
	\begin{tabular}[t]{@{}lll@{}}
\toprule
Language &                                         \gl{s_a_} &                                             \gl{s_p_} \\
\midrule
\kaxui  &       \obj{o-\emp{w-}ehurka-tpɨrɨ} \qu{your fall} &            \obj{o-onenmehɨ-tpɨrɨ} \qu{your waking up} \\
\arara  &        \obj{\emp{w-}orik-tubo} \qu{dancing place} &              \obj{ereŋmɨ-tpo} \qu{killing instrument} \\
\trio   &   \obj{ji-\emp{w-}əturu-to} \qu{(for) my talking} &              \obj{j-emamina-to} \qu{(for) my playing} \\
\wayana &          \obj{ɨ-\emp{w-}əturu-topo} \qu{my story} &        \obj{j-ɨnɨkɨ-topo} \qu{my object for sleeping} \\
\apalai &            \obj{j-epɨ-topo} \qu{my bathing place} &   \obj{j-enuru-topõ-pɨrɨ} \qu{the place of my birth} \\
\kalina &  \obj{a-\emp{w-}ekupi-rɨ} \qu{your taking a bath} &                  \obj{aj-ereʔna-∅} \qu{your fainting} \\
\panare &     \obj{j-\emp{u-}t͡ʃireema-n} \qu{their eating} &  \obj{tj-arunkampətɨ-n} \qu{his hair standing on end} \\
\bottomrule
\end{tabular}

\end{table}

\begin{table}
	\centering
	\caption{Nominalizations of \gl{s_a_} and \gl{s_p_} verbs}
	\label{tab:nmlz}
	\begin{tabular}{@{}llll@{}}
	\mytoprule
Language & \gl{s_a_} verb & \gl{s_p_} verb & Source(s)\\
	\mymidrule
\kaxui & \obj{o-\emp{w-}ehurka-tpɨrɨ} \qu{your fall} & \obj{o-onenmehɨ-tpɨrɨ} \qu{your waking up} & \textcite[49, 74]{schuring2018kaxuyana}\\
\arara & \obj{\emp{w-}orik-tubo} \qu{dancing place} & \obj{ereŋmɨ-tpo} \qu{killing instrument} & \textcite[97]{alves2017arara} \\
\trio & \obj{ji-\emp{w-}əturu-to} \qu{(for) my talking} & \obj{j-emamina-to} \qu{(for) my playing} & \textcite[246]{triomeira1999} \\
\wayana & \obj{ɨ-\emp{w-}əturu-topo} \qu{my story} & \obj{j-ɨnɨkɨ-topo} \qu{my object for sleeping} & \textcite[409, 130]{wayanatavares2005}\\
\kalina & \obj{a-\emp{w-}ekupi-rɨ} \qu{your taking a bath} & \obj{aj-ereʔna-Ø} \qu{your fainting} & \textcite[392, 135]{hoff1968carib}\\
\apalai & \obj{j-epɨ-topo} \qu{my bath place} & \obj{j-enuru-topõ-pɨrɨ} \qu{the place of my birth} & \textcites[90]{koehn1986apalai}[ner2/1]{koehns1994textos}\\
\panare & \obj{j-\emp{u-}t͡ʃireema-n} \qu{their eating} & \obj{tj-arunkampətɨ-n} \qu{his hair standing on end} & \textcites[390]{panarepayne2013}[23]{mattei1994diccionario}\\
	\mybottomrule
	\end{tabular}
\end{table}

The distinction between \gl{s_a_} and \gl{s_p_} is also borne out in imperatives, the suffix for which can be reconstructed as \PC \rc{-kə}.
Here, \gl{s_p_} verbs took the \gl{p}-oriented second person prefix \rc{ə(j)-}, while \gl{s_a_} verbs were unprefixed.
This is illustrated with reflexes in various modern languages in \cref{tab:imp}.
As in the case of the \gl{s_a_} marker \rc{w-} participles and nominalizations, some languages have lost the distinction between \gl{s_a_} and \gl{s_p_} verbs in imperatives, for example \panare, or \kaxui (not shown in \cref{tab:imp}).

\begin{table}
	\centering
	\caption{Imperatives of \gl{s_a_} and \gl{s_p_} verbs}
	\label{tab:imp}
	\begin{tabular}{@{}llll@{}}
	\mytoprule
Language & \gl{s_a_} & \gl{s_p_} & Source(s)\\
	\mymidrule
\hixka & \obj{omoh-ko} \qu{come!} & \obj{\emp{oj-}okajɨm-ko} \qu{go up!} & \textcite[89, 44]{derbyshire1965textos}\\
\arara & \obj{odotpot-ko} \qu{come back!} & \obj{\emp{o-}alum-ko} \qu{jump!} & \textcite[161]{alves2017arara}\\
\trio & \obj{epɨ-kə} \qu{bathe!} & \obj{\emp{ə-}eremina-kə} \qu{sing!} & \textcite[323]{triomeira1999}\\
\wayana & \obj{əməm-kə} \qu{enter!} & \obj{\emp{əw-}eremi-kə} \qu{sing!} &  \textcite[227]{wayanatavares2005} \\
\kalina & \obj{oʔmaʔ-ko} \qu{stop!} & \obj{\emp{aj-}awon-ko} \qu{get up!} & \textcite[190]{hoff1968carib}\\
\apalai & \obj{otuʔ-ko} \qu{eat!} & \obj{\emp{o-}nɨʔ-ko} \qu{sleep!} & \textcites[62]{koehn1986apalai}[Mopo/20]{koehns1994textos} \\
\panare & \obj{ape-ʔ} \qu{flee!} & \obj{ahpən-kə} \qu{jump!} & \textcite[17, 5]{mattei1994diccionario}\\
	\mybottomrule
	\end{tabular}
\end{table}

%\subsubsection{The detransitivizing prefix}
There is one further property uniting \gl{s_a_} verbs, which is not based on inflectional morphology.
As mentioned in \cref{sec:pc_person}, mismatches between the semantics of intransitive verbs and their \gl{a}- or \gl{p}-oriented inflectional morphology are common.
However, the Cariban split-\gl{s} system goes further than all other known such systems, in that its division of the verbal lexicon does not follow any discernible semantic criteria whatsoever.
\textcite{meira2000split} takes a sizable corpus of intransitive verbs from \trio, \kalina, \apalai, and \wayana, and categorizes them by applying different criteria commonly encountered in split-\gl{s} systems.
He shows that neither (non\-)activities, % \parencite[210]{meira2000split}, 
(non-)agency, % \citeyearpar[212]{meira2000split}, 
(in-)animacy, % \citeyearpar[213]{meira2000split},
nor Aktionsart % \citeyearpar[215]{meira2000split} 
satisfactorily predict the class membership of intransitive verbs.

Rather, the reason for a verb to take the \gl{a}- or \gl{p}-oriented prefix is (at least diachronically) a morphological one.
\textcite[217--221]{meira2000split} demonstrates that those intransitive verbs which (etymologically) have a detransitivizing prefix are treated as \gl{s_a_} verbs, while essentially all others are \gl{s_p_} verbs:
\begin{quotation}
Almost all verbs in the \gl{s_a_} class are detransitivized forms of transitive verbs, either synchronically (with still exisiting transitive sources) or diachronically (with reconstructible but no longer existing transitive sources) \parencite[201]{meira2000split}
\end{quotation}
\textcite[221--223]{meira2000split} also argues that the detransitivizing prefixes are indeed deriving \gl{s_a_} verbs, rather than being inflectional in nature:
\begin{inlinelist}
	\item there are a few underived \gl{s_a_} verbs, with no detransitivizing prefix
	\item \gl{s_a_} verbs can develop irregular semantics compared to their transitive counterparts
	\item it is unpredictable whether the \gl{a} or \gl{p} argument of the underlying transitive verb becomes the \gl{s} of the derived \gl{s_a_} verb
	\item some originally derived \gl{s_a_} verbs have lost their transitive counterparts
	\item \dbqu{basic} concepts are expressed as derivations of more complex concepts, like \qu{to dance (\gl{s_a_})} from \qu{to dance with (\gl{tr})}
\end{inlinelist}.
He also notes that this leads to an inflectional split not based in meaning, but rather morphology:

\begin{quotation}
Apparently, the morphological behavior of the \gl{s_a_} verb class is an accidental consequence of the fact that detransitivization, as far back as we can reconstruct, entails all the morphology described […] as typical of \gl{s_a_} verbs. The alignment of person-marking prefixes appears not to be driven by any semantic forces in the language; it is as though they were being dragged by the evolution of the reflexive marker. \parencite[226]{meira2000split}
\end{quotation}

As for the form of this reflexive marker, \textcite[505--512]{meira2010origin} reconstruct two distinct  prefixes for \PC: reciprocal \rc{əte-} and reflexive \rc{e-}, although they have since merged into a single morpheme in modern Cariban languages.
Reflexes of \detrz show diverse meanings, which are best subsumed under \dbqu{detransitive}; this range is illustrated with \trio examples in \exref{triodetrz}.

\ex<triodetrz> \trio \parencites[218--219]{meira2000split}[128, 256]{triomeira1999}\\
\begin{tabular}[t]{@{}llll@{}}
\\
\obj{nonta}  & → & \obj{e-nonta}, & \qu{abandon each other}\\
\qu{abandon} & & \obj{əi-nonta} &  (reciprocal) \\
\\
\obj{suka} & → & \obj{e-suka}, & \qu{wash self}\\
\qu{wash} & & \obj{əi-suka} & (reflexive)\\
\\
\obj{pahka} & → & \obj{e-pahka} & \qu{break (\gl{intr})}\\
\qu{break (\gl{tr})} & & & (anticausative)\\
\\
\obj{puunəpɨ} & → & \obj{əh-puunəpɨ}, & \qu{think, meditate}\\
\qu{think about} & & \obj{əi-puunəpɨ} & (antipassive)\\
\end{tabular}
\xe

%\pex<triodetrz> Semantics of the \trio detransitivizer \parencite[218--219]{meira2000split}
%\a \obj{e-suka}, \obj{əi-suka} \qu{to wash oneself} (reflexive)
%\a \obj{e-nonta}, \obj{əi-nonta} \qu{to abandon each other} (reciprocal)
%\a \obj{e-pahka} \qu{to break} (anticausative)
%\a \obj{əi-puunəəpɨ} \qu{to think} (antipassive)
%\xe
%
The morphological variation featured in \qu{to wash oneself} and \qu{to abandon each other} is due to the mentioned collapse between the two \PC prefixes:
\obj{e-} is a reflex of the reflexive prefix \rc{e-}, while the form \obj{əi-} originates in reciprocal \rc{əte-}.
However, both can occur with either meaning -- at least for these two verbs.

%\subsubsection{Underived \gl{s_a_} verbs}
As mentioned, there are a few \gl{s_a_} verbs that are not derived from transitives, and do not have a reflex of \detrz, either.
\textcite[221]{meira2000split} characterizes these as very old and exceptional, and counts between 5 and 7 verbs, depending on the language.
\textcite{gildea2007greenberg} identify 7 such underived \gl{s_a_} verbs as reconstructible to \PC, shown in \cref{tab:underived2007}.
These verbs will be discussed in more detail in \cref{sec:verbs}.

\begin{table}
	\centering
	\caption{Underived \PC \gl{s_a_} verbs \parencite[30]{gildea2007greenberg}}
	\label{tab:underived2007}
	\begin{tabular}{@{}lll@{}}
	\mytoprule
Form & Meaning\\
\mymidrule
\rc{tə(mə)} & \qu{to go}\\
\rc{ətepɨ} & \qu{to come\subs{1}}\\
\rc{ka(ti)} & \qu{to say}\\
\rc{əmə(mɨ)} & \qu{to enter}\\
\rc{eti} & \qu{to dwell, be\subs{2}}\\
\rc{a(p)} & \qu{to be\subs{1}, say}\\
\rc{əməkɨ} & \qu{to come\subs{2}}\\
	\mybottomrule
	\end{tabular}
\end{table}

\subsection{Person marker extensions in intransitive verbs}
\label{sec:extensions_intro}
The original \PC split-\gl{s} system has been subject to change in many languages, in some even to the point of total loss.
Besides such modifications as for example losing the \gl{s_a_} class marker \rc{w-} \pcref{sec:split} or losing certain person values altogether, these changes have largely been due to person prefixes being extended to new verbs.\footnote{
One could also imagine a scenario whereby verbs of one of the classes are gradually replaced with innovative verbs from the other class, as suggested for \panare by \textcite[225]{meira2000split}.
However, the verbs he sampled came from the \emp{A}-section of \posscite{mattei1994diccionario} dictionary \perscommpar{Spike Gildea}, and \obj{at-}\slash{}\obj{at͡ʃ-}\slash{}\obj{as-} is a frequent reflex of \detrz in \panare, meaning that many \gl{s_a_} verbs are \obj{a}-initial and explaining the 80:40 \gl{s_a_}:\gl{s_p_} ratio in Meira's sample.
Thus, the hypothesis that \panare has largely replaced \gl{s_p_} with \gl{s_a_} verbs remains to be thoroughly tested.
For the moment, the most likely scenario leading to the eventual loss of the split-\gl{s} system includes markers being extended to new verbs.}
There have been many such person marker extensions in Cariban languages, and some are still ongoing.
This is shown by \textcite{gildea1998}, using the Parukotoan languages as an example.
I have reproduced his tables as a tree diagram in \cref{fig:par_ext}, with adapted transcription -- and in the case of \kaxui, the addition of ∅/\obj{j-} as an alternative \gl{1}\gl{s_p_} marker \perscommpar{Spike Gildea}.
Apart from segmental changes to individual morphemes, the following restructuring innovations happened in the \setone paradigm in Parukotoan:

\begin{enumerate}
\item \PPar \begin{enumerate}
	\item \gl{1}\gl{s_a_} \rc{w-} to \gl{1}>\gl{3}
	\item \gl{1+2} \rc{k-} to \gl{1}\gl{s_p_} (completed in \PWai, ongoing in \kaxui)
	\item \gl{1+2} \rc{kɨt-} to \gl{1+2}\gl{s_p_} (completed in \PWai, ongoing in \kaxui)
\end{enumerate}
\item \PWai \begin{enumerate}
\item \gl{1}\gl{s_p_} \rc{k-} to \gl{1}\gl{s_a_}
\item innovative \rc{owɨro j-} \qu{\gl{1}\gl{pro} \gl{lk}} for \gl{1}\gl{p}
\end{enumerate}
\item \waiwai \begin{enumerate}
\item \gl{2}\gl{s_a_} \obj{m-} to \gl{2}\gl{s_p_}\end{enumerate}
\end{enumerate}

\begin{figure}[hbt]
\centering
\setlength{\tabcolsep}{2pt}
\begin{tikzpicture}[
	every node/.append style={align=center},
	level distance=80pt,
]
\Tree[.{\PC{}\pcone}
[.{\PPar{}\ppone}
{\kaxui{}\kaxone}
[.{\PWai{}\pwone}
{\hixka{}\hixone}
{\waiwai{}\waione}
]
]
]
\end{tikzpicture}
  \caption{Person marking extensions in Parukotoan, after \textcite[94]{gildea1998}}
  \label{fig:par_ext}
\end{figure}

\hixka has preserved split-\gl{s} only in the second person prefixes, while \kaxui still shows variation in the first person and \gl{1+2} prefixes.
\waiwai, on the other hand, has lost the system entirely, which notably happened via distinct innovations at three different diachronic stages.
In this case, the loss of the inflectional classes also entailed the loss of the other morphological traces of the system.
That is, the \gl{s_a_} class marker \rc{w-} \pcref{sec:split} was lost in \waiwai, as evidenced by the contrast between the \waiwai and \kaxui deverbal forms in \exref{wailoss.wai-158} and \exref{wailoss.kax-138}.
Similarly, the \gl{2}\gl{s_p_} prefix \obj{a-} was extended to imperatives of (former) \gl{s_a_} verbs \exref[wailoss.wai-112]{wailoss.hix-43}, but only C-initial ones \parencite[62]{waiwaihawkins1998}.

\pex<wailoss>
\a<wai-158> \waiwai \parencite[][98]{waiwaihawkins1998}\\
\begingl
\gla k-eɸɨrka-t͡ʃhe//
\glb \gl{1+2}-fall-\gl{advz}.after//
\glft \qu{after our fall}//
\endgl
\a<kax-138> \kaxui \parencite[][49]{schuring2018kaxuyana}\\
\begingl
\gla ku-w-ehurka-tpɨrɨ//
\glb \gl{1+2}-\gl{s_a_}-fall-\gl{nmlz}.\gl{pst}.\gl{pert}//
\glft \qu{our fall}//
\endgl
\a<wai-112> \waiwai \parencite[][177]{waiwaihawkins1998}\\
\begingl
\gla a-mo-ko//
\glb \gl{2}-come-\gl{imp}//
\glft \qu{come!}//
\endgl
\a<hix-43> \hixka \parencite[][191]{hixkaryanaderby1985}\\
\begingl
\gla m-omokɨ-no//
\glb \gl{2}\gl{s_a_}-come-\gl{imm}//
\glft \qu{You have come.}//
\endgl
\xe

%Such extensions, often only affecting one person value at a time, and the eventual loss of split-\gl{s} are not entirely unexpected, given that the system lacks any semantic basis.

While different cases of loss of split-\gl{s} are discussed by \textcite[91--96]{gildea1998}, this paper focuses on a so far neglected aspect of these person marking extensions.
I argue that they are executed via lexical diffusion, characterized as a type of extension by \textcite[106--115]{harris1995historical}; this hypothesis is supported by three facts.

First of all, the variation in first person and \gl{1+2} prefixes described above for \kaxui is not completely free.
Rather, some verbs only allow for example first person \obj{k-}, but not \obj{j-}, while others can occur with both, which is the expected pattern in a lexical diffusion scenario.
In addition, this is speaker-dependent \perscommpar{Spike Gildea}, which is what one would expect from a change in progress.
%todo put in GOOD kaxui data
Second, while there is no detailed diachronic scenario for the switch of \gl{1}>\gl{3} \rc{t-} and \gl{1}\gl{s_a_} in the Tiriyoan languages \pcref{sec:taranoan}, \textcite[111--112]{meira1998proto} argues that it must have happened gradually rather than instantaneously, and entailed both markers spreading at the same time.
Whether this gradual switch was along ordered lines or not, lexical diffusion must have played a role.
Finally, the process of lexical diffusion is seen most clearly where it was incomplete.
Not all person marker extensions spread through the entire lexicon, but rather stopped short of some verbs.
This is illustrated for \trio first person forms of \gl{s_a_} verbs in \exref{trio1}; regular \gl{s_a_} verbs take phonologically determined allomorphs \obj{s-} (\envr{}{\obj{e}}) and \obj{t-} (\envr{}{\obj{ə}}), but \obj{tən} \qu{to go} takes an \dbqu{irregular} or \dbqu{archaic} prefix \obj{wɨ-}; it was not affected by the spread of innovative \obj{t-}/\obj{s-} and therefore preserves the old prefix.

\ex<trio1> \trio \gl{1}\gl{s_a_} verbs \parencite[292, 294]{triomeira1999}\\
\begin{tabular}[t]{@{}ll@{}}
\obj{s-epɨ} & \qu{I bathed} \\
\obj{s-entapo} & \qu{I yawned}\\
\obj{t-əturu} & \qu{I talked}\\
\obj{t-əənɨkɨ} & \qu{I slept}\\
\obj{wɨ-tən} & \qu{I went}\\
\end{tabular}
\xe
%
This paper is primarily about these incomplete extensions, and the verbs that were not affected by them.

%
%
%For instance, while most \gl{s_a_} verbs in \hixka and \waiwai have a first person marker \obj{k-}, not all do:
%
%\ex<TAG> \waiwai \qu{to say} \perscommpar{Spike Gildea}\\
%\begin{tabular}[t]{@{}ll@{}}
%\gl{1} & \obj{wɨɨ-ka-}\\
%\gl{2} & \obj{mɨɨ-ka-}\\
%\gl{1+2} & \obj{tɨt-ka-}\\
%\gl{3} & \obj{nɨɨ-ka-}\\
%\end{tabular}
%\xe
%%
%From a synchronic perspective, \waiwai \qu{to say} has an irregular first person marker \obj{wɨ-} instead of regular \obj{kɨ-}.
%From a diachronic perspective, this verb is archaic, preserving the old \gl{1}\gl{s_a_} marker \rc{w-}.

That is, apparently ongoing changes like the situation discussed above for \kaxui will not be investigated in detail.
The same is true for extensions that affected the entire verbal lexicon, although we will briefly comment on them.
We have investigated the 19 person marker extensions we are aware of, but only 6 of them left a group of irregularly or archaically inflected verbs.
Interestingly, these verbs are always \gl{s_a_} verbs, and the irregular marker is always a first person one.
While there have been extensions for other person values as well, they never affect \gl{s_a_} verbs, only \gl{s_p_} ones, and they always affect the entire lexicon, at least based on the attested data.
Besides the completed extension of \rc{k-} to \gl{1}\gl{s_p_} and of \rc{tɨt-} to \gl{1+2}\gl{s_p_} in \PWai \pcref{fig:par_ext}, examples include the extension of \gl{1+2}\gl{s_a_} \obj{s(ɨ)-} (< \rc{kɨt-}) in \apalai \pcref{tab:apalai}, of \gl{2}\gl{s_a_} \obj{m(ɨ)-} in \panare \pcref{tab:panare},\footnote{The presence of the third person marker \obj{n-} for \gl{1+2} is due to the wholesale loss of that inflectional value.} or the extension of the entire \gl{s_a_} set in \waimiri \pcref{tab:waimiri}.

\begin{table}[h]
\caption{Some examples for completed extensions \parencite[90--92]{gildea1998}}
\label{tab:completed}
\begin{subtable}[h]{0.24\textwidth}
\centering
\caption{\apalai}
\label{tab:apalai}
\begin{tabular}{@{}lll@{}}
\mytoprule
& \gl{s_a_} & \gl{s_p_}\\
\mymidrule
\gl{1} & ∅/\obj{ɨ-} & \obj{ɨ-}/\obj{j-}\\
\gl{2} & \obj{m(ɨ)-} & \obj{o-}\\
\gl{1+2} & \multicolumn{2}{c}{\emp{\obj{s(ɨ)-}}}\\
\gl{3} & \multicolumn{2}{c}{\obj{n(ɨ)-}}\\
\mybottomrule
\end{tabular}
\end{subtable}
\hfill
\begin{subtable}[h]{0.24\textwidth}
\centering
\caption{\panare}
\label{tab:panare}
\begin{tabular}{@{}lll@{}}
\mytoprule
& \gl{s_a_} & \gl{s_p_}\\
\mymidrule
\gl{1} & \obj{w(ɨ)-} & ∅/\obj{j-}\\
\gl{2} & \multicolumn{2}{c}{\emp{\obj{m(ɨ)-}}}\\
\gl{1+2} & \multicolumn{2}{c}{\obj{n(ɨ)-}}\\
\gl{3} & \multicolumn{2}{c}{\obj{n(ɨ)-}}\\
\mybottomrule
\end{tabular}
\end{subtable}
\hfill
%\begin{subtable}[h]{0.24\textwidth}
%\centering
%\caption{\pemon}
%\label{tab:pemon}
%\begin{tabular}{@{}lll@{}}
%\mytoprule
%& \gl{s_a_} & \gl{s_p_}\\
%\mymidrule
%\gl{1} & \multicolumn{2}{c}{\emp{∅}}\\
%\gl{2} & \multicolumn{2}{c}{\emp{\obj{m-}}}\\
%\gl{1+2} & \multicolumn{2}{c}{\obj{n-}}\\
%\gl{3} & \multicolumn{2}{c}{\obj{n-}}\\
%\mybottomrule
%\end{tabular}
%\end{subtable}
%\hfill
\begin{subtable}[h]{0.24\textwidth}
\centering
\caption{\waimiri}
\label{tab:waimiri}
\begin{tabular}{@{}ll@{}}
\mytoprule
& \gl{s}\\
\mymidrule
\gl{1} & \emp{\obj{w(ɨ)-}/\obj{i-}}\\
\gl{2} & \emp{\obj{m(ɨ)-}}\\
\gl{1+2} & \emp{\obj{h(ɨ)-}}\\
\gl{3} & \obj{n-}/∅\\
\mybottomrule
\end{tabular}
\end{subtable}
\end{table}

The remainder of this paper is structured as follows:
\cref{sec:extensions} investigates each of these six incomplete extensions individually, listing unaffected verbs and reconstructing proto-paradigms, where necessary.
\cref{sec:verbs} takes a comparative look at the verbs that were not affected by these extensions and searches for possible motivations.
Finally, \cref{sec:discussion} discusses the findings and puts them in a general context of language change and morphology.

%For instance, while the extension of \PWai \gl{1}\gl{s_p_} \rc{k-} affected most verbs, like \rc{eɸurka} \qu{to fall}, verbs like \rc{ka(ti)} \qu{to say} were not affected \exref{pwaiext}.


%\ex<pwaiext> Incomplete extension of \PWai \rc{k-} to \gl{1}\gl{s_a_}\\
%Sources: \textcites[510]{howard2001wrought}[227, 39, 4, 198, 249]{hixkaryanaderby1985}[175, 174, 43, 55, 60]{waiwaihawkins1998} \\
%\begin{tabular}[t]{@{}llll@{}}
%& \PWai & \hixka & \waiwai \\
%\qu{fall} & \rc{k-eɸurka-} & \obj{k-ehurka-} & \obj{k-eɸɨrka}\\
%\qu{choke self} & \rc{k-oseʃewnuk-} & \obj{k-oseʃemnuk-} & \obj{k-eʃeʃewnuk-}\\
%\qu{treat self medically} & \rc{k-osoht͡ʃema-} & \obj{k-osoht͡ʃema-} & \obj{k-eseht͡ʃem-}\\
%\qu{come} & \rc{k-omokɨ-} & \obj{k-omokɨ-} & \obj{k-mok-}\\
%\qu{say} & \rc{wɨ-ka-} & \obj{ɨ-ka-} & \obj{wɨɨ-ka-}\\
%\qu{be} & \rc{w-eʃi-} & \obj{w-eʃe-} & \obj{w-eeʃi-}\\
%\end{tabular}
%\xe
%
