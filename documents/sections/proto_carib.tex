\section{The origins of irregular first person inflections}
\label{sec:background}
The irregular first person prefixes from \cref{sec:intro} are relics, inherited from the ancestral \PC system \pcref{sec:pc_person}.
That system underwent much innovation; the mechanism responsible for the irregular forms involves person marker extensions not spreading through the entire \gl{sa} lexicon \pcref{sec:extensions_intro}.
A specific aspect of the system, the \gl{sa} vs \gl{sp} distinction, plays a role in incomplete extensions and is discussed in \cref{sec:split}.
%\cref{sec:outlook} summarizes the main points of this section and provides an outlook for the rest of the paper.

\subsection{\PC person marking and inflectional relics}
\label{sec:pc_person}
\PC is reconstructed by \textcite{gildea1998} as using a person paradigm called \setone in its independent verb forms, shown in \cref{tab:pcpers}.
Person indexation in transitive verbs was conditioned by a basic hierarchy \fbox{\gl{1}/\gl{2} > \gl{3}}.
\gl{sap} markers had two forms, an \gl{a}-oriented one for direct (\gl{sap}>\gl{3}) scenarios and a \gl{p}-oriented one for inverse (\gl{3}>\gl{sap}) scenarios.
There was a single third person marker \rc{n(i)-}, which only surfaced in nonlocal (\gl{3}>\gl{3}) scenarios, without morphologically expressed distinctions between different third person referents.
Local scenarios were marked identically and non-transparently with the \gl{1+2} prefix \rc{k-}.

\begin{table}
	\centering
	\caption[\PC \setone (main clause) person markers]{\PC \setone (main clause) person markers \parencites[495]{meira2010origin}[497]{gildea2016referential}}
	\label{tab:pcpers}
	\begin{subtable}[b]{.49\linewidth}
		\caption{Transitive}
		\label{tab:pctrans}
		\centering
		\begin{tabular}{@{}lllll@{}}
			\mytoprule
			\gl{a}/\gl{p}&		\gl{1}	&	\gl{2}		&	\gl{1+2}	&	\gl{3}	\\
			\mymidrule
			\gl{1}	&		&	\rc{k-}	&				&	\rc{t(i)-}		\\	
			\gl{2}	&	\rc{k-}			&&				&	\rc{m(i)-}		\\
			\gl{1+2}&		&				&				&	\rc{kɨt(i)-}		\\
			\gl{3}	&	\rc{u(j)-}	&	\rc{ə(j)-}	&	\rc{k-}			&	\rc{n(i)-}		\\
			\mybottomrule
		\end{tabular}
	\end{subtable}%
	\begin{subtable}[b]{.49\linewidth}
		\caption{Intransitive}
		\label{tab:pcintrans}
		\centering
		\begin{tabular}{@{}lll@{}}
			\mytoprule
			& \gl{s_a_} & \gl{s_p_}  \\
			\mymidrule
			\gl{1} & \rc{w-} & \rc{u(j)-} \\
			\gl{2} & \rc{m-} & \rc{ə(j)-}\\
			\gl{1+2} & \rc{kɨt-} & \rc{k-}\\
			\gl{3} & \rc{n-} & \rc{n(i)-}\\
			\mybottomrule
		\end{tabular}	
	\end{subtable}
\end{table}

Formally identical or etymologically related markers occured in intransitive verbs, which showed a split-\gl{s} system \pcref{tab:pcintrans}.
\gl{sa} verbs took similar markers as the \gl{a}-oriented ones in transitive verbs, with the exception of first person (\gl{1}\gl{sa} \rc{w-} vs \gl{1}>\gl{3} \rc{t[i]-}) and the absence of \rc{i} after all \gl{sa} prefixes.
On the other hand, \gl{sp} verbs took markers fully identical to the \gl{p}-oriented ones, with \gl{3}\gl{sp} \rc{n(i)-} aligning with \gl{3}>\gl{3} scenarios.

Knowledge about the ancestral system makes it clear that the divergent \hixka and \trio forms in \cref{tab:hixintro,tab:triintro} behave irregularly because they preserve the original \PC \gl{1}\gl{sa} prefix \rc{w-}; they are therefore \firstmention{conservative}.
They contrast with regular \gl{sa} verbs, which are innovative in both languages.
The reflexes of \rc{w-} are \firstmention{relics}, old and restricted to a few lexemes, contrasting with the innovative prefixes found elsewhere.
These verbs and their prefixes are comparable with the few English nouns like \obj{ɒks}, which preserve the old plural suffix \obj{-ən}.
It was once more widespread as the normal plural suffix of the weak inflection, compare German \obj{ɔks-ən} \qu{oxen}, \obj{naːmə-n} \qu{names}, \obj{haːzə-n} \qu{hares}, \obj{bɛːʁ-ən} \qu{bears}.
Since the regular \hixka and \trio prefixes are innovative, the question arises how they developed.


%\begin{table}
\centering
\caption[Some \maqui verbs]{Some \maqui verbs \parencites[98, 180, 183, 223, 224, 363]{maquiritaricaceres2011}}
\label{tab:makintro}
\begin{tabular}[t]{@{}llll@{}}
\mytoprule
{} &                    eat & \qu{to arrive} &    \qu{to go} \\
\midrule
\gl{1}   &  \obj{w-ətəwasint͡ʃə-} &  \obj{w-əʔrɨ-} &  \obj{w-ɨtə-} \\
\gl{2}   &  \obj{m-ətəwasint͡ʃə-} &  \obj{m-əʔrɨ-} &  \obj{m-ɨtə-} \\
\gl{1+2} &  \obj{k-ətəwasint͡ʃə-} &  \obj{k-əʔrɨ-} &  \obj{k-ɨtə-} \\
\gl{3}   &  \obj{n-ətəwasint͡ʃə-} &  \obj{n-əʔrɨ-} &  \obj{n-ɨtə-} \\
\bottomrule
\end{tabular}
\end{table}

\subsection{Person marker extensions and lexical diffusion}
\label{sec:extensions_intro}
In his discussion of the \PC split-\gl{s} system \pcref{sec:split} and reconstruction of the intransitive person prefixes, \textcite[88--96]{gildea1998} shows that the system has undergone many different changes in daughter languages.
The main mechanism of these changes are \firstmention{person marker extensions}, person prefixes being extended to verbal paradigm cells previously occupied by other prefixes.
%\footnote{
%In principle, one could also imagine a scenario in which verbs of one of the classes are gradually replaced with innovative verbs from the other class, as suggested for \panare by \textcite[225]{meira2000split}.
%However, the verbs Meira sampled came from the \emp{A}-section of \posscite{mattei1994diccionario} dictionary \perscommpar{Spike Gildea}, and \obj{at-}\slash{}\obj{at͡ʃ-}\slash{}\obj{as-} is a frequent reflex of \detrz in \panare, meaning that many \gl{sa} verbs are \obj{a}-initial, thus explaining the 80:40 \gl{sa}:\gl{sp} ratio in Meira's sample.
%Thus, the hypothesis that \panare has largely replaced \gl{sp} with \gl{sa} verbs remains to be thoroughly tested.
%For the moment, the most likely scenario leading to the eventual loss of the split-\gl{s} system involves markers being extended to new verbs.}
There have been many person marker extensions in Cariban languages, some still ongoing.
\textcite{gildea1998} illustrates this with the three Parukotoan languages \kaxui, \hixka, and \waiwai.
Apart from segmental changes to individual morphemes, the following innovations happened in the \setone paradigm in Parukotoan:

\ex
    \begin{threeparttable}
    	\begin{tabular}[t]{ll}
    		\PPar & \gl{1}\gl{sa} \rc{w-} to \gl{1}>\gl{3}\\
    		& \gl{1+2} \rc{k-} to \gl{1}\gl{sp}\tnote{a}\\
    		& \gl{1+2} \rc{kɨt-} to \gl{1+2}\gl{sp}\tnote{a}\\
    		\PWai & \gl{1}\gl{sp} \rc{k-} to \gl{1}\gl{sa}\\
    		& \rc{owɨ(ro) (j-)} \qu{\gl{1}\gl{pro} (\gl{lk})} for \gl{1}\gl{p}\\
    		\waiwai & \gl{2}\gl{sa} \obj{m-} to \gl{2}\gl{sp}
    	\end{tabular}
    \begin{tablenotes}
    	\footnotesize
    	\item[a] Completed in \PWai, ongoing in \kaxui.
    \end{tablenotes}
    	\end{threeparttable}
\xe
%
%\begin{enumerate}
%	\item \PPar \begin{enumerate}
%		\item \gl{1}\gl{sa} \rc{w-} to \gl{1}>\gl{3}
%		\item \gl{1+2} \rc{k-} to \gl{1}\gl{sp} (completed in \PWai, ongoing in \kaxui)
%		\item \gl{1+2} \rc{kɨt-} to \gl{1+2}\gl{sp} (completed in \PWai, ongoing in \kaxui)
%	\end{enumerate}
%	\item \PWai \begin{enumerate}
%		\item \gl{1}\gl{sp} \rc{k-} to \gl{1}\gl{sa}
%		\item innovative \rc{owɨro j-} \qu{\gl{1}\gl{pro} \gl{lk}} for \gl{1}\gl{p}
%	\end{enumerate}
%	\item \waiwai \begin{enumerate}
%		\item \gl{2}\gl{sa} \obj{m-} to \gl{2}\gl{sp}\end{enumerate}
%\end{enumerate}
%
All innovations are person marker extensions except the innovative first person P form, which consists of a pronoun, combined with the linker \rc{j-} for V-initial stems.
\cref{fig:par_ext} shows them in bold and reproduces \posscite{gildea1998} tables as a tree diagram, with adapted transcription and an additional \kaxui \gl{1}\gl{sp} marker ∅/\obj{j-} \perscommpar{Spike Gildea}.
%
\begin{figure}[hbt]
	\centering
	\setlength{\tabcolsep}{2pt}
	\fbox{\begin{tikzpicture}[
			every node/.append style={align=center},
			level distance=80pt,
			]
			\Tree[.{\PC{}\pcone}
			[.{\PPar{}\ppone}
			{\kaxui{}\kaxone}
			[.{\PWai{}\pwone}
			{\hixka{}\hixone}
			{\waiwai{}\waione}
			]
			]
			]
	\end{tikzpicture}}
	\caption[Person marker extensions in Parukotoan]{Person marker extensions in Parukotoan, after \textcite[94]{gildea1998}}
	\label{fig:par_ext}
\end{figure}
%
\hixka has preserved split-\gl{s} only in the second person prefixes, while \kaxui still shows the variation reconstructible to \PPar in its first person and \gl{1+2} prefixes.
\waiwai has lost the system entirely, which notably happened via three diachronically distinct innovations.
%In this case, the loss of the inflectional classes also entailed the loss of the other morphological traces of the split-\gl{s} system.
%That is, the \gl{sa} class marker \rc{w-} \pcref{sec:split} was lost in \waiwai, as evidenced by the contrast between the \waiwai and \kaxui deverbal forms in \exref{wailoss.wai-158} and \exref{wailoss.kax-138}.
%Similarly, the \gl{2}\gl{sp} prefix \obj{a-} was extended to imperatives of (former) \gl{sa} verbs \exref[wailoss.wai-112]{wailoss.hix-43}, but only C-initial ones \parencite[62]{waiwaihawkins1998}.


%Such extensions, often only affecting one person value at a time, and the eventual loss of split-\gl{s} are not entirely unexpected, given that the system lacks any semantic basis.

For \textcite{gildea1998}, person marker extensions are relevant for the loss of split-\gl{s} and the resulting changes to indexing alignment, whereas this study focuses on a different aspect.
Namely, they most likely took place via lexical diffusion, characterized as a type of extension by \textcite[106--115]{harris1995historical}, a hypothesis supported by three facts.
First, the variation in first person and \gl{1+2} prefixes described above for \kaxui is not completely free.
Some verbs only allow e.g.\ first person \obj{k-}, but not \obj{j-}, while others can occur with both, a pattern expected in a lexical diffusion scenario.
In addition, this is speaker-dependent \perscommpar{Spike Gildea}, which points to an ongoing change.
%todo maybe put in GOOD kaxui data
Second, while there is no detailed diachronic account of the switch of \gl{1}>\gl{3} \rc{t-} and \gl{1}\gl{sa} \rc{w-} in the Tiriyoan languages \pcref{sec:taranoan}, \textcite[111--112]{meira1998proto} argues that it must have happened gradually rather than instantaneously, and entailed both markers spreading simultaneously.
Whether or not this gradual switch followed ordered lines, lexical diffusion must have played a role.

The third argument in favor of the lexical diffusion scenario goes back to conservative forms like those in \hixka and \trio.
In both languages, the innovative \gl{1}\gl{sa} prefixes were introduced by a person marker extension spreading via lexical diffusion.
The continued presence of the old \gl{1}\gl{sa} prefix in a few verbs is the result of the extension stopping short of these verbs, rather than spreading through the entire \gl{sa} lexicon.
In a family-wide search, 18 distinct extensions affecting intransitive verbs were identified, 6 of them incomplete.
The latter have left between 1 and 7 conservatively inflected verbs in 9 Cariban languages \pcref{sec:data}.

Interestingly, all six featured innovative first person markers on \gl{sa} verbs.
All other (complete) extensions%
\footnote{As an honorable mention, when \ikpeng replaced third person \setone with \settwo prefixes, \obj{a} \qu{to be} and \obj{ke} \qu{to say} retained \obj{n-} \parencite[12]{matter2019arara}.
	However, innovative markers were not spreading within a paradigm, but rather from former subordinate to main clauses.} either occurred with other person values and/or targeted \gl{sp} verbs.
Illustrative examples for complete extensions are shown in \cref{tab:completed}: the extension of \gl{1+2}\gl{sa} \obj{s(ɨ)-} (< \rc{kɨt-}) to \gl{sp} verbs in \apalai \pcref{tab:apalai}, of \gl{2}\gl{sa} \obj{m(ɨ)-} in to \gl{sp} verbs in \panare \pcref{tab:panare}, and the extension of the entire \gl{sa} set to \gl{sp} verbs in \waimiri \pcref{tab:waimiri}. %, as well as the extension of \rc{k-} to \gl{1}\gl{sp} and of \rc{tɨt-} to \gl{1+2}\gl{sp} in \PWai \pcref{fig:par_ext}.
The starkly different behavior of \gl{sa} and \gl{sp} verbs regarding extensions points to the split-\gl{s} system playing a role, so its main properties will be discussed in \cref{sec:split}.
This will also clarify how the \gl{sa}/\gl{sp} distinction can be lost for a single person, or how \gl{sp} verbs can take on \gl{sa} markers with apparent semantic impunity.

\begin{table}
	\caption[Some examples for completed extensions]{Some examples for completed extensions \parencite[90--92]{gildea1998}}
	\label{tab:completed}
	\begin{subtable}[t]{0.24\textwidth}
		\centering
		\caption{\apalai}
		\label{tab:apalai}
		\begin{tabular}{@{}lll@{}}
			\mytoprule
			& \gl{sa} & \gl{sp}\\
			\mymidrule
			\gl{1} & \obj{ɨ-}/∅ & \obj{ɨ-}/\obj{j-}\\
			\gl{2} & \obj{m(ɨ)-} & \obj{o-}\\
			\gl{1+2} & \multicolumn{2}{c}{\emp{\obj{s(ɨ)-}}}\\
			\gl{3} & \multicolumn{2}{c}{\obj{n(ɨ)-}}\\
			\mybottomrule
		\end{tabular}
	\end{subtable}
	\hfill
	\begin{subtable}[t]{0.24\textwidth}
	\begin{threeparttable}
		\centering
		\caption{\panare}
		\label{tab:panare}
		\begin{tabular}{@{}lll@{}}
			\mytoprule
			& \gl{sa} & \gl{sp}\\
			\mymidrule
			\gl{1} & \obj{w(ɨ)-} & ∅/\obj{j-}\\
			\gl{2} & \multicolumn{2}{c}{\emp{\obj{m(ɨ)-}}}\\
			\gl{1+2} & \multicolumn{2}{c}{\obj{n(ɨ)-}\tnote{a}}\\
			\gl{3} & \multicolumn{2}{c}{\obj{n(ɨ)-}}\\
			\mybottomrule
		\end{tabular}
	\begin{tablenotes}\footnotesize
		\item[a] \gl{1+2} was lost as a person value.
	\end{tablenotes}
	\end{threeparttable}
	\end{subtable}
	\hfill
	%\begin{subtable}[h]{0.24\textwidth}
	%\centering
	%\caption{\pemon}
	%\label{tab:pemon}
	%\begin{tabular}{@{}lll@{}}
	%\mytoprule
	%& \gl{sa} & \gl{sp}\\
	%\mymidrule
	%\gl{1} & \multicolumn{2}{c}{\emp{∅}}\\
	%\gl{2} & \multicolumn{2}{c}{\emp{\obj{m-}}}\\
	%\gl{1+2} & \multicolumn{2}{c}{\obj{n-}}\\
	%\gl{3} & \multicolumn{2}{c}{\obj{n-}}\\
	%\mybottomrule
	%\end{tabular}
	%\end{subtable}
	%\hfill
	\begin{subtable}[t]{0.24\textwidth}
		\centering
		\caption{\waimiri}
		\label{tab:waimiri}
		\begin{tabular}{@{}ll@{}}
			\mytoprule
			& \gl{s}\\
			\mymidrule
			\gl{1} & \emp{\obj{w(ɨ)-}/\obj{i-}}\\
			\gl{2} & \emp{\obj{m(ɨ)-}}\\
			\gl{1+2} & \emp{\obj{h(ɨ)-}}\\
			\gl{3} & \obj{n-}/∅\\
			\mybottomrule
		\end{tabular}
	\end{subtable}
\end{table}

\subsection{The Cariban split-\gl{s} system}
\label{sec:split}
As seen in \cref{sec:pc_person}, the split between \gl{sa} and \gl{sp} verbs was instantiated by inflection classes within the \PC \setone person paradigm, but this was not the only difference:
In nonfinite forms (nominalizations and adverbializations), \gl{sa} verbs took a prefix \rc{w-}, while \gl{sp} verbs lacked that prefix \parencites[89, 141--142]{gildea1998}[208]{meira2000split}.
Also, \gl{sp} verbs took the \gl{2}\gl{sp} prefix \rc{ə(j)-} in imperatives, while \gl{sa} verbs were unprefixed \parencites[89]{gildea1998}[208]{meira2000split}.

%\parencite[208]{meira2000split} Many languages show reflexes \gl{sa}  in deverbalized forms, originating in .
%With \gl{sa} verbs, \rc{w-} occurred immediately between the possessive prefixes and the verb stem, while \gl{sp} verbs took the bare prefixes \pcref{tab:participles,tab:nominalizations}.
%Reflexes of \rc{w-} in languages from different branches are illustrated in \cref{tab:participles,tab:nominalizations} for participles and nominalizations.%
%
%\begin{table}
\centering
\caption[Participles of \gl{s_a_} and \gl{s_p_} verbs]{Participles of \gl{s_a_} and \gl{s_p_} verbs \parencites[39]{schuring2018kaxuyana}[118, 207]{alves2017arara}[333, 334]{triomeira1999}[400]{wayanatavares2005}[35]{koehn1986apalai}[kuruaz-154]{koehns1994textos}[430, 433]{hoff1968carib}[232, 244]{panarepayne2013}}
\label{tab:participles}
\begin{tabular}[t]{@{}lll@{}}
\toprule
Language &                               \gl{s_a_} &                        \gl{s_p_} \\
\midrule
\kaxui  &         \obj{t-ehurka-t͡ʃe} \qu{fallen} &       \obj{tɨ-jaʔ-so} \qu{burnt} \\
\arara  &         \obj{t-\emp{o-}ep-te} \qu{come} &      \obj{t-oregrum-te} \qu{sad} \\
\trio   &    \obj{tɨ-\emp{w-}əturu-e} \qu{talked} &       \obj{t-əpəə-se} \qu{tired} \\
\wayana &     \obj{tə-\emp{w-}epɨ-he} \qu{bathed} &    \obj{t-onopɨ-he} \qu{painted} \\
\apalai &        \obj{t-\emp{o-}ɨto-se} \qu{gone} &   \obj{t-ɨhto-se} \qu{gone down} \\
\kalina &  \obj{tu-\emp{w-}oʔka-se} \qu{come out} &       \obj{t-okari-se} \qu{told} \\
\panare &  \obj{t-\emp{o-}tatɨhpə-se} \qu{wailed} &  \obj{tɨ-sɨrɨke-t͡ʃe} \qu{tired} \\
\bottomrule
\end{tabular}
\end{table}%
%
%\begin{tabular}[t]{@{}lll@{}}
\toprule
Language &                                         \gl{s_a_} &                                             \gl{s_p_} \\
\midrule
\kaxui  &       \obj{o-\emp{w-}ehurka-tpɨrɨ} \qu{your fall} &            \obj{o-onenmehɨ-tpɨrɨ} \qu{your waking up} \\
\arara  &        \obj{\emp{w-}orik-tubo} \qu{dancing place} &              \obj{ereŋmɨ-tpo} \qu{killing instrument} \\
\trio   &   \obj{ji-\emp{w-}əturu-to} \qu{(for) my talking} &              \obj{j-emamina-to} \qu{(for) my playing} \\
\wayana &          \obj{ɨ-\emp{w-}əturu-topo} \qu{my story} &        \obj{j-ɨnɨkɨ-topo} \qu{my object for sleeping} \\
\apalai &            \obj{j-epɨ-topo} \qu{my bathing place} &   \obj{j-enuru-topõ-pɨrɨ} \qu{the place of my birth} \\
\kalina &  \obj{a-\emp{w-}ekupi-rɨ} \qu{your taking a bath} &                  \obj{aj-ereʔna-∅} \qu{your fainting} \\
\panare &     \obj{j-\emp{u-}t͡ʃireema-n} \qu{their eating} &  \obj{tj-arunkampətɨ-n} \qu{his hair standing on end} \\
\bottomrule
\end{tabular}
%
%
%\begin{table}
\centering
\caption[Imperatives of \gl{s_a_} and \gl{s_p_} verbs]{Imperatives of \gl{s_a_} and \gl{s_p_} verbs \parencites[44, 89]{derbyshire1965textos}[161]{alves2017arara}[323]{triomeira1999}[227]{wayanatavares2005}[62]{koehn1986apalai}[Mopo/20]{koehns1994textos}[190]{hoff1968carib}[5, 17]{mattei1994diccionario}}
\label{tab:imperatives}
\begin{tabular}[t]{@{}lll@{}}
\mytoprule
Language &                         \gl{s_a_} &                             \gl{s_p_} \\
\mymidrule
\hixka  &          \obj{omoh-ko} \qu{come!} &  \obj{\emp{oj-}okajɨm-ko} \qu{go up!} \\
\arara  &  \obj{odotpot-ko} \qu{come back!} &      \obj{\emp{o-}alum-ko} \qu{jump!} \\
\trio   &          \obj{epɨ-kə} \qu{bathe!} &   \obj{\emp{ə-}eremina-kə} \qu{sing!} \\
\wayana &         \obj{əməm-kə} \qu{enter!} &    \obj{\emp{əw-}eremi-kə} \qu{sing!} \\
\apalai &           \obj{otuʔ-ko} \qu{eat!} &      \obj{\emp{o-}nɨʔ-ko} \qu{sleep!} \\
\kalina &         \obj{oʔmaʔ-ko} \qu{stop!} &   \obj{\emp{aj-}awon-ko} \qu{get up!} \\
\panare &            \obj{ape-ʔ} \qu{flee!} &             \obj{ahpən-kə} \qu{jump!} \\
\mybottomrule
\end{tabular}
\end{table}%
%
%The distinction between \gl{sa} and \gl{sp} was also reflected in imperatives, where the latter took the \gl{2}\gl{sp} prefix \rc{ə(j)-} while the former were unprefixed .
%This is illustrated with reflexes in various modern languages in \pcref{tab:imperatives}.
%Both the \gl{sa} marker \rc{w-} and the prefix pattern in imperatives have been lost in some languages, but are reconstructible to \PC.
%However, there was one further property uniting most \gl{sa} verbs, not based on inflectional morphology.


%In modern instantiations of Cariban split-\gl{s}, mismatches between the semantics of verbs and their \gl{sa} or \gl{sp} status are common, exemplified with \kalina data in \exref{kar2}.
%
%\input{examples/kar2.tex}
%
%The \gl{sa} verb \obj{ekema} \qu{to be afraid} takes an \gl{a}-oriented marker \exref{kar2.kar-59}, while the \gl{sp} verb \obj{awomɨ} \qu{to get up} takes a \gl{p}-oriented marker \exref{kar2.kar-61}.
%In both cases, the prefix does not appear to contribute to the semantics of the predicate, since there are clear mismatches:
%\qu{to be afraid} with an \dbqu{agentive} marker can hardly be considered a volitional act, while  \qu{to get up} with a \dbqu{patientive} marker is clearly volitional.
%\gl{a}- or \gl{p}-oriented inflectional morphology are common, exemplified with \kalina data in \exref[kar1]{kar2}.
%
%\pex<kar1> \kalina
%\a<kar-60>
%\begingl
%\gla mi-kupi-ja//
%\glb \gl{2}>\gl{3}-bathe-\gl{prs}//
%\glft \qu{You bathe him/her.} \parencite[][160]{hoff1968carib}//
%\endgl
%\a<kar-62>
%\begingl
%\gla a-kupi-ja//
%\glb \gl{3}>\gl{2}-bathe-\gl{prs}//
%\glft \qu{S/he bathes you.} \parencite[][63]{yamada2011evidentiality}//
%\endgl
%\xe
%%
%In \exref{kar1}, the choice between the second person \gl{a}- and \gl{p}-oriented markers \obj{mi-} and \obj{a-} depends on the scenario:
%The transitive verb \obj{kupi} \qu{to bathe} takes \obj{mi-} in \gl{2}>\gl{3} scenarios \exref{kar1.kar-60}, but \obj{a-} in \gl{3}>\gl{2} scenarios \exref{kar1.kar-62}.
%The intransitive verbs in \exref{kar2} show the same person markers, but the choice of marker depends on the verb rather than semantics.
%

\textcite{meira2000split} investigates a corpus of intransitive verbs from \trio, \kalina, \apalai, and \wayana, and categorizes them by applying different criteria commonly encountered in split-\gl{s} systems.
He shows that neither (non\-)activities, % \parencite[210]{meira2000split}, 
(non-)agency, % \citeyearpar[212]{meira2000split}, 
(in-)animacy, % \citeyearpar[213]{meira2000split},
nor Aktionsart % \citeyearpar[215]{meira2000split} 
satisfactorily predict the class membership of intransitive verbs in any of the languages.
Rather, the reason for a verb to take \gl{a}- or \gl{p}-oriented prefixes is (at least diachronically) a morphological one.
\textcite[217--221]{meira2000split} demonstrates that those intransitive verbs which (etymologically) have a derivational detransitivizing prefix are treated as \gl{sa} verbs, while almost all others are \gl{sp} verbs:
\begin{quotebox}{\parencite[201]{meira2000split}}
	Almost all verbs in the \gl{sa} class are detransitivized forms of transitive verbs, either synchronically (with still existing transitive sources) or diachronically (with reconstructible but no longer existing transitive sources)
\end{quotebox}
%\textcite[221--223]{meira2000split} also argues that the detransitivizing prefixes are indeed deriving \gl{sa} verbs, rather than being inflectional in nature:
%\begin{inlinelist}
%	\item there are a few underived \gl{sa} verbs, with no detransitivizing prefix
%	\item \gl{sa} verbs can develop irregular semantics compared to their transitive counterparts
%	\item it is unpredictable whether the \gl{a} or \gl{p} argument of the underlying transitive verb becomes the \gl{s} of the derived \gl{sa} verb
%	\item some originally derived \gl{sa} verbs have lost their transitive counterparts
%	\item \dbqu{basic} concepts are expressed as derivations of more complex concepts, like \qu{to dance (\gl{sa})} from \qu{to dance with (\gl{tr})}
%\end{inlinelist}.
He notes that this leads to an inflectional split not based in meaning, but rather morphology:

\begin{quotebox}{\parencite[226]{meira2000split}}
	Apparently, the morphological behavior of the \gl{sa} verb class is an accidental consequence of the fact that detransitivization, as far back as we can reconstruct, entails all the morphology described […] as typical of \gl{sa} verbs. The alignment of person-marking prefixes appears not to be driven by any semantic forces in the language; it is as though they were being dragged by the evolution of the reflexive marker.
\end{quotebox}

Regarding the form of this marker, \textcite[505--512]{meira2010origin} reconstruct two distinct prefixes for \PC: reciprocal \rc{əte-} and reflexive \rc{e-}, although their reflexes on verbs have been merged into a single morpheme in modern languages.%
\footnote{In the \PC transitive verb template, derivational prefixes were in a paradigmatic relationship with the earlier third person marker \rc{i-}:
\rc{m-i-}V \qu{you V it}, \rc{m-e-}V \qu{you V yourself}, \rc{m-əte-}V \qu{you V each other}.
This analysis is applied to \trio by \textcite[268--269]{triocarlin2004}, who interprets \obj{i-} as marking transitive diathesis.}
Reflexes of \detrz show a range of meanings summarizable as \dbqu{detransitive}, illustrated with \trio \gl{sa} verbs in \exref{triodetrz}.

\ex<triodetrz> \trio \parencites[218--219]{meira2000split}[128, 256]{triomeira1999}\\
\begin{tabular}[t]{@{}llll@{}}
	\\
	\obj{nonta}  & → & \obj{e-nonta}, & \qu{abandon each other}\\
	\qu{abandon} & & \obj{əi-nonta} &  (reciprocal) \\
	\\
	\obj{suka} & → & \obj{e-suka}, & \qu{wash self}\\
	\qu{wash} & & \obj{əi-suka} & (reflexive)\\
	\\
	\obj{pahka} & → & \obj{e-pahka} & \qu{break (\gl{intr})}\\
	\qu{break (\gl{tr})} & & & (anticausative)\\
	\\
	\obj{puunəpɨ} & → & \obj{əh-puunəpɨ}, & \qu{think, meditate}\\
	\qu{think about} & & \obj{əi-puunəpɨ} & (antipassive)\\
\end{tabular}
\xe
%
The morphological variation in \qu{abandon each other} and \qu{wash self} is due to the collapse of the two \PC prefixes:
\obj{e-} comes from the reflexive prefix \rc{e-}, while the form \obj{əi-} originates in reciprocal \rc{əte-}.
However, both can occur with either meaning -- at least for these two verbs.
In the next section, it will be seen that many of the verbs not affected by person marker extensions belong to the small group of \gl{sa} verbs without a reflex of \detrz.
%As mentioned, there are a few \gl{sa} verbs that are not derived from transitive verbs, and do not have a reflex of \detrz, either.
%\textcite[221]{meira2000split} characterizes these as very old and exceptional, and counts between 5 and 7 verbs, depending on the language.
%\textcite{gildea2007greenberg} identify 7 such underived \gl{sa} verbs as reconstructible to \PC, shown in \cref{tab:underived2007}.
%Many of these verbs will be discussed in \cref{sec:verbs}.

%\begin{table}
	\centering
	\caption[Underived \PC \gl{s_a_} verbs]{Underived \PC \gl{s_a_} verbs \parencite[30]{gildea2007greenberg}}
	\label{tab:underived2007}
	\begin{tabular}{@{}lll@{}}
		\mytoprule
		Form & Meaning\\
		\mymidrule
		\rc{tə[mə]} & \qu{to go}\\
		\rc{ətepɨ} & \qu{to come\subs{1}}\\
		\rc{ka[ti]} & \qu{to say}\\
		\rc{əmə[mɨ]} & \qu{to enter}\\
		\rc{eti} & \qu{to dwell, be\subs{2}}\\
		\rc{a[p]} & \qu{to be\subs{1}, say}\\
		\rc{əməkɨ} & \qu{to come\subs{2}}\\
		\mybottomrule
	\end{tabular}
\end{table}
%
%
%\subsection{Summary and outlook}
%\label{sec:outlook}
%\gl{sa} verbs irregularly inflected for first person were identified in 9 Cariban languages.
%These irregular forms are actually conservative, belonging to a few verbs unaffected by person marker extensions spread via lexical diffusion.
%Such conservative forms are only found among the first person forms of  (etymological) \gl{sa} verbs.
