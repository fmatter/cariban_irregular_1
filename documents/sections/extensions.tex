\section{The innovative \gl{1}\gl{s_a_} markers}
\label{sec:extensions}
As stated in \cref{sec:extensions_intro}, the person marker extensions which did not spread through the entire lexicon all have in common that they feature innovative \gl{1}\gl{s_a_} markers.
There are six attested such innovations, three of which can be reconstructed to intermediate proto-languages.
The new \gl{1}\gl{s_a_} prefix is formally identical to the \gl{1+2} marker (\PC \rc{k-}) in three cases, to the \gl{1}\gl{s_p_} marker (\PC \rc{u(j)-}) in two cases, and to the \gl{1}>\gl{3} marker (\PC \rc{t-}) in one case.
\cref{sec:pekodian} investigates the innovation of \rc{k-} in \PPek, reflected in the three daughter languages \arara, \ikpeng, and \bakairi.
\cref{sec:waiwaian} takes a closer look at the extension of \rc{k-} in \PWai, which was briefly shown in \cref{sec:extensions_intro}.
\cref{sec:taranoan} concerns the extension of \rc{t-} in \PTir or \PTar, reflected in modern \trio and \akuriyo.
\crefrange{sec:akuriyo}{sec:yukpa} look at innovative first person markers which are only attested in single modern languages:
\obj{k-} in \akuriyo, and \obj{j-} in \carijo and \yukpa.
 
\subsection{\PPek \rc{k-}}
\label{sec:pekodian}
The Pekodian branch was suggested by \textcite{meira2005southern}, as the result of fieldwork on \bakairi by Meira and the availability of more material on \ikpeng. % \parencites{pacheco1997relativizacao}{ikpengpacheco1997}{campetela1997analise}{pacheco1998nominalizacao}{ikpengpacheco2001}{campetela2002aspectos}{pacheco2002aspectos}{pacheco2003intransitivos}.
It consists of closely related \arara and \ikpeng, with \bakairi as a more distant member.
\textcite{meira2005southern} focused on phonological and lexical properties, so no reconstructive work has been done on \PPek morphosyntax.
All three Pekodian languages have a regular \gl{1}\gl{s_a_} marker \obj{k-}, as evidenced by the paradigms in \exref{pekreg}.
Its presence can thus be reconstructed to \PPek.

\ex<pekreg>
\begin{tabular}[t]{@{}llll@{}}
& \bakairi \qu{to go up}  & \arara \qu{to dance}  & \ikpeng \qu{to run} \\
& \parencite[4]{meira2003bakairi} & \parencite[150]{alves2017arara} &  \parencite[52]{ikpengpacheco2001}\\
\gl{1}\gl{s_a_} & \obj{\emp{k-}əku-} & \obj{\emp{k-}origu-} & \obj{\emp{k-}aranme-} \\
\gl{2}\gl{s_a_} & \obj{m-əku-} & \obj{m-origu-} & \obj{m-aranme-} \\
\gl{1+2}\gl{s_a_} & \obj{kɨd-əku-} & \obj{kud-origu-} & \obj{kw-aranme-} \\
\gl{3}\gl{s_a_} & \obj{n-əku-} & ∅\obj{-origu-} & ∅\obj{-aranme-} \\
\end{tabular}
\xe

For \arara, \textcite{alves2017arara} reports six\footnote{Seven under her analysis, which treats the two meanings of \obj{it͡ʃi} \qu{to be, to lie down} as different verbs.} \gl{s_a_} verbs which have a first person marker \obj{w(ɨ)-} rather than \obj{k-}, listed in \exref{arairr}.
We take these \arara verbs as a starting point for the other Pekodian languages, as \textcite{alves2017arara} gives the most detailed description of person markers for any language of the branch.

\ex<arairr> \arara \parencite[153]{alves2017arara} \\
\begin{tabular}[t]{@{}ll@{}}
\obj{wɨ-genɨ} & \qu{I said}\\
\obj{w-it͡ʃinɨ} & \qu{I was, lied down}\\
\obj{w-ebɨnɨ} & \qu{I came}\\
\obj{w-ibɨnɨ} & \qu{I bathed}\\
\obj{w-iptoŋrɨ} & \qu{I went down}\\
\obj{w-ɨdolɨ} & \qu{I went}\\
\end{tabular}
\xe
%
From a comparative view, this list is not quite complete, as there is also a reflex of the copula \rc{a(p)}, which is analayzed as a postposition by \textcite[199--201]{alves2017arara}; it is however inflected with \setone prefixes and also shows first person \obj{w-} \exref{araap}.

\ex<araap>
\begin{tabular}[t]{@{}ll@{}}
\gl{1} & \obj{w-aptam}\\
\gl{2} & \obj{m-od-aptam}\\
\gl{1+2} & \obj{kud-aptam}\\
\gl{3} & ∅\obj{-aptam}\\
\end{tabular}
\xe

In his brief discussion of \bakairi verbal person marking, \textcite{meira2003bakairi} reports the existence of two subclasses of \gl{s_a_} verbs, one taking first person \obj{w-}, and one \obj{k-}.
The verb used to illustrate the first group is \obj{i} \qu{to bathe} \exref{bak-33}, contrasting with regular \obj{əku} \qu{to go up} in \exref{pekreg} above.
\qu{to bathe} is also found in the \obj{w-}list for \arara.

\ex<bak-33>\bakairi \parencite[][4]{meira2003bakairi}\\
\begingl
\gla w-i-də//
\glb \gl{1}\gl{s_a_}-bathe-\gl{imm}//
\glft \qu{I bathed}//
\endgl
\xe
%
While \textcite[4]{meira2003bakairi} does list \obj{ge} \qu{to say}, \obj{tə} \qu{to go}, and \obj{əe(wɨ)} \qu{to come} as examples of \gl{s_a_} verbs, he does not indicate whether they belong to the class of \gl{s_a_}-1 verbs, with first person \obj{k-}, or the \gl{s_a_}-2 verbs, with \obj{w-}.\footnote{It should be noted at this point that \textcite{meira2003bakairi} indicates that the same verbs which take first person \obj{w-} in \bakairi also take a \gl{1+2} marker \obj{k-}. However, this marker is only illustrated for \qu{to bathe}, both by \textcite{meira2003bakairi} and \textcite{von1892bakairi}. Given the lack of data for other verbs, we will not further discuss this potential additional pattern. If the characterization by \citeauthor{meira2003bakairi} is accurate, then the pattern is fully parallel to the distribution of the first person prefixes.}
Luckily, while \textcite{von1892bakairi} did not accurately record all phonemic distinctions in \bakairi \parencite{meira2005bakairi}, he does provide inflected forms of cognates to the \arara verbs in \exref{arairr}.
We present them according to the phonemic analysis of \bakairi by \textcites{wheatley1969bakairi}{meira2003bakairi}{meira2005bakairi} in \exref{bakverbs}.

\pex[everyglpreamble=]<bakverbs> \bakairi \parencite[][131, 397, 76, 137, 374, 130]{von1892bakairi}
\begin{multicols}{2}
\a<bak-3>
\begingl
\glpreamble \ort{u-ɣépa} //
\gla u-ge-pa//
\glb \gl{1}\gl{s_a_}-say-\gl{neg}//
\glft \qu{I don't say.}//
\endgl
\a<bak-5>
\begingl
\glpreamble \ort{wi-táki} / \ort{wi-tági} //
\gla w-i-taki//
\glb \gl{1}\gl{s_a_}-be-\gl{int}//
\glft \qu{I was.}//
\endgl
\a<bak-4>
\begingl
\glpreamble \ort{kχaewí-le} //
\gla k-əewɨ-lɨ//
\glb \gl{1}\gl{s_a_}-come-\gl{imm}//
\glft \qu{I came.}//
\endgl
\a<bak-6>
\begingl
\glpreamble \ort{kχ-itaké-he} //
\gla k-ɨtəgɨ-se//
\glb \gl{1}\gl{s_a_}-go.down-\gl{npst}?//
\glft \qu{I go down.}//
\endgl
\a<bak-2>
\begingl
\glpreamble \ort{ū́ta} / \ort{uúta}//
\gla u-tə//
\glb \gl{1}\gl{s_a_}-go//
\glft \qu{I go.}//
\endgl
\a<bak-35>
\begingl
\glpreamble \ort{töre-w-akine}//
\gla tərə w-a-kɨne//
\glb there \gl{1}\gl{s_a_}-be-\gl{pst}.\gl{cont}//
\glft \qu{I was there.}//
\endgl
\end{multicols}
\xe

All available descriptions of the third Pekodian language, \ikpeng, list \obj{k-} as the only \gl{1}\gl{s_a_} marker \parencites[55]{ikpengpacheco1997}[105]{campetela1997analise}[64]{ikpengpacheco2001}[205]{alves2013verbo}.
However, most \ikpeng cognates of the \arara verbs with \gl{1}\gl{s_a_} \obj{w-} actually do not take \obj{k-}, but rather \obj{ɨ-} or Ø, as shown in \exref{ikpw}.
The exception is \qu{to go}, which has \obj{k-} \exref{ikp-170}.
There is a formally identical \ikpeng cognate of \arara \obj{iptoŋ} \qu{to go down}, but no first person forms are attested \perscommpar{Angela Chagas}.
While there are reflexes of \rc{a(p)} \qu{to be} in \ikpeng, only \rc{eti} forms are attested inflected for first person \parencite[401]{gildea2018reconstructing}.

\pex<ikpw>\ikpeng
\a<ikp-168>
\begingl
\gla ɨ-ge-lɨ//
\glb \gl{1}-say-\gl{rec}//
\glft \qu{I said.} \parencite[][209]{ikpengpacheco2001}//
\endgl
\a<ikp-169>
\begingl
\gla Ø-et͡ʃi-lɨ//
\glb \gl{1}-be-\gl{rec}//
\glft \qu{I was.} \parencite[][139]{ikpengpacheco2001}//
\endgl
\a<ikp-167>
\begingl
\gla at͡ʃagotpop Ø-ip-t͡ʃi ik-gwa-kt͡ʃi//
\glb always \gl{1}-bathe-\gl{npst} river-\gl{loc}.aquatic-\gl{all}//
\glft \qu{I always bathe in this river.} \parencite[][68]{ikpengpacheco1997}//
\endgl
\xe

\ex<ikp-170>\ikpeng \parencite[][80]{ikpengpacheco2001}\\
\begingl
\gla k-aran-t͡ʃi//
\glb \gl{1}-go-\gl{npst}//
\glft \qu{I'm going.}//
\endgl
\xe

%\ex<ikp-182>\ikpeng \parencite[][320]{chagas2019ikpeng}\\
%\begingl
%\gla ∅-iptoŋ-lan//
%\glb \gl{3}\gl{s_a_}-sit.down-\gl{pst}//
%\glft \qu{She sat down.}//
%\endgl
%\xe
%
While clearly not a regular sound change, \PXin \rc{w} is attested as sometimes being lost in \ikpeng, as evidenced by the correspondences in \exref{ikpwloss}.
Thus, we suggest that the \ikpeng \gl{1}\gl{s_a_} marker \obj{ɨ-}/Ø is cognate with \arara \gl{1}\gl{s_a_} \obj{w(ɨ)-}, and was subject to phonological erosion of \rc{w}, a phenomenon which will also be seen in other languages discussed in following sections.

\ex<ikpwloss> %\arara-\ikpeng \obj{w}:Ø correspondences\\
\begin{tabular}[t]{@{}lll@{}}
\PXin & \arara & \ikpeng\\
\rc{wɨna} \qu{\gl{dat}} & \obj{wɨna} & \obj{ɨna}\\
\rc{wokori} \qu{dog, jaguar} & \obj{wokori} & \obj{akari}\\
\rc{tawe} \qu{capuchin monkey} & \obj{tawe} & \obj{tae}\\
\rc{watke} \qu{to defecate} & \obj{watke} & \obj{atke}\\
\rc{wɨnkɨ} \qu{to sleep} & \obj{wɨnkɨ} & \obj{ɨnkɨ}\\
& (\obj{o-wɨnkɨ} \qu{sleep!}) & (\obj{j-ɨnkɨ-lɨ} \qu{s/he slept})\\
\end{tabular}\\
\parencites[80]{alves2017arara}[164, 67, 150]{ikpengpacheco2001}[9]{desouza2010arara}[68]{souza1993arara}[91]{alves2013verbo}[40]{campetela1997analise} 
\xe

\begin{table}
	\centering
	\caption{Verbs preserving \gl{1}\gl{s_a_} in \PPek}
	\label{tab:pekverbs}
	\begin{tabular}{@{}lllll@{}}
	\mytoprule
& \PPek & \arara & \ikpeng & \bakairi \\
\mytoprule
\qu{say} & \rc{wɨ-ge-} & \obj{wɨ-ge-} & \obj{ɨ-ge-} & \obj{u-ge-} \\
\qu{bathe} & \rc{w-ipɨ-} & \obj{w-ibɨ-} & \obj{Ø-ip-} & \obj{w-i-} \\
\qu{be-1} & \rc{w-ap-} & \obj{w-ap-} & -- & \obj{w-a-} \\
\qu{be-2} & \rc{w-et͡ʃi-} & \obj{w-it͡ʃi-} & \obj{Ø-et͡ʃi-} & \obj{w-i-} \\
\qu{come} & \rc{w-epɨ}, \rc{k-əd-epɨ-} & \obj{w-ebɨ-} & \obj{k-arep-} & \obj{k-əewɨ-} \\
\qu{go down} & \rc{w-[ɨ/i]ptə-} & \obj{w-iptoŋ-} & ?\obj{-iptoŋ-} & \obj{k-ɨtəgɨ-} \\
\qu{go} & \rc{w-ɨtən-} & \obj{w-ɨdo-} & \obj{k-aran-} & \obj{u-tə-} \\
	\mybottomrule
	\end{tabular}
\end{table}


\Cref{tab:pekverbs} gives an overview of the first person forms of the seven verbs under discussion, along with our \PPek reconstruction.
The form for \rc{ge} \qu{to say} is straightforward to reconstruct; the \bakairi change of \rc{wɨ-} to \obj{u-} is not unexpected given correspondences like \bakairi \obj{udo} \parencite{meira2005southern} from \PC \rc{wɨtoto} \qu{person} \parencite[4]{gildea2007greenberg}.
\qu{to bathe} is straightforwardly reconstructible as \rc{ipɨ}.
For \qu{to be}, \ikpeng \obj{e} is very likely the original vowel, given the \PC form \rc{eti} \pcref{sec:be}.

The forms for \qu{to come} are not fully cognate; \ikpeng and \bakairi both show a reflex of the detransitivizer in combination with a root reconstructible as \rc{epɨ}.
In contrast, the \arara first person form is directly based on this root \rc{epɨ}.
However, other person values show a reflex of \detrz in \arara, too \exref{ara-123}.

\ex<ara-123>\arara \parencite[][150]{alves2017arara}\\
\begingl
\gla m-odebɨ-nɨ//
\glb \gl{2}\gl{s_a_}-come-\gl{rec}//
\glft \qu{You came.}//
\endgl
\xe
%
On the other hand, both \ikpeng and \bakairi show reflexes of \rc{əd-ebɨ} throughout the whole paradigm.
Following the line of reasoning used by \textcite[114]{meira1998proto} (see also \cref{sec:taranoan}) for a similar pattern in the three Taranoan languages, we suggest that the idiosyncratic pattern in \arara is reconstructible to \PPek, and that \bakairi and \ikpeng independently regularized the paradigm.

The forms for \qu{to go down} are also not fully cognate; \textcite{meira2005southern} make no mention of a regular correspondence between \bakairi \obj{gɨ} and \ikpeng \obj{ŋ}.
Rather, it seems that some roots gained a final \obj{ŋ} in \PXin, for so far unclear reasons: \begin{inlinelist}
\item \PC \rc{əne} \qu{to see}, \arara and \ikpeng \obj{eneŋ} \parencites[8]{gildea2007greenberg}[56]{alves2017arara}[25]{ikpengpacheco2001}
\item \PC \rc{əta} \qu{to hear}, \arara \obj{taŋ}, \ikpeng \obj{iraŋ} \parencites[8]{gildea2007greenberg}[144]{alves2017arara}[270]{ikpengpacheco2001}
\item \PC \rc{ənə} \qu{to eat meat}, \arara \obj{oŋoŋ} \qu{to bite} \parencites[8]{gildea2007greenberg}[57]{alves2017arara}
\end{inlinelist}.
Also, based on the fact \bakairi has generally lost much segmental material, we suggest that \bakairi \obj{gɨ} is yet another addition to a root \rc{ɨptə} or \rc{iptə}, rather than a conservative form.

The forms for \qu{to go} are all cognate, but while \arara and \bakairi point to \rc{ɨtə}, the \ikpeng form seems rather divergent.
However, it is in fact compatible with a reconstruction \rc{ɨtən}, when one considers that \ikpeng \obj{a} is an attested outcome of \rc{ə} (\PXin \rc{o}/\obj{e}), for example \obj{akari} \qu{dog} in \exref{ikpwloss} above, \obj{anma} \qu{path} \parencite[24]{ikpengpacheco2001} from \PC \rc{ətema} \parencite[12]{gildea2007greenberg}, or \obj{jaj} \qu{tree} from \PC \rc{jəje}.
This attested change of \rc{ə} to \obj{a} need only be preceded by a assimilatory lowering of initial \rc{ɨ} to \rc{ə}, to yield the form \obj{aran} from \rc{ɨtən}.
Other \ikpeng reflexes of \qu{to go} offer evidence for the suggested intermediate stage \rc{ətən}: \obj{ero-lɨ} \qu{s/he went} \parencite[25]{ikpengpacheco2001}.

Summing up, an innovative \gl{1}\gl{s_a_} marker \rc{k-} is reconstructible to \PPek.
Seven verbs can be reconstructed as having resisted this innovation, and preserving \gl{1}\gl{s_a_} \rc{w(ɨ)-} in \PPek.
For \qu{to come}, two distinct forms \rc{w-epɨ-} and \rc{k-əd-epɨ-} can be reconstructed, although the latter may be two parallel developments in \bakairi and \ikpeng.
In later, individual developments, \bakairi introduced \obj{k-} to \qu{to go down}, and \ikpeng to \qu{to go}.

\subsection{\PWai \rc{k-}}
\label{sec:waiwaian}
This extension was briefly discussed in \cref{sec:extensions_intro}; the new \gl{1}\gl{s_p_} prefix \rc{k-}, already innovated at the \PPar stage, was later extended to \gl{1}\gl{s_a_} in \PWai.
This created a unified \gl{1}\gl{s} category, reflected in both \hixka and \waiwai \exref{wais}.

\ex<wais> Waiwaian \rc{eɸurka} \qu{to fall} (\gl{s_a_}) and \rc{wɨnɨkɨ} \qu{to sleep} (\gl{s_p_})\\
\begin{tabular}[t]{@{}lllllll@{}}
& \multicolumn{2}{l}{\PWai} & \multicolumn{2}{l}{\hixka}  & \multicolumn{2}{l}{\waiwai} \\
 & \gl{s_a_} & \gl{s_p_} & \gl{s_a_} & \gl{s_p_} & \multicolumn{2}{l}{\gl{s}} \\
\gl{1} & \rc{\emp{k-}eɸurka-} & \rc{kɨ-wɨnɨkɨ-} & \obj{\emp{k-}ehurka-} & \obj{kɨ-nɨkɨ-} & \obj{\emp{k-}eɸɨrka-} & \obj{kɨ-wɨnk-} \\
\gl{2} & \rc{m-eɸurka-} & \rc{o-wɨnɨkɨ-} & \obj{m-ehurka-} & \obj{o-wnɨkɨ-} & \obj{m-eɸɨrka-} & \obj{mɨ-wɨnk-} \\
\gl{1+2} & \rc{t-eɸurka-} & \rc{tɨt-wɨnɨkɨ-} & \obj{t-ehurka-} & \obj{tɨ-nɨkɨ-} & \obj{t͡ʃ-eɸɨrka-} & \obj{tɨt-wɨnk-} \\
\gl{3} & \rc{ɲ-eɸurka-} & \rc{nɨ-wɨnɨkɨ-} & \obj{ɲ-ehurka-} & \obj{nɨ-nɨkɨ-} & \obj{ɲ-eɸɨrka-} & \obj{nɨ-wɨnk-} \\
\end{tabular}\\
\parencites[510]{howard2001wrought}[189--191]{hixkaryanaderby1985}[209--211]{hawkins1953waiwai}[50]{waiwaihawkins1998}
\xe
%
\waiwai \obj{ka} \qu{to say} does not take \obj{kɨ-}, but rather conservative \obj{wɨ-} \exref{hixɨ.wai-108}.
Its \hixka counterpart also takes a reflex of \gl{1}\gl{s_a_} \rc{w-}, which has the form \obj{ɨ-} \exref{hixɨ.hix-118}.
A formally identical prefix is found on transitive verbs in \gl{1}>\gl{3} scenarios \exref{hixɨ.hix-17}, where it also corresponds to \waiwai \obj{w(ɨ)-} \exref{hixɨ.wai-159}.

\pex<hixɨ>
\a<wai-108> \waiwai \parencite[][71]{waiwaihawkins1998}
\begingl
\glpreamble wɨɨkekɲe//
\gla wɨ-ka-jakɲe//
\glb \gl{1}-say-\gl{pst}//
\glft \qu{I said.}//
\endgl
\a<hix-118> \hixka \parencite[][124]{hixkaryanaderby1985}
\begingl
\glpreamble roxehra nay hamɨ Kaywerye ɨkekonɨ//
\gla ro-ʃe-hɨra n-a-je hamɨ kajwerʲe ɨ-ka-jakonɨ//
\glb \gl{1}-\gl{des}-\gl{neg} \gl{3}-be-\gl{npst}.\gl{uncert} \gl{evid} K. \gl{1}\gl{s_a_}-say-\gl{rem}.\gl{cont}//
\glft \qu{I said (to myself), “Kaywerye evidently doesn't like me”.}//
\endgl
\a<hix-17> \hixka \parencite[][191]{hixkaryanaderby1985}
\begingl
\gla ɨ-koroka-no//
\glb \gl{1}>\gl{3}-wash-\gl{imm}//
\glft \qu{I washed him.}//
\endgl
\a<wai-159> \waiwai \parencite[][192]{waiwaihawkins1998}
\begingl
\glpreamble wîyesî//
\gla wɨ-jo-jasɨ//
\glb \gl{1}>\gl{3}-boil-\gl{npst}//
\glft \qu{I will boil it.}//
\endgl
\xe
%
Thus, similar to the the case of \ikpeng, \hixka has lost the \rc{w} of the \envr{}{C} allomorph \rc{wɨ-}.
Notably, \textcite{hixkaryanaderby1985} does not see this \obj{ɨ-} as an irregular \gl{1}\gl{s_a_}, but as the regular \gl{1}>\gl{3} marker, because he considers \hixka \obj{ka} \qu{to say} to be transitive, an issue which will be discussed in \cref{sec:say}.

There are three more verbs which did not take innovative \rc{k-} in \PWai.
Two roots are part of the copula \rc{eʃi}/\obj{a(h)}, straightforwardly reconstructed in \exref{parpar.parcop}.
The other is \qu{to go} \exref{parpar.pargo}, somewhat of a special case.

\pex<parpar>
\a<parcop> Parukotoan \qu{to be}\\
\begin{tabular}[t]{@{}llll@{}}
& \PPar & \hixka & \waiwai \\
\gl{1}\gl{s_a_} & \rc{\emp{w-}eʃi-}/\obj{\emp{w-}ah-} & \obj{\emp{w-}eʃe-}/\obj{\emp{w-}ah-} & \obj{\emp{w-}eeʃi-}/\obj{\emp{w-}a-}\\
\gl{2}\gl{s_a_} & \rc{m-eʃi-}/\obj{m-ah-} & \obj{m-eʃe-}/\obj{m-ah-} & \obj{m-eeʃi-}/\obj{m-a-}\\
\gl{1+2}\gl{s_a_} & \rc{t-eʃi-}/\obj{t-ah-} & \obj{t-eʃe-}/\obj{t-ah-} & \obj{t͡ʃ-eeʃi-}/\obj{t-a-}\\
\gl{3}\gl{s_a_} & \rc{n-eʃi-}/\obj{n-ah-} & \obj{n-eʃe-}/\obj{n-ah-} & \obj{n-eeʃi-}/\obj{n-a-}\\
\end{tabular}\\
(\cite[197--198]{hixkaryanaderby1985} and Spike Gildea, p.c.)
\a<pargo> Parukotoan \qu{to go}\\
\begin{tabular}[t]{@{}llll@{}}
& \PPar & \hixka & \waiwai\\
\gl{1}\gl{s_a_} & \rc{\emp{wɨ-}to-} & \obj{\emp{ɨ-}to-} & \obj{kɨ\emp{w-}to-}\\
\gl{2}\gl{s_a_} & \rc{mɨ-to-} & \obj{mɨ-to-} & \obj{mɨɨ-to-}\\
\gl{1+2}\gl{s_a_} & \rc{tɨt-to-} & \obj{tɨ-to-} & \obj{tɨ-to-}/\obj{tɨh-t͡ʃe-}\\
\gl{3}\gl{s_a_} & \rc{nɨ-to-} & \obj{n-to-} & \obj{nɨɨ-to-}\\
\end{tabular}\\
(\cites[69--70, 211]{hixkaryanaderby1985}[179]{waiwaihawkins1998} and Spike Gildea, p.c.)
\xe
%
While \hixka has the expected \obj{ɨ-}, \waiwai seems to have combined innovative \gl{1}\gl{s_a_} \obj{k-} with the old \rc{w-}, an analysis also considered by \textcite[90]{gildea1998}.
Alternatively, this form was influenced by deverbalized forms of \qu{to go}, where a reflex of the \gl{s_a_} class marker \rc{w-} has become fossilized \exref{waiwto}.

\pex<waiwto> \waiwai reflexes of the \gl{s_a_} class marker \rc{w-}
\a \obj{o-\emp{w}to-topo-nho} \qu{my trip} \parencite[92]{waiwaihawkins1998}
\a \obj{o-\emp{w}to-t͡ʃhe} \qu{after I went} \parencite[165]{waiwaihawkins1998}
\a \obj{kɨ-\emp{w}to-me} \qu{for us to go} \parencite[204]{waiwaihawkins1998}
\xe
%
In any case, \hixka \qu{to go} was clearly not affected by the extension of \rc{k-}, allowing us to reconstruct a \PWai first person form \rc{wɨ-to-}.

Summing up, we reconstruct the three verbs \rc{et͡ʃi}/\obj{a(h)} \qu{to be}, \rc{ka(s)} \qu{to say}, and \rc{to} \qu{to go} as preserving the old \gl{1}\gl{s_a_} marker \rc{w-} in \PWai, while the rest took on innovative \rc{k-}.

\subsection{\PTir \rc{t-}}
\label{sec:taranoan}
The Taranoan languages were already grouped together by \textcite{girard1971proto}, consisting of closely related \trio and \akuriyo (subsumed here under the moniker Tiriyoan), and the more distant member \carijo.
\textcite{meira1998proto} provides an extensive phonological, morphological, and lexical reconstruction of \PTar.
The \setone paradigms of \trio and \akuriyo contain an interesting puzzle: \PC \gl{1}>\gl{3} \rc{t-} and \gl{1}\gl{s_a_} \rc{w-} seem to have switched places.
%This is illustrated by comparing reconstructible \PTir forms with more conservative \wayana \exref{sat} and \kalina \exref{13w}, which can be taken as representative for \PC.
%
%\ex<sat>
%\begin{tabular}[t]{@{}llllll@{}}
%\gl{1}\gl{s_a_} & \PC & \wayana & \PTir & \trio & \akuriyo\\
%\qu{yawn} & \rc{w-e-mɨta-po-} & \obj{w-emtapɨ-} & \rc{t͡ʃ-entapo-} & \obj{s-entapo-} & \obj{t͡ʃ-entapo-}\\
%\qu{bathe} & \rc{w-e-pɨ-} & \obj{w-epɨ-} & \rc{t͡ʃ-epɨ-} & \obj{s-epɨ-} & \obj{t͡ʃ-epɨ-}\\
%\qu{talk} & \rc{w-ət-uru} & \obj{w-əturu-} & \rc{t-əturu-} & \obj{t-əturu-} & \obj{k-əturu-}\footnotemark\\
%\qu{learn} & \rc{w-ət-eme-pa} & \obj{w-əhepa-} & \rc{t-əempa-}\footnotemark & \obj{t-əenpa-} & \obj{k-əempa-}\\
%\end{tabular}\\
%\parencites[28]{camargo2010wayana}[292, 664]{triomeira1999}[77, 86, 82]{gildea1994akuriyo}[50, 208, 213]{wayanatavares2005}
%\xe
%\addtocounter{footnote}{-1}
%\footnotetext{\PTir \rc{t-} was subsequently replaced by \obj{k-} in \akuriyo \pcref{sec:akuriyo}.}
%\addtocounter{footnote}{1}
%\footnotetext{
%The \trio form is not due to dissimilation from \rc{mp} to \obj{np}, but rather a phonemic transcription rendering the syllable-final archiphoneme \pnm{N} as \obj{n} \parencite[31]{triomeira1999}.
%Further, \PTar intervocalic \rc{t͡ʃ} was lost entirely in \trio (except after \rc{i}) and is reflected as \obj{ʔ} in \akuriyo \parencite[31--32]{meira1998proto}.
%Given the original shape of the detransitivizers \rc{ət(e)-} and \rc{e-}, \trio \obj{ə-enpa} \qu{to learn} contains a reflex of the former with loss of \rc{t}, so \akuriyo \obj{əempa} would be expected to have a glottal stop: ?\obj{əʔempa}.
%Its absence could either be due to \trio influence or not being transcribed by \textcite{gildea1994akuriyo}.
%Since there is no evidence for a glottal stop in \PTir (\rc{əʔempa}), I do not reconstruct it.}
%
%\ex<13w> 
%\begin{tabular}[t]{@{}llllll@{}}
%\gl{1}>\gl{3}& \PC & \kalina & \PTir & \trio & \akuriyo \\
%\qu{see} & \rc{t-ene-} & \obj{s-ene-} & \rc{w-ene-} & \obj{w-ene-} & \obj{(w-)ene-} \\
%\qu{bring} & \rc{t-enepɨ-} & \obj{s-enepɨ-} & \rc{w-enepɨ-} & \obj{w-enepɨ-} & Ø\obj{-enepɨ-} \\
%\end{tabular}\\
%\parencites[425--426]{hoff1968carib}[28, 304]{triomeira1999}[86--87]{gildea1994akuriyo}
%\xe
%%
%Thus, \wayana preserves the \PC \gl{1}\gl{s_a_} marker \rc{w-}, while it was replaced by \rc{t-} (\phonr{}{\rc{t͡ʃ-}}{\rc{e}}) in \PTir \exref{sat}.
%%Both \qu{to yawn} and \qu{to bathe} are transparently derived \gl{s_a_} verbs, from \rc{mɨta-po} \qu{to open the mouth of} and \rc{pɨ} \qu{to wash}, respectively.
%On the other hand, \gl{1}>\gl{3} scenarios show a reflex of \rc{t-} in \kalina, while they took on \rc{w-} in \PTir \exref{13w}.
%
The question of how this happened in detail \parencite[107--112]{meira1998proto} still has no answer, although it seems necessary to assume a scenario whereby both \rc{t-} and \rc{w-} for a time occurred on both transitive and intransitive verbs \parencite[112]{meira1998proto}.\footnote{In fact, the issue of \emph{when} this happened is also open. It could have happened at the \PTar stage, but the subsequent introduction of \obj{j-} in \carijo \pcref{sec:carijo} would have erased any traces of such an innovation. Accordingly, \textcite{meira1998proto} hesitates to assign this extension to a specific proto-language.}
As for verbs unaffected by the spread of \rc{t-}, \textcite{meira1998proto} lists the four verbs in \exref{ptirw} as being reconstructible to \PTar with \rc{w-}; we offer our reconstructed \PTir forms.
In addition, \trio \obj{ei} \qu{to be}, the other copular root, also shows first person \obj{w-} \parencite[294]{triomeira1999}, allowing us to reconstruct a first-person \rc{w-eʔi-} for \PTir.

\pex<ptirw> \PTir \gl{1}\gl{s_a_} \rc{w-} \parencites[112--115]{meira1998proto}[85]{gildea1994akuriyo}\\
\begin{tabular}[t]{@{}llll@{}}
& \PTir & \trio & \akuriyo\\
\qu{go} & \rc{wɨ-tə(mɨ)-} & \obj{wɨ-tə(mɨ)-} & \obj{ə-təmɨ-}/\obj{wɨ-təemɨ-}  \\
\qu{say} & \rc{wɨ-ka-} & \obj{wɨ-ka-} & \obj{wɨ-ka-} \\
\qu{come} & \rc{w-əʔepɨ-} & \obj{w-əepɨ-} & \obj{Ø-eepɨ-} \\
\qu{be} & \rc{w-ae-} & \obj{w-ae-} & Ø\obj{-aʔe-} \\
\end{tabular}
%\a<aku12> \akuriyo \obj{kəʔ-eepɨ} \qu{we came} \parencite[114]{meira1998proto}
%\a<aku13> \akuriyo \obj{(w)i-toka} \qu{I hit him/her} \parencite[86]{gildea1994akuriyo}
%\a<tricome> \trio \obj{n-epɨ} \qu{s/he came} \parencite[114]{meira1998proto}
\xe

For \akuriyo \qu{to go}, \textcite{gildea1994akuriyo} registered a different prefix \obj{wɨ-}, rather than \posscite{meira1998proto} \obj{ə-}.\footnote{The \akuriyo recorded by Gildea potentially has strong \trio and/or \wayana influence \parencite[253]{gildea1998}.}
The two forms can be reconciled by a specific idiosyncratic combination of sound changes:
We suggest that \obj{ə} is the outcome of the lowering of the \rc{ɨ} in the prefix \rc{wɨ-}; the same has happened to the vowel in the \gl{1+2} prefix \rc{kɨt-} \exref{exp.aku-159}.
Also, \obj{w-} appears to have been subject to ongoing erosion in \akuriyo, also evidenced by its absence in \qu{to come} and \qu{to be}, but its presence in \qu{to say} \exref{ptirw}.
This erosion is also found in the etymologically related \gl{1}>\gl{3} prefix \exref{exp.aku-164}, as well as in other Cariban languages, like \ikpeng \pcref{sec:pekodian} or \hixka \pcref{sec:waiwaian}.

\pex<exp>
\a<aku-159> \akuriyo \parencite[][114]{meira1998proto}\\
\begingl
\gla kəʔ-eepɨ//
\glb \gl{1+2}-come//
\glft \qu{We came.}//
\endgl
\a<aku-164> \akuriyo \parencite[][86]{gildea1994akuriyo}\\
\begingl
\gla (w)i-toka//
\glb \gl{1}>\gl{3}-hit//
\glft \qu{I hit him/her.}//
\endgl
\xe
%
For the form of the verb, \textcite[114--115]{meira1998proto} reconstructs \PTar \rc{əepɨ} for the first person, and \rc{eepɨ} for the others, based on the idiosyncratic form in \trio and the vowel length in \akuriyo.
\akuriyo (and \carijo) would then have levelled the original distribution of \rc{əepɨ} and \rc{eepɨ}, similar to what we have suggested for Pekodian in \cref{sec:pekodian}.
We agree with this scenario, with the exception that \trio \obj{əepɨ} is clearly cognate with \rc{ət-epɨ} \pcref{sec:come}, meaning that the \PTir form would have been \rc{əʔepɨ}.

%\ex<tri-130> \trio \parencite[][114]{triomeira1999}\\
%\begingl
%\gla n-epɨ//
%\glb \gl{3}-come//
%\glft \qu{S/he came.}//
%\endgl
%\xe

\textcite{gildea1994akuriyo} also recorded four more \akuriyo verbs seemingly not affected by innovative \rc{t-}.
They are listed in \exref{akumov.aku}.
We have only found a \trio cognate for \obj{erama} \qu{to return}, which behaves like a regular \gl{s_a_} verb in taking \obj{s-} \exref{akumov.tri}.
Thus, it is uncertain whether these verbs can be reconstructed as taking \rc{w-} in \PTir, although its presence in \akuriyo does suggest so.

\pex<akumov>
\a<aku> \akuriyo \gl{1}\gl{s_a_} \rc{w-} \parencite[84--86]{gildea1994akuriyo}\\
\begin{tabular}[t]{@{}ll@{}}
\qu{return} & Ø\obj{-erama-}\\
\qu{get up} & Ø\obj{-eokahtə-}\\
\qu{jump} & \obj{w-ejahka-}\\
\qu{go out} & \obj{w-ekɨrɨka-}\\
\end{tabular}
\a<tri> \trio \obj{s-erama-} \parencite[301]{triomeira1999}
\xe

Finally, there are two irregularly inflected \gl{s_a_} verbs in \trio, \obj{ɨhtə} \qu{to go down} and \obj{weka}/\obj{oeka} \qu{to defecate} \exref{downshit}.
They have \gl{1}\gl{s_a_} markers \obj{p-} and \obj{k-}, which are otherwise entirely unattested in \trio.\footnote{Although both elements also occur in other irregular forms of these verbs \parencite[180, 325, 331]{triomeira1999}.}
At least for \qu{to go down}, the \akuriyo cognate suggests that the irregular first person form \rc{p-ɨhtə-} can be reconstructed to \PTir.
Whatever their origins, they were not affected by the extension of \rc{t-}.

\ex<downshit> Idiosyncratic \gl{1}\gl{s_a_} prefixes \parencites[294]{triomeira1999}[84]{gildea1994akuriyo}\\
\begin{tabular}[t]{@{}lll@{}}
& \trio & \akuriyo\\
\qu{go down} & \obj{p-ɨhtə-} & \obj{p-ɨtə-}\\
\qu{defecate} & \obj{k-oeka-} & ?\\
\end{tabular}
\xe

In summary, in \PTir{} -- at the latest -- the \PC \gl{1}>\gl{3} marker \rc{t-} largely replaced \gl{1}\gl{s_a_} \rc{w-}.
Four verbs are solidly reconstructible as preserving \rc{w-} in \PTir, and \rc{ɨhtə} \qu{to go down} with irregular \rc{p-} was also not affected.
There are four movement verbs in \akuriyo which appear to not have been affected by the extension of \rc{t-}, as well as another irregular verb \obj{weka} \qu{to defecate} in \trio, with first person \obj{k-oeka-}.

\subsection{\akuriyo \obj{k-}}
\label{sec:akuriyo}
After the split-up of \PTir, when \rc{t-} had largely replaced \rc{w-}, \akuriyo innovated yet another \gl{1}\gl{s_a_} marker: \obj{k-}.
It seems to have replaced \rc{t-} only in specific places, with the two markers showing a clear phonologically conditioned distribution in the \akuriyo data available to us \parencite{gildea1994akuriyo}, with all relevant verbs shown in \Cref{tab:aku1sa}.
\textcite[107]{meira1998proto} largely confirms the distribution shown here, but mentions \dbqu{several cases of first person \obj{t-} in \akuriyo{}} (on \obj{ə}-initial verbs), albeit without any examples.
He also suggests that \obj{k-} could be more recent, with which we agree -- since phonologically distributed \rc{t-}/\rc{t͡ʃ-} is reconstructible to \PTir, a scenario whereby \obj{k-} replaces \rc{t-}, but not \rc{t͡ʃ-}, is the most straightforward.
The few \obj{t-} mentioned by \textcite{meira1998proto} were then either reintroduced under \trio influence, or are the last remnants of the replacement of \rc{t-}.

\begin{table}
	\centering
	\caption{\akuriyo \gl{1}\gl{s_a_} markers in Gildea's fieldnotes}
	\label{tab:aku1sa}
	\begin{tabular}{@{}lll@{}}
	\mytoprule
first person \obj{k-} & first person \obj{t͡ʃ-} \\
\mymidrule
\obj{ənɨkɨ} \qu{to sleep} & \obj{eepɨ} \qu{to bathe} \\
\obj{əməmɨ} \qu{to enter} & \obj{ewai} \qu{to sit down} \\
\obj{əturu} \qu{to talk} & \obj{etonema} \qu{to lie down} \\
\obj{əət͡ʃena} \qu{to cry} & \obj{ekɨɨrɨka} \qu{to stay back} \\
\obj{ətajiŋka} \qu{to run} & \obj{entapo} \qu{to yawn} \\
\obj{əiwa} \qu{to tremble} &  \\
\obj{əempa} \qu{to learn} \\
	\mybottomrule
	\end{tabular}
\end{table}

%Let us first consider a scenario where \obj{k-} was the older \gl{1}\gl{s_a_} marker, occurring on both \obj{e}- and \obj{ə}-initial verbs.
%Next, \obj{t-}/\obj{t͡ʃ-} would have come in and replaced \obj{k-} before \obj{e}, but (mostly) not before \obj{ə}, resulting in the observable distribution.
%This scenario is not very satisfactory, since \obj{k-} is replaced in one completely unmotivated phonological environment, but not in the other.
%
%The other scenario is that \rc{t-} was the old first person marker occurring on the vast majority of \gl{s_a_} verbs.
%Subsequently, it developed into \rc{t͡ʃ-} / \_\obj{e}, but remained \rc{t-} / \_\obj{ə}; \obj{t͡ʃ}\goodtilde\obj{ʃ}\goodtilde\obj{s} and \obj{t} are phonemically contrastive in \akuriyo \parencite[16]{meira1998proto}.
%Thus, \obj{k-} can be seen as replacing \obj{t-}, but not \obj{t͡ʃ-}.
%The occurrences of \obj{t-} / \_\obj{ə} mentioned by \citeauthor{meira1998proto} are then remnants of the earlier prevalence of \obj{t-}.
%While this scenario also lacks a clear motivation, it at least explains the largely phonologically complementary distribution of \obj{k-} and \obj{t-} by the fact that the marker which was replaced had a phonemically different form than the one which was preserved.
%The other scenario which sees \rc{k-} as older would require \rc{t-} replacing \obj{k-} only in a specific environment.\footnote{Note that this process would be plausible even if the instances of \obj{t-} / \_\obj{ə} were due to \trio influence, as suggested by \textcite[108]{meira1998proto} -- these would then be a later reintroduction, replacing \obj{k-}, instead of remnants of the introduction of it.}

It has been speculated that the \trio \gl{s_a_} verb with a first person \obj{k-} marker, \obj{weka}/\obj{oeka} \qu{to defecate} might have something to do with \akuriyo \obj{k-}, which would potentially make the innovation of \rc{k-} a \PTir matter \parencite[116]{meira1998proto}.
However, this hypothesis faces a problem in that the first person form for \akuriyo \qu{to defecate} is \obj{j-ereina-} \parencite[88]{gildea1994akuriyo}.
Another hypothesis is that it originated in the corresponding \kaxui form \obj{ku-weka-} (form provided by Spike Gildea, p.c.), although the occurrence of \obj{o} in \trio would still need explanation.
%Rather, it seems likely that this irregular first person form originated in a class switch from \gl{s_p_} to \gl{s_a_} (\Cref{sec:resistant_verbs}), although it is unclear how exactly that switch resulted in the irregular form.

\subsection{\carijo \obj{j-}}
\label{sec:carijo}
As discussed by \textcite[105--107]{meira1998proto}, \carijo has extended the \gl{1}\gl{s_p_} marker \obj{j(i)-} to \gl{s_a_} verbs.
This, in combination with the extension of \gl{2}\gl{s_a_} \obj{m-} and \gl{1+2}\gl{s_a_} \obj{kɨt-}/\obj{kɨs-} to \gl{s_p_} verbs, results in a single unified \gl{s} category \exref{carpar}.

\ex<carpar> \carijo \parencites[106]{meira1998proto}[173]{robayo2000avance}\\
\begin{tabular}[t]{@{}lll@{}}
& \obj{tuda} \qu{to arrive} & \obj{eharaga} \qu{to dance}\\
\gl{1} & \obj{ji-tuda-} & \obj{j-eharaga-}\\
\gl{2} & \obj{mɨ-tuda-} & \obj{m-egaraga-}\\
\gl{1+2} & \obj{kɨsi-tuda-} & \obj{kɨs-eharaga-}\\
\gl{3} & \obj{ni-tuda-} & \obj{n-eharaga-}\\
\end{tabular}
\xe
%
Thus, also detransitivized verbs like \obj{ehɨnəhɨ} \qu{to fight}, which are \gl{s_a_} verbs in other languages, take \obj{j-} \exref{car-30}.
%The same marker is also found in \gl{3}>\gl{1} scenarios \exref{car1p.car-1} and possessed nouns \exref{car1p.car-31}.

\ex<car-30>\carijo \parencite[][79]{koch1908hiana}\\
\begingl
\glpreamble hẹnẹ(x)tónoko-mā́ɽẹ y-e̱-hẹnẹ́(x)yai̯//
\gla hɨnəhtono-ko=marə j-e-hɨnəh-jai//
\glb enemy-\gl{pl}=with \gl{1}-\gl{detrz}-kill-\gl{npst}.\gl{cert}//
\glft \qu{I fight with the enemies.}//
\endgl
\xe
%

%\pex<car1p>\carijo \parencite[][59, 53]{guerrero2016carijo}
%\begin{multicols}{2}
%\a<car-1>
%\begingl
%\gla jɨ-hɨnəh-ɨ//
%\glb \gl{3}>\gl{1}-kill-\gl{pfv}//
%\glft \qu{S/he killed me.}//
%\endgl
%\a<car-31>
%\begingl
%\gla j-owo-rɨ//
%\glb \gl{1}\gl{poss}-uncle-\gl{pert}//
%\glft \qu{my uncle}//
%\endgl
%\end{multicols}
%\xe
%

As noted in \cref{sec:taranoan}, this extension also erased any traces of a putative \PTar \gl{1}\gl{s_a_} marker \rc{t-}.
However, it did not fully eclipse the old \gl{1}\gl{s_a_} marker \rc{w-}, which is attested as being preserved in the verbs \obj{tə} \qu{to go} \exref{car1w.car-24} and \obj{a} \qu{to be} \exref{car1w.car-25}.

\pex<car1w>\carijo \parencite[][5, 42]{guerrero2016karihona}
\a<car-24>
\begingl
\gla wɨ-tə-e=rehe//
\glb \gl{1}-go-\gl{npst}=\gl{frust}//
\glft \qu{I almost go (but I am not going to go).}//
\endgl
\a<car-25>
\begingl
\gla əji-marə-ne w-a-e//
\glb \gl{2}-with-\gl{pl} \gl{1}-be-\gl{npst}//
\glft \qu{I am with you all.}//
\endgl
\xe
%
%Interestingly enough, the verb \qu{to be} only shows \obj{w-} in its \obj{a} root allomorph, but not in the \obj{et͡ʃi} root allomorph, where it has \obj{j-} \exref{car1y.mivida-12}.
%In a somewhat foreshadowing move, I will note that \PTar \rc{(ət)epɨ} \qu{to come} was also affected by the \carijo extension, as seen in \exref{car1y.car-18}.
%As discussed in \cref{sec:taranoan}, this verb preserves \rc{w-} in \akuriyo and \trio \exref{tri-105}.
%
%\pex<car1y>\carijo
%\a<mivida-12>
%\begingl
%\glpreamble iretibə et͡ʃinəme gərə jet͡ʃiɨ//
%\gla ireti-bə et͡ʃi-nə=me gərə j-et͡ʃi-ɨ//
%\glb then-from be-\gl{inf}=\gl{attrz} still \gl{1}-be-\gl{pfv}//
%\glft \qu{Then I was already grown up.} \parencite[][177]{robayo1989rame}//
%\endgl
%\a<car-18>
%\begingl
%\gla əji-wa-e j-eh-ɨ//
%\glb \gl{2}-search-\gl{sup} \gl{1}-come-\gl{pfv}//
%\glft \qu{I came looking for you.} \parencite[][102]{guerrero2019carijo}//
%\endgl
%\xe
%
%\ex<tri-105>\trio \parencite[][337]{triocarlin2004}\\
%\begingl
%\gla ene w-əe//
%\glb see.\gl{sup} \gl{1}\gl{s_a_}-come//
%\glft \qu{I came to see him/her.}//
%\endgl
%\xe

\subsection{\yukpa \obj{j-}}
\label{sec:yukpa}
\yukpa has lost most of its \setone constructions, mostly employing innovative ones \parencite{meira2006syntactic}.
It preserves the \setone prefixes in the immediate past, where the \gl{1}\gl{s_p_} marker appears to have been extended to \gl{s_a_} verbs, as evidenced by the paradigms in \exref{yukpaintr}.
The loss of split-\gl{s} in favor of the \gl{1}\gl{s_p_} marker is accompanied by the \gl{2}\gl{s_a_} marker \obj{m(ɨ)-} displacing \gl{2}\gl{s_p_} \rc{ə(j)-}, and \gl{1+2} being lost as an inflectional value.
%It has lost the split-\gl{s} system in general

\ex<yukpaintr> \yukpa \parencites[72, 76]{largo2011yukpa}[139]{meira2006syntactic}\\
\begin{tabular}[t]{@{}llll@{}}
& \obj{otum} \qu{to wash self} & \obj{nɨ} \qu{to sleep} & \obj{ata} \qu{to fall}\\
\gl{1} & \obj{j-otum-} & \obj{jɨ-nɨ-} & \obj{j-ata-}\\
\gl{2} & \obj{m-otum-} & \obj{mɨ-nɨ-} & \obj{m-ata-}\\
\gl{3} & \obj{n-otum-} & \obj{nɨ-nɨ-} & \obj{n-ata-}\\
\end{tabular}
\xe
%
The normal intransitive first person marker is \obj{j(ɨ)-}, including those with clear reflexes of \detrz, like \obj{otum} \qu{to wash self} in \exref{yukpaintr}.
The analysis that this is a reflex of the \gl{1}\gl{s_p_} marker \rc{u(j)-} is supported by the same form occurring in \gl{3}>\gl{1} scenarios \exref{yukpatr.yuk-5}.

\pex<yukpatr>\yukpa \parencite[][139]{meira2006syntactic}
\a<yuk-5>
\begingl
\gla aw j-esare//
\glb \gl{1}\gl{pro} \gl{3}>\gl{1}-see//
\glft \qu{S/he saw me.}//
\endgl
\a<yuk-1>
\begingl
\gla aw {\normalfont ∅}-esare//
\glb \gl{1}\gl{pro} \gl{1}>\gl{3}-see//
\glft \qu{I saw it.}//
\endgl
\xe
On the other hand, \gl{1}>\gl{3} scenarios are zero-marked \exref{yukpatr.yuk-1}.
Since \PC \gl{1}\gl{s_a_} \rc{w(ɨ)-} was extended to \gl{1}>\gl{3} scenarios in most languages \parencite[81--82]{gildea1998}, and since it is prone to phonological erosion across the family \pcref{sec:pekodian,sec:waiwaian,sec:akuriyo}, we argue that the zero in \gl{1}>\gl{3} scenarios is in fact the \yukpa reflex of \rc{w-}.
The same zero marking is attested for a single intransitive verb: \obj{to} \qu{to go} \exref{yuk-7}.
Based on regular C-initial verbs taking \obj{jɨ-}, like \qu{to sleep} in \exref{yukpaintr}, it is clear that also C-initial \obj{to} \qu{to go} preserved the old \gl{1}\gl{s_a_} marker \rc{w(ɨ)-}, which was lost due to phonological erosion.

\ex<yuk-7>\yukpa \parencite[][139]{meira2006syntactic}\\
\begingl
\gla aw {\normalfont ∅}-to//
\glb \gl{1}\gl{pro} \gl{1}\gl{s_a_}-go//
\glft \qu{I went.}//
\endgl
\xe