\section{Prefixes and verbs: innovation and resistance}
\label{sec:data}
As shown in \cref{sec:extensions_intro}, irregularly inflected first person forms are leftovers of incomplete person marker extensions.
\cref{sec:extensions} presents the six identified incomplete extensions, the prefixes they introduced and the verbs they spared.
Since languages show considerable etymological overlap in their conservative verbs, they are compared and reconstructed in \cref{sec:verbs}.
Further, their reflexes which did get affected by an incomplete extension are identified.
%The main goal of this section is to identify the distributional patterns of innovative and conservative prefixes in the \gl{s_a_} lexicon, in order to search for potential explanations \pcref{sec:motivations}.

\subsection{Incomplete extensions: the innovative \gl{1}\texorpdfstring{\gl{s_a_}}{Sa} markers}
\label{sec:extensions}
As stated in \cref{sec:extensions_intro}, the six person marker extensions which left a group of verbs untouched all introduced innovative first person markers on \gl{s_a_} verbs.
Of these extensions, half can be reconstructed to intermediate proto-languages, and half happened in pre-modern stages of single languages.
The sources of innovative markers vary, but not much: the innovative \gl{1}\gl{s_a_} prefix is formally identical to the \gl{1+2}\gl{p}/\gl{s_p_} marker (\PC \rc{k-}) in three cases, to the \gl{1}\gl{p}/\gl{s_p_} marker (\PC \rc{u(j)-}) in two cases, and to the \gl{1}>\gl{3} marker (\PC \rc{t-}) in one case.
For each extension, regular (innovative) verbs are contrasted with irregular (conservative) ones, and verb forms are reconstructed where necessary.
\cref{sec:pekodian} demonstrates the extension of \rc{k-} in \PPek, reflected in the three daughter languages \arara, \ikpeng, and \bakairi.
\cref{sec:waiwaian} details the extension of \rc{k-} in \PWai, briefly shown in \cref{sec:extensions_intro}.
\cref{sec:taranoan} focuses on innovative \rc{t-} in \PTir, reflected in modern \trio and \akuriyo.
The topic of \crefrange{sec:akuriyo}{sec:yukpa} are innovative \gl{1}\gl{s_a_} markers found in single languages:
\obj{k-} in \akuriyo, and \obj{j-} in \carijo and \yukpa.
 
\subsubsection{\PPek \rc{k-}}
\label{sec:pekodian}
%The Pekodian branch was suggested by \textcite{meira2005southern}, as the result of fieldwork on \bakairi by Meira and the availability of more material on \ikpeng. % \parencites{pacheco1997relativizacao}{ikpengpacheco1997}{campetela1997analise}{pacheco1998nominalizacao}{ikpengpacheco2001}{campetela2002aspectos}{pacheco2002aspectos}{pacheco2003intransitivos}.
The Pekodian branch consists of closely related \arara and \ikpeng, with \bakairi as a more distant member.
The contribution establishing the branch \parencite{meira2005southern} focused on phonology and lexicon, so no reconstructions of \PPek morphosyntax are found in the literature.
However, all three Pekodian languages have a regular \gl{1}\gl{s_a_} marker \obj{k-} \pcref{tab:pekreg}, allowing the reconstruction of a \PPek \gl{1}\gl{s_a_} marker \rc{k-}.

%\ex<pekreg>
%\begin{tabular}[t]{@{}llll@{}}
%& \bakairi \qu{to go up}  & \arara \qu{to dance}  & \ikpeng \qu{to run} \\
%& \parencite[4]{meira2003bakairi} & \parencite[150]{alves2017arara} &  \parencite[52]{ikpengpacheco2001}\\
%\gl{1}\gl{s_a_} & \obj{\emp{k-}əku-} & \obj{\emp{k-}origu-} & \obj{\emp{k-}aranme-} \\
%\gl{2}\gl{s_a_} & \obj{m-əku-} & \obj{m-origu-} & \obj{m-aranme-} \\
%\gl{1+2}\gl{s_a_} & \obj{kɨd-əku-} & \obj{kud-origu-} & \obj{kw-aranme-} \\
%\gl{3}\gl{s_a_} & \obj{n-əku-} & ∅\obj{-origu-} & ∅\obj{-aranme-} \\
%\end{tabular}
%\xe

\begin{table}
\centering
\caption[Regular Pekodian \gl{s_a_} verbs]{Regular Pekodian \gl{s_a_} verbs \parencites[4]{meira2003bakairi}[150]{alves2017arara}[52]{ikpengpacheco2001}}
\label{tab:pekreg}
\begin{tabular}[t]{@{}llll@{}}
\mytoprule
{} & \bakairi \qu{to go up} &         \arara \qu{to dance} &            \ikpeng \qu{to run} \\
\mymidrule
\gl{1}   &           \obj{k-əku-} &               \obj{k-origu-} &                \obj{k-aranme-} \\
\gl{2}   &           \obj{m-əku-} &               \obj{m-origu-} &                \obj{m-aranme-} \\
\gl{1+2} &         \obj{kɨd-əku-} &             \obj{kud-origu-} &               \obj{kw-aranme-} \\
\gl{3}   &           \obj{n-əku-} &  \obj{{\normalfont ∅}-origu} &  \obj{{\normalfont ∅}-aranme-} \\
\mybottomrule
\end{tabular}
\end{table}

The most detailed description of a Pekodian language \parencite{alves2017arara} names six%
\footnote{Seven in her analysis, treating the two meanings of \obj{it͡ʃi} \qu{to be, to lie down} as different verbs.}
\arara \gl{s_a_} verbs forming a subclass defined by a first person marker \obj{w(ɨ)-} rather than \obj{k-}, shown in \exref{arairr}.
There is also a reflex of the copula \rc{a[p]} \cref{sec:be}, serving syntactically as a postposition introducing adverbial clauses meaning \qu{if} or \qu{when} \parencite[199--201]{alves2017arara}.
However, it inflects with verbal \setone prefixes, including first person \obj{w-} \exref{araap}.
%We take these \arara verbs as a starting point for the other Pekodian languages, as \textcite{alves2017arara} gives the most detailed description of person markers for any language of the branch.

\begin{multicols}{2}
\ex<arairr> \arara \parencite[153]{alves2017arara}\\
\begin{tabular}[t]{@{}ll@{}}
\obj{wɨ-genɨ} & \qu{I said}\\
\obj{w-it͡ʃinɨ} & \qu{I was, lied down}\\
\obj{w-ebɨnɨ} & \qu{I came}\\
\obj{w-ibɨnɨ} & \qu{I bathed}\\
\obj{w-iptoŋrɨ} & \qu{I went down}\\
\obj{w-ɨdolɨ} & \qu{I went}\\
\end{tabular}
\xe

\ex<araap> \arara \parencite[200]{alves2017arara}\\
\begin{tabular}[t]{@{}ll@{}}
\gl{1} & \obj{w-aptam} \qu{when/if I was}\\
\gl{2} & \obj{m-od-aptam}\\
\gl{1+2} & \obj{kud-aptam}\\
\gl{3} & ∅\obj{-aptam}\\
\end{tabular}
\xe
\end{multicols}

In his brief but precise discussion of \bakairi verbal person marking, \textcite{meira2003bakairi} reports the existence of two \gl{s_a_} subclasses, defined by \gl{1}\gl{s_a_}\obj{w-} and \obj{k-}, respectively.\footnote{\textcite{meira2003bakairi} indicates that the same verbs which take first person \obj{w-} in \bakairi also take a \gl{1+2} marker \obj{k-}.
	However, this marker is only illustrated for \qu{to bathe}, both by \textcite{meira2003bakairi} and \textcite{von1892bakairi}.
	Given the lack of data for other verbs, this potential additional pattern will not be further discussed.
	If the characterization by \citeauthor{meira2003bakairi} is accurate, then verbs with innovative first person prefixes have conservative \gl{1+2} prefixes, and vice versa.
}
The first group is illustrated with \obj{i} \qu{to bathe} \exref{bak-33}.

\ex<bak-33>\bakairi \parencite[][4]{meira2003bakairi}\\
\begingl
\gla w-i-də//
\glb \gl{1}\gl{s_a_}-bathe-\gl{imm}//
\glft \qu{I bathed}//
\endgl
\xe
%
%Since \qu{to bathe} is also found in the \obj{w-}list for \arara, other \bakairi cognates of these verbs are of interest.
While \textcite[4]{meira2003bakairi} lists some \bakairi cognates of the \arara verbs in \exref{arairr} as \gl{s_a_} verbs, he does not indicate whether they belong to the \gl{s_a_}-1 class with \obj{k-}, or the \gl{s_a_}-2 class with \obj{w-}.
However, inflected forms can be found in \textcite{von1892bakairi}, represented in \exref{bakverbs} based on the analyses of \bakairi phonology and verbal morphology by \textcites{wheatley1969bakairi}{meira2003bakairi}{meira2005bakairi}{franchetto2016classes}.

\pex[everyglpreamble=]<bakverbs> \bakairi \parencite[][131, 397, 76, 137, 374, 130]{von1892bakairi}
\begin{multicols}{2}
\a<bak-3>
\begingl
\glpreamble \ort{u-ɣépa} //
\gla u-ge-pa//
\glb \gl{1}\gl{s_a_}-say-\gl{neg}//
\glft \qu{I don't say.}//
\endgl
\a<bak-5>
\begingl
\glpreamble \ort{wi-táki} / \ort{wi-tági} //
\gla w-i-taki//
\glb \gl{1}\gl{s_a_}-be-\gl{int}//
\glft \qu{I was.}//
\endgl
\a<bak-4>
\begingl
\glpreamble \ort{kχaewí-le} //
\gla k-əewɨ-lɨ//
\glb \gl{1}\gl{s_a_}-come-\gl{imm}//
\glft \qu{I came.}//
\endgl
\a<bak-6>
\begingl
\glpreamble \ort{kχ-itaké-he} //
\gla k-ɨtəgɨ-se//
\glb \gl{1}\gl{s_a_}-go.down-\gl{npst}?//
\glft \qu{I go down.}//
\endgl
\a<bak-2>
\begingl
\glpreamble \ort{ū́ta} / \ort{uúta}//
\gla u-tə//
\glb \gl{1}\gl{s_a_}-go//
\glft \qu{I go.}//
\endgl
\a<bak-35>
\begingl
\glpreamble \ort{töre-w-akine}//
\gla tərə w-a-kɨne//
\glb there \gl{1}\gl{s_a_}-be-\gl{pst}.\gl{cont}//
\glft \qu{I was there.}//
\endgl
\end{multicols}
\xe

All descriptions of \ikpeng list \obj{k-} as the only \gl{1}\gl{s_a_} marker \parencites[55]{ikpengpacheco1997}[105]{campetela1997analise}[64]{ikpengpacheco2001}[205]{alves2013verbo}.
However, most \ikpeng cognates of the verbs in question do not take \obj{k-}, but rather \obj{ɨ-} or Ø \exref{ikpw}, with the exception of \obj{k-}prefixed \qu{to go} \exref{ikp-170}.
There is a formally identical \ikpeng cognate of \arara \obj{iptoŋ} \qu{to go down}, but no first person forms are attested \perscommpar{Angela Chagas}.
While reflexes of \rc{a[p]} \qu{be-1} do exist in \ikpeng, apparently only reflexes of \rc{eti} \qu{be-2} occur with first person prefixes \parencite[401]{gildea2018reconstructing}.

\pex<ikpw>\ikpeng
\a<ikp-168>
\begingl
\gla ɨ-ge-lɨ//
\glb \gl{1}-say-\gl{rec}//
\glft \qu{I said.} \parencite[][209]{ikpengpacheco2001}//
\endgl
\a<ikp-169>
\begingl
\gla Ø-et͡ʃi-lɨ//
\glb \gl{1}-be-\gl{rec}//
\glft \qu{I was.} \parencite[][139]{ikpengpacheco2001}//
\endgl
\a<ikp-167>
\begingl
\gla at͡ʃagotpop Ø-ip-t͡ʃi ik-gwa-kt͡ʃi//
\glb always \gl{1}-bathe-\gl{npst} river-\gl{loc}.aquatic-\gl{all}//
\glft \qu{I always bathe in this river.} \parencite[][68]{ikpengpacheco1997}//
\endgl
\xe

\ex<ikp-170>\ikpeng \parencite[][80]{ikpengpacheco2001}\\
\begingl
\gla k-aran-t͡ʃi//
\glb \gl{1}-go-\gl{npst}//
\glft \qu{I'm going.}//
\endgl
\xe

\begin{table}
\centering
\caption[Verbs preserving \gl{1}\gl{s_a_} \rc{w-} in \PPek]{Verbs preserving \gl{1}\gl{s_a_} \rc{w-} in \PPek \parencites[153, 200]{alves2017arara}[76, 130, 131, 374, 397]{von1892bakairi}[42, 80, 139, 209]{ikpengpacheco2001}[68]{ikpengpacheco1997}[4]{meira2003bakairi}}
\label{tab:ppekverbs}
\begin{tabular}[t]{@{}lllll@{}}
\mytoprule
{} &          \PPek &          \arara &                       \ikpeng &       \bakairi \\
\mymidrule
\qu{be-1}    &     \rc{w-ap-} &     \obj{w-ap-} &                             – &     \obj{w-a-} \\
\qu{be-2}    &  \rc{w-et͡ʃi-} &  \obj{w-it͡ʃi-} &  \obj{{\normalfont ∅}-et͡ʃi-} &     \obj{w-i-} \\
\qu{say}     &    \rc{wɨ-ge-} &    \obj{wɨ-ge-} &                   \obj{ɨ-ge-} &    \obj{u-ge-} \\
\qu{go}      &   \rc{w-ɨtən-} &    \obj{w-ɨdo-} &                 \obj{k-aran-} &    \obj{u-tə-} \\
\qu{come}    &    \rc{w-epɨ-} &    \obj{w-ebɨ-} &                 \obj{k-arep-} &  \obj{k-əewɨ-} \\
\qu{go down} &   \rc{w-ɨptə-} &                 &                               &                \\
\qu{bathe}   &    \rc{w-ipɨ-} &    \obj{w-ibɨ-} &     \obj{{\normalfont ∅}-ip-} &     \obj{w-i-} \\
\mybottomrule
\end{tabular}
\end{table}

Reconstructed \PPek forms of conservatively inflected verbs are given in \Cref{tab:ppekverbs}.
Newly identified \ikpeng \obj{ɨ-}/Ø is demonstrably a reflex of \PXin \rc{w(ɨ)-}, based on other (albeit irregular) cases of loss of \rc{w} \pcref{tab:pxinw}.
%The presence and distribution of the \ikpeng \gl{1}\gl{s_a_} marker \obj{ɨ-}/Ø suggests that it is cognate with \arara \gl{1}\gl{s_a_} \obj{w(ɨ)-}.
%Indeed, \PXin \rc{w} is attested as sometimes being lost in \ikpeng, as evidenced by the correspondences in .
%While it is by no means a regular sound change, it permits identifying the two prefixes as cognate.
Similarly, the change of \rc{wɨ} to \bakairi \obj{u} is found in correspondences like \obj{udo} \parencite{meira2005southern} from \PC \rc{wɨtoto} \qu{person} \parencite[4]{gildea2007greenberg}.
Thus, a \gl{1}\gl{s_a_} prefix \rc{w(ɨ)-} can securely be reconstructed to \PPek, identical to its \arara reflex in form and distribution.
In later, individual developments, \bakairi extended \obj{k-} to \qu{to go down}, and \ikpeng to \qu{to go}.

\begin{table}
\centering
\caption[Loss of \rc{w} in \ikpeng]{Loss of \rc{w} in \ikpeng \parencites[44, 70]{souza1993arara}[118]{alves2013verbo}[143]{alves2017arara}[21, 164]{ikpengpacheco2001}[9]{desouza2010arara}[40]{campetela1997analise}}
\label{tab:pxinw}
\begin{tabular}[t]{@{}lll@{}}
\mytoprule
             Meaning &       \arara &     \ikpeng \\
\midrule
    \qu{to defecate} &  \obj{watke} &  \obj{atke} \\
       \qu{\gl{dat}} &   \obj{wɨna} &   \obj{ɨna} \\
            \qu{dog} & \obj{wokori} & \obj{akari} \\
\qu{capuchin monkey} &   \obj{tawe} &   \obj{tae} \\
       \qu{to sleep} &  \obj{wɨnkɨ} &  \obj{ɨnkɨ} \\
\bottomrule
\end{tabular}
\end{table}

Reconstructions of verb stems are deferred to \cref{sec:verbs}, but a comment on \qu{to come} is in order:
The stems are not fully cognate, as \ikpeng and \bakairi both show a reflex of the \PPek detransitivizer \rc{əd-} in combination with a root reconstructible as \rc{epɨ} \pcref{sec:come}.
In contrast, the \arara first person form has a bare reflex of \rc{epɨ}.
While reflexes of \rc{əd-epɨ} can be found elsewhere in the \arara paradigm \exref{ara-123}, \ikpeng and \bakairi uniformly reflect \rc{əd-ebɨ}.

\ex<ara-123>\arara \parencite[][150]{alves2017arara}\\
\begingl
\gla m-odebɨ-nɨ//
\glb \gl{2}\gl{s_a_}-come-\gl{rec}//
\glft \qu{You came.}//
\endgl
\xe
%
Following \posscite[114]{meira1998proto} line of reasoning for a similar pattern in Taranoan (see also \cref{sec:taranoan}), the idiosyncratic pattern in \arara can be reconstructed to \PPek, with \bakairi and \ikpeng independently leveling the paradigm in favor of \rc{əd-epɨ}.

\subsubsection{\PWai \rc{k-}}
\label{sec:waiwaian}
This extension, one of the Parukotoan innovations shown in \cref{sec:extensions_intro}, resulted in the \hixka patterns from \cref{sec:intro}.
\PWai further extended the \gl{1}\gl{s_p_} prefix \rc{k-} (innovated in \PPar) to \gl{1}\gl{s_a_}.
For regularly inflected verbs, this created a unified \gl{1}\gl{s} category \pcref{tab:pwaireg}.
\begin{table}
\centering
\caption[Regular \PWai verbs]{Regular \qu{to fall} (\gl{s_a_}) and \qu{to sleep} (\gl{s_p_}) in \PWai \parencites[150]{howard2001wrought}[30]{waiwaihawkins1998}[189, 190, 196]{hixkaryanaderby1985}[209, 211]{hawkins1953waiwai}}
\label{tab:pwaireg}
\begin{tabular}[t]{@{}lllllll@{}}
\mytoprule
{} & \multicolumn{2}{l}{\PWai} & \multicolumn{2}{l}{\hixka} & \multicolumn{2}{l}{\waiwai} \\
{} &    \qu{to fall} &     \qu{to sleep} &     \qu{to fall} &   \qu{to sleep} &       \qu{to fall} &      \qu{to sleep} \\
\midrule
\gl{1}   &  \rc{k-eɸurka-} &   \rc{kɨ-wɨnɨkɨ-} &  \obj{k-ehurka-} &  \obj{kɨ-nɨkɨ-} &    \obj{k-eɸɨrka-} &   \obj{kɨ-wɨnɨkɨ-} \\
\gl{2}   &  \rc{m-eɸurka-} &    \rc{o-wɨnɨkɨ-} &  \obj{m-ehurka-} &  \obj{o-wnɨkɨ-} &    \obj{m-eɸɨrka-} &   \obj{mɨ-wɨnɨkɨ-} \\
\gl{1+2} &  \rc{t-eɸurka-} &  \rc{tɨt-wɨnɨkɨ-} &  \obj{t-ehurka-} &  \obj{tɨ-nɨkɨ-} &  \obj{t͡ʃ-eɸɨrka-} &  \obj{tɨt-wɨnɨkɨ-} \\
\gl{3}   &  \rc{ɲ-eɸurka-} &   \rc{nɨ-wɨnɨkɨ-} &  \obj{ɲ-ehurka-} &  \obj{nɨ-nɨkɨ-} &    \obj{ɲ-eɸɨrka-} &   \obj{nɨ-wɨnɨkɨ-} \\
\bottomrule
\end{tabular}
\end{table}

Not all \gl{s_a_} verbs were affected: \waiwai \obj{ka} \qu{to say} does not take \obj{kɨ-}, but rather conservative \obj{wɨ-} \exref{hixɨ.wai-108}.
Its \hixka counterpart has a prefix \obj{ɨ-} \exref{hixɨ.hix-118}, which also occurs in \gl{1}>\gl{3} scenarios in \hixka \exref{hixɨ.hix-17}, corresponding to \waiwai \obj{w(ɨ)-} \exref{hixɨ.wai-159}.

\pex<hixɨ>
\a<wai-108> \waiwai \parencite[][71]{waiwaihawkins1998}\\
\begingl
\glpreamble wɨɨkekɲe//
\gla wɨ-ka-jakɲe//
\glb \gl{1}-say-\gl{pst}//
\glft \qu{I said.}//
\endgl
\a<hix-118> \hixka \parencite[][124]{hixkaryanaderby1985}\\
\begingl
\glpreamble roxehra nay hamɨ Kaywerye ɨkekonɨ//
\gla ro-ʃe-hɨra n-a-je hamɨ kajwerʲe ɨ-ka-jakonɨ//
\glb \gl{1}-\gl{des}-\gl{neg} \gl{3}-be-\gl{npst}.\gl{uncert} \gl{evid} K. \gl{1}\gl{s_a_}-say-\gl{rem}.\gl{cont}//
\glft \qu{I said (to myself), “Kaywerye evidently doesn't like me”.}//
\endgl
\a<hix-17> \hixka \parencite[][191]{hixkaryanaderby1985}\\
\begingl
\gla ɨ-koroka-no//
\glb \gl{1}>\gl{3}-wash-\gl{imm}//
\glft \qu{I washed him.}//
\endgl
\a<wai-159> \waiwai \parencite[][192]{waiwaihawkins1998}\\
\begingl
\glpreamble wîyesî//
\gla wɨ-jo-jasɨ//
\glb \gl{1}>\gl{3}-boil-\gl{npst}//
\glft \qu{I will boil it.}//
\endgl
\xe
%
The regular correspondence in transitive verbs points to \hixka \obj{ɨ-} on intransitive verbs as another reflex of \rc{wɨ-}, with a similar phonological reduction as in \ikpeng \pcref{sec:pekodian}.
Notably, \textcite{hixkaryanaderby1985} analyzes this \obj{ɨ-} as the regular \gl{1}>\gl{3} prefix, because he considers \hixka \obj{ka} \qu{to say} to be transitive \pcref{sec:say}.

There are three more verbs which did not take innovative \rc{k-} in \PWai \pcref{tab:pwaiverbs}.
The two forms for \qu{to be} are unproblematic, whereas \qu{to go} is a special case.
While \hixka has the expected \obj{ɨ-}, \waiwai seems to have combined innovative \obj{k-} with  old \rc{w-}, an analysis also considered by \textcite[90]{gildea1998}.
Alternatively, this form may have been influenced by deverbalized forms of \qu{to go}, which show a fossilized of the \gl{s_a_} class marker \rc{w-}  \parentext{e.g., \obj{o-\emp{w}to-topo-nho} \qu{my trip} \parencite[92]{waiwaihawkins1998}}.
%
%\pex<waiwto> \waiwai reflexes of the \gl{s_a_} class marker \rc{w-}
%\a \obj{o-\emp{w}to-topo-nho} \qu{my trip} \parencite[92]{waiwaihawkins1998}
%\a \obj{o-\emp{w}to-t͡ʃhe} \qu{after I went} \parencite[165]{waiwaihawkins1998}
%\a \obj{kɨ-\emp{w}to-me} \qu{for us to go} \parencite[204]{waiwaihawkins1998}
%\xe
%
Either way, the first person form \hixka \qu{to go} clearly points to \PWai \rc{wɨ-tom-}.

\begin{table}
\centering
\caption[Verbs preserving \gl{1}\gl{s_a_} \rc{w-} in \PWai]{Verbs preserving \gl{1}\gl{s_a_} \rc{w-} in \PWai \parencites[4]{hixkaryanaderby1979}[71, 85]{waiwaihawkins1998}[70, 197]{hixkaryanaderby1985}[Spike Gildea]{pc}}
\label{tab:pwaiverbs}
\begin{tabular}[t]{@{}llll@{}}
\mytoprule
{} &         \PWai &        \hixka &         \waiwai \\
\mymidrule
\qu{say}  &   \rc{wɨ-ka-} &   \obj{ɨ-ka-} &    \obj{wɨ-ka-} \\
\qu{be-1} &    \rc{w-ah-} &   \obj{w-ah-} &      \obj{w-a-} \\
\qu{be-2} &   \rc{w-eʃi-} &  \obj{w-eʃe-} &   \obj{w-eeʃi-} \\
\qu{go}   &  \rc{wɨ-tom-} &   \obj{ɨ-to-} &  \obj{kɨw-tom-} \\
\mybottomrule
\end{tabular}
\end{table}

%Summing up, we reconstruct the four verbs \rc{eʃi} and \rc{a[h]} \qu{to be}, \rc{ka[s]} \qu{to say}, and \rc{[ɨ]to[m]} \qu{to go} as preserving the old \gl{1}\gl{s_a_} marker \rc{w-} in \PWai, while the rest took on innovative \rc{k-}.

\subsubsection{\PTir \rc{t-}}
\label{sec:taranoan}
The moniker Tiriyoan \parencite{glottolog} subsumes \trio and \akuriyo, the more closely related of the three Taranoan languages identified by \textcite{girard1971proto}, with \carijo as a sister.
\textcite{meira1998proto} provides an extensive phonological, morphological, and lexical reconstruction of \PTar, facing an interesting puzzle in the \setone paradigms of \trio and \akuriyo: \PC \gl{1}>\gl{3} \rc{t-} and \gl{1}\gl{s_a_} \rc{w-} seem to have switched places, creating a regular \gl{1}\gl{s_a_} marker of the form \envr{\rc{t͡ʃ-}}{\obj{e}}, \envr{\rc{t-}}{\obj{ə}} \pcref{tab:ptirreg}.%
\footnote{The latter allomorph was subsequently replaced by \obj{k-} in \akuriyo \pcref{sec:akuriyo}.}
\begin{table}[htbp]
\centering
\caption[Regular \PTir \gl{s_a_} verbs]{Regular \PTir \gl{s_a_} verbs \parencites[292, 294]{triomeira1999}[87]{gildea1994akuriyo}}
\label{tab:ptirreg}
\begin{tabular}[t]{@{}lllllll@{}}
\mytoprule
{} & \qu{to bathe (\gl{intr})} &                &                 &     \qu{to sleep} &                    &                    \\
{} &                     \PTir &          \trio &        \akuriyo &             \PTir &              \trio &           \akuriyo \\
\mymidrule
\gl{1}   &             \rc{t͡ʃ-epɨ-} &   \obj{s-epɨ-} &  \obj{t͡ʃ-epɨ-} &    \rc{t-əənɨkɨ-} &    \obj{t-əənɨkɨ-} &    \obj{k-əənɨkɨ-} \\
\gl{2}   &               \rc{m-epɨ-} &   \obj{m-epɨ-} &    \obj{m-epɨ-} &    \rc{m-əənɨkɨ-} &    \obj{m-əənɨkɨ-} &    \obj{m-əənɨkɨ-} \\
\gl{1+2} &              \rc{ke-epɨ-} &  \obj{ke-epɨ-} &   \obj{ke-epɨ-} &  \rc{kɨt-əənɨkɨ-} &  \obj{kɨt-əənɨkɨ-} &  \obj{kəʔ-əənɨkɨ-} \\
\gl{3}   &               \rc{n-epɨ-} &   \obj{n-epɨ-} &    \obj{n-epɨ-} &    \rc{n-əənɨkɨ-} &    \obj{n-əənɨkɨ-} &    \obj{n-əənɨkɨ-} \\
\mybottomrule
\end{tabular}
\end{table}
The question of how this switch happened in detail \parencite[107--112]{meira1998proto} is still open, although a scenario seems necessary in which both \rc{t-} and \rc{w-} for a time occurred on both transitive and intransitive verbs \parencite[112]{meira1998proto}.\footnote{
In fact, even the issue of \emph{when} this happened is open.
It could have happened at the \PTar stage, but the subsequent introduction of \obj{j-} in \carijo \pcref{sec:carijo} would have erased any traces of such an innovation.
Accordingly, \textcite{meira1998proto} hesitates to assign this extension to a specific proto-language.
Here, a conservative stance is taken and the innovation is arbitrarily assumed to be \PTir.
This decision does not affect the results of this study.}

\begin{table}
\centering
\caption[Verbs preserving \gl{1}\gl{s_a_} \rc{w-} in \PTir]{Verbs preserving \gl{1}\gl{s_a_} \rc{w-} in \PTir \parencites[292, 294, 339]{triomeira1999}[112, 113, 114, 115, 165]{meira1998proto}}
\label{tab:ptirverbs}
\begin{tabular}[t]{@{}llll@{}}
\toprule
{} &          \PTir &          \trio &                     \akuriyo \\
\midrule
\qu{go}   &  \rc{wɨ-təmɨ-} &  \obj{wɨ-tən-} &                \obj{ə-təmɨ-} \\
\qu{say}  &    \rc{wɨ-ka-} &   \obj{wɨ-ka-} &                 \obj{wɨ-ka-} \\
\qu{come} &  \rc{w-əʔepɨ-} &  \obj{w-əepɨ-} &  \obj{{\normalfont ∅}-eepɨ-} \\
\qu{be-1} &      \rc{w-a-} &     \obj{w-a-} &     \obj{{\normalfont ∅}-a-} \\
\qu{be-2} &    \rc{w-eʔi-} &    \obj{w-ei-} &   \obj{{\normalfont ?}-eʔi-} \\
\bottomrule
\end{tabular}
\end{table}

As for verbs unaffected by the spread of \rc{t-}, \textcite{meira1998proto} reconstructs four of the items in \cref{tab:ptirverbs} as taking \rc{w-} in \PTar, for which reconstructed \PTir forms are substituted here.\footnote{The present reconstruction of \qu{to come} diverges from \posscite[114--115]{meira1998proto}, who reconstructs \PTar \rc{əepɨ} for first, but \rc{eepɨ} for other persons, based on the paradigmatic pattern in \trio and the vowel length in \akuriyo.
\akuriyo and \carijo would then have levelled that pattern, similar to what was suggested for the Pekodian languages \pcref{sec:pekodian}.
Here, the length in \akuriyo \obj{eepɨ} is identified as resulting from coalescence of \rc{əe}, and \trio \obj{əepɨ} as reflecting \rc{ətepɨ} \pcref{sec:come}.
This yields \PTir \rc{(əʔ)epɨ} and \PTar \rc{(ət͡ʃ-)epɨ}.}
As a fifth verb \rc{eʔi} \qu{be-1} can be added, whose \trio reflex retains \obj{w-}.
The idiosyncratic \akuriyo first person prefix \obj{ə-} on \qu{to go} is identified as a reflex of \rc{wɨ-} by 
\textcite[113]{meira1998proto}, which is supported the fact that both components of the idiosyncratic change \rc{wɨ-} > \obj{ə-} (\phon{\rc{w}}{∅} and \phon{\rc{ɨ}}{\obj{ə}}) are found in other person prefixes \exref[exp.aku-164]{exp.aku-159}.

\pex<exp>\akuriyo
\begin{multicols}{2}
\a<aku-164>
\begingl
\gla wi-toka//
\glb \gl{1}>\gl{3}-hit//
\glft \qu{I hit him/her.} \parencite[][86]{gildea1994akuriyo}//
\endgl
\a<aku-159>
\begingl
\gla kəʔ-eepɨ//
\glb \gl{1+2}-come//
\glft \qu{We came.} \parencite[][114]{meira1998proto}//
\endgl
\end{multicols}
\xe
%
%For \akuriyo \qu{to go}, \textcite{gildea1994akuriyo} registered a different prefix \obj{wɨ-}, rather than \posscite{meira1998proto} \obj{ə-}.\footnote{The \akuriyo recorded by Gildea potentially has strong \trio and/or \wayana influence \parencite[253]{gildea1998}.}
%The two forms can be reconciled by a specific idiosyncratic combination of sound changes:
%We suggest that \obj{ə} is the outcome of the lowering of the \rc{ɨ} in the prefix \rc{wɨ-}; the same has happened to the vowel in the \gl{1+2} prefix \rc{kɨt-} \exref{exp.aku-159}.
%Also, \obj{w-} appears to have been subject to ongoing erosion in \akuriyo, also evidenced by its absence in \qu{to come} and \qu{to be}, but its presence in \qu{to say} \exref{ptirw}.
%This erosion is also found in the etymologically related \gl{1}>\gl{3} prefix \exref{exp.aku-164}, as well as in other Cariban languages, like \ikpeng \pcref{sec:pekodian} or \hixka \pcref{sec:waiwaian}.
%
%
%

%For \qu{to come}, \textcite[114--115]{meira1998proto} reconstructs \PTar \rc{əepɨ} for first person, and \rc{eepɨ} for other persons, based on such a paradigmatic pattern in \trio and the vowel length in \akuriyo.
%Both \akuriyo and \carijo then levelled this original distribution, similar to what was suggested for the Pekodian languages \pcref{sec:pekodian}.
%This scenario is plausible, with the exception that \trio \obj{əepɨ} is a reflex of \rc{ət-jəpɨ} \pcref{sec:come}, meaning that the \PTir form would have been \rc{əʔepɨ} (\PTar \rc{ət͡ʃepɨ}).

%Apart from these \PTir verbs with \rc{w(ɨ)-}, there are two irregularly inflected \gl{s_a_} verbs in \trio, \obj{ɨhtə} \qu{to go down} and \obj{weka}/\obj{oeka} \qu{to defecate} \exref{downshit}.
%They have \gl{1}\gl{s_a_} markers \obj{p-} and \obj{k-}, which are otherwise entirely unattested in \trio.\footnote{Although both elements also occur in other irregular forms of these verbs \parencite[180, 325, 331]{triomeira1999}.}
%At least for \qu{to go down}, the \akuriyo cognate suggests that the irregular first person form \rc{p-ɨhtə-} can be reconstructed to \PTir.
%Whatever their origins, they were not affected by the extension of \rc{t-}.
%
%\ex<downshit> Idiosyncratic \gl{1}\gl{s_a_} prefixes \parencites[294]{triomeira1999}[84]{gildea1994akuriyo}\\
%\begin{tabular}[t]{@{}lll@{}}
%& \trio & \akuriyo\\
%\qu{go down} & \obj{p-ɨhtə-} & \obj{p-ɨtə-}\\
%\qu{defecate} & \obj{k-oeka-} & ?\\
%\end{tabular}
%\xe

%In addition, \textcite{gildea1994akuriyo} recorded four more \akuriyo verbs seemingly not affected by innovative \rc{t-} \exref{akumov.aku}, all \obj{e}-initial movement verbs.
%Only \obj{erama} \qu{to return} has an attested \trio cognate, which behaves like a regular \gl{s_a_} verb in taking \obj{s-} \exref{akumov.tri}.
%Further, these verbs are not mentioned by \textcite{meira1998proto}, who was also working with \posscite{gildea1994akuriyo} data.
%Given that this data potentially has strong \trio and/or \wayana influence \parencite[253]{gildea1998} and the lack of support by the available part of \posscite{meira1998proto} data, these verbs cannot be confidently reconstructed as resisting the extension of \PTir \rc{t-}.
%
%\pex<akumov>
%\a<aku> \akuriyo \gl{1}\gl{s_a_} \rc{w-} \parencite[84--86]{gildea1994akuriyo}\\
%\begin{tabular}[t]{@{}ll@{}}
%\qu{return} & Ø\obj{-erama-}\\
%\qu{get up} & Ø\obj{-eokahtə-}\\
%\qu{jump} & \obj{w-ejahka-}\\
%\qu{go out} & \obj{w-ekɨrɨka-}\\
%\end{tabular}
%\a<tri> \trio \obj{s-erama-} \parencite[301]{triomeira1999}
%\xe

%In summary, in \PTir{} -- at the latest -- the \PC \gl{1}>\gl{3} marker \rc{t-} largely replaced \gl{1}\gl{s_a_} \rc{w-}.
%Five verbs are solidly reconstructible as preserving \rc{w-} in \PTir, and \rc{ɨhtə} \qu{to go down} with irregular \rc{p-} was also not affected.
%There is another irregular verb \obj{weka} \qu{to defecate} in \trio, with first person \obj{k-oeka-}.

\subsubsection{\akuriyo \obj{k-}}
\label{sec:akuriyo}
After the split of \PTir, when \rc{t-} and \rc{t͡ʃ-} had largely replaced \rc{w-}, \akuriyo innovated a third \gl{1}\gl{s_a_} marker \obj{k-}.
It seems to have replaced \rc{t-} only in specific environments, with \obj{k-} and \obj{t͡ʃ-} showing a clear phonologically conditioned distribution in \posscite{gildea1994akuriyo} \akuriyo data \pcref{tab:aku1sa}.
\textcite[107]{meira1998proto} largely confirms that distribution, but mentions \dbqu{several cases of first person \obj{t-} in \akuriyo{}} (on \obj{ə}-initial verbs), albeit without any examples.
He also suggests that \obj{k-} may be more recent, which is plausible: since the distribution \envr{\rc{t-}}{\obj{ə}} / \envr{\rc{t͡ʃ-}}{\obj{e}} is reconstructible to \PTir, the most straightforward scenario is \obj{k-} replacing \rc{t-} but not \rc{t͡ʃ-} in \akuriyo.
The few \obj{t-} mentioned by \textcite{meira1998proto} were then perhaps reintroduced under \trio influence.
However, since there are no examples of -- or more information about -- \obj{ə}-initial verbs with \obj{t-}, these cases cannot be discussed further.

%\begin{table}
%	\centering
%	\caption{\akuriyo \gl{1}\gl{s_a_} markers in Gildea's fieldnotes}
%	\label{tab:aku1sa}
%	\begin{tabular}{@{}lll@{}}
%	\mytoprule
%first person \obj{k-} & first person \obj{t͡ʃ-} \\
%\mymidrule
%\obj{ənɨkɨ} \qu{to sleep} & \obj{eepɨ} \qu{to bathe} \\
%\obj{əməmɨ} \qu{to enter} & \obj{ewai} \qu{to sit down} \\
%\obj{əturu} \qu{to talk} & \obj{etonema} \qu{to lie down} \\
%\obj{əət͡ʃena} \qu{to cry} & \obj{ekɨɨrɨka} \qu{to stay back} \\
%\obj{ətajiŋka} \qu{to run} & \obj{entapo} \qu{to yawn} \\
%\obj{əiwa} \qu{to tremble} &  \\
%\obj{əempa} \qu{to learn} \\
%	\mybottomrule
%	\end{tabular}
%\end{table}

\begin{table}
\centering
\caption[Regular \akuriyo \gl{1}\gl{s_a_} markers]{Regular \akuriyo \gl{1}\gl{s_a_} markers \parencites[77, 79, 82, 84-87]{gildea1994akuriyo}}
\label{tab:aku1sa}
\begin{tabular}[t]{@{}ll@{}}
\mytoprule
      first person \obj{k-} &               first person \obj{t͡ʃ-} \\
\mymidrule
 \obj{əempa-} \qu{to learn} &  \obj{epɨ-} \qu{to bathe (\gl{intr})} \\
\obj{əət͡ʃena-} \qu{to cry} &      \obj{ekɨrɨka-} \qu{to stay back} \\
\obj{əiwa-} \qu{to tremble} &            \obj{entapo-} \qu{to yawn} \\
 \obj{əməmɨ-} \qu{to enter} &       \obj{etonema-} \qu{to lie down} \\
\obj{ətajiŋka-} \qu{to run} &          \obj{ewai-} \qu{to sit down} \\
  \obj{əturu-} \qu{to talk} & \obj{ehpa-} \qu{to bathe (\gl{intr})} \\
\obj{əənɨkɨ-} \qu{to sleep} &                                       \\
\mybottomrule
\end{tabular}
\end{table}

The verbs listed for \PTir in \cref{tab:ptirverbs} in \cref{sec:taranoan} of course also resisted the extension of \obj{k-} in \akuriyo, although the first-person form of the copular verb \obj{eʔi} is unknown.
In addition, there is an \gl{s_a_} verb \obj{ɨ(h)tə} \qu{to go down}, which has an irregular first person marker \obj{p-}, apparently reconstructible to \PTir \exref{tridown}.
It was not affected by the extension of \akuriyo \obj{k-}, but whether it was an \gl{s_a_} verb when \PTir \rc{t-} was introduced is unclear (see \cref{sec:godown}).

\ex<tridown> First person forms of \qu{to go down}\\
\begin{tabular}[t]{@{}lll@{}}
\trio & \obj{p-ɨhtə-} & \parencites[294]{triomeira1999} \\
\akuriyo & \obj{p-ɨtə-} & \parencite[84]{gildea1994akuriyo}\\
\end{tabular}
\xe


%Let us first consider a scenario where \obj{k-} was the older \gl{1}\gl{s_a_} marker, occurring on both \obj{e}- and \obj{ə}-initial verbs.
%Next, \obj{t-}/\obj{t͡ʃ-} would have come in and replaced \obj{k-} before \obj{e}, but (mostly) not before \obj{ə}, resulting in the observable distribution.
%This scenario is not very satisfactory, since \obj{k-} is replaced in one completely unmotivated phonological environment, but not in the other.
%
%The other scenario is that \rc{t-} was the old first person marker occurring on the vast majority of \gl{s_a_} verbs.
%Subsequently, it developed into \rc{t͡ʃ-} / \_\obj{e}, but remained \rc{t-} / \_\obj{ə}; \obj{t͡ʃ}\goodtilde\obj{ʃ}\goodtilde\obj{s} and \obj{t} are phonemically contrastive in \akuriyo \parencite[16]{meira1998proto}.
%Thus, \obj{k-} can be seen as replacing \obj{t-}, but not \obj{t͡ʃ-}.
%The occurrences of \obj{t-} / \_\obj{ə} mentioned by \citeauthor{meira1998proto} are then remnants of the earlier prevalence of \obj{t-}.
%While this scenario also lacks a clear motivation, it at least explains the largely phonologically complementary distribution of \obj{k-} and \obj{t-} by the fact that the marker which was replaced had a phonemically different form than the one which was preserved.
%The other scenario which sees \rc{k-} as older would require \rc{t-} replacing \obj{k-} only in a specific environment.\footnote{Note that this process would be plausible even if the instances of \obj{t-} / \_\obj{ə} were due to \trio influence, as suggested by \textcite[108]{meira1998proto} -- these would then be a later reintroduction, replacing \obj{k-}, instead of remnants of the introduction of it.}

%It has been speculated that the \trio \gl{s_a_} verb with a first person \obj{k-} marker, \obj{weka}/\obj{oeka} \qu{to defecate} might have something to do with \akuriyo \obj{k-}, which would potentially make the innovation of \rc{k-} a \PTir matter \parencite[116]{meira1998proto}.
%However, this hypothesis faces a problem in that the first person form for \akuriyo \qu{to defecate} is \obj{j-ereina-} \parencite[88]{gildea1994akuriyo}.
%Another hypothesis is that it originated in the corresponding \kaxui form \obj{ku-weka-} (form provided by Spike Gildea, p.c.), although the occurrence of \obj{o} in \trio would still need explanation.
%Rather, it seems likely that this irregular first person form originated in a class switch from \gl{s_p_} to \gl{s_a_} (\Cref{sec:resistant_verbs}), although it is unclear how exactly that switch resulted in the irregular form.

\subsubsection{\carijo \obj{j-}}
\label{sec:carijo}
\carijo, the third Taranoan language, extended the \gl{1}\gl{s_p_} marker \obj{j(i)-}\footnote{Since all affected \gl{s_a_} verbs are V-initial, only the \envr{}{V} allomorph \obj{j-} occurs in that context.} to \gl{s_a_} verbs \parencite[105--107]{meira1998proto}.
Combined with the extension of \gl{2}\gl{s_a_} \obj{m-} and \gl{1+2}\gl{s_a_} \obj{kɨt-}/\obj{kɨs-} to \gl{s_p_} verbs, this created a single unified \gl{s} category for regular verbs \pcref{tab:carreg}.
%
\begin{table}[h]
\centering
\caption[Regular \carijo verbs]{Regular \carijo verbs \parencites[106]{meira1998proto}[173]{robayo2000avance}}
\label{tab:carreg}
\begin{tabular}[t]{@{}lll@{}}
\mytoprule
{} &    \qu{to arrive} &       \qu{to dance} \\
\mymidrule
\gl{1}   &    \obj{ji-tuda-} &    \obj{j-eharaga-} \\
\gl{2}   &    \obj{mɨ-tuda-} &    \obj{m-eharaga-} \\
\gl{1+2} &  \obj{kɨsi-tuda-} &  \obj{kɨs-eharaga-} \\
\gl{3}   &    \obj{ni-tuda-} &    \obj{n-eharaga-} \\
\mybottomrule
\end{tabular}
\end{table}%
%
Although the split-\gl{s} system was lost entirely, former \gl{s_a_} verbs can be identified by the presence of a detransitivizer, like \obj{ehɨnəhɨ} \qu{to fight} \exref{car-30}, derived from \obj{hɨnəhɨ} \qu{to kill} \parencite[179]{robayo2000avance}.
%The same marker is also found in \gl{3}>\gl{1} scenarios \exref{car1p.car-1} and possessed nouns \exref{car1p.car-31}.

\ex<car-30>\carijo \parencite[][79]{koch1908hiana}\\
\begingl
\glpreamble hẹnẹ(x)tónoko-mā́ɽẹ y-e̱-hẹnẹ́(x)yai̯//
\gla hɨnəhtono-ko=marə j-e-hɨnəhɨ-jai//
\glb enemy-\gl{pl}=with \gl{1}-\gl{detrz}-kill-\gl{npst}.\gl{cert}//
\glft \qu{I fight with the enemies.}//
\endgl
\xe
%


As noted in \cref{sec:taranoan}, this extension also erased any traces of putative \PTar \gl{1}\gl{s_a_} \rc{t-}.
However, it did not fully eclipse the old \gl{1}\gl{s_a_} marker \rc{w-}, which is attested as being preserved in the verbs \obj{tə} \qu{to go} \exref{car1w.car-24} and \obj{a} \qu{to be} \exref{car1w.car-25}.
In addition, the verb \obj{ka} \qu{to say} has a zero-marked first-person form \exref{car1w.car-33}.

\pex<car1w>\carijo \parencite[][5, 42, personal communication]{guerrero2016karihona}
\a<car-24>
\begingl
\gla wɨ-tə-e=rehe//
\glb \gl{1}-go-\gl{npst}=\gl{frust}//
\glft \qu{I almost go (but I am not going to go).}//
\endgl
\a<car-25>
\begingl
\gla əji-marə-ne w-a-e//
\glb \gl{2}-with-\gl{pl} \gl{1}-be-\gl{npst}//
\glft \qu{I am with you all.}//
\endgl
\a<car-33>
\begingl
\glpreamble dëmëmara kae ëwï iya//
\gla n-tə-mə=mara ∅-ka-e əwɨ i-ja//
\glb \gl{3}-go-\gl{pst}=\gl{dub} \gl{1}-say-\gl{npst}.\gl{cert} \gl{1}\gl{pro} \gl{3}-\gl{obl}//
\glft \qu{“Did s/he leave?”, I say to him.}//
\endgl
\xe
%We interpret this ∅ as a reflex of \rc{wɨ-}, a result of irregular phonological erosion.
Based on other C-initial verbs like \obj{tə} \qu{to go} or \obj{tuda} \qu{to arrive}, \obj{ka} \qu{to say} should either have conservative \obj{wɨ-} or innovative \obj{ji-}, so the zero is unexpected.
It is analysed here as a reflex of \rc{wɨ-}, primarily due to the loss of \rc{w} in \ikpeng and \hixka.
While those developments were more regular, an already irregular marker undergoing idiosyncratic phonological erosion is not that surprising, see \akuriyo \rc{wɨ-} > \obj{ə-} in the preceding section.
Alternatively, the divergent development of \rc{w-} on \obj{ka} \qu{to say} and \obj{tə} \qu{to go} may be due to the latter's originally V-initial nature \pcref{sec:go}.

\subsubsection{\yukpa \obj{j-}}
\label{sec:yukpa}
The divergent nature of \yukpa\footnote{Very little is known about its sister \japreria, whose status as a dialect of \yukpa is contested by \textcite{oquendo2004japreria}.} is \textit{inter alia}t visible in the loss of many \setone forms and the formation of non-cognate innovative constructions \parencite{meira2006syntactic}.
However, it does preserve the \setone prefixes in the immediate past, which shows a unified intransitive paradigm \pcref{tab:yukreg}.
The wholesale loss of \gl{1+2} as an inflectional value was combined with the extension of \gl{2}\gl{s_a_} \obj{m(ɨ)-} to (now former) \gl{s_p_} verbs like \obj{nɨ} \qu{to sleep}.
%
\begin{table}
\centering
\caption[Regular \yukpa verbs]{Regular \yukpa verbs \parencites[139]{meira2006syntactic}[72, 76]{largo2011yukpa}}
\label{tab:yukreg}
\begin{tabular}[t]{@{}llll@{}}
\mytoprule
{} &  \qu{to fall} & \qu{to sleep} & \qu{to wash self} \\
\mymidrule
\gl{1} &  \obj{j-ata-} &  \obj{jɨ-nɨ-} &     \obj{j-otum-} \\
\gl{2} &  \obj{m-ata-} &  \obj{mɨ-nɨ-} &     \obj{m-otum-} \\
\gl{3} &  \obj{n-ata-} &  \obj{nɨ-nɨ-} &     \obj{n-otum-} \\
\mybottomrule
\end{tabular}
\end{table}%
%
These verbs share their first person marker \obj{j(ɨ)-} with former \gl{s_a_} verbs like \obj{otum} \qu{to wash self}, identifiable by their semantics and the reflex of \detrz.
Since \obj{j(ɨ)-} is the reflex of the \PC \gl{1}S\textsubscript{(P)} marker \rc{u(j)-} \parencite[92]{gildea1998}, it also occurs on transitive verbs in \gl{3}>\gl{1} scenarios \exref{yukpatr.yuk-5}.
In contrast, \gl{1}>\gl{3} scenarios are zero-marked \exref{yukpatr.yuk-1}.

\pex<yukpatr>\yukpa \parencite[][139]{meira2006syntactic}
\a<yuk-5>
\begingl
\gla aw j-esare//
\glb \gl{1}\gl{pro} \gl{3}>\gl{1}-see//
\glft \qu{S/he saw me.}//
\endgl
\a<yuk-1>
\begingl
\gla aw {\normalfont ∅}-esare//
\glb \gl{1}\gl{pro} \gl{1}>\gl{3}-see//
\glft \qu{I saw it.}//
\endgl
\xe
%
Since \PC \gl{1}\gl{s_a_} \rc{w(ɨ)-} was extended to \gl{1}>\gl{3} scenarios in most languages \parencite[81--82]{gildea1998}, and given its inclination for phonological erosion \pcref{sec:pekodian,sec:waiwaian}, the zero marking in \gl{1}>\gl{3} scenarios can be identified as the \yukpa reflex of \gl{1}\gl{s_a_} \rc{w-}.

In intransitive verbs, this first-person zero marking is attested with a single verb, \obj{to} \qu{to go} \exref{yuk-7}.
It diverges from regular C-initial verbs with \obj{jɨ-}, like \qu{to sleep} \pcref{tab:yukreg}.
It can thus be identified as having resisted the extension of \obj{j-} in \yukpa.

\ex<yuk-7>\yukpa \parencite[][139]{meira2006syntactic}\\
\begingl
\gla aw {\normalfont ∅}-to//
\glb \gl{1}\gl{pro} \gl{1}\gl{s_a_}-go//
\glft \qu{I went.}//
\endgl
\xe

