\section{Discussion}
\label{sec:discussion}
In \cref{sec:verbs}, we reconstructed the verbs which were unaffected by the incomplete extensions discussed in \cref{sec:extensions}.
\cref{tab:overview} shows the correlation between the extensions and verbs; only verbs that were unaffected in at least two cases are listed.
That is, \trio \obj{weka}/\obj{oeka} \qu{to defecate} and the \akuriyo movement verbs are not shown.
We will discuss

\begin{table}
\centering
\caption{Overview of extensions and (un-)affected verbs}
\label{tab:overview}
\begin{tabular}[t]{@{}llllllll@{}}
\mytoprule
{} & \rc{ka[ti]} &  \rc{ɨtə[mə]} &   \rc{a[p]} &    \rc{eti} & \rc{(ət-)jəpɨ} &    \rc{ɨpɨtə} &   \rc{e-pɨ} \\
{} &    \qu{say} &       \qu{go} &   \qu{be-1} &   \qu{be-2} &      \qu{come} &  \qu{go down} &  \qu{bathe} \\
\midrule
\PWai \rc{k-}     &           × &             × &           × &           × &              – &  \textsc{n/a} &  \checkmark \\
\quad \hixka      &           × &             × &           × &           × &              – &  \textsc{n/a} &  \checkmark \\
\quad \waiwai     &           × &  (\checkmark) &           × &           × &              – &  \textsc{n/a} &  \checkmark \\
\PPek \rc{k-}     &           × &             × &           × &           × &              × &             × &           × \\
\quad \arara      &           × &             × &           × &           × &              × &             × &           × \\
\quad \ikpeng     &           × &    \checkmark &           – &           × &     \checkmark &             ? &           × \\
\quad \bakairi    &           × &             × &           × &           × &     \checkmark &    \checkmark &           × \\
\PTir \rc{t-}     &           × &             × &           × &           × &              × &  \textsc{n/a} &  \checkmark \\
\quad \trio       &           × &             × &           × &           × &              × &  \textsc{n/a} &  \checkmark \\
\quad \akuriyo    &           × &             × &           × &           ? &              × &  \textsc{n/a} &  \checkmark \\
\akuriyo \obj{k-} &           × &             × &           × &           ? &              × &             × &           × \\
\carijo \obj{j-}  &           × &             × &           × &  \checkmark &     \checkmark &  \textsc{n/a} &           ? \\
\yukpa \obj{j-}   &           ? &             × &  \checkmark &  \checkmark &              – &  \textsc{n/a} &           – \\
\bottomrule
\end{tabular}
\begin{legendlist}\item[\checkmark] affected
\item[×] not affected
\item[?] unknown first person prefix
\item[–] does not occur
\item[(\checkmark)] affected with surviving old marker
\item[\textsc{n/a}] not meaningfully answerable
\end{legendlist}\end{table}

\subsection{Possible motivations: \posscite{bybee1985morphology} network model}
\label{sec:motivations}
The fact that reflexes of several reconstructible verbs emerged as being unaffected by different extensions (\crefrange{sec:be}{sec:godown}) suggests that there is some strong motivation for these verbs to do so.
%From this list of verbs, two salient properties emerge: they are very frequent, and they do not have a reflex of the detransitivizer \rc{ət(e)-}/\obj{e-}.
The most well-known contribution regarding irregularity in the lexicon is \textcite{bybee1985morphology}, with her network model of morphology, which is well-suited for the data at hand.
It aims \dbqu{to account for cross-linguistic, diachronic and acquisition patterns in complex morphological systems} \parencite[428]{bybee1995regular}.
It does so by modeling shared morphological properties such as inflectional patterns as emerging from connections of differing strength between related words in the mental lexicon.
For example, a large group of connected \dbqu{strong} English verbs with \obj{strɪŋ}--\obj{strʌŋ} at its center and pairs like \obj{rɪŋ}--\obj{rʌŋ}, \obj{spɪn}--\obj{spʌn}, or \obj{stɪk}--\obj{stʌk} at its periphery is attracting more verbs in certain dialects: \obj{sniːk}--\obj{snʌk} or \obj{brɪŋ}--\obj{brʌŋ} \parencite[129--130]{bybee1985morphology}.
These verbs are recruited based on the lexical connection they form with prototypical members of the group, and accordingly develop \dbqu{irregular} past tense forms.

For the causes of these lexical connections, \textcite[118]{bybee1985morphology} suggests the criteria of semantic similarity, phonological similarity, and morphological similarity.
Another important factor in her model is frequency, since more frequent words have a higher lexical strength \parencite[119]{bybee1985morphology}.
This higher lexical strength results in less influence from other lexemes, meaning that irregular forms are more likely to be preserved in high-frequency items.
Thus, from a diachronic perspective, the prediction is a) that semantically\slash{} phonologically\slash{}morphologically similar verbs adapt the same morphological properties, and b) that frequent verbs show a certain immunity to changes.

When considering the resistant verbs in our Cariban case, reiterated in \cref{tab:resistant}, a very salient property emerges:
Most of the verbs lack a reflex of the detransitivizer \rc{ət(e)-}/\obj{e-} usually found in  \gl{s_a_} verbs.
That is, there is an apparent connection between presence of the detransitivizer and innovating new \gl{1}\gl{s_a_} markers.
This was already noted for a group of \PTar verbs taking irregular \rc{w-} by \textcite[112]{meira1998proto}:
\begin{quotation}
	This category includes a small number of stems, among which ‘to go’, ‘to come’. ‘to say’, ‘to go down’, ‘to defecate', and the copula. These are basically the verbs that are not synchronically or diachronically detransitivized. yet belong to the A conjugation.
\end{quotation}
The characterization of the absence of \detrz resulting in irregular verbs is also applicable to other languages and branches, most clearly so for Pekodian, which I will discuss in \cref{sec:morphology}.

On the other hand, the fact that reflexes of \rc{ət(e)-}/\obj{e-} are found on \dbqu{normal} \gl{s_a_} verbs also means that they are all \rc{ə-} or \rc{e-} initial.
That is, the morphologically caused lexical connection between regular \gl{s_a_} verbs is also phonological in nature.
In some cases, phonological conditioning seems to have indeed been the crucial factor, discussed in detail in \cref{sec:phonology}.

Finally, the first four verbs in \cref{tab:resistant} are also united by the fact that they are usually among the most frequent ones.
This has e.g.\ been noted for \kalina by \textcite[75]{courtz2008carib}: \dbqu{It is difficult […] to imagine an intransitive or transitive origin for some of the most frequent middle verbs}.
Such frequency effects are discussed in \cref{sec:frequency}.

Semantic connections do not appear to play a role, except potentially in the case of the ill-understood movement verbs in understudied \akuriyo.

In many cases it is difficult to decide which of the three relevant factors best explains the pattern in a specific language.
Rather, it seems that the three factors largely converge in the Cariban case, as discussed in \cref{sec:convergence}.


\subsubsection{Morphology: \PPek}
\label{sec:morphology}
A clear-cut example where morphology is the sole deciding factor is the introduction of \PPek \rc{k-}.
As I have argued in \cref{sec:pekodian}, there were two forms for \qu{to come}, \rc{epɨ} and \rc{əd-epɨ}, the latter with a detransitivizing prefix.
The \arara reflex shows a first person form \obj{w-ebɨ-}, while its sister language \ikpeng and the cousin \bakairi have forms based on \rc{k-ədepɨ}: \obj{k-arep-} and \obj{k-əewɨ-}.
For \PPek, the group of verbs which resisted the extension of \rc{k-} can succinctly be defined as those without a detransitivizing prefix, fully accounting for the group of unaffected verbs.

\subsubsection{Phonology: \akuriyo, \carijo, \yukpa}
\label{sec:phonology}
There is one case where phonological connections account for the lexical distribution of the innovative marker.
That is the introduction of \akuriyo \obj{k-}, which only affected \obj{ə}-initial verbs -- but not \obj{e}-initial ones, which kept \obj{t͡ʃ-} \pcref{sec:akuriyo}.
Using \posscite{bybee1985morphology} network model, these classes form a consistent lexical group, based on their phonological form.
The \akuriyo case is rather different from the others under discussion here, as the unaffected group of verbs is fairly large, since it includes regular \gl{s_a_} verbs with a reflex of the detransitivizer \rc{e-}.
There are two other cases, namely \carijo \obj{j-} \pcref{sec:carijo} and \yukpa \obj{j-} \pcref{sec:yukpa}.

In the case of \carijo, the group of affected verbs can be characterized as being \obj{e-}/\obj{ə-}initial.
That is, as in other languages, regular derived \gl{s_a_} verbs -- those with a reflex of \rc{ət(e)-}/\obj{e-} underwent the innovation, as shown in \cref{sec:carijo}.
However, underived \obj{e}-initial \gl{s_a_} verbs also took on new markers, as shown in \exref{care}.

\pex<care>\carijo
\a<car-18>
\begingl
\gla əji-wa-e j-eh-ɨ//
\glb \gl{2}-search-\gl{sup} \gl{1}-come-\gl{pfv}//
\glft \qu{I came looking for you.} \parencite[][102]{guerrero2019carijo}//
\endgl
\a<car-32>
\begingl
\gla irə wat͡ʃinakano tae j-ehɨtə-e//
\glb \gl{inan}.\gl{ana} body.of.water along.bounded \gl{1}-go.down-\gl{npst}//
\glft \qu{…I go down through that guachinacán.} \perscomm{David Felipe Guerrero}//
\endgl
\a<mivida-12>
\begingl
\glpreamble iretibə et͡ʃinəme gərə jet͡ʃiɨ//
\gla ireti-bə et͡ʃi-nə=me gərə j-et͡ʃi-ɨ//
\glb then-from be-\gl{inf}=\gl{attrz} still \gl{1}-be-\gl{pfv}//
\glft \qu{Then I was already grown up.} \parencite[][177]{robayo1989rame}//
\endgl
\xe
%
\exref{care.car-18} shows the verb \qu{to come}, which in \carijo is a reflex of \rc{jəpɨ} (> \rc{epɨ}), not of \rc{ətepɨ}.
\exref{care.car-32} shows the verb \qu{to go down}, which has acquired an unexpected \obj{e} in \carijo, compare the reconstruction in \cref{tab:descend_cog}.
Whether this development was distinct from the introduction of the new prefix \obj{j-} or whether it was a result of regularization is impossible to say.
Finally, \exref{care.mivida-12} shows the verb \obj{et͡ʃi} \qu{to be}, which also takes \obj{j-}.
Interestingly, this verb shows a suppletive alternation between \obj{et͡ʃi} and \obj{a}, where the old marker \obj{w-} is preserved with the latter root allomorph \exref{car-25}.

\ex<car-25>\carijo \parencite[][42]{guerrero2016karihona}\\
\begingl
\gla əji-marə-ne w-a-e//
\glb \gl{2}-\gl{com}-\gl{pl} \gl{1}-be-\gl{npst}//
\glft \qu{I am with you all.}//
\endgl
\xe
%
The only other place where this \obj{w-} is attested is with C-initial \obj{tə} \qu{to go} \exref{car-24}, meaning that the property of being \obj{ə}- or \obj{e}-initial fully accounts for the distribution of innovative \obj{j-}.

\ex<car-24>\carijo \parencite[][5]{guerrero2016karihona}\\
\begingl
\gla wɨ-tə-e=rehe//
\glb \gl{1}-go-\gl{npst}=\gl{frust}//
\glft \qu{I almost go (but I am not going to go).}//
\endgl
\xe

An apparently more extreme case is \yukpa, where \obj{j-} is not only found on the reflex of \rc{eti} \exref{yukbe.yuk-13}, but also on that of \rc{a(p)} \exref{yukbe.yuk-12}.

\pex<yukbe>\yukpa \parencite[][142, 143]{meira2006syntactic}
\a<yuk-13>
\begingl
\gla aw utuwanpa=p=j-e//
\glb \gl{1}\gl{pro} study=\gl{prog}=\gl{1}-be//
\glft \qu{I was studying.}//
\endgl
\a<yuk-12>
\begingl
\gla aw juwatpɨ=p=j-a-s//
\glb \gl{1}\gl{pro} chief=\gl{ess}=\gl{1}-be-\gl{npst}//
\glft \qu{I am the chief.}//
\endgl
\xe
%
The only place where I have identified a reflex of \rc{w-} is on C-initial \obj{to} \exref{yuk-7}.
This makes it possible to characterize the extension as affecting all V-initial verbs.

\ex<yuk-7>\yukpa \parencite[][139]{meira2006syntactic}\\
\begingl
\gla aw Ø-to//
\glb \gl{1}\gl{pro} \gl{1}\gl{s_a_}-go//
\glft \qu{I went.}//
\endgl
\xe

%Finally, a possible phonological motivation also comes into play with verbs that do not have a detransitivizing prefix synchronically, but only etymologically.
%TODO CONTINUE HERE with a good example of a synchronically underived verb

\subsubsection{Frequency: \bakairi?}
\label{sec:frequency}
Among the investigated cases of incomplete extensions, \bakairi is the only language where one might suggest frequency effects that are not coupled to something else, but the evidence is scarce.
In \cref{sec:morphology}, I argued that the \PPek verbs \rc{ipɨ} \qu{to bathe} and \rc{[ɨ/i]ptə} \qu{to go down} resisted the introduction of \rc{k-} because they did not have a reflex of \detrz.
However, while \bakairi \obj{i} \qu{to bathe} preserved the \PPek pattern, \obj{ɨtəgɨ} \qu{to go down} subsequently innovated \obj{k-}.
While it is possible that \obj{i}, which also means \qu{to wash} \parencite[105]{von1892bakairi}, is more frequent in \bakairi discourse than \qu{to go down}, such a claim would have to be supported by corpus data, which are not available.

\subsubsection{Converging factors}
\label{sec:convergence}
I have shown that in one case, morphological criteria account for the distribution of conservative and innovative prefixes, and that three cases can adequately be explained by purely phonological criteria.
I suggested that the development in \bakairi after the \PPek stage may be due to frequency, but only very speculatively so.
No semantic patterns have emerged as a conditioning factor for preserving old \gl{1}\gl{s_a_} markers in any of the cases under study.
As for the other three factors, the main conclusion is that they largely converge in the case of Cariban \gl{s_a_} verbs.
Take for example the \dbqu{most} resistant verb, which is not attested as having taken on a new first person marker in any language, \rc{ka(ti)} \qu{to say}.
It is at the same time: \begin{inlinelist}
	\item highly frequent
	\item the only C-initial \gl{s_a_} verb in \PC
	\item one of the few \gl{s_a_} verbs without a reflex of \rc{ət(e)-}/\obj{e-}
\end{inlinelist}.
That is, one would expect it to resist morphological innovation based on all three factors: frequency, phonological form, and morphological makeup.

The same kind of convergence is found for most other verbs consistently emerging as resistant across the family.
They are more frequent than other \gl{s_a_} verbs, and diverge phonologically and morphologically from them.
This results in an overall picture where the factors leading to irregular or archaic morphological patterns strongly overlap, to a degree where one cannot simply decide which factor ultimately contributed.
While this means that \posscite{bybee1985morphology} model for the most part nicely accounts for the Cariban patterns, the conditioning factors she suggests are highly interrelated.

\subsection{Conclusion}

\begin{itemize}
	\item why are the most irregular verbs all underived \gl{s_a_} verbs? \textbf{something} is there
	\item ultimately plays into the mystery of how the hell the split-\gl{s} system actually came into being
	\item not surprising that more frequent \gl{s_a_} verbs have no \rc{ət-}, but definitely surprising that \qu{say}, \qu{go}, and \qu{be} are \gl{s_a_} verbs in the first place!
\end{itemize}

%Also: \dbqu{lexical […] splits constitute important evidence about the structure of the lexicon.} \parencite[119]{bybee1985morphology}.
%This suggests that the unaffected verbs form their own distinct part of the lexicon.
%In the case of the ones without \detrz, this means that they are distinct from the \dbqu{regular} ones (open class) in the lexicon.