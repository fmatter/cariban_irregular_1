\section{Conclusion}
\label{sec:discussion}
The main research questions of this study were 
\begin{inlinelistplain}
	\item where did irregular first person prefixes originate?
	\item which verbs are irregular in what language?
	\item why are they irregular?
\end{inlinelistplain}
The main findings can be summarized as follows:
Irregular first person inflections are conservative, leftovers of person marker extensions which did not affect some verbs.
Some of these extensions are reconstructible to proto-languages, while others happened in pre-modern stages of single languages.
The same 1-7 verbs are conservative in all languages, and are often irregular in other ways.
\posscite{bybee1985morphology} network model delivers explanations for distribution of innovative markers, but in in 4 of 6 cases, multiple explanations predict the same outcome.
This is due to the situation reconstructible to \PC, where only a small group of frequent \gl{sa} verbs had no detransitivizer \detrz.
%Findings:
%\begin{itemize}
%	\item conservative verbs are leftovers of incomplete extensions
%	\item some are reconstructible to proto-languages
%	\item resistant verbs show a great degree of overlap between languages, and are often irregular in other ways, too
%	\item the 3/6 potential explanations identifiable with the network model have no unambiguous winner in 4/6 cases
%	\item this is due to what is reconstructible already for \PC
%\end{itemize}

It was the association of \detrz with \gl{a}-oriented prefixes that led to the split-\gl{s} system \parencite{meira2000split}.
An open question is why the few underived \gl{sa} verbs that are reconstructible to \PC are not only the most frequent \gl{sa} verbs, but arguably the most frequent intransitive verbs (\qu{be}, \qu{go}, \qu{say}).
At least the first two meanings could just as well be expressed with \gl{sp} verbs.
Further comparative work on the detransitivizer might answer this question.


%Family context:
%\begin{itemize}
%	\item the association of \gl{a}-oriented prefixes with detransitivized verbs is somewhat surprising, given passive readings
%	\item even more surprising is that underived \gl{sa} verbs are not only conservative, but are also among the most irregular verbs overall (copula, say, go down...)
%	\item origins of \gl{sa} system still unclear
%	\item why are these verbs \gl{sa} verbs in the first place? we don't know
%\end{itemize}

\posscite{bybee1985morphology} network model held its promise of explaining irregularities in inflectional patterns.
It  did overshoot its goal somewhat in that only two of six cases had an unambiguous answer -- phonology.
Notably, the two phonology-based hypotheses (with and without frequency) together correctly predicted 100\% of patterns.
Since the crude frequency model increased overall prediction accuracy, but decreased it in the case of \carijo and \yukpa, language-specific counts of \gl{sa} verbs would be interesting.
Regardless of the interpretation of the results, the inconclusiveness of the network model's answers is due to a specific pattern in \PC.
This raises the general question how applicable the model is when ambiguity exists.

%
%
%Theory context:
%\begin{itemize}
%	\item network model did not disappoint in delivering attractive hypotheses for morphological patterns
%	\item but no unambiguous answers, unfortunately
%	\item however, phonology with optional frequency fares best overall; frequency counts may not be representative where frequency fares worse
%	\item what to do when not obvious?
%\end{itemize}

As mentioned, additional and more extensive counts of \gl{sa} verbs in different Cariban languages would be important not only for comparison with the \apalai sample, but also as better input for the network model.
Apart from this specific purpose, there is a general need for corpora of Cariban languages accompanying the improving descriptive coverage, allowing studies like \textcite{sapien2021antipassive}.
Finally, this study could have benefitted from more extensive descriptive work on \yukpa and \carijo specifically, while such work on underdescribed languages would benefit Cariban linguistics as a whole.
%
%Follow-up:
%\begin{itemize}
%	\item confirmation of frequency ratios in other languages, relation to \gl{sp} and transitive verbs
%	\item generally, corpus-based investigations like \textcite{sapien2021antipassive} are good and needed, much to be found
%	\item more extensive descriptions of \yukpa and \carijo (and \akuriyo) please
%\end{itemize}

%Also: \dbqu{lexical […] splits constitute important evidence about the structure of the lexicon.} \parencite[119]{bybee1985morphology}.
%This suggests that the unaffected verbs form their own distinct part of the lexicon.
%In the case of the ones without \detrz, this means that they are distinct from the \dbqu{regular} ones (open class) in the lexicon.