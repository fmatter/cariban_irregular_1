\section{Discussion}
\label{sec:discussion}

\begin{itemize}
	\item why are the most irregular verbs all underived \gl{s_a_} verbs? \textbf{something} is there
	\item ultimately plays into the mystery of how the hell the split-\gl{s} system actually came into being
	\item not surprising that more frequent \gl{s_a_} verbs have no \rc{ət-}, but definitely surprising that \qu{say}, \qu{go}, and \qu{be} are \gl{s_a_} verbs in the first place!
\end{itemize}

%Also: \dbqu{lexical […] splits constitute important evidence about the structure of the lexicon.} \parencite[119]{bybee1985morphology}.
%This suggests that the unaffected verbs form their own distinct part of the lexicon.
%In the case of the ones without \detrz, this means that they are distinct from the \dbqu{regular} ones (open class) in the lexicon.