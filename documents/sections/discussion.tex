\section{Discussion}
\label{sec:discussion}
In \cref{sec:verbs}, we reconstructed the verbs which were untouched by the incomplete person marker extensions discussed in \cref{sec:extensions}.
\cref{tab:overview} gives an overview of what verbs were affected by which extensions, except for \obj{e}-initial \akuriyo verbs unaffected by the extension of \obj{k-}, as they are a large and predictable group.
In a few cases we do not know the first person form (?), in others we have reason to believe that the verb does not occur at all or at least not inflected for first person (–), and in the case of \qu{to go down} we often do not know when the switch to \gl{s_a_} happened, if at all (\textsc{n/a}).
Every \checkmark stands for a verb affected by an extension, while × represents conservatively inflected verbs.
This overview makes clear just how pervasive the tendency for these verbs to resist innovative markers is, as they do so in different languages.

\begin{table}
\centering
\caption{Overview of extensions and (un-)affected verbs}
\label{tab:overview}
\begin{tabular}[t]{@{}llllllll@{}}
\mytoprule
{} & \rc{ka[ti]} &  \rc{ɨtə[mə]} &   \rc{a[p]} &    \rc{eti} & \rc{(ət-)jəpɨ} &    \rc{ɨpɨtə} &   \rc{e-pɨ} \\
{} &    \qu{say} &       \qu{go} &   \qu{be-1} &   \qu{be-2} &      \qu{come} &  \qu{go down} &  \qu{bathe} \\
\midrule
\PWai \rc{k-}     &           × &             × &           × &           × &              – &  \textsc{n/a} &  \checkmark \\
\quad \hixka      &           × &             × &           × &           × &              – &  \textsc{n/a} &  \checkmark \\
\quad \waiwai     &           × &  (\checkmark) &           × &           × &              – &  \textsc{n/a} &  \checkmark \\
\PPek \rc{k-}     &           × &             × &           × &           × &              × &             × &           × \\
\quad \arara      &           × &             × &           × &           × &              × &             × &           × \\
\quad \ikpeng     &           × &    \checkmark &           – &           × &     \checkmark &             ? &           × \\
\quad \bakairi    &           × &             × &           × &           × &     \checkmark &    \checkmark &           × \\
\PTir \rc{t-}     &           × &             × &           × &           × &              × &  \textsc{n/a} &  \checkmark \\
\quad \trio       &           × &             × &           × &           × &              × &  \textsc{n/a} &  \checkmark \\
\quad \akuriyo    &           × &             × &           × &           ? &              × &  \textsc{n/a} &  \checkmark \\
\akuriyo \obj{k-} &           × &             × &           × &           ? &              × &             × &           × \\
\carijo \obj{j-}  &           × &             × &           × &  \checkmark &     \checkmark &  \textsc{n/a} &           ? \\
\yukpa \obj{j-}   &           ? &             × &  \checkmark &  \checkmark &              – &  \textsc{n/a} &           – \\
\bottomrule
\end{tabular}
\begin{legendlist}\item[\checkmark] affected
\item[×] not affected
\item[?] unknown first person prefix
\item[–] does not occur
\item[(\checkmark)] affected with surviving old marker
\item[\textsc{n/a}] not meaningfully answerable
\end{legendlist}\end{table}

It is astonishing that the same 1-7 verbs retain their old first person marker in 6 distinct developments, while a plethora of regular \gl{s}\textsubscript{(\gl{a})} verbs take on innovative markers.
This suggest that there is some strong motivation for these verbs to not be affected by innovative markers.
The question arises what properties unite these verbs and make them so conservative across different Cariban languages.
We will discuss possible answers to this question in \cref{sec:motivations}, using \posscite{bybee1985morphology} network model of morphology.

\subsection{Reasons for conservativeness}
\label{sec:motivations}
Perhaps the most well-known contribution regarding conservativeness, innovativeness, and (ir-){}re\-gu\-la\-ri\-ty in the lexicon is \textcite{bybee1985morphology} with her network model of morphology, which seems well-suited for the data at hand.
It aims \dbqu{to account for cross-linguistic, diachronic and acquisition patterns in complex morphological systems} \parencite[428]{bybee1995regular}.
It does so by modeling shared morphological properties such as inflectional patterns as emerging from connections of differing strength between lexemes.
A classic example is a network of \dbqu{strong} English verbs with \obj{strɪŋ}--\obj{strʌŋ} at the center and pairs like \obj{rɪŋ}--\obj{rʌŋ}, \obj{spɪn}--\obj{spʌn}, or \obj{stɪk}--\obj{stʌk} at its periphery.
This network is attracting new verbs in certain dialects, like \obj{sniːk}--\obj{snʌk} or \obj{brɪŋ}--\obj{brʌŋ} \parencite[129--130]{bybee1985morphology}.
These verbs are recruited based on the lexical connection they form with prototypical members of the group, and accordingly develop irregular or \dbqu{strong} past tense forms.

As possible bases of these connections between lexemes, \textcite[118]{bybee1985morphology} suggests the criteria of semantic, phonological, and morphological similarity; the English strong verbs are an example for a phonologically motivated network.
Another important factor in the model is frequency, since more frequent words have a higher lexical strength \parencite[119]{bybee1985morphology}.
This higher lexical strength diminishes the influence from other lexemes, meaning that high-frequency items are more likely to resist innovations.
For our diachronic study of Cariban inflectional patterns, the model would predict that a) semantically\slash{}phonologically\slash\hspace{0pt}morphologically similar verbs will be affected by person marker extensions, and b) high-frequency verbs will tend to resist these extensions and thus remain conservative.

When one considers the groups of verbs with innovative first person markers (those not in \cref{tab:overview}), one can perceive multiple factors potentially serving as the thread connecting a lexical network.
Perhaps the most obvious one is that they all have a reflex of the detransitivizer \detrz, a hallmark of \gl{s_a_} verbs \pcref{sec:split}.
%This was already noted by \textcite{meira1998proto} for the group of \PTar verbs taking irregular first person \rc{w-}:
%\begin{quotebox}{\parencite[112]{meira1998proto}}
%	This category includes a small number of stems, among which ‘to go’, ‘to come’, ‘to say’, [...]\footnote{The original list includes \qu{to go down} and \qu{to defecate}. While these verbs are indeed underived \gl{s_a_} verbs in \trio, no irregular first person \rc{w-} can be reconstructed. As their inclusion in the list was most likely an error, we omit them here.} and the copula. These are basically the verbs that are not synchronically or diachronically detransitivized, yet belong to the A conjugation.
%\end{quotebox} % ‘to go down’, ‘to defecate’,
This also has a phonological consequence: all affected \gl{s_a_} verbs begin with reflexes of \rc{ə} or \rc{e}, meaning that networks with a phonological basis are also plausible.
A more trivial connection between these other verbs is that they are all \gl{s_a_} verbs, and thus share inflectional morphological patterns.
To make this potential network factor more specific, we will restrict it to the \gl{1}\gl{s_a_} prefix (pre-innovation).
These two criteria based on inflectional morphology predict the exact same verbs except in \akuriyo, where there was already innovative \gl{1}\gl{s_a_} morphology, with \obj{k-} and idiosyncratic \obj{p-} on \qu{to go down}.
There are no semantic patterns that would be obvious, which is not unsurprising given the lack of semantic patterns in the split-\gl{s} system overall \pcref{sec:split}.
For each extension, this leaves us with three hypotheses as to what connected the members of the responsible network: a reflex of \gl{detrz}, their stem-initial phoneme, or a specific \gl{1}\gl{s_a_} prefix.

It is intuitively obvious that many of the conservative verbs in \cref{tab:overview} are high-frequency verbs, which would mean high lexical strength and conservativeness according to the network model.
A major obstacle to confirming this intuition is the lack of frequency counts for individual lexemes for any Cariban language.
We are only aware of \posscite[75]{courtz2008carib} claim of \kalina underived \gl{s_a_} verbs being the most frequent ones: \dbqu{It is difficult […] to imagine an intransitive or transitive origin for some of the most frequent middle verbs}.
This claim is supported neither by frequency counts nor accompanied by a list of verbs, although it seems likely that these underived \gl{s_a_} verbs refer to the two roots for \qu{to be}, as well as \qu{to say}, \qu{to go}, and \qu{to come}.
Given this dearth of data, we conducted a count of \gl{s_a_} verbs in three glossed texts from \textcite{koehns1994textos}, the results of which are shown in \cref{tab:apalaicounts}.
The \apalai data agree with our interpretation of \posscite{courtz2008carib} claim; defining \dbqu{high frequency} as having an above average count produces the exact same five verbs.
While it is not at all certain that this small \apalai sample is really representative of discourse patterns in the Cariban (proto-)languages under discussion, the absence of alternatives led us to use it as a tool for categorizing verbs as high-frequency.

\begin{table}[h]
\centering
\caption[Frequency counts of \gl{s_a_} verbs in \apalai]{Frequency counts of \gl{s_a_} verbs in three \apalai texts from \textcite{koehns1994textos} (163 \gl{s_a_} verbs, 1070 words)}
\label{tab:apalaicounts}
\begin{tabular}[t]{@{}lrr}
\mytoprule
                              Verb &  Count & \% \gl{s_a_} verb tokens \\
\mymidrule
                 \obj{a} \qu{be-1} &     49 &                  30.06\% \\
               \obj{eʃi} \qu{be-2} &     30 &                  18.40\% \\
                 \obj{ka} \qu{say} &     26 &                  15.95\% \\
                 \obj{ɨto} \qu{go} &     23 &                  14.11\% \\
              \obj{oepɨ} \qu{come} &     13 &                   7.98\% \\
       \obj{e-poreʔka} \qu{arrive} &      3 &                   1.84\% \\
           \obj{ot-urupo} \qu{ask} &      2 &                   1.23\% \\
              \obj{ot-uʔ} \qu{eat} &      2 &                   1.23\% \\
     \obj{os-enakũnuʔ} \qu{choke} &      2 &                   1.23\% \\
          \obj{e-unopɨ} \qu{laugh} &      1 &                   0.61\% \\
    \obj{at-akĩma} \qu{pack bags} &      1 &                   0.61\% \\
    \obj{at-ankɨema} \qu{be happy} &      1 &                   0.61\% \\
      \obj{os-ereh} \qu{be amazed} &      1 &                   0.61\% \\
\obj{e-metɨka} \qu{lose loincloth} &      1 &                   0.61\% \\
       \obj{e-tuarima} \qu{suffer} &      1 &                   0.61\% \\
            \obj{e-puka} \qu{fall} &      1 &                   0.61\% \\
          \obj{os-eporɨ} \qu{meet} &      1 &                   0.61\% \\
         \obj{ot-ɨrɨʔka} \qu{land} &      1 &                   0.61\% \\
         \obj{ot-ɨʔka} \qu{finish} &      1 &                   0.61\% \\
            \obj{ot-uru} \qu{talk} &      1 &                   0.61\% \\
    \obj{at-apiaka} \qu{divide up} &      1 &                   0.61\% \\
         \obj{e-sɨrɨʔma} \qu{move} &      1 &                   0.61\% \\
\mybottomrule
\end{tabular}
\end{table}

%\textcite[153]{alves2017arara}: \dbqu{Observe que todos os 12 verbos indexados pelo morfema \obj{k-} [\gl{s_a_}] apresentam a raiz iniciada pela vogal /o/}.

Thus, in addition to the three hypotheses for possible network factors, each can be combined with frequency; high-frequency verbs are predicted to not undergo innovation, even though the factor under investigation would put them in the same lexical network as regular \gl{s_a_} verbs.
For each extension, this leaves us with six possible explanations for which verbs are affected and which are not.
First, we established for each explanation what behavior it would predict for each verb, illustrated in \cref{tab:ptir-predictions.tex} for \PTir.
For example, \rc{eʔi} \qu{to be} is expected to participate in innovations spreading in a phonologically defined network, being \rc{e}-initial, as well as in an inflectionally defined one, since it shared \rc{w-} with other \gl{s_a_} verbs.
However, it would not have belonged to a network defined by the presence of a detransitivizer; if frequency is taken into account, it is expected to remain conservative regardless of the nature of the network.
We then checked these predictions against the data in \cref{tab:overview}, to see how many potentially conservative verb each explanation predicted correctly.
This gave us a score of what proportion of potentially conservative verbs had their behavior predicted accurately, illustrated for \PTir in \cref{tab:ptir-evaluations} and summed up in \cref{tab:resultsoverview}.

\begin{table}
\centering
\caption{Predictions for \PTir}
\label{tab:ptir-predictions}
\begin{tabular}[t]{@{}lllllll@{}}
\mytoprule
{} &      \rc{a} &    \rc{eʔi} &  \rc{əʔepɨ} &   \rc{təmɨ} &     \rc{ka} &               \rc{epɨ} \\
{} &     \qu{be} &     \qu{be} &   \qu{come} &     \qu{go} &    \qu{say} & \qu{bathe (\gl{intr})} \\
\mymidrule
\gl{detrz}                       &           × &           × &  \checkmark &           × &           × &             \checkmark \\
\gl{detrz}+freq                  &           × &           × &           × &           × &           × &             \checkmark \\
phono (\envr{}{\rc{ə}, \obj{e}}) &           × &  \checkmark &  \checkmark &           × &           × &             \checkmark \\
phono+freq                       &           × &           × &           × &           × &           × &             \checkmark \\
infl (\rc{w-})                   &  \checkmark &  \checkmark &  \checkmark &  \checkmark &  \checkmark &             \checkmark \\
infl+freq                        &           × &           × &           × &           × &           × &             \checkmark \\
\mybottomrule
\end{tabular}
\end{table}
\begin{table}[h]
\centering
\caption{Evaluating predictions for \PTir}
\label{tab:ptir-evaluations}
\begin{tabular}[t]{@{}lllllllr}
\mytoprule
{} &     \rc{ka} & \rc{ɨtə[mɨ]} &      \rc{a} &    \rc{eʔi} &  \rc{əʔepɨ} &    \rc{epɨ} &  Score \\
{} &    \qu{say} &      \qu{go} &   \qu{be-1} &   \qu{be-2} &   \qu{come} &  \qu{bathe} &        \\
\mymidrule
\gl{detrz}+freq &  \checkmark &   \checkmark &  \checkmark &  \checkmark &  \checkmark &  \checkmark & 100.0\% \\
phono+freq      &  \checkmark &   \checkmark &  \checkmark &  \checkmark &  \checkmark &  \checkmark & 100.0\% \\
infl+freq       &  \checkmark &   \checkmark &  \checkmark &  \checkmark &  \checkmark &  \checkmark & 100.0\% \\
\gl{detrz}      &  \checkmark &   \checkmark &           × &  \checkmark &  \checkmark &  \checkmark &  83.3\% \\
phono           &  \checkmark &            × &           × &  \checkmark &  \checkmark &  \checkmark &  66.7\% \\
infl            &           × &            × &           × &           × &           × &  \checkmark &  16.7\% \\
\mybottomrule
\end{tabular}
\end{table}


\begin{table}[h]
\centering
\caption{Overview of prediction accuracy}
\label{tab:resultsoverview}
\begin{tabular}[t]{@{}lrrrrrr}
\mytoprule
{} &  \gl{detrz} &  \gl{detrz}+freq &  phono &  phono+freq &   infl &  infl+freq \\
\mymidrule
\PWai \rc{k-}     &      100.0\% &           100.0\% &  60.0\% &      100.0\% &  20.0\% &     100.0\% \\
\PPek \rc{k-}     &      100.0\% &           100.0\% &  71.4\% &      100.0\% &   0.0\% &      71.4\% \\
\PTir \rc{t-}     &       83.3\% &           100.0\% &  66.7\% &      100.0\% &  16.7\% &     100.0\% \\
\akuriyo \obj{k-} &       66.7\% &            83.3\% & 100.0\% &      100.0\% & 100.0\% &     100.0\% \\
\carijo \obj{j-}  &       60.0\% &            60.0\% & 100.0\% &       60.0\% &  40.0\% &      60.0\% \\
\yukpa \obj{j-}   &       33.3\% &            33.3\% & 100.0\% &       33.3\% &  66.7\% &      33.3\% \\
\mybottomrule
\end{tabular}
\end{table}

It is important to understand that the scores in \cref{tab:resultsoverview} only refer to the group of seven verbs in \cref{tab:overview}, i.e. those that are attested as resisting at least one extension.
For each extension, there were also many run-of-the-mill \gl{s_a_} verbs, all taking on the new person marker, except for the \akuriyo \obj{e}-initial verbs.%
\footnote{While there are a few \gl{s_a_} verbs not transparently derived from transitive verbs \parencites[252]{triomeira1999}[222]{meira2000split}[30]{gildea2007greenberg}, which are not featured in \cref{tab:overview}, these are mostly \rc{ə}-initial and were likely productively derived at some point.
The verbs to which this does not apply, like \trio \obj{wa} \qu{to dance} \parencites[252]{triomeira1999}, are all instances of \gl{s_p_} verbs switching classes.
Since none of them is attested as being an \gl{s_a_} verb at the time of a person marker extension, they are not relevant for our discussion of conservative verbs.}
To illustrate, if one adds 1'000 regular \gl{s_a_} verbs -- a conservative estimate based on \posscite{courtz2008carib} \kalina dictionary -- all six explanations consistently predict the behavior of 99.99+\% verbs correctly.
However, the available data simply does not allow such large-scale tests for Cariban languages, so we restrict our investigation to the edge cases.


%\begin{table}
\centering
\caption{Ratio of (un-)affected verbs explained by possible factors}
\label{tab:factors}
\begin{tabular}[t]{@{}llll@{}}
\mytoprule
{} & morphological & phonological &   frequency \\
\midrule
\PWai \rc{k-}     &    5/5 (100\%) &    4/5 (80\%) &  5/5 (100\%) \\
\PPek \rc{k-}     &    7/7 (100\%) &    5/7 (71\%) &   4/7 (57\%) \\
\PTir \rc{t-}     &     5/6 (83\%) &    4/6 (67\%) &   5/6 (83\%) \\
\akuriyo \obj{k-} &     5/6 (83\%) &   6/6 (100\%) &   3/6 (50\%) \\
\carijo \obj{j-}  &     3/5 (60\%) &   5/5 (100\%) &   4/5 (80\%) \\
\yukpa \obj{j-}   &     1/3 (33\%) &   3/3 (100\%) &   1/3 (33\%) \\
\bottomrule
\end{tabular}
\end{table}

The extent of the extensions in both \PWai and \PPek is fully predicted by the presence or absence of a detransitivizer.
In both cases, only the underived%
\footnote{Note that for \PPek, we assumed that the idiosyncratic evolution of \rc{e-pɨ} \qu{to bathe (\gl{intr})} to \rc{ipɨ}  made the verb morphologically opaque.}
 \gl{s_a_} verbs were not affected, all other \gl{s_a_} verbs taking \rc{k-}.
Not shown in \cref{tab:resultsoverview.tex} are subsequent evolutions in the Pekodian daughter languages, which can largely be argued to also be due to the detransitivizer:
First, we argued that both \ikpeng and \bakairi regularized the paradigm to use forms with detransitivizer for first person, which in both languages led to an introduction of \obj{k-}.%
\footnote{If one instead assumes that first person \rc{w-ebɨ-} and \rc{k-əd-ebɨ-} already co-existed in \PPek, the clear correlation between \rc{k-} and the detransitivizer remains.}
Second, the subsequent introduction of \obj{k-} to \ikpeng \obj{aran} \qu{to go} (< \rc{ɨtən}) potentially suggests a reanalysis of \obj{ar} as a detransitivizer.


Three extensions are fully predicted by phonological criteria, those in \akuriyo, \carijo, and \yukpa.
We have already discussed \akuriyo \obj{k-} \pcref{sec:akuriyo}, which only appears on \obj{ə}-initial verbs.
In \carijo, the extension of \obj{j-} affected \obj{e}- and \obj{ə}-initial verbs, including \obj{eh} \qu{to come} or \obj{et͡ʃi} \qu{to be}, which do not have a detransitivizing prefix.
Only \obj{ka} \qu{say}, \obj{təmə} \qu{go}, and \obj{a} \qu{be-1} did not take on \obj{j-}.
Similarly, the extension of \yukpa \obj{j-} can succinctly be characterized as affecting all vowel-initial verbs; the only verb attested as unaffected is C-initial \obj{to} \qu{to go}.
Inflectional morphology as a network basis only played a potential role in the case of \akuriyo, but it must be noted that we treated the first person markers \obj{t-} and \obj{t͡ʃ-} as distinct, which were of course phonologically conditioned.

When additionally considering putative conservatory frequency effects, prediction accuracy was improved in 8 cases, stagnated in 7 cases, and worsened in 3 cases.
The three cases where our rough model of verb frequency arrives at incorrect predictions are found in \carijo and \yukpa, the only languages to feature innovative markers on the reflexes of \rc{eti} \qu{be-2}, \yukpa also on \rc{a[p]} \qu{be-1}.
Including frequency in the model led to overall improvements, resulting in a 100\% prediction accuracy for all three potential factors in \PTir, as well as for the inflection criterion in \PWai.

Overall, the patterns of most extensions are correctly predicted not by a single explanation, but usually between 3 and 4, except those in \carijo and \yukpa.
Here, a lexical network with a phonological basis emerges as an unambiguous winner, while frequency-based explanations fare much worse.
%It may be worth noting that the three extensions which we do not reconstruct to a proto-language (ti.e., which are more recent) are best explained by phonological conditioning factors.
For the other extensions, the network model gives no unambiguous answer to the question of what combination of factors caused the innovative markers to spread the way they did.
This in turn is due to the fact that three of the factors we used to account for morphological behavior -- detransitivizer, phonology, frequency -- largely converge in their predictions:
The most frequent verbs are at the same time those without a detransitivizer, and therefore mostly of a different phonological shape than regular \gl{s_a_} verbs.





\subsection{Conclusion}

\begin{itemize}
	\item why are the most irregular verbs all underived \gl{s_a_} verbs? \textbf{something} is there
	\item ultimately plays into the mystery of how the hell the split-\gl{s} system actually came into being
	\item not surprising that more frequent \gl{s_a_} verbs have no \rc{ət-}, but definitely surprising that \qu{say}, \qu{go}, and \qu{be} are \gl{s_a_} verbs in the first place!
\end{itemize}

%Also: \dbqu{lexical […] splits constitute important evidence about the structure of the lexicon.} \parencite[119]{bybee1985morphology}.
%This suggests that the unaffected verbs form their own distinct part of the lexicon.
%In the case of the ones without \detrz, this means that they are distinct from the \dbqu{regular} ones (open class) in the lexicon.