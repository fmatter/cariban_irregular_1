\section{Conclusion}
\label{sec:discussion}
The first research question of this study asked where the irregularly inflected first person forms in some Cariban languages came from.
A second question was what verbs are irregular in what languages.
Finally, reasons for their irregularness were sought.

The main findings can be summarized as follows:
Verbs irregularly inflected for first person are conservative, leftovers of person marker extensions which left some verbs untouched.
Some of these extensions are reconstructible to proto-languages, while others happened in pre-modern stages of single languages.
Conservatively inflected verbs show a great degree of overlap between languages, and often behave irregularly in other ways, too.
While \posscite{bybee1985morphology} network model offers eplanations for the lexical extent of innovative markers, in 4 of 6 cases it gives no unambiguous answer, as multiple factors predict the same outcome.
This is due to the situation reconstructible to \PC, where only a small group of frequent \gl{s_a_} verbs had no prefix \detrz.
%Findings:
%\begin{itemize}
%	\item conservative verbs are leftovers of incomplete extensions
%	\item some are reconstructible to proto-languages
%	\item resistant verbs show a great degree of overlap between languages, and are often irregular in other ways, too
%	\item the 3/6 potential explanations identifiable with the network model have no unambiguous winner in 4/6 cases
%	\item this is due to what is reconstructible already for \PC
%\end{itemize}

While the origins of the split-\gl{s} system clearly have to do wih the detransitivizer, the question why \detrz is associated \gl{a}-oriented prefixes \parencite{meira2000split} is still not answered.
Another question to be asked is why the few underived \gl{s_a_} verbs that are reconstructible to \PC are not only the most frequent \gl{s_a_} verbs, but arguably the most frequent intransitive verbs (\qu{be}, \qu{say}, \qu{go}).
There is no reason why they could not be simple \gl{s_p_} verbs, except maybe in the case of \rc{ka[ti]} \qu{to say} with its transitive tendencies.
The answer might be found in the origins of \detrz and its association with \gl{a}-oriented prefixes.


%Family context:
%\begin{itemize}
%	\item the association of \gl{a}-oriented prefixes with detransitivized verbs is somewhat surprising, given passive readings
%	\item even more surprising is that underived \gl{s_a_} verbs are not only conservative, but are also among the most irregular verbs overall (copula, say, go down...)
%	\item origins of \gl{s_a_} system still unclear
%	\item why are these verbs \gl{s_a_} verbs in the first place? we don't know
%\end{itemize}

As for \posscite{bybee1985morphology} network model of morphology, it fulfilled the promise of delivering attractive explanations for irregularities in inflectional patterns.
If anything, it overshot its goal somewhat; only two of the four investigated innovations had an unambiguous answer -- phonology.
It may be noted that across all six hypotheses, the two featuring phonology (one with, once without frequency) together correctly predicted 100\% of patterns.
Since frequency decreased prediction accuracy in the case of \carijo and \yukpa, it would be interesting to see \gl{s_a_} verb frequency statistics from corpora of these languages.
However one interprets the results of the network model investigation, it needs to be pointed out that the ambiguity of its answers are due to the special circumstances in \PC, rather than a flaw in the model.
Still, the results raise the question how applicable the model is in circumstances where ambiguity arises.

%
%
%Theory context:
%\begin{itemize}
%	\item network model did not disappoint in delivering attractive hypotheses for morphological patterns
%	\item but no unambiguous answers, unfortunately
%	\item however, phonology with optional frequency fares best overall; frequency counts may not be representative where frequency fares worse
%	\item what to do when not obvious?
%\end{itemize}

As mentioned, other and more extensive counts of \gl{s_a_} verbs in different Cariban languages would be important not only for comparison with the \apalai sample, but also providing better input for the network model.
Apart from this specific purpose, corpus-based investigations like \textcite{sapien2021antipassive} in Cariban languages are direly needed, to accompany the improving descriptive side.
Concerning the latter, more extensive descriptive work on \yukpa and \carijo would not only tremendously benefit this paper, but Cariban studies as a whole.
%
%Follow-up:
%\begin{itemize}
%	\item confirmation of frequency ratios in other languages, relation to \gl{s_p_} and transitive verbs
%	\item generally, corpus-based investigations like \textcite{sapien2021antipassive} are good and needed, much to be found
%	\item more extensive descriptions of \yukpa and \carijo (and \akuriyo) please
%\end{itemize}

%Also: \dbqu{lexical […] splits constitute important evidence about the structure of the lexicon.} \parencite[119]{bybee1985morphology}.
%This suggests that the unaffected verbs form their own distinct part of the lexicon.
%In the case of the ones without \detrz, this means that they are distinct from the \dbqu{regular} ones (open class) in the lexicon.