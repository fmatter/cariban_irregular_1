\section{Discussion}
\label{sec:discussion}
In \cref{sec:verbs}, we reconstructed the verbs which were unaffected by the incomplete extensions discussed in \cref{sec:extensions}.
\cref{tab:overview} shows what verbs were affected by which extensions, except for the \obj{e}-initial \akuriyo verbs unaffected by the extension of \obj{k-}.
In a few cases we do not know the first person form, in others we have reason to believe that the verb does not occur at all or at least not inflected for first person, and in the case of \qu{to go down} we do not know whether or when the switch to \gl{s_a_} happened.
The patterns in this table make clear just how pervasive the tendency for certain verbs to resist innovative markers is.
Every × stands for a language in which this particular \gl{s_a_} verb has an archaic first person form, while regularly inflected \gl{s_a_} verbs have an innovative form.

\begin{table}
\centering
\caption{Overview of extensions and (un-)affected verbs}
\label{tab:overview}
\begin{tabular}[t]{@{}llllllll@{}}
\mytoprule
{} & \rc{ka[ti]} &  \rc{ɨtə[mə]} &   \rc{a[p]} &    \rc{eti} & \rc{(ət-)jəpɨ} &    \rc{ɨpɨtə} &   \rc{e-pɨ} \\
{} &    \qu{say} &       \qu{go} &   \qu{be-1} &   \qu{be-2} &      \qu{come} &  \qu{go down} &  \qu{bathe} \\
\midrule
\PWai \rc{k-}     &           × &             × &           × &           × &              – &  \textsc{n/a} &  \checkmark \\
\quad \hixka      &           × &             × &           × &           × &              – &  \textsc{n/a} &  \checkmark \\
\quad \waiwai     &           × &  (\checkmark) &           × &           × &              – &  \textsc{n/a} &  \checkmark \\
\PPek \rc{k-}     &           × &             × &           × &           × &              × &             × &           × \\
\quad \arara      &           × &             × &           × &           × &              × &             × &           × \\
\quad \ikpeng     &           × &    \checkmark &           – &           × &     \checkmark &             ? &           × \\
\quad \bakairi    &           × &             × &           × &           × &     \checkmark &    \checkmark &           × \\
\PTir \rc{t-}     &           × &             × &           × &           × &              × &  \textsc{n/a} &  \checkmark \\
\quad \trio       &           × &             × &           × &           × &              × &  \textsc{n/a} &  \checkmark \\
\quad \akuriyo    &           × &             × &           × &           ? &              × &  \textsc{n/a} &  \checkmark \\
\akuriyo \obj{k-} &           × &             × &           × &           ? &              × &             × &           × \\
\carijo \obj{j-}  &           × &             × &           × &  \checkmark &     \checkmark &  \textsc{n/a} &           ? \\
\yukpa \obj{j-}   &           ? &             × &  \checkmark &  \checkmark &              – &  \textsc{n/a} &           – \\
\bottomrule
\end{tabular}
\begin{legendlist}\item[\checkmark] affected
\item[×] not affected
\item[?] unknown first person prefix
\item[–] does not occur
\item[(\checkmark)] affected with surviving old marker
\item[\textsc{n/a}] not meaningfully answerable
\end{legendlist}\end{table}

These patterns suggest that there is some strong motivation for these verbs to not be affected by innovative markers.
To illustrate: the first three verbs all retained their old marker in at least five individual innovations.
The question arises what properties unite these verbs and make them so conservative.
We will discuss possible answers to this question in \cref{sec:motivations}, using \posscite{bybee1985morphology} network model of morphology.

\subsection{Possible motivations for archaicity}
\label{sec:motivations}
The most well-known contribution regarding irregularity in the lexicon is \textcite{bybee1985morphology} with her network model of morphology, which seems well-suited for the data at hand.
It aims \dbqu{to account for cross-linguistic, diachronic and acquisition patterns in complex morphological systems} \parencite[428]{bybee1995regular}.
It does so by modeling shared morphological properties such as inflectional patterns as emerging from connections of differing strength between related words in the mental lexicon.
For example, a large group of connected \dbqu{strong} English verbs with \obj{strɪŋ}--\obj{strʌŋ} at its center and pairs like \obj{rɪŋ}--\obj{rʌŋ}, \obj{spɪn}--\obj{spʌn}, or \obj{stɪk}--\obj{stʌk} at its periphery is attracting new verbs in certain dialects: \obj{sniːk}--\obj{snʌk} or \obj{brɪŋ}--\obj{brʌŋ} \parencite[129--130]{bybee1985morphology}.
These verbs are recruited based on the lexical connection they form with prototypical members of the group, and accordingly develop \dbqu{irregular} past tense forms.

For the causes of these lexical connections, \textcite[118]{bybee1985morphology} suggests the criteria of semantic similarity, phonological similarity, and morphological similarity -- the English strong verbs are an example for a phonologically motivated network.
Another important factor in the model is frequency, since more frequent words have a higher lexical strength \parencite[119]{bybee1985morphology}.
This higher lexical strength results in less influence from other lexemes, meaning that irregular forms are more likely to be preserved in high-frequency items.
Thus, from a diachronic perspective, the prediction is that a) semantically\slash{}phonologically\slash\hspace{0pt}morphologically similar verbs adapt the same morphological properties, and b) frequent verbs are more resistant to changes.

When one considers the resistant verbs in our Cariban problem, a rather salient morphological property emerges:
Most of the verbs lack a reflex of the detransitivizer \detrz, which normal \gl{s_a_} verbs do possess \pcref{sec:split}.
That is, there is a possible connection between presence of the detransitivizer and innovating new \gl{1}\gl{s_a_} markers.
This was already noted by \textcite{meira1998proto} for the group of \PTar verbs taking irregular first person \rc{w-}:
\begin{quotebox}{\parencite[112]{meira1998proto}}
	This category includes a small number of stems, among which ‘to go’, ‘to come’. ‘to say’, [...]\footnote{The original list includes \qu{to go down} and \qu{to defecate}. While these verbs are indeed underived \gl{s_a_} verbs in \trio, no irregular first person \rc{w-} can be reconstructed. As their inclusion in the list was most likely an error, we have omitted them here.} and the copula. These are basically the verbs that are not synchronically or diachronically detransitivized. yet belong to the A conjugation.
\end{quotebox} % ‘to go down’, ‘to defecate’,
%How well this morphological property explains which patterns will be discussedss in \cref{sec:morphology}.

Since regular \gl{s_a_} verbs all have reflexes of \detrz, they all begin with reflexes of \rc{ə} or \rc{e}.
That is, they are not only connected by their morphological makeup, but also by phonological criteria.
%In some cases, the initial segment of the verb stem seems to have indeed been the crucial factor \pcref{sec:phonology}.
Finally, at least the first four verbs in \cref{tab:overview} are also united by the fact that they are very likely the most frequent ones.
This has e.g.\ been noted for \kalina by \textcite[75]{courtz2008carib}: \dbqu{It is difficult […] to imagine an intransitive or transitive origin for some of the most frequent middle verbs}.
%Such possible frequency effects are discussed in \cref{sec:frequency}.

Semantic connections do not seem to play any roles, leaving us with three possibly relevant factors.
%It is often difficult to decide which of these factors best explains the pattern in a specific language.
%Rather, it seems that the three factors converge to a considerable degree in the data examined \pcref{sec:convergence}.

%\subsubsection{Morphology: \PWai and \PPek}
%\label{sec:morphology}
When considering the six individual extensions, the unaffected verbs can be characterized as lacking \detrz in the extensions described for \PWai and \PPek.
In \PWai, only four verbs were not affected, all without \detrz, while all other \gl{s_a_} verbs ()with \detrz) have regular \rc{k-}.
The case for a morphologically conditioned lexical network is even stronger in \PPek, where all seven unaffected verbs have no detransitivizing prefix, contrasting with a plethora of derived \gl{s_a_} verbs with \rc{k-}.
It is necessary for this analysis to assume that the idiosyncratic evolution of \rc{e-pɨ} \qu{to bathe (\gl{intr})} to \rc{ipɨ} effectively removed the transparent derivational prefix.
Further evidence comes from evolutions in the daughter languages.
First, we argued that both \ikpeng and \bakairi regularized the paradigm to use forms with detransitivizer for first person, which in both languages led to an introduction of \obj{k-}.%
\footnote{If one instead assumes that first person \rc{w-ebɨ-} and \rc{k-əd-ebɨ-} already co-existed in \PPek, the clear correlation remains.}
Second, the subsequent introduction of \obj{k-} to \ikpeng \obj{aran} \qu{to go} (< \rc{ɨtən}) could well be due to the reanalysis of \obj{ar} as a detransitivizer.

%\subsubsection{Phonology: \akuriyo, \carijo, \yukpa}
%\label{sec:phonology}
We have already discussed a case of a phonologically conditioned distribution of an innovative marker, \akuriyo \obj{k-} \pcref{sec:akuriyo}, which only appears on \obj{ə}-initial verbs.
In \carijo, the extension of \obj{j-} affected \obj{e}- and \obj{ə}-initial verbs, including \obj{eh} \qu{to come} or \obj{et͡ʃi} \qu{to be}, which do not have a detransitivizing prefix.
Only \obj{ka} \qu{say}, \obj{təmə} \qu{go}, and \obj{a} \qu{be-1} did not take on \obj{j-}.
Similarly, the extension of \yukpa \obj{j-} can succinctly be characterized as affecting all vowel-initial verbs; the only verb attested as unaffected is C-initial \obj{to} \qu{to go}.

%\subsubsection{Frequency: \PWai?}
%\label{sec:frequency}
\posscite{bybee1985morphology} model would predict high-frequency verbs to resist innovative person markers and thus remain archaic.
A major obstacle to investigating frequency effects is that we are not aware of frequency counts of individual lexemes for any Cariban language.
In absence of such data, we investigated frequency effects based on the following assumptions:
\begin{inlinelist}
	\item \qu{to say}, \qu{to go}, and \qu{to be} are the most frequent \gl{s_a_} verbs
	\item \qu{to come}, \qu{to go down} and \qu{to bathe} are much less frequent
\end{inlinelist}.
It should be kept in mind that in what follows, any statements about frequency effects are based on these assumptions.
One would expect the four frequent verbs to retain their old marker, and the three infrequent ones to innovate.
The extension of \PWai \rc{k-} perfectly fits this pattern; the four unaffected \gl{s_a_} verbs are the ones we assume to be the most frequent.

%\subsubsection{Comparing the factors}
%\label{sec:convergence}
We have shown that the extent of some extensions was determined by the presence or absence of a detransitivizing prefix, while others were phonologically conditioned.
Also, we have suggested that frequency effects may have led to verbs retaining their old marker in two cases, although our frequency grouping is very tentative and rudimentary.
However, we have only discussed cases where a given factor was able to account for all affected or unaffected verbs in a given extension.
\cref{tab:factors} shows, for each extension, how many cases of affected or unaffected verbs could be accounted for by each factor.
Of course, these counts only reflect the group of verbs in \cref{tab:overview}, i.e. those verbs that emerged as unaffected in any of the languages under study.
In each case, there are many run-of-the-mill \gl{s_a_} verbs which are regularly inflected.
While for most extensions there is a factor outperforming all others, these other factors are often also able to account for a large portion of (un-)affected verbs.
In one case, \PWai, both morphological properties and frequency effects account for 100\%.
However, it should be kept in mind that our interpretation of frequency effects

\begin{table}
\centering
\caption{Ratio of (un-)affected verbs explained by possible factors}
\label{tab:factors}
\begin{tabular}[t]{@{}llll@{}}
\mytoprule
{} & morphological & phonological &   frequency \\
\midrule
\PWai \rc{k-}     &    5/5 (100\%) &    4/5 (80\%) &  5/5 (100\%) \\
\PPek \rc{k-}     &    7/7 (100\%) &    5/7 (71\%) &   4/7 (57\%) \\
\PTir \rc{t-}     &     5/6 (83\%) &    4/6 (67\%) &   5/6 (83\%) \\
\akuriyo \obj{k-} &     5/6 (83\%) &   6/6 (100\%) &   3/6 (50\%) \\
\carijo \obj{j-}  &     3/5 (60\%) &   5/5 (100\%) &   4/5 (80\%) \\
\yukpa \obj{j-}   &     1/3 (33\%) &   3/3 (100\%) &   1/3 (33\%) \\
\bottomrule
\end{tabular}
\end{table}

Take for example the \dbqu{most} resistant verb, which is not attested as having taken on a new first person marker in any language, \rc{ka[ti]} \qu{to say}.
It is at the same time: \begin{inlinelist}
	\item highly frequent
	\item the only C-initial \gl{s_a_} verb in \PC
	\item one of the few \gl{s_a_} verbs without a reflex of \rc{ət(e)-}/\obj{e-}
\end{inlinelist}.
That is, one would expect it to resist morphological innovation based on all three factors: frequency, phonological form, and morphological makeup.

The same kind of convergence is found for most other verbs consistently emerging as resistant across the family.
They are more frequent than other \gl{s_a_} verbs, and diverge phonologically and morphologically from them.
This results in an overall picture where the factors leading to irregular or archaic morphological patterns strongly overlap, to a degree where one cannot simply decide which factor ultimately contributed.
While this means that \posscite{bybee1985morphology} model for the most part nicely accounts for the Cariban patterns, the conditioning factors she suggests are highly interrelated.

\subsection{Conclusion}

\begin{itemize}
	\item why are the most irregular verbs all underived \gl{s_a_} verbs? \textbf{something} is there
	\item ultimately plays into the mystery of how the hell the split-\gl{s} system actually came into being
	\item not surprising that more frequent \gl{s_a_} verbs have no \rc{ət-}, but definitely surprising that \qu{say}, \qu{go}, and \qu{be} are \gl{s_a_} verbs in the first place!
\end{itemize}

%Also: \dbqu{lexical […] splits constitute important evidence about the structure of the lexicon.} \parencite[119]{bybee1985morphology}.
%This suggests that the unaffected verbs form their own distinct part of the lexicon.
%In the case of the ones without \detrz, this means that they are distinct from the \dbqu{regular} ones (open class) in the lexicon.
