%The used theme. Options: No page number, progress bar at the bottom, no standalone frames for section titles
\usetheme[numbering=fraction,progressbar=none,sectionpage=none,block=fill,numbering=none]{metropolis}
%%This is used for smoothly blending the backgrounds of navigation and frame title
\useoutertheme[subsection=true]{smoothbars}
%This is to smoothly integrate the sections in the navigation
\setbeamercolor{section in head/foot}{fg=white,bg=mDarkTeal}
\setbeamercolor{subsection in head/foot}{fg=white,bg=mDarkTeal}
%
%
%%I don't really remember what this does. I guess it makes the header look good?
%\setbeamertemplate{headline}{%
%\begin{beamercolorbox}[colsep=1.5pt]{upper separation line head}
%\end{beamercolorbox}
%\begin{beamercolorbox}{section in head/foot}
%\insertsectionnavigationhorizontal{\paperwidth}{}{}
%\end{beamercolorbox}
%\begin{beamercolorbox}{subsection in head/foot}
%\insertsubsectionnavigationhorizontal{\paperwidth}{}{}
%\end{beamercolorbox}
%}
\setbeamertemplate{headline}{%
\begin{beamercolorbox}[colsep=1.5pt]{upper separation line head}
\end{beamercolorbox}
\begin{beamercolorbox}{section in head/foot}
    \vskip2pt\insertsectionnavigationhorizontal{\paperwidth}{}{\hskip0pt plus1filll}\vskip2pt
\end{beamercolorbox}%
\begin{beamercolorbox}[ht=10pt]{subsection in head/foot}%
    \vskip2pt\insertsubsectionnavigationhorizontal{\paperwidth}{}{\hskip0pt plus1filll}\vskip2pt
\end{beamercolorbox}%
\begin{beamercolorbox}[colsep=1.5pt]{lower separation line head}
\end{beamercolorbox}
}
%
%%Numbered definitions
%\setbeamertemplate{theorems}[numbered]
%
%%Puts the section numbers in circles in the ToC
%\setbeamertemplate{section in toc}[circle]
%We don't want frames that are broken into multiple slides to have these slides numbered
\setbeamertemplate{frametitle continuation}{}

%Changing fonts
\usepackage{fontspec}
%Not using the official Uni Bern font since there's gonna be lots of IPA. Using Calibri for true smallcaps.
%\setsansfont{Frutiger LT Com 55 Roman}
\setsansfont[SmallCapsFont=Calibri,SmallCapsFeatures={RawFeature=+smcp,Scale=1.1}]{DejaVu Sans}

%Drawing facilities
\usepackage{tikz}
\usetikzlibrary{positioning}
%For consistent quotation marks
\providecommand{\dbqu}[1]{``#1''}
\providecommand{\subs}[1]{\textsubscript{#1}}

%For formatting URLs
\usepackage{url}

%This package provides symbols
\usepackage[misc]{ifsym}
\usepackage{fontawesome}
%Author definition contains a mailto link
\author{Florian Matter \href{mailto:florian.matter@isw.unibe.ch}{\faEnvelopeO}}
\institute{Institut für Sprachwissenschaft, Universität Bern}

%For intelligent cross-references
\usepackage[noabbrev,nameinlink]{cleveref}

\usepackage{tabularx}
\usepackage{booktabs}
%Configure table rules for easier adaptation
\newcommand{\mytoprule}{\toprule}
\newcommand{\mymidrule}{\midrule}
\newcommand{\mybottomrule}{\bottomrule}

\usepackage{xspace}
\providecommand{\goodtilde}{\char`~\xspace}

%For multicolumn lists on slides
\usepackage{multicol}

\usepackage{tikz-qtree}
\usetikzlibrary{tikzmark}

%Just for emphasis
\newcommand{\emp}[1]{\textbf{#1}}

\hypersetup{
    colorlinks = true,
    linkcolor = {orange},
    urlcolor = {orange},
    citecolor = {orange},
    filecolor = {orange},
    linkbordercolor = {orange},
}

%Enables underlined and colored links
\makeatletter
\Hy@AtBeginDocument{%
  \def\@pdfborder{0 0 1}% Overrides border definition set with colorlinks=true
  \def\@pdfborderstyle{/S/U/W 1}% Overrides border style set with colorlinks=true
                                % Hyperlink border style will be underline of width 1pt
}
\makeatother